\section{Heat Diffusion Bencmark}
\subsection{Problem description}
This example tests the ability of a code to represent heat dissipation by diffusion alone. The problem is defined for a solid with Dirichlet conditions on the left and right sides of the \SI[scientific-notation=false, round-precision=2]{100.0}{\metre} long by \SI[scientific-notation=false, round-precision=2]{10.0}{\metre} thick domain (Figure~\ref{fig:heatDiffFig}<++>). The initial initial condition is uniform temperature of \SI[scientific-notation=false, round-precision=2]{0.0}{\degreeCelsius}. Output data for x-axis pressures are calculated for times of \SIlist[round-precision=2]{1.0e6; 1.0e7}{\second}. %$10^6$ and $10^7$ seconds.1E6 and 1E7 seconds.
\begin{figure}[h]
	\centering
	\import{./figures/}{heatDiffFig.pdf_tex}
	\caption{Model domain}
	\label{fig:heatDiffFig}
\end{figure}

The analytical solution applied is for diffusion in a plane sheet in a semi-infiite domain:
\begin{equation}
	T(x,t)=\erf\left(\frac{x}{2\sqrt{D_{\text{eff}}t}} \right)
	\label{eq:errorFunc}
\end{equation}
The solution was computed using Mathcad, Version 15, using parameter values given below.

\begin{table}[h]
	\caption{Material properties}
	\begin{center}
	\begin{tabular}{lrcSs}
		Thermal conductivity of fluid & $\lambda_f$ & $\coloneqq$ & 0.68 & \si{\watt\per\metre\per\kelvin} \\
		Thermal conductivity of rock & $\lambda_r$ & $\coloneqq$ & 2.955 & \si{\watt\per\metre\per\kelvin} \\
		Specific heat of fluid & $c_{p,f}$ & $\coloneqq$ &4186.0  & \si{\joule\per\kilogram\per\kelvin} \\
		Specific heat of rock & $c_{p,r}$ & $\coloneqq$ & 920.0 & \si{\joule\per\kilogram\kelvin} \\
		Density of fluid & $\rho_f$ & $\coloneqq$ & 1000.0 & \si{\kilogram\per\metre\cubed} \\
		Density of rock & $\rho_r$ & $\coloneqq$ &  2500.0 & \si{\kilogram\per\metre\cubed} \\
		Porosity & $\theta$ & $\coloneqq$ & 0.20 &  
	\end{tabular}
	\end{center}
	\label{tab:heatDiffResPar}
\end{table}

\begin{table}[h]
	\caption{Domain parameters}
	\begin{center}
	\begin{tabular}{lrcSs}
		Initial temperature & $T_0$ & $\coloneqq$ & 0.0 & \si{\degreeCelsius} \\
		Injection temperature & $T_{\text{inj}}$ & $\coloneqq$ & 1.0 & \si{\degreeCelsius} \\
		Flow length & $X$ & $\coloneqq$ & 10.0 & \si{\metre} \\
		Simulation end time& $t_1$ & $\coloneqq$ & 1.0e5  & \si{\second} \\
		& $t_2$ & $\coloneqq$ & 5.0e5  & \si{\second} 
	\end{tabular}
	\end{center}
	\label{tab:heatDiffDomPar}
\end{table}

\setlength{\tabcolsep}{3pt}
\begin{table}[h]
	\caption{Derived parameters}
	\begin{center}
		\begin{tabular}{lrclcSs}
			Eff. thermal cond. & $\lambda_{\text{eff}}$ & $\coloneqq$ & $\lambda_f\theta+\lambda_r(1-\theta)$ & $=$ &2.5 & \si{\watt\per\metre\per\kelvin} \\
 		Eff. heat capacity & $c_{p,\text{eff}}$ & $\coloneqq$ & $\rho_fc_{p,f}\theta+\rho_r c_{p,r}(1-\theta)$ & $=$ & 2677200.0 & \si{\joule\per\metre\cubed\per\kelvin} \\
		Thermal diffusivity & $D_{\text{eff}}$ & $\coloneqq$ & $\dfrac{\lambda_{\text{eff}}}{\rho_fc_{p,f}} $ & $=$ & 5.972288580984234e-07 & \si{\metre\squared\per\second} \\
		Thermal retardation & $R$ & $\coloneqq$ & $\dfrac{c_{p,\text{eff}}}{\theta\rho_fc_{p,f}}$ & $=$ & 3.197802197802198 & \\
	\end{tabular}
	\end{center}
	\label{tab:heatDiffDerivPar}
\end{table}

\subsection{Files}
\begin{itemize}
	\item Analytical results are stored in \verb|HeatDiffusionBM-analytical.xlsx|
	\item Numerical results are stored in \verb|HeatDiffusionBM-FALCON.xlsx|
	\item The numerical solution input file is \verb|H1_heat_diffusion.i|
	\item The numerical solution on exodus output file is \verb|H1_heat_diffusion_out.e|
\end{itemize}

\subsection{Results}
\begin{figure}[h!]
	\begin{center}
		\setlength\figureheight{8cm} 
		\setlength\figurewidth{0.8\textwidth} 
		% This file was created by matplotlib v0.1.0.
% Copyright (c) 2010--2014, Nico Schlömer <nico.schloemer@gmail.com>
% All rights reserved.
% 

\begin{tikzpicture}

\begin{axis}[
xlabel={Distance [m]},
ylabel={Temperature [$^\circ$C]},
xmin=0, xmax=10,
ymin=0, ymax=1,
axis on top,
width=\figurewidth,
height=\figureheight,
legend entries={
{Analytic, $t=$\SI[round-precision=2]{1.0e6}{\second}},
{Analytic, $t=$\SI[round-precision=2]{1.0e7}{\second} },
{FALCON, $t=$\SI[round-precision=2]{1.0e6}{\second}},
{FALCON, $t=$\SI[round-precision=2]{1.0e7}{\second}}},
legend style={at={(0.97,0.03)}, anchor=south east}
]
\addplot [red]
coordinates {
(0,0)
(0.01001001001001,0.00584421380581384)
(0.02002002002002,0.0116881140720713)
(0.03003003003003,0.0175313873096787)
(0.04004004004004,0.023373720130455)
(0.05005005005005,0.0292147992975531)
(0.0600600600600601,0.0350543117758427)
(0.0700700700700701,0.0408919447822369)
(0.0800800800800801,0.0467273858359527)
(0.0900900900900901,0.0525603228086901)
(0.1001001001001,0.0583904439747161)
(0.11011011011011,0.0642174380608422)
(0.12012012012012,0.0700409942962786)
(0.13013013013013,0.0758608024623565)
(0.14014014014014,0.0816765529421002)
(0.15015015015015,0.0874879367696396)
(0.16016016016016,0.0932946456794483)
(0.17017017017017,0.0990963721553944)
(0.18018018018018,0.104892809479591)
(0.19019019019019,0.110683651781034)
(0.2002002002002,0.116468594084012)
(0.21021021021021,0.122247332356282)
(0.22022022022022,0.128019563556988)
(0.23023023023023,0.133784985684321)
(0.24024024024024,0.139543297822894)
(0.25025025025025,0.145294200190841)
(0.26026026026026,0.151037394186604)
(0.27027027027027,0.156772582435411)
(0.28028028028028,0.162499468835427)
(0.29029029029029,0.168217758603568)
(0.3003003003003,0.173927158320971)
(0.31031031031031,0.179627375978094)
(0.32032032032032,0.185318121019449)
(0.33033033033033,0.190999104387955)
(0.34034034034034,0.196670038568885)
(0.35035035035035,0.20233063763342)
(0.36036036036036,0.207980617281774)
(0.37037037037037,0.213619694885901)
(0.38038038038038,0.219247589531759)
(0.39039039039039,0.224864022061121)
(0.4004004004004,0.230468715112935)
(0.41041041041041,0.236061393164209)
(0.42042042042042,0.241641782570419)
(0.43043043043043,0.247209611605422)
(0.44044044044044,0.252764610500881)
(0.45045045045045,0.258306511485172)
(0.46046046046046,0.263835048821775)
(0.47047047047047,0.269349958847144)
(0.48048048048048,0.274850980008033)
(0.49049049049049,0.280337852898284)
(0.500500500500501,0.285810320295061)
(0.510510510510511,0.29126812719452)
(0.520520520520521,0.296711020846914)
(0.530530530530531,0.302138750791115)
(0.540540540540541,0.307551068888558)
(0.550550550550551,0.312947729356591)
(0.560560560560561,0.318328488801219)
(0.570570570570571,0.323693106249249)
(0.580580580580581,0.329041343179823)
(0.590590590590591,0.33437296355532)
(0.600600600600601,0.339687733851642)
(0.610610610610611,0.344985423087866)
(0.620620620620621,0.350265802855249)
(0.630630630630631,0.355528647345604)
(0.640640640640641,0.360773733379007)
(0.650650650650651,0.366000840430869)
(0.660660660660661,0.371209750658328)
(0.670670670670671,0.376400248925994)
(0.680680680680681,0.381572122831013)
(0.690690690690691,0.386725162727459)
(0.700700700700701,0.39185916175005)
(0.710710710710711,0.396973915837183)
(0.720720720720721,0.402069223753279)
(0.730730730730731,0.407144887110445)
(0.740740740740741,0.412200710389435)
(0.750750750750751,0.417236500959926)
(0.760760760760761,0.422252069100087)
(0.770770770770771,0.427247228015454)
(0.780780780780781,0.4322217938571)
(0.790790790790791,0.437175585739099)
(0.800800800800801,0.442108425755283)
(0.810810810810811,0.447020138995296)
(0.820820820820821,0.451910553559931)
(0.830830830830831,0.456779500575766)
(0.840840840840841,0.461626814209076)
(0.850850850850851,0.466452331679046)
(0.860860860860861,0.471255893270264)
(0.870870870870871,0.476037342344501)
(0.880880880880881,0.480796525351785)
(0.890890890890891,0.485533291840756)
(0.900900900900901,0.490247494468313)
(0.910910910910911,0.494938989008547)
(0.920920920920921,0.499607634360971)
(0.930930930930931,0.50425329255803)
(0.940940940940941,0.508875828771919)
(0.950950950950951,0.513475111320676)
(0.960960960960961,0.518051011673591)
(0.970970970970971,0.522603404455901)
(0.980980980980981,0.527132167452786)
(0.990990990990991,0.531637181612674)
(1.001001001001,0.536118331049849)
(1.01101101101101,0.540575503046365)
(1.02102102102102,0.545008588053276)
(1.03103103103103,0.549417479691179)
(1.04104104104104,0.553802074750076)
(1.05105105105105,0.558162273188559)
(1.06106106106106,0.562497978132319)
(1.07107107107107,0.566809095871992)
(1.08108108108108,0.57109553586033)
(1.09109109109109,0.575357210708723)
(1.1011011011011,0.579594036183052)
(1.11111111111111,0.583805931198902)
(1.12112112112112,0.587992817816126)
(1.13113113113113,0.592154621232762)
(1.14114114114114,0.596291269778325)
(1.15115115115115,0.60040269490646)
(1.16116116116116,0.604488831186977)
(1.17117117117117,0.608549616297265)
(1.18118118118118,0.612584991013095)
(1.19119119119119,0.616594899198817)
(1.2012012012012,0.620579287796957)
(1.21121121121121,0.624538106817224)
(1.22122122122122,0.628471309324925)
(1.23123123123123,0.632378851428808)
(1.24124124124124,0.636260692268326)
(1.25125125125125,0.640116794000342)
(1.26126126126126,0.643947121785269)
(1.27127127127127,0.647751643772669)
(1.28128128128128,0.651530331086296)
(1.29129129129129,0.655283157808618)
(1.3013013013013,0.659010100964802)
(1.31131131131131,0.662711140506179)
(1.32132132132132,0.666386259293203)
(1.33133133133133,0.670035443077903)
(1.34134134134134,0.673658680485838)
(1.35135135135135,0.677255962997569)
(1.36136136136136,0.680827284929648)
(1.37137137137137,0.684372643415139)
(1.38138138138138,0.687892038383679)
(1.39139139139139,0.691385472541082)
(1.4014014014014,0.6948529513485)
(1.41141141141141,0.698294483001152)
(1.42142142142142,0.701710078406623)
(1.43143143143143,0.705099751162746)
(1.44144144144144,0.708463517535077)
(1.45145145145145,0.711801396433966)
(1.46146146146146,0.715113409391248)
(1.47147147147147,0.718399580536542)
(1.48148148148148,0.721659936573181)
(1.49149149149149,0.724894506753789)
(1.5015015015015,0.728103322855488)
(1.51151151151151,0.731286419154778)
(1.52152152152152,0.734443832402073)
(1.53153153153153,0.737575601795916)
(1.54154154154154,0.740681768956876)
(1.55155155155155,0.743762377901142)
(1.56156156156156,0.746817475013823)
(1.57157157157157,0.749847109021952)
(1.58158158158158,0.752851330967228)
(1.59159159159159,0.755830194178475)
(1.6016016016016,0.758783754243854)
(1.61161161161161,0.761712068982822)
(1.62162162162162,0.764615198417853)
(1.63163163163163,0.767493204745933)
(1.64164164164164,0.77034615230983)
(1.65165165165165,0.77317410756916)
(1.66166166166166,0.775977139071248)
(1.67167167167167,0.778755317421799)
(1.68168168168168,0.781508715255392)
(1.69169169169169,0.784237407205795)
(1.7017017017017,0.786941469876126)
(1.71171171171171,0.789620981808854)
(1.72172172172172,0.792276023455657)
(1.73173173173173,0.794906677147152)
(1.74174174174174,0.797513027062495)
(1.75175175175175,0.800095159198864)
(1.76176176176176,0.80265316134084)
(1.77177177177177,0.805187123029688)
(1.78178178178178,0.80769713553255)
(1.79179179179179,0.810183291811561)
(1.8018018018018,0.812645686492893)
(1.81181181181181,0.815084415835734)
(1.82182182182182,0.817499577701225)
(1.83183183183183,0.819891271521344)
(1.84184184184184,0.822259598267759)
(1.85185185185185,0.824604660420651)
(1.86186186186186,0.826926561937531)
(1.87187187187187,0.82922540822203)
(1.88188188188188,0.831501306092705)
(1.89189189189189,0.833754363751842)
(1.9019019019019,0.83598469075428)
(1.91191191191191,0.838192397976258)
(1.92192192192192,0.840377597584293)
(1.93193193193193,0.842540403004098)
(1.94194194194194,0.844680928889552)
(1.95195195195195,0.846799291091721)
(1.96196196196196,0.848895606627943)
(1.97197197197197,0.850969993650989)
(1.98198198198198,0.853022571418295)
(1.99199199199199,0.855053460261287)
(2.002002002002,0.857062781554796)
(2.01201201201201,0.859050657686571)
(2.02202202202202,0.861017212026906)
(2.03203203203203,0.862962568898377)
(2.04204204204204,0.864886853545698)
(2.05205205205205,0.866790192105711)
(2.06206206206206,0.868672711577508)
(2.07207207207207,0.870534539792692)
(2.08208208208208,0.87237580538579)
(2.09209209209209,0.874196637764816)
(2.1021021021021,0.875997167081994)
(2.11211211211211,0.877777524204652)
(2.12212212212212,0.879537840686284)
(2.13213213213213,0.881278248737787)
(2.14214214214214,0.882998881198887)
(2.15215215215215,0.884699871509753)
(2.16216216216216,0.886381353682803)
(2.17217217217217,0.888043462274709)
(2.18218218218218,0.889686332358612)
(2.19219219219219,0.891310099496542)
(2.2022022022022,0.892914899712056)
(2.21221221221221,0.894500869463092)
(2.22222222222222,0.896068145615053)
(2.23223223223223,0.897616865414113)
(2.24224224224224,0.899147166460764)
(2.25225225225225,0.900659186683593)
(2.26226226226226,0.90215306431331)
(2.27227227227227,0.903628937857009)
(2.28228228228228,0.905086946072691)
(2.29229229229229,0.906527227944038)
(2.3023023023023,0.907949922655434)
(2.31231231231231,0.909355169567264)
(2.32232232232232,0.910743108191462)
(2.33233233233233,0.912113878167335)
(2.34234234234234,0.913467619237654)
(2.35235235235235,0.91480447122502)
(2.36236236236236,0.916124574008509)
(2.37237237237237,0.917428067500587)
(2.38238238238238,0.918715091624324)
(2.39239239239239,0.919985786290871)
(2.4024024024024,0.921240291377244)
(2.41241241241241,0.922478746704385)
(2.42242242242242,0.923701292015517)
(2.43243243243243,0.924908066954796)
(2.44244244244244,0.926099211046254)
(2.45245245245245,0.927274863673043)
(2.46246246246246,0.928435164056971)
(2.47247247247247,0.92958025123835)
(2.48248248248248,0.930710264056135)
(2.49249249249249,0.931825341128373)
(2.5025025025025,0.932925620832952)
(2.51251251251251,0.93401124128866)
(2.52252252252252,0.935082340336545)
(2.53253253253253,0.936139055521589)
(2.54254254254254,0.937181524074687)
(2.55255255255255,0.938209882894932)
(2.56256256256256,0.939224268532217)
(2.57257257257257,0.940224817170139)
(2.58258258258258,0.941211664609217)
(2.59259259259259,0.942184946250423)
(2.6026026026026,0.943144797079019)
(2.61261261261261,0.944091351648708)
(2.62262262262262,0.945024744066093)
(2.63263263263263,0.94594510797545)
(2.64264264264264,0.946852576543807)
(2.65265265265265,0.947747282446339)
(2.66266266266266,0.948629357852064)
(2.67267267267267,0.949498934409857)
(2.68268268268268,0.950356143234766)
(2.69269269269269,0.951201114894637)
(2.7027027027027,0.952033979397051)
(2.71271271271271,0.952854866176554)
(2.72272272272272,0.953663904082208)
(2.73273273273273,0.954461221365436)
(2.74274274274274,0.955246945668171)
(2.75275275275275,0.956021204011306)
(2.76276276276276,0.95678412278345)
(2.77277277277277,0.957535827729975)
(2.78278278278278,0.958276443942366)
(2.79279279279279,0.959006095847862)
(2.8028028028028,0.959724907199398)
(2.81281281281281,0.960433001065831)
(2.82282282282282,0.961130499822465)
(2.83283283283283,0.961817525141862)
(2.84284284284284,0.962494197984934)
(2.85285285285285,0.963160638592334)
(2.86286286286286,0.963816966476115)
(2.87287287287287,0.964463300411683)
(2.88288288288288,0.965099758430018)
(2.89289289289289,0.965726457810183)
(2.9029029029029,0.966343515072095)
(2.91291291291291,0.966951045969584)
(2.92292292292292,0.967549165483706)
(2.93293293293293,0.968137987816335)
(2.94294294294294,0.968717626384014)
(2.95295295295295,0.969288193812074)
(2.96296296296296,0.969849801929007)
(2.97297297297297,0.970402561761102)
(2.98298298298298,0.970946583527335)
(2.99299299299299,0.971481976634506)
(3.003003003003,0.972008849672636)
(3.01301301301301,0.972527310410602)
(3.02302302302302,0.973037465792017)
(3.03303303303303,0.97353942193136)
(3.04304304304304,0.974033284110332)
(3.05305305305305,0.974519156774457)
(3.06306306306306,0.974997143529916)
(3.07307307307307,0.975467347140599)
(3.08308308308308,0.975929869525403)
(3.09309309309309,0.976384811755738)
(3.1031031031031,0.976832274053261)
(3.11311311311311,0.977272355787826)
(3.12312312312312,0.977705155475653)
(3.13313313313313,0.978130770777702)
(3.14314314314314,0.978549298498263)
(3.15315315315315,0.978960834583745)
(3.16316316316316,0.979365474121673)
(3.17317317317317,0.979763311339884)
(3.18318318318318,0.980154439605915)
(3.19319319319319,0.980538951426591)
(3.2032032032032,0.980916938447793)
(3.21321321321321,0.98128849145443)
(3.22322322322322,0.981653700370573)
(3.23323323323323,0.982012654259789)
(3.24324324324324,0.982365441325641)
(3.25325325325325,0.982712148912367)
(3.26326326326326,0.98305286350573)
(3.27327327327327,0.983387670734032)
(3.28328328328328,0.983716655369298)
(3.29329329329329,0.984039901328621)
(3.3033033033033,0.98435749167566)
(3.31331331331331,0.984669508622303)
(3.32332332332332,0.98497603353047)
(3.33333333333333,0.985277146914077)
(3.34334334334334,0.985572928441139)
(3.35335335335335,0.985863456936015)
(3.36336336336336,0.986148810381799)
(3.37337337337337,0.98642906592284)
(3.38338338338338,0.986704299867399)
(3.39339339339339,0.98697458769044)
(3.4034034034034,0.98724000403654)
(3.41341341341341,0.98750062272293)
(3.42342342342342,0.987756516742656)
(3.43343343343343,0.988007758267854)
(3.44344344344344,0.988254418653145)
(3.45345345345345,0.98849656843914)
(3.46346346346346,0.988734277356057)
(3.47347347347347,0.988967614327436)
(3.48348348348348,0.989196647473968)
(3.49349349349349,0.989421444117419)
(3.5035035035035,0.989642070784648)
(3.51351351351351,0.989858593211729)
(3.52352352352352,0.990071076348153)
(3.53353353353353,0.990279584361132)
(3.54354354354354,0.990484180639976)
(3.55355355355355,0.990684927800564)
(3.56356356356356,0.990881887689889)
(3.57357357357357,0.991075121390684)
(3.58358358358358,0.991264689226123)
(3.59359359359359,0.991450650764592)
(3.6036036036036,0.991633064824533)
(3.61361361361361,0.991811989479356)
(3.62362362362362,0.99198748206241)
(3.63363363363363,0.992159599172021)
(3.64364364364364,0.992328396676592)
(3.65365365365365,0.99249392971975)
(3.66366366366366,0.992656252725561)
(3.67367367367367,0.992815419403786)
(3.68368368368368,0.992971482755193)
(3.69369369369369,0.993124495076914)
(3.7037037037037,0.993274507967848)
(3.71371371371371,0.993421572334105)
(3.72372372372372,0.993565738394496)
(3.73373373373373,0.99370705568605)
(3.74374374374374,0.993845573069583)
(3.75375375375375,0.993981338735283)
(3.76376376376376,0.994114400208343)
(3.77377377377377,0.994244804354611)
(3.78378378378378,0.994372597386276)
(3.79379379379379,0.994497824867571)
(3.8038038038038,0.994620531720511)
(3.81381381381381,0.994740762230642)
(3.82382382382382,0.99485856005281)
(3.83383383383383,0.994973968216957)
(3.84384384384384,0.995087029133922)
(3.85385385385385,0.995197784601261)
(3.86386386386386,0.995306275809076)
(3.87387387387387,0.995412543345861)
(3.88388388388388,0.995516627204344)
(3.89389389389389,0.99561856678735)
(3.9039039039039,0.995718400913656)
(3.91391391391391,0.995816167823861)
(3.92392392392392,0.995911905186246)
(3.93393393393393,0.996005650102644)
(3.94394394394394,0.996097439114303)
(3.95395395395395,0.99618730820775)
(3.96396396396396,0.996275292820641)
(3.97397397397397,0.996361427847621)
(3.98398398398398,0.996445747646162)
(3.99399399399399,0.996528286042398)
(4.004004004004,0.996609076336952)
(4.01401401401401,0.996688151310749)
(4.02402402402402,0.996765543230813)
(4.03403403403403,0.99684128385606)
(4.04404404404404,0.996915404443061)
(4.05405405405405,0.996987935751803)
(4.06406406406406,0.997058908051424)
(4.07407407407407,0.99712835112593)
(4.08408408408408,0.997196294279893)
(4.09409409409409,0.997262766344132)
(4.1041041041041,0.997327795681364)
(4.11411411411411,0.997391410191837)
(4.12412412412412,0.997453637318939)
(4.13413413413413,0.997514504054783)
(4.14414414414414,0.997574036945758)
(4.15415415415415,0.997632262098065)
(4.16416416416416,0.997689205183217)
(4.17417417417417,0.99774489144351)
(4.18418418418418,0.997799345697475)
(4.19419419419419,0.997852592345283)
(4.2042042042042,0.997904655374134)
(4.21421421421421,0.997955558363607)
(4.22422422422422,0.998005324490978)
(4.23423423423423,0.998053976536506)
(4.24424424424424,0.998101536888686)
(4.25425425425425,0.998148027549464)
(4.26426426426426,0.998193470139423)
(4.27427427427427,0.998237885902927)
(4.28428428428428,0.998281295713233)
(4.29429429429429,0.998323720077563)
(4.3043043043043,0.998365179142143)
(4.31431431431431,0.998405692697201)
(4.32432432432432,0.998445280181928)
(4.33433433433433,0.998483960689396)
(4.34434434434434,0.998521752971449)
(4.35435435435435,0.998558675443538)
(4.36436436436436,0.99859474618953)
(4.37437437437437,0.998629982966468)
(4.38438438438438,0.998664403209297)
(4.39439439439439,0.998698024035543)
(4.4044044044044,0.998730862249956)
(4.41441441441441,0.998762934349105)
(4.42442442442442,0.998794256525944)
(4.43443443443443,0.998824844674318)
(4.44444444444444,0.998854714393442)
(4.45445445445445,0.998883880992332)
(4.46446446446446,0.998912359494192)
(4.47447447447447,0.998940164640761)
(4.48448448448448,0.998967310896614)
(4.49449449449449,0.998993812453426)
(4.5045045045045,0.999019683234186)
(4.51451451451451,0.999044936897371)
(4.52452452452452,0.999069586841077)
(4.53453453453453,0.999093646207108)
(4.54454454454454,0.999117127885018)
(4.55455455455455,0.999140044516111)
(4.56456456456456,0.9991624084974)
(4.57457457457457,0.999184231985518)
(4.58458458458458,0.999205526900592)
(4.59459459459459,0.999226304930066)
(4.6046046046046,0.999246577532487)
(4.61461461461461,0.999266355941245)
(4.62462462462462,0.999285651168267)
(4.63463463463463,0.999304474007676)
(4.64464464464464,0.999322835039395)
(4.65465465465465,0.999340744632722)
(4.66466466466466,0.999358212949848)
(4.67467467467467,0.999375249949344)
(4.68468468468468,0.999391865389596)
(4.69469469469469,0.999408068832208)
(4.7047047047047,0.99942386964535)
(4.71471471471471,0.999439277007072)
(4.72472472472472,0.999454299908579)
(4.73473473473473,0.999468947157453)
(4.74474474474474,0.999483227380844)
(4.75475475475475,0.999497149028614)
(4.76476476476476,0.999510720376441)
(4.77477477477477,0.999523949528881)
(4.78478478478478,0.999536844422395)
(4.79479479479479,0.999549412828324)
(4.8048048048048,0.999561662355835)
(4.81481481481481,0.999573600454822)
(4.82482482482482,0.999585234418764)
(4.83483483483483,0.999596571387551)
(4.84484484484484,0.999607618350262)
(4.85485485485485,0.999618382147914)
(4.86486486486486,0.999628869476162)
(4.87487487487487,0.999639086887964)
(4.88488488488488,0.999649040796214)
(4.89489489489489,0.999658737476328)
(4.9049049049049,0.999668183068797)
(4.91491491491491,0.9996773835817)
(4.92492492492492,0.999686344893186)
(4.93493493493493,0.99969507275391)
(4.94494494494494,0.999703572789441)
(4.95495495495495,0.999711850502628)
(4.96496496496496,0.999719911275935)
(4.97497497497497,0.999727760373735)
(4.98498498498498,0.999735402944574)
(4.99499499499499,0.999742844023397)
(5.00500500500501,0.999750088533739)
(5.01501501501502,0.999757141289886)
(5.02502502502503,0.999764006999)
(5.03503503503504,0.999770690263203)
(5.04504504504505,0.999777195581642)
(5.05505505505506,0.999783527352512)
(5.06506506506507,0.999789689875045)
(5.07507507507508,0.999795687351472)
(5.08508508508509,0.999801523888952)
(5.0950950950951,0.999807203501466)
(5.10510510510511,0.999812730111687)
(5.11511511511512,0.999818107552808)
(5.12512512512513,0.999823339570352)
(5.13513513513514,0.999828429823943)
(5.14514514514515,0.999833381889052)
(5.15515515515516,0.999838199258708)
(5.16516516516517,0.999842885345188)
(5.17517517517518,0.999847443481672)
(5.18518518518519,0.999851876923872)
(5.1951951951952,0.999856188851632)
(5.20520520520521,0.999860382370498)
(5.21521521521522,0.999864460513269)
(5.22522522522523,0.999868426241512)
(5.23523523523524,0.999872282447052)
(5.24524524524525,0.999876031953441)
(5.25525525525526,0.999879677517393)
(5.26526526526527,0.999883221830202)
(5.27527527527528,0.999886667519127)
(5.28528528528529,0.999890017148754)
(5.2952952952953,0.99989327322234)
(5.30530530530531,0.999896438183122)
(5.31531531531532,0.999899514415608)
(5.32532532532533,0.999902504246846)
(5.33533533533534,0.999905409947665)
(5.34534534534535,0.999908233733895)
(5.35535535535536,0.999910977767566)
(5.36536536536537,0.99991364415808)
(5.37537537537538,0.999916234963369)
(5.38538538538539,0.999918752191022)
(5.3953953953954,0.999921197799394)
(5.40540540540541,0.999923573698701)
(5.41541541541542,0.999925881752081)
(5.42542542542543,0.999928123776644)
(5.43543543543544,0.999930301544501)
(5.44544544544545,0.999932416783771)
(5.45545545545546,0.999934471179564)
(5.46546546546547,0.999936466374958)
(5.47547547547548,0.999938403971939)
(5.48548548548549,0.999940285532343)
(5.4954954954955,0.999942112578756)
(5.50550550550551,0.99994388659542)
(5.51551551551552,0.999945609029103)
(5.52552552552553,0.999947281289958)
(5.53553553553554,0.99994890475237)
(5.54554554554555,0.999950480755774)
(5.55555555555556,0.99995201060547)
(5.56556556556557,0.99995349557341)
(5.57557557557558,0.999954936898975)
(5.58558558558559,0.999956335789737)
(5.5955955955956,0.9999576934222)
(5.60560560560561,0.999959010942531)
(5.61561561561562,0.999960289467272)
(5.62562562562563,0.999961530084043)
(5.63563563563564,0.99996273385222)
(5.64564564564565,0.999963901803611)
(5.65565565565566,0.999965034943109)
(5.66566566566567,0.999966134249335)
(5.67567567567568,0.999967200675267)
(5.68568568568569,0.999968235148852)
(5.6956956956957,0.999969238573614)
(5.70570570570571,0.999970211829239)
(5.71571571571572,0.999971155772152)
(5.72572572572573,0.999972071236082)
(5.73573573573574,0.999972959032616)
(5.74574574574575,0.999973819951733)
(5.75575575575576,0.999974654762339)
(5.76576576576577,0.99997546421278)
(5.77577577577578,0.99997624903135)
(5.78578578578579,0.999977009926779)
(5.7957957957958,0.999977747588726)
(5.80580580580581,0.999978462688245)
(5.81581581581582,0.999979155878246)
(5.82582582582583,0.999979827793954)
(5.83583583583584,0.999980479053344)
(5.84584584584585,0.999981110257574)
(5.85585585585586,0.999981721991413)
(5.86586586586587,0.999982314823644)
(5.87587587587588,0.999982889307474)
(5.88588588588589,0.999983445980928)
(5.8958958958959,0.99998398536723)
(5.90590590590591,0.999984507975184)
(5.91591591591592,0.999985014299536)
(5.92592592592593,0.99998550482134)
(5.93593593593594,0.999985980008303)
(5.94594594594595,0.999986440315131)
(5.95595595595596,0.999986886183863)
(5.96596596596597,0.999987318044197)
(5.97597597597598,0.99998773631381)
(5.98598598598599,0.999988141398672)
(5.995995995996,0.999988533693347)
(6.00600600600601,0.99998891358129)
(6.01601601601602,0.999989281435142)
(6.02602602602603,0.999989637617009)
(6.03603603603604,0.999989982478739)
(6.04604604604605,0.999990316362194)
(6.05605605605606,0.999990639599513)
(6.06606606606607,0.999990952513367)
(6.07607607607608,0.999991255417213)
(6.08608608608609,0.999991548615536)
(6.0960960960961,0.999991832404092)
(6.10610610610611,0.999992107070136)
(6.11611611611612,0.999992372892654)
(6.12612612612613,0.999992630142581)
(6.13613613613614,0.999992879083022)
(6.14614614614615,0.999993119969459)
(6.15615615615616,0.999993353049958)
(6.16616616616617,0.999993578565374)
(6.17617617617618,0.999993796749537)
(6.18618618618619,0.999994007829456)
(6.1961961961962,0.999994212025493)
(6.20620620620621,0.999994409551553)
(6.21621621621622,0.999994600615256)
(6.22622622622623,0.999994785418114)
(6.23623623623624,0.999994964155693)
(6.24624624624625,0.999995137017784)
(6.25625625625626,0.999995304188558)
(6.26626626626627,0.999995465846722)
(6.27627627627628,0.999995622165672)
(6.28628628628629,0.999995773313642)
(6.2962962962963,0.999995919453845)
(6.30630630630631,0.999996060744616)
(6.31631631631632,0.999996197339547)
(6.32632632632633,0.999996329387621)
(6.33633633633634,0.999996457033341)
(6.34634634634635,0.999996580416857)
(6.35635635635636,0.999996699674091)
(6.36636636636637,0.999996814936851)
(6.37637637637638,0.999996926332955)
(6.38638638638639,0.999997033986339)
(6.3963963963964,0.999997138017171)
(6.40640640640641,0.999997238541956)
(6.41641641641642,0.999997335673645)
(6.42642642642643,0.999997429521732)
(6.43643643643644,0.999997520192357)
(6.44644644644645,0.9999976077884)
(6.45645645645646,0.99999769240958)
(6.46646646646647,0.999997774152539)
(6.47647647647648,0.999997853110939)
(6.48648648648649,0.999997929375543)
(6.4964964964965,0.999998003034303)
(6.50650650650651,0.999998074172439)
(6.51651651651652,0.999998142872522)
(6.52652652652653,0.99999820921455)
(6.53653653653654,0.999998273276024)
(6.54654654654655,0.99999833513202)
(6.55655655655656,0.999998394855263)
(6.56656656656657,0.999998452516196)
(6.57657657657658,0.999998508183045)
(6.58658658658659,0.999998561921889)
(6.5965965965966,0.999998613796721)
(6.60660660660661,0.999998663869509)
(6.61661661661662,0.999998712200261)
(6.62662662662663,0.99999875884708)
(6.63663663663664,0.999998803866222)
(6.64664664664665,0.99999884731215)
(6.65665665665666,0.999998889237593)
(6.66666666666667,0.999998929693593)
(6.67667667667668,0.999998968729557)
(6.68668668668669,0.99999900639331)
(6.6966966966967,0.999999042731137)
(6.70670670670671,0.999999077787837)
(6.71671671671672,0.999999111606763)
(6.72672672672673,0.999999144229868)
(6.73673673673674,0.999999175697746)
(6.74674674674675,0.999999206049677)
(6.75675675675676,0.999999235323666)
(6.76676676676677,0.99999926355648)
(6.77677677677678,0.999999290783691)
(6.78678678678679,0.999999317039706)
(6.7967967967968,0.999999342357811)
(6.80680680680681,0.999999366770199)
(6.81681681681682,0.999999390308008)
(6.82682682682683,0.999999413001352)
(6.83683683683684,0.999999434879355)
(6.84684684684685,0.999999455970179)
(6.85685685685686,0.999999476301056)
(6.86686686686687,0.999999495898317)
(6.87687687687688,0.99999951478742)
(6.88688688688689,0.999999532992979)
(6.8968968968969,0.999999550538788)
(6.90690690690691,0.999999567447848)
(6.91691691691692,0.999999583742392)
(6.92692692692693,0.999999599443912)
(6.93693693693694,0.999999614573178)
(6.94694694694695,0.999999629150264)
(6.95695695695696,0.99999964319457)
(6.96696696696697,0.999999656724843)
(6.97697697697698,0.999999669759197)
(6.98698698698699,0.999999682315135)
(6.996996996997,0.999999694409569)
(7.00700700700701,0.999999706058837)
(7.01701701701702,0.999999717278721)
(7.02702702702703,0.999999728084469)
(7.03703703703704,0.999999738490809)
(7.04704704704705,0.999999748511966)
(7.05705705705706,0.99999975816168)
(7.06706706706707,0.99999976745322)
(7.07707707707708,0.9999997763994)
(7.08708708708709,0.999999785012596)
(7.0970970970971,0.999999793304756)
(7.10710710710711,0.999999801287418)
(7.11711711711712,0.999999808971721)
(7.12712712712713,0.99999981636842)
(7.13713713713714,0.999999823487898)
(7.14714714714715,0.999999830340176)
(7.15715715715716,0.999999836934929)
(7.16716716716717,0.999999843281495)
(7.17717717717718,0.999999849388886)
(7.18718718718719,0.999999855265801)
(7.1971971971972,0.999999860920634)
(7.20720720720721,0.999999866361485)
(7.21721721721722,0.99999987159617)
(7.22722722722723,0.999999876632232)
(7.23723723723724,0.999999881476947)
(7.24724724724725,0.999999886137334)
(7.25725725725726,0.999999890620168)
(7.26726726726727,0.99999989493198)
(7.27727727727728,0.999999899079074)
(7.28728728728729,0.999999903067527)
(7.2972972972973,0.999999906903202)
(7.30730730730731,0.999999910591754)
(7.31731731731732,0.999999914138635)
(7.32732732732733,0.999999917549104)
(7.33733733733734,0.999999920828231)
(7.34734734734735,0.999999923980905)
(7.35735735735736,0.99999992701184)
(7.36736736736737,0.999999929925581)
(7.37737737737738,0.999999932726509)
(7.38738738738739,0.999999935418846)
(7.3973973973974,0.999999938006665)
(7.40740740740741,0.999999940493889)
(7.41741741741742,0.9999999428843)
(7.42742742742743,0.999999945181544)
(7.43743743743744,0.999999947389134)
(7.44744744744745,0.999999949510453)
(7.45745745745746,0.999999951548765)
(7.46746746746747,0.999999953507213)
(7.47747747747748,0.999999955388824)
(7.48748748748749,0.999999957196516)
(7.4974974974975,0.9999999589331)
(7.50750750750751,0.999999960601284)
(7.51751751751752,0.999999962203675)
(7.52752752752753,0.999999963742787)
(7.53753753753754,0.999999965221038)
(7.54754754754755,0.99999996664076)
(7.55755755755756,0.999999968004195)
(7.56756756756757,0.999999969313506)
(7.57757757757758,0.999999970570774)
(7.58758758758759,0.999999971778003)
(7.5975975975976,0.999999972937121)
(7.60760760760761,0.999999974049987)
(7.61761761761762,0.999999975118389)
(7.62762762762763,0.999999976144048)
(7.63763763763764,0.999999977128622)
(7.64764764764765,0.999999978073705)
(7.65765765765766,0.999999978980833)
(7.66766766766767,0.999999979851483)
(7.67767767767768,0.999999980687078)
(7.68768768768769,0.999999981488986)
(7.6976976976977,0.999999982258523)
(7.70770770770771,0.999999982996957)
(7.71771771771772,0.999999983705507)
(7.72772772772773,0.999999984385346)
(7.73773773773774,0.999999985037602)
(7.74774774774775,0.999999985663361)
(7.75775775775776,0.999999986263667)
(7.76776776776777,0.999999986839524)
(7.77777777777778,0.999999987391898)
(7.78778778778779,0.999999987921719)
(7.7977977977978,0.999999988429881)
(7.80780780780781,0.999999988917242)
(7.81781781781782,0.999999989384628)
(7.82782782782783,0.999999989832836)
(7.83783783783784,0.999999990262627)
(7.84784784784785,0.999999990674738)
(7.85785785785786,0.999999991069874)
(7.86786786786787,0.999999991448713)
(7.87787787787788,0.99999999181191)
(7.88788788788789,0.99999999216009)
(7.8978978978979,0.999999992493857)
(7.90790790790791,0.99999999281379)
(7.91791791791792,0.999999993120447)
(7.92792792792793,0.999999993414362)
(7.93793793793794,0.99999999369605)
(7.94794794794795,0.999999993966004)
(7.95795795795796,0.9999999942247)
(7.96796796796797,0.999999994472595)
(7.97797797797798,0.999999994710125)
(7.98798798798799,0.999999994937713)
(7.997997997998,0.999999995155763)
(8.00800800800801,0.999999995364663)
(8.01801801801802,0.999999995564786)
(8.02802802802803,0.999999995756492)
(8.03803803803804,0.999999995940123)
(8.04804804804805,0.999999996116012)
(8.05805805805806,0.999999996284475)
(8.06806806806807,0.999999996445816)
(8.07807807807808,0.99999999660033)
(8.08808808808809,0.999999996748297)
(8.0980980980981,0.999999996889986)
(8.10810810810811,0.999999997025657)
(8.11811811811812,0.999999997155559)
(8.12812812812813,0.999999997279929)
(8.13813813813814,0.999999997398998)
(8.14814814814815,0.999999997512984)
(8.15815815815816,0.9999999976221)
(8.16816816816817,0.999999997726548)
(8.17817817817818,0.999999997826521)
(8.18818818818819,0.999999997922207)
(8.1981981981982,0.999999998013785)
(8.20820820820821,0.999999998101425)
(8.21821821821822,0.999999998185294)
(8.22822822822823,0.999999998265549)
(8.23823823823824,0.999999998342341)
(8.24824824824825,0.999999998415816)
(8.25825825825826,0.999999998486114)
(8.26826826826827,0.999999998553368)
(8.27827827827828,0.999999998617707)
(8.28828828828829,0.999999998679253)
(8.2982982982983,0.999999998738126)
(8.30830830830831,0.999999998794437)
(8.31831831831832,0.999999998848295)
(8.32832832832833,0.999999998899806)
(8.33833833833834,0.999999998949067)
(8.34834834834835,0.999999998996175)
(8.35835835835836,0.999999999041222)
(8.36836836836837,0.999999999084296)
(8.37837837837838,0.99999999912548)
(8.38838838838839,0.999999999164855)
(8.3983983983984,0.9999999992025)
(8.40840840840841,0.999999999238488)
(8.41841841841842,0.999999999272889)
(8.42842842842843,0.999999999305773)
(8.43843843843844,0.999999999337205)
(8.44844844844845,0.999999999367247)
(8.45845845845846,0.999999999395958)
(8.46846846846847,0.999999999423398)
(8.47847847847848,0.999999999449619)
(8.48848848848849,0.999999999474676)
(8.4984984984985,0.999999999498618)
(8.50850850850851,0.999999999521494)
(8.51851851851852,0.99999999954335)
(8.52852852852853,0.999999999564231)
(8.53853853853854,0.999999999584179)
(8.54854854854855,0.999999999603234)
(8.55855855855856,0.999999999621437)
(8.56856856856857,0.999999999638823)
(8.57857857857858,0.999999999655428)
(8.58858858858859,0.999999999671288)
(8.5985985985986,0.999999999686434)
(8.60860860860861,0.999999999700897)
(8.61861861861862,0.999999999714709)
(8.62862862862863,0.999999999727897)
(8.63863863863864,0.999999999740489)
(8.64864864864865,0.999999999752511)
(8.65865865865866,0.999999999763989)
(8.66866866866867,0.999999999774946)
(8.67867867867868,0.999999999785406)
(8.68868868868869,0.999999999795391)
(8.6986986986987,0.999999999804921)
(8.70870870870871,0.999999999814017)
(8.71871871871872,0.999999999822698)
(8.72872872872873,0.999999999830983)
(8.73873873873874,0.999999999838889)
(8.74874874874875,0.999999999846433)
(8.75875875875876,0.999999999853632)
(8.76876876876877,0.999999999860501)
(8.77877877877878,0.999999999867054)
(8.78878878878879,0.999999999873306)
(8.7987987987988,0.99999999987927)
(8.80880880880881,0.99999999988496)
(8.81881881881882,0.999999999890387)
(8.82882882882883,0.999999999895564)
(8.83883883883884,0.999999999900502)
(8.84884884884885,0.999999999905211)
(8.85885885885886,0.999999999909702)
(8.86886886886887,0.999999999913984)
(8.87887887887888,0.999999999918068)
(8.88888888888889,0.999999999921962)
(8.8988988988989,0.999999999925675)
(8.90890890890891,0.999999999929215)
(8.91891891891892,0.99999999993259)
(8.92892892892893,0.999999999935807)
(8.93893893893894,0.999999999938874)
(8.94894894894895,0.999999999941798)
(8.95895895895896,0.999999999944584)
(8.96896896896897,0.99999999994724)
(8.97897897897898,0.999999999949772)
(8.98898898898899,0.999999999952184)
(8.998998998999,0.999999999954483)
(9.00900900900901,0.999999999956673)
(9.01901901901902,0.999999999958761)
(9.02902902902903,0.99999999996075)
(9.03903903903904,0.999999999962645)
(9.04904904904905,0.99999999996445)
(9.05905905905906,0.99999999996617)
(9.06906906906907,0.999999999967808)
(9.07907907907908,0.999999999969369)
(9.08908908908909,0.999999999970855)
(9.0990990990991,0.999999999972271)
(9.10910910910911,0.99999999997362)
(9.11911911911912,0.999999999974904)
(9.12912912912913,0.999999999976127)
(9.13913913913914,0.999999999977291)
(9.14914914914915,0.9999999999784)
(9.15915915915916,0.999999999979456)
(9.16916916916917,0.999999999980461)
(9.17917917917918,0.999999999981418)
(9.18918918918919,0.999999999982329)
(9.1991991991992,0.999999999983196)
(9.20920920920921,0.999999999984022)
(9.21921921921922,0.999999999984808)
(9.22922922922923,0.999999999985555)
(9.23923923923924,0.999999999986267)
(9.24924924924925,0.999999999986945)
(9.25925925925926,0.999999999987589)
(9.26926926926927,0.999999999988203)
(9.27927927927928,0.999999999988787)
(9.28928928928929,0.999999999989342)
(9.2992992992993,0.99999999998987)
(9.30930930930931,0.999999999990373)
(9.31931931931932,0.999999999990851)
(9.32932932932933,0.999999999991306)
(9.33933933933934,0.999999999991739)
(9.34934934934935,0.999999999992151)
(9.35935935935936,0.999999999992542)
(9.36936936936937,0.999999999992915)
(9.37937937937938,0.999999999993269)
(9.38938938938939,0.999999999993605)
(9.3993993993994,0.999999999993926)
(9.40940940940941,0.99999999999423)
(9.41941941941942,0.99999999999452)
(9.42942942942943,0.999999999994795)
(9.43943943943944,0.999999999995057)
(9.44944944944945,0.999999999995305)
(9.45945945945946,0.999999999995542)
(9.46946946946947,0.999999999995767)
(9.47947947947948,0.99999999999598)
(9.48948948948949,0.999999999996183)
(9.4994994994995,0.999999999996377)
(9.50950950950951,0.99999999999656)
(9.51951951951952,0.999999999996734)
(9.52952952952953,0.9999999999969)
(9.53953953953954,0.999999999997057)
(9.54954954954955,0.999999999997207)
(9.55955955955956,0.999999999997349)
(9.56956956956957,0.999999999997484)
(9.57957957957958,0.999999999997612)
(9.58958958958959,0.999999999997734)
(9.5995995995996,0.99999999999785)
(9.60960960960961,0.99999999999796)
(9.61961961961962,0.999999999998064)
(9.62962962962963,0.999999999998163)
(9.63963963963964,0.999999999998258)
(9.64964964964965,0.999999999998347)
(9.65965965965966,0.999999999998432)
(9.66966966966967,0.999999999998513)
(9.67967967967968,0.999999999998589)
(9.68968968968969,0.999999999998662)
(9.6996996996997,0.999999999998731)
(9.70970970970971,0.999999999998796)
(9.71971971971972,0.999999999998859)
(9.72972972972973,0.999999999998918)
(9.73973973973974,0.999999999998974)
(9.74974974974975,0.999999999999027)
(9.75975975975976,0.999999999999077)
(9.76976976976977,0.999999999999125)
(9.77977977977978,0.999999999999171)
(9.78978978978979,0.999999999999214)
(9.7997997997998,0.999999999999255)
(9.80980980980981,0.999999999999294)
(9.81981981981982,0.999999999999331)
(9.82982982982983,0.999999999999366)
(9.83983983983984,0.999999999999399)
(9.84984984984985,0.99999999999943)
(9.85985985985986,0.99999999999946)
(9.86986986986987,0.999999999999488)
(9.87987987987988,0.999999999999515)
(9.88988988988989,0.999999999999541)
(9.8998998998999,0.999999999999565)
(9.90990990990991,0.999999999999588)
(9.91991991991992,0.999999999999609)
(9.92992992992993,0.99999999999963)
(9.93993993993994,0.99999999999965)
(9.94994994994995,0.999999999999668)
(9.95995995995996,0.999999999999686)
(9.96996996996997,0.999999999999702)
(9.97997997997998,0.999999999999718)
(9.98998998998999,0.999999999999733)
(10,0.999999999999747)

};
\addplot [green!50.0!black]
coordinates {
(0,0)
(0.01001001001001,0.00184811754889109)
(0.02002002002002,0.00369622518243176)
(0.03003003003003,0.00554431298543117)
(0.04004004004004,0.00739237104301769)
(0.05005005005005,0.00924038944079841)
(0.0600600600600601,0.0110883582650188)
(0.0700700700700701,0.012936267602722)
(0.0800800800800801,0.0147841075419088)
(0.0900900900900901,0.0166318681716969)
(0.1001001001001,0.01847953958248)
(0.11011011011011,0.0203271118660879)
(0.12012012012012,0.0221745751159457)
(0.13013013013013,0.0240219194272327)
(0.14014014014014,0.0258691348970423)
(0.15015015015015,0.0277162116245411)
(0.16016016016016,0.0295631397111277)
(0.17017017017017,0.0314099092605925)
(0.18018018018018,0.0332565103792763)
(0.19019019019019,0.0351029331762293)
(0.2002002002002,0.0369491677633704)
(0.21021021021021,0.0387952042556458)
(0.22022022022022,0.0406410327711877)
(0.23023023023023,0.0424866434314733)
(0.24024024024024,0.0443320263614831)
(0.25025025025025,0.0461771716898594)
(0.26026026026026,0.0480220695490651)
(0.27027027027027,0.0498667100755417)
(0.28028028028028,0.0517110834098672)
(0.29029029029029,0.0535551796969151)
(0.3003003003003,0.0553989890860113)
(0.31031031031031,0.0572425017310928)
(0.32032032032032,0.0590857077908649)
(0.33033033033033,0.0609285974289592)
(0.34034034034034,0.0627711608140907)
(0.35035035035035,0.0646133881202155)
(0.36036036036036,0.0664552695266876)
(0.37037037037037,0.0682967952184164)
(0.38038038038038,0.0701379553860232)
(0.39039039039039,0.0719787402259982)
(0.4004004004004,0.073819139940857)
(0.41041041041041,0.0756591447392971)
(0.42042042042042,0.0774987448363541)
(0.43043043043043,0.0793379304535577)
(0.44044044044044,0.081176691819088)
(0.45045045045045,0.0830150191679307)
(0.46046046046046,0.0848529027420331)
(0.47047047047047,0.0866903327904594)
(0.48048048048048,0.0885272995695456)
(0.49049049049049,0.090363793343055)
(0.500500500500501,0.0921998043823325)
(0.510510510510511,0.0940353229664593)
(0.520520520520521,0.0958703393824079)
(0.530530530530531,0.0977048439251951)
(0.540540540540541,0.099538826898037)
(0.550550550550551,0.101372278612502)
(0.560560560560561,0.103205189388666)
(0.570570570570571,0.105037549555262)
(0.580580580580581,0.106869349449838)
(0.590590590590591,0.108700579418906)
(0.600600600600601,0.110531229818096)
(0.610610610610611,0.112361291012309)
(0.620620620620621,0.114190753375866)
(0.630630630630631,0.116019607292667)
(0.640640640640641,0.117847843156333)
(0.650650650650651,0.119675451370365)
(0.660660660660661,0.12150242234829)
(0.670670670670671,0.123328746513817)
(0.680680680680681,0.125154414300982)
(0.690690690690691,0.126979416154303)
(0.700700700700701,0.128803742528924)
(0.710710710710711,0.130627383890773)
(0.720720720720721,0.132450330716705)
(0.730730730730731,0.134272573494652)
(0.740740740740741,0.136094102723773)
(0.750750750750751,0.137914908914605)
(0.760760760760761,0.139734982589205)
(0.770770770770771,0.141554314281304)
(0.780780780780781,0.143372894536451)
(0.790790790790791,0.145190713912161)
(0.800800800800801,0.147007762978063)
(0.810810810810811,0.148824032316047)
(0.820820820820821,0.150639512520407)
(0.830830830830831,0.152454194197992)
(0.840840840840841,0.154268067968346)
(0.850850850850851,0.156081124463859)
(0.860860860860861,0.157893354329908)
(0.870870870870871,0.159704748225002)
(0.880880880880881,0.16151529682093)
(0.890890890890891,0.1633249908029)
(0.900900900900901,0.165133820869686)
(0.910910910910911,0.166941777733769)
(0.920920920920921,0.16874885212148)
(0.930930930930931,0.170555034773147)
(0.940940940940941,0.17236031644323)
(0.950950950950951,0.174164687900468)
(0.960960960960961,0.175968139928018)
(0.970970970970971,0.177770663323597)
(0.980980980980981,0.179572248899621)
(0.990990990990991,0.18137288748335)
(1.001001001001,0.18317256991702)
(1.01101101101101,0.18497128705799)
(1.02102102102102,0.186769029778876)
(1.03103103103103,0.188565788967692)
(1.04104104104104,0.190361555527989)
(1.05105105105105,0.19215632037899)
(1.06106106106106,0.193950074455729)
(1.07107107107107,0.195742808709188)
(1.08108108108108,0.197534514106435)
(1.09109109109109,0.199325181630758)
(1.1011011011011,0.2011148022818)
(1.11111111111111,0.202903367075696)
(1.12112112112112,0.204690867045211)
(1.13113113113113,0.206477293239867)
(1.14114114114114,0.208262636726083)
(1.15115115115115,0.210046888587305)
(1.16116116116116,0.211830039924144)
(1.17117117117117,0.213612081854501)
(1.18118118118118,0.215393005513708)
(1.19119119119119,0.217172802054651)
(1.2012012012012,0.218951462647909)
(1.21121121121121,0.22072897848188)
(1.22122122122122,0.222505340762912)
(1.23123123123123,0.224280540715436)
(1.24124124124124,0.22605456958209)
(1.25125125125125,0.227827418623853)
(1.26126126126126,0.229599079120173)
(1.27127127127127,0.231369542369089)
(1.28128128128128,0.233138799687367)
(1.29129129129129,0.234906842410621)
(1.3013013013013,0.236673661893442)
(1.31131131131131,0.238439249509524)
(1.32132132132132,0.240203596651789)
(1.33133133133133,0.241966694732511)
(1.34134134134134,0.243728535183442)
(1.35135135135135,0.245489109455936)
(1.36136136136136,0.247248409021073)
(1.37137137137137,0.249006425369779)
(1.38138138138138,0.250763150012952)
(1.39139139139139,0.252518574481582)
(1.4014014014014,0.254272690326874)
(1.41141141141141,0.256025489120367)
(1.42142142142142,0.257776962454056)
(1.43143143143143,0.25952710194051)
(1.44144144144144,0.261275899212994)
(1.45145145145145,0.263023345925585)
(1.46146146146146,0.264769433753292)
(1.47147147147147,0.266514154392175)
(1.48148148148148,0.268257499559457)
(1.49149149149149,0.269999460993648)
(1.5015015015015,0.271740030454654)
(1.51151151151151,0.273479199723898)
(1.52152152152152,0.275216960604431)
(1.53153153153153,0.276953304921049)
(1.54154154154154,0.278688224520407)
(1.55155155155155,0.280421711271129)
(1.56156156156156,0.282153757063924)
(1.57157157157157,0.283884353811698)
(1.58158158158158,0.285613493449664)
(1.59159159159159,0.287341167935454)
(1.6016016016016,0.28906736924923)
(1.61161161161161,0.290792089393792)
(1.62162162162162,0.292515320394688)
(1.63163163163163,0.294237054300326)
(1.64164164164164,0.295957283182075)
(1.65165165165165,0.297675999134381)
(1.66166166166166,0.299393194274868)
(1.67167167167167,0.301108860744445)
(1.68168168168168,0.302822990707416)
(1.69169169169169,0.304535576351578)
(1.7017017017017,0.306246609888334)
(1.71171171171171,0.30795608355279)
(1.72172172172172,0.309663989603861)
(1.73173173173173,0.311370320324374)
(1.74174174174174,0.313075068021173)
(1.75175175175175,0.314778225025212)
(1.76176176176176,0.316479783691666)
(1.77177177177177,0.318179736400026)
(1.78178178178178,0.319878075554196)
(1.79179179179179,0.321574793582601)
(1.8018018018018,0.323269882938277)
(1.81181181181181,0.324963336098971)
(1.82182182182182,0.326655145567241)
(1.83183183183183,0.328345303870551)
(1.84184184184184,0.330033803561365)
(1.85185185185185,0.331720637217245)
(1.86186186186186,0.333405797440946)
(1.87187187187187,0.335089276860506)
(1.88188188188188,0.336771068129346)
(1.89189189189189,0.338451163926355)
(1.9019019019019,0.34012955695599)
(1.91191191191191,0.34180623994836)
(1.92192192192192,0.343481205659325)
(1.93193193193193,0.345154446870579)
(1.94194194194194,0.346825956389743)
(1.95195195195195,0.348495727050454)
(1.96196196196196,0.350163751712452)
(1.97197197197197,0.351830023261669)
(1.98198198198198,0.353494534610315)
(1.99199199199199,0.355157278696965)
(2.002002002002,0.356818248486644)
(2.01201201201201,0.358477436970912)
(2.02202202202202,0.360134837167951)
(2.03203203203203,0.361790442122643)
(2.04204204204204,0.363444244906659)
(2.05205205205205,0.365096238618539)
(2.06206206206206,0.36674641638377)
(2.07207207207207,0.368394771354876)
(2.08208208208208,0.370041296711488)
(2.09209209209209,0.371685985660431)
(2.1021021021021,0.373328831435801)
(2.11211211211211,0.374969827299043)
(2.12212212212212,0.376608966539026)
(2.13213213213213,0.378246242472126)
(2.14214214214214,0.379881648442298)
(2.15215215215215,0.381515177821151)
(2.16216216216216,0.383146824008027)
(2.17217217217217,0.384776580430071)
(2.18218218218218,0.386404440542308)
(2.19219219219219,0.388030397827713)
(2.2022022022022,0.389654445797286)
(2.21221221221221,0.391276577990122)
(2.22222222222222,0.39289678797348)
(2.23223223223223,0.394515069342858)
(2.24224224224224,0.396131415722059)
(2.25225225225225,0.397745820763259)
(2.26226226226226,0.399358278147077)
(2.27227227227227,0.400968781582645)
(2.28228228228228,0.402577324807668)
(2.29229229229229,0.404183901588495)
(2.3023023023023,0.405788505720184)
(2.31231231231231,0.407391131026565)
(2.32232232232232,0.408991771360304)
(2.33233233233233,0.410590420602969)
(2.34234234234234,0.412187072665088)
(2.35235235235235,0.413781721486216)
(2.36236236236236,0.415374361034991)
(2.37237237237237,0.416964985309198)
(2.38238238238238,0.418553588335827)
(2.39239239239239,0.420140164171134)
(2.4024024024024,0.421724706900698)
(2.41241241241241,0.423307210639476)
(2.42242242242242,0.424887669531867)
(2.43243243243243,0.426466077751759)
(2.44244244244244,0.428042429502594)
(2.45245245245245,0.429616719017414)
(2.46246246246246,0.431188940558923)
(2.47247247247247,0.432759088419534)
(2.48248248248248,0.434327156921426)
(2.49249249249249,0.435893140416592)
(2.5025025025025,0.437457033286897)
(2.51251251251251,0.439018829944119)
(2.52252252252252,0.440578524830009)
(2.53253253253253,0.442136112416331)
(2.54254254254254,0.443691587204918)
(2.55255255255255,0.445244943727715)
(2.56256256256256,0.446796176546826)
(2.57257257257257,0.448345280254565)
(2.58258258258258,0.449892249473496)
(2.59259259259259,0.45143707885648)
(2.6026026026026,0.452979763086722)
(2.61261261261261,0.454520296877808)
(2.62262262262262,0.456058674973756)
(2.63263263263263,0.457594892149049)
(2.64264264264264,0.459128943208686)
(2.65265265265265,0.460660822988213)
(2.66266266266266,0.462190526353772)
(2.67267267267267,0.463718048202131)
(2.68268268268268,0.46524338346073)
(2.69269269269269,0.466766527087716)
(2.7027027027027,0.46828747407198)
(2.71271271271271,0.469806219433192)
(2.72272272272272,0.471322758221839)
(2.73273273273273,0.472837085519259)
(2.74274274274274,0.474349196437675)
(2.75275275275275,0.475859086120229)
(2.76276276276276,0.477366749741013)
(2.77277277277277,0.478872182505103)
(2.78278278278278,0.480375379648591)
(2.79279279279279,0.481876336438614)
(2.8028028028028,0.483375048173383)
(2.81281281281281,0.484871510182214)
(2.82282282282282,0.486365717825559)
(2.83283283283283,0.487857666495027)
(2.84284284284284,0.489347351613416)
(2.85285285285285,0.490834768634741)
(2.86286286286286,0.492319913044253)
(2.87287287287287,0.493802780358472)
(2.88288288288288,0.495283366125205)
(2.89289289289289,0.496761665923571)
(2.9029029029029,0.498237675364028)
(2.91291291291291,0.499711390088387)
(2.92292292292292,0.501182805769844)
(2.93293293293293,0.502651918112989)
(2.94294294294294,0.504118722853837)
(2.95295295295295,0.50558321575984)
(2.96296296296296,0.507045392629906)
(2.97297297297297,0.508505249294423)
(2.98298298298298,0.509962781615269)
(2.99299299299299,0.511417985485831)
(3.003003003003,0.512870856831024)
(3.01301301301301,0.514321391607302)
(3.02302302302302,0.515769585802672)
(3.03303303303303,0.517215435436711)
(3.04304304304304,0.518658936560579)
(3.05305305305305,0.520100085257026)
(3.06306306306306,0.521538877640411)
(3.07307307307307,0.522975309856707)
(3.08308308308308,0.524409378083514)
(3.09309309309309,0.525841078530067)
(3.1031031031031,0.527270407437248)
(3.11311311311311,0.528697361077589)
(3.12312312312312,0.530121935755284)
(3.13313313313313,0.531544127806195)
(3.14314314314314,0.532963933597854)
(3.15315315315315,0.534381349529474)
(3.16316316316316,0.535796372031951)
(3.17317317317317,0.537208997567866)
(3.18318318318318,0.538619222631493)
(3.19319319319319,0.540027043748796)
(3.2032032032032,0.541432457477437)
(3.21321321321321,0.542835460406772)
(3.22322322322322,0.544236049157856)
(3.23323323323323,0.545634220383436)
(3.24324324324324,0.547029970767961)
(3.25325325325325,0.548423297027569)
(3.26326326326326,0.549814195910094)
(3.27327327327327,0.551202664195056)
(3.28328328328328,0.552588698693664)
(3.29329329329329,0.553972296248806)
(3.3033033033033,0.555353453735049)
(3.31331331331331,0.55673216805863)
(3.32332332332332,0.558108436157451)
(3.33333333333333,0.559482255001073)
(3.34334334334334,0.560853621590707)
(3.35335335335335,0.562222532959207)
(3.36336336336336,0.563588986171061)
(3.37337337337337,0.564952978322381)
(3.38338338338338,0.566314506540892)
(3.39339339339339,0.567673567985924)
(3.4034034034034,0.569030159848397)
(3.41341341341341,0.57038427935081)
(3.42342342342342,0.571735923747232)
(3.43343343343343,0.573085090323282)
(3.44344344344344,0.574431776396119)
(3.45345345345345,0.575775979314428)
(3.46346346346346,0.5771176964584)
(3.47347347347347,0.578456925239721)
(3.48348348348348,0.579793663101554)
(3.49349349349349,0.581127907518518)
(3.5035035035035,0.582459655996676)
(3.51351351351351,0.583788906073511)
(3.52352352352352,0.58511565531791)
(3.53353353353353,0.586439901330142)
(3.54354354354354,0.587761641741842)
(3.55355355355355,0.589080874215982)
(3.56356356356356,0.590397596446855)
(3.57357357357357,0.591711806160053)
(3.58358358358358,0.59302350111244)
(3.59359359359359,0.59433267909213)
(3.6036036036036,0.595639337918466)
(3.61361361361361,0.596943475441989)
(3.62362362362362,0.598245089544417)
(3.63363363363363,0.599544178138617)
(3.64364364364364,0.60084073916858)
(3.65365365365365,0.602134770609389)
(3.66366366366366,0.603426270467198)
(3.67367367367367,0.604715236779197)
(3.68368368368368,0.606001667613586)
(3.69369369369369,0.607285561069544)
(3.7037037037037,0.6085669152772)
(3.71371371371371,0.609845728397599)
(3.72372372372372,0.611121998622676)
(3.73373373373373,0.612395724175215)
(3.74374374374374,0.613666903308826)
(3.75375375375375,0.614935534307903)
(3.76376376376376,0.616201615487595)
(3.77377377377377,0.61746514519377)
(3.78378378378378,0.61872612180298)
(3.79379379379379,0.619984543722426)
(3.8038038038038,0.621240409389917)
(3.81381381381381,0.622493717273841)
(3.82382382382382,0.623744465873119)
(3.83383383383383,0.624992653717174)
(3.84384384384384,0.626238279365887)
(3.85385385385385,0.627481341409562)
(3.86386386386386,0.628721838468882)
(3.87387387387387,0.629959769194872)
(3.88388388388388,0.631195132268855)
(3.89389389389389,0.632427926402414)
(3.9039039039039,0.633658150337346)
(3.91391391391391,0.634885802845622)
(3.92392392392392,0.636110882729344)
(3.93393393393393,0.637333388820699)
(3.94394394394394,0.638553319981915)
(3.95395395395395,0.639770675105219)
(3.96396396396396,0.640985453112787)
(3.97397397397397,0.642197652956704)
(3.98398398398398,0.643407273618911)
(3.99399399399399,0.64461431411116)
(4.004004004004,0.645818773474971)
(4.01401401401401,0.647020650781578)
(4.02402402402402,0.64821994513188)
(4.03403403403403,0.649416655656399)
(4.04404404404404,0.650610781515221)
(4.05405405405405,0.651802321897953)
(4.06406406406406,0.652991276023668)
(4.07407407407407,0.654177643140855)
(4.08408408408408,0.655361422527369)
(4.09409409409409,0.656542613490373)
(4.1041041041041,0.657721215366292)
(4.11411411411411,0.658897227520756)
(4.12412412412412,0.660070649348547)
(4.13413413413413,0.661241480273541)
(4.14414414414414,0.662409719748662)
(4.15415415415415,0.663575367255816)
(4.16416416416416,0.664738422305841)
(4.17417417417417,0.665898884438453)
(4.18418418418418,0.66705675322218)
(4.19419419419419,0.668212028254316)
(4.2042042042042,0.669364709160854)
(4.21421421421421,0.670514795596432)
(4.22422422422422,0.671662287244273)
(4.23423423423423,0.672807183816126)
(4.24424424424424,0.673949485052207)
(4.25425425425425,0.675089190721137)
(4.26426426426426,0.676226300619881)
(4.27427427427427,0.677360814573691)
(4.28428428428428,0.678492732436038)
(4.29429429429429,0.679622054088555)
(4.3043043043043,0.680748779440972)
(4.31431431431431,0.681872908431052)
(4.32432432432432,0.682994441024531)
(4.33433433433433,0.684113377215051)
(4.34434434434434,0.685229717024096)
(4.35435435435435,0.686343460500927)
(4.36436436436436,0.68745460772252)
(4.37437437437437,0.688563158793494)
(4.38438438438438,0.68966911384605)
(4.39439439439439,0.690772473039904)
(4.4044044044044,0.691873236562215)
(4.41441441441441,0.692971404627524)
(4.42442442442442,0.694066977477683)
(4.43443443443443,0.695159955381785)
(4.44444444444444,0.696250338636098)
(4.45445445445445,0.697338127563995)
(4.46446446446446,0.698423322515884)
(4.47447447447447,0.699505923869138)
(4.48448448448448,0.700585932028025)
(4.49449449449449,0.701663347423636)
(4.5045045045045,0.702738170513816)
(4.51451451451451,0.70381040178309)
(4.52452452452452,0.704880041742592)
(4.53453453453453,0.705947090929993)
(4.54454454454454,0.707011549909426)
(4.55455455455455,0.708073419271416)
(4.56456456456456,0.709132699632804)
(4.57457457457457,0.710189391636674)
(4.58458458458458,0.711243495952279)
(4.59459459459459,0.712295013274963)
(4.6046046046046,0.713343944326092)
(4.61461461461461,0.714390289852973)
(4.62462462462462,0.715434050628778)
(4.63463463463463,0.716475227452473)
(4.64464464464464,0.717513821148738)
(4.65465465465465,0.718549832567887)
(4.66466466466466,0.719583262585798)
(4.67467467467467,0.720614112103827)
(4.68468468468468,0.721642382048736)
(4.69469469469469,0.722668073372614)
(4.7047047047047,0.723691187052795)
(4.71471471471471,0.724711724091781)
(4.72472472472472,0.725729685517162)
(4.73473473473473,0.726745072381538)
(4.74474474474474,0.727757885762437)
(4.75475475475475,0.728768126762234)
(4.76476476476476,0.729775796508073)
(4.77477477477477,0.730780896151783)
(4.78478478478478,0.731783426869797)
(4.79479479479479,0.732783389863074)
(4.8048048048048,0.733780786357011)
(4.81481481481481,0.734775617601365)
(4.82482482482482,0.735767884870169)
(4.83483483483483,0.736757589461649)
(4.84484484484484,0.73774473269814)
(4.85485485485485,0.738729315926004)
(4.86486486486486,0.739711340515545)
(4.87487487487487,0.740690807860925)
(4.88488488488488,0.741667719380078)
(4.89489489489489,0.742642076514631)
(4.9049049049049,0.74361388072981)
(4.91491491491491,0.744583133514362)
(4.92492492492492,0.745549836380468)
(4.93493493493493,0.746513990863652)
(4.94494494494494,0.747475598522702)
(4.95495495495495,0.74843466093958)
(4.96496496496496,0.749391179719334)
(4.97497497497497,0.750345156490014)
(4.98498498498498,0.751296592902583)
(4.99499499499499,0.752245490630831)
(5.00500500500501,0.753191851371284)
(5.01501501501502,0.75413567684312)
(5.02502502502503,0.755076968788078)
(5.03503503503504,0.756015728970371)
(5.04504504504505,0.756951959176596)
(5.05505505505506,0.757885661215646)
(5.06506506506507,0.758816836918622)
(5.07507507507508,0.759745488138738)
(5.08508508508509,0.76067161675124)
(5.0950950950951,0.761595224653309)
(5.10510510510511,0.762516313763973)
(5.11511511511512,0.763434886024018)
(5.12512512512513,0.764350943395895)
(5.13513513513514,0.765264487863632)
(5.14514514514515,0.76617552143274)
(5.15515515515516,0.767084046130125)
(5.16516516516517,0.767990064003994)
(5.17517517517518,0.768893577123763)
(5.18518518518519,0.769794587579969)
(5.1951951951952,0.770693097484172)
(5.20520520520521,0.771589108968868)
(5.21521521521522,0.772482624187396)
(5.22522522522523,0.77337364531384)
(5.23523523523524,0.774262174542945)
(5.24524524524525,0.775148214090016)
(5.25525525525526,0.776031766190828)
(5.26526526526527,0.776912833101535)
(5.27527527527528,0.777791417098571)
(5.28528528528529,0.778667520478564)
(5.2952952952953,0.779541145558233)
(5.30530530530531,0.780412294674301)
(5.31531531531532,0.781280970183398)
(5.32532532532533,0.782147174461966)
(5.33533533533534,0.783010909906169)
(5.34534534534535,0.78387217893179)
(5.35535535535536,0.784730983974145)
(5.36536536536537,0.785587327487981)
(5.37537537537538,0.786441211947388)
(5.38538538538539,0.787292639845696)
(5.3953953953954,0.788141613695385)
(5.40540540540541,0.788988136027987)
(5.41541541541542,0.789832209393994)
(5.42542542542543,0.790673836362755)
(5.43543543543544,0.791513019522389)
(5.44544544544545,0.792349761479681)
(5.45545545545546,0.79318406485999)
(5.46546546546547,0.794015932307155)
(5.47547547547548,0.794845366483391)
(5.48548548548549,0.795672370069202)
(5.4954954954955,0.796496945763275)
(5.50550550550551,0.79731909628239)
(5.51551551551552,0.798138824361323)
(5.52552552552553,0.798956132752743)
(5.53553553553554,0.799771024227122)
(5.54554554554555,0.800583501572634)
(5.55555555555556,0.801393567595059)
(5.56556556556557,0.802201225117685)
(5.57557557557558,0.803006476981211)
(5.58558558558559,0.803809326043651)
(5.5955955955956,0.804609775180233)
(5.60560560560561,0.805407827283305)
(5.61561561561562,0.806203485262234)
(5.62562562562563,0.806996752043312)
(5.63563563563564,0.807787630569655)
(5.64564564564565,0.808576123801107)
(5.65565565565566,0.80936223471414)
(5.66566566566567,0.810145966301757)
(5.67567567567568,0.810927321573396)
(5.68568568568569,0.811706303554829)
(5.6956956956957,0.812482915288063)
(5.70570570570571,0.813257159831247)
(5.71571571571572,0.814029040258568)
(5.72572572572573,0.814798559660154)
(5.73573573573574,0.815565721141978)
(5.74574574574575,0.816330527825759)
(5.75575575575576,0.817092982848861)
(5.76576576576577,0.817853089364195)
(5.77577577577578,0.818610850540124)
(5.78578578578579,0.819366269560362)
(5.7957957957958,0.820119349623874)
(5.80580580580581,0.82087009394478)
(5.81581581581582,0.821618505752253)
(5.82582582582583,0.822364588290427)
(5.83583583583584,0.82310834481829)
(5.84584584584585,0.823849778609592)
(5.85585585585586,0.824588892952742)
(5.86586586586587,0.825325691150711)
(5.87587587587588,0.826060176520934)
(5.88588588588589,0.826792352395212)
(5.8958958958959,0.827522222119609)
(5.90590590590591,0.82824978905436)
(5.91591591591592,0.828975056573764)
(5.92592592592593,0.829698028066095)
(5.93593593593594,0.830418706933496)
(5.94594594594595,0.831137096591881)
(5.95595595595596,0.831853200470842)
(5.96596596596597,0.832567022013544)
(5.97597597597598,0.833278564676629)
(5.98598598598599,0.833987831930119)
(5.995995995996,0.834694827257314)
(6.00600600600601,0.835399554154698)
(6.01601601601602,0.836102016131836)
(6.02602602602603,0.836802216711278)
(6.03603603603604,0.837500159428462)
(6.04604604604605,0.83819584783161)
(6.05605605605606,0.838889285481639)
(6.06606606606607,0.839580475952052)
(6.07607607607608,0.84026942282885)
(6.08608608608609,0.840956129710426)
(6.0960960960961,0.84164060020747)
(6.10610610610611,0.842322837942873)
(6.11611611611612,0.843002846551626)
(6.12612612612613,0.843680629680722)
(6.13613613613614,0.84435619098906)
(6.14614614614615,0.845029534147347)
(6.15615615615616,0.845700662837999)
(6.16616616616617,0.846369580755044)
(6.17617617617618,0.847036291604025)
(6.18618618618619,0.847700799101902)
(6.1961961961962,0.848363106976953)
(6.20620620620621,0.84902321896868)
(6.21621621621622,0.849681138827709)
(6.22622622622623,0.850336870315694)
(6.23623623623624,0.850990417205221)
(6.24624624624625,0.851641783279706)
(6.25625625625626,0.852290972333305)
(6.26626626626627,0.852937988170814)
(6.27627627627628,0.853582834607569)
(6.28628628628629,0.854225515469355)
(6.2962962962963,0.854866034592308)
(6.30630630630631,0.855504395822815)
(6.31631631631632,0.856140603017422)
(6.32632632632633,0.856774660042735)
(6.33633633633634,0.857406570775326)
(6.34634634634635,0.858036339101635)
(6.35635635635636,0.858663968917876)
(6.36636636636637,0.859289464129938)
(6.37637637637638,0.859912828653296)
(6.38638638638639,0.860534066412907)
(6.3963963963964,0.861153181343121)
(6.40640640640641,0.861770177387582)
(6.41641641641642,0.862385058499135)
(6.42642642642643,0.862997828639732)
(6.43643643643644,0.863608491780331)
(6.44644644644645,0.864217051900811)
(6.45645645645646,0.864823512989867)
(6.46646646646647,0.865427879044925)
(6.47647647647648,0.866030154072039)
(6.48648648648649,0.866630342085804)
(6.4964964964965,0.867228447109258)
(6.50650650650651,0.867824473173787)
(6.51651651651652,0.868418424319036)
(6.52652652652653,0.869010304592812)
(6.53653653653654,0.869600118050989)
(6.54654654654655,0.870187868757419)
(6.55655655655656,0.870773560783836)
(6.56656656656657,0.871357198209764)
(6.57657657657658,0.871938785122421)
(6.58658658658659,0.872518325616633)
(6.5965965965966,0.873095823794734)
(6.60660660660661,0.873671283766481)
(6.61661661661662,0.874244709648954)
(6.62662662662663,0.874816105566469)
(6.63663663663664,0.875385475650488)
(6.64664664664665,0.87595282403952)
(6.65665665665666,0.876518154879037)
(6.66666666666667,0.877081472321378)
(6.67667667667668,0.87764278052566)
(6.68668668668669,0.878202083657686)
(6.6966966966967,0.878759385889855)
(6.70670670670671,0.87931469140107)
(6.71671671671672,0.87986800437665)
(6.72672672672673,0.880419329008238)
(6.73673673673674,0.880968669493711)
(6.74674674674675,0.881516030037089)
(6.75675675675676,0.882061414848449)
(6.76676676676677,0.882604828143832)
(6.77677677677678,0.883146274145155)
(6.78678678678679,0.883685757080122)
(6.7967967967968,0.884223281182136)
(6.80680680680681,0.884758850690207)
(6.81681681681682,0.885292469848868)
(6.82682682682683,0.885824142908083)
(6.83683683683684,0.886353874123161)
(6.84684684684685,0.886881667754669)
(6.85685685685686,0.88740752806834)
(6.86686686686687,0.887931459334991)
(6.87687687687688,0.888453465830432)
(6.88688688688689,0.88897355183538)
(6.8968968968969,0.889491721635373)
(6.90690690690691,0.890007979520684)
(6.91691691691692,0.89052232978623)
(6.92692692692693,0.891034776731493)
(6.93693693693694,0.891545324660427)
(6.94694694694695,0.892053977881378)
(6.95695695695696,0.892560740706994)
(6.96696696696697,0.893065617454145)
(6.97697697697698,0.893568612443832)
(6.98698698698699,0.894069730001105)
(6.996996996997,0.89456897445498)
(7.00700700700701,0.895066350138351)
(7.01701701701702,0.895561861387911)
(7.02702702702703,0.896055512544062)
(7.03703703703704,0.896547307950835)
(7.04704704704705,0.897037251955807)
(7.05705705705706,0.897525348910016)
(7.06706706706707,0.898011603167879)
(7.07707707707708,0.898496019087109)
(7.08708708708709,0.898978601028632)
(7.0970970970971,0.899459353356508)
(7.10710710710711,0.899938280437843)
(7.11711711711712,0.900415386642713)
(7.12712712712713,0.90089067634408)
(7.13713713713714,0.90136415391771)
(7.14714714714715,0.901835823742094)
(7.15715715715716,0.902305690198367)
(7.16716716716717,0.902773757670224)
(7.17717717717718,0.903240030543847)
(7.18718718718719,0.903704513207816)
(7.1971971971972,0.904167210053037)
(7.20720720720721,0.904628125472659)
(7.21721721721722,0.905087263861994)
(7.22722722722723,0.905544629618441)
(7.23723723723724,0.906000227141404)
(7.24724724724725,0.906454060832216)
(7.25725725725726,0.906906135094061)
(7.26726726726727,0.907356454331893)
(7.27727727727728,0.907805022952362)
(7.28728728728729,0.908251845363734)
(7.2972972972973,0.908696925975818)
(7.30730730730731,0.909140269199881)
(7.31731731731732,0.909581879448582)
(7.32732732732733,0.910021761135887)
(7.33733733733734,0.910459918676996)
(7.34734734734735,0.910896356488271)
(7.35735735735736,0.911331078987153)
(7.36736736736737,0.911764090592094)
(7.37737737737738,0.912195395722476)
(7.38738738738739,0.912624998798541)
(7.3973973973974,0.913052904241314)
(7.40740740740741,0.913479116472532)
(7.41741741741742,0.913903639914565)
(7.42742742742743,0.914326478990345)
(7.43743743743744,0.914747638123297)
(7.44744744744745,0.915167121737258)
(7.45745745745746,0.91558493425641)
(7.46746746746747,0.916001080105206)
(7.47747747747748,0.916415563708297)
(7.48748748748749,0.916828389490462)
(7.4974974974975,0.917239561876534)
(7.50750750750751,0.917649085291331)
(7.51751751751752,0.918056964159583)
(7.52752752752753,0.918463202905865)
(7.53753753753754,0.918867805954519)
(7.54754754754755,0.919270777729593)
(7.55755755755756,0.919672122654766)
(7.56756756756757,0.920071845153276)
(7.57757757757758,0.920469949647859)
(7.58758758758759,0.92086644056067)
(7.5975975975976,0.921261322313224)
(7.60760760760761,0.921654599326319)
(7.61761761761762,0.922046276019975)
(7.62762762762763,0.922436356813359)
(7.63763763763764,0.922824846124725)
(7.64764764764765,0.923211748371342)
(7.65765765765766,0.923597067969428)
(7.66766766766767,0.923980809334083)
(7.67767767767768,0.924362976879224)
(7.68768768768769,0.924743575017519)
(7.6976976976977,0.925122608160319)
(7.70770770770771,0.925500080717595)
(7.71771771771772,0.925875997097871)
(7.72772772772773,0.926250361708161)
(7.73773773773774,0.926623178953903)
(7.74774774774775,0.926994453238895)
(7.75775775775776,0.927364188965231)
(7.76776776776777,0.927732390533239)
(7.77777777777778,0.928099062341415)
(7.78778778778779,0.928464208786361)
(7.7977977977978,0.928827834262723)
(7.80780780780781,0.929189943163129)
(7.81781781781782,0.929550539878123)
(7.82782782782783,0.92990962879611)
(7.83783783783784,0.930267214303288)
(7.84784784784785,0.930623300783588)
(7.85785785785786,0.930977892618618)
(7.86786786786787,0.931330994187597)
(7.87787787787788,0.931682609867295)
(7.88788788788789,0.932032744031976)
(7.8978978978979,0.932381401053337)
(7.90790790790791,0.932728585300449)
(7.91791791791792,0.933074301139693)
(7.92792792792793,0.933418552934712)
(7.93793793793794,0.933761345046341)
(7.94794794794795,0.934102681832555)
(7.95795795795796,0.934442567648412)
(7.96796796796797,0.93478100684599)
(7.97797797797798,0.935118003774337)
(7.98798798798799,0.935453562779408)
(7.997997997998,0.935787688204011)
(8.00800800800801,0.93612038438775)
(8.01801801801802,0.936451655666971)
(8.02802802802803,0.936781506374702)
(8.03803803803804,0.937109940840603)
(8.04804804804805,0.937436963390907)
(8.05805805805806,0.937762578348365)
(8.06806806806807,0.938086790032196)
(8.07807807807808,0.938409602758028)
(8.08808808808809,0.938731020837845)
(8.0980980980981,0.939051048579937)
(8.10810810810811,0.939369690288843)
(8.11811811811812,0.939686950265299)
(8.12812812812813,0.940002832806186)
(8.13813813813814,0.940317342204478)
(8.14814814814815,0.940630482749188)
(8.15815815815816,0.940942258725321)
(8.16816816816817,0.941252674413816)
(8.17817817817818,0.941561734091501)
(8.18818818818819,0.941869442031038)
(8.1981981981982,0.942175802500875)
(8.20820820820821,0.942480819765195)
(8.21821821821822,0.942784498083868)
(8.22822822822823,0.943086841712398)
(8.23823823823824,0.943387854901875)
(8.24824824824825,0.943687541898931)
(8.25825825825826,0.943985906945683)
(8.26826826826827,0.94428295427969)
(8.27827827827828,0.944578688133907)
(8.28828828828829,0.94487311273663)
(8.2982982982983,0.945166232311458)
(8.30830830830831,0.945458051077238)
(8.31831831831832,0.945748573248022)
(8.32832832832833,0.946037803033022)
(8.33833833833834,0.946325744636559)
(8.34834834834835,0.946612402258024)
(8.35835835835836,0.946897780091825)
(8.36836836836837,0.947181882327349)
(8.37837837837838,0.947464713148912)
(8.38838838838839,0.947746276735717)
(8.3983983983984,0.948026577261809)
(8.40840840840841,0.948305618896032)
(8.41841841841842,0.948583405801984)
(8.42842842842843,0.948859942137973)
(8.43843843843844,0.949135232056978)
(8.44844844844845,0.949409279706602)
(8.45845845845846,0.949682089229032)
(8.46846846846847,0.949953664760995)
(8.47847847847848,0.95022401043372)
(8.48848848848849,0.950493130372892)
(8.4984984984985,0.950761028698613)
(8.50850850850851,0.951027709525363)
(8.51851851851852,0.951293176961954)
(8.52852852852853,0.951557435111497)
(8.53853853853854,0.951820488071356)
(8.54854854854855,0.952082339933113)
(8.55855855855856,0.952342994782524)
(8.56856856856857,0.952602456699485)
(8.57857857857858,0.952860729757987)
(8.58858858858859,0.953117818026086)
(8.5985985985986,0.953373725565858)
(8.60860860860861,0.953628456433362)
(8.61861861861862,0.953882014678607)
(8.62862862862863,0.95413440434551)
(8.63863863863864,0.954385629471861)
(8.64864864864865,0.954635694089288)
(8.65865865865866,0.954884602223216)
(8.66866866866867,0.955132357892838)
(8.67867867867868,0.955378965111071)
(8.68868868868869,0.955624427884528)
(8.6986986986987,0.955868750213479)
(8.70870870870871,0.956111936091819)
(8.71871871871872,0.956353989507027)
(8.72872872872873,0.956594914440142)
(8.73873873873874,0.95683471486572)
(8.74874874874875,0.957073394751804)
(8.75875875875876,0.957310958059893)
(8.76876876876877,0.957547408744904)
(8.77877877877878,0.957782750755145)
(8.78878878878879,0.958016988032276)
(8.7987987987988,0.958250124511282)
(8.80880880880881,0.958482164120441)
(8.81881881881882,0.958713110781289)
(8.82882882882883,0.958942968408592)
(8.83883883883884,0.959171740910313)
(8.84884884884885,0.959399432187583)
(8.85885885885886,0.959626046134669)
(8.86886886886887,0.959851586638945)
(8.87887887887888,0.960076057580863)
(8.88888888888889,0.960299462833922)
(8.8988988988989,0.960521806264637)
(8.90890890890891,0.960743091732516)
(8.91891891891892,0.960963323090026)
(8.92892892892893,0.961182504182566)
(8.93893893893894,0.961400638848439)
(8.94894894894895,0.961617730918826)
(8.95895895895896,0.961833784217757)
(8.96896896896897,0.962048802562081)
(8.97897897897898,0.962262789761446)
(8.98898898898899,0.962475749618265)
(8.998998998999,0.962687685927694)
(9.00900900900901,0.962898602477607)
(9.01901901901902,0.963108503048565)
(9.02902902902903,0.963317391413795)
(9.03903903903904,0.963525271339162)
(9.04904904904905,0.963732146583148)
(9.05905905905906,0.963938020896822)
(9.06906906906907,0.96414289802382)
(9.07907907907908,0.964346781700319)
(9.08908908908909,0.964549675655012)
(9.0990990990991,0.964751583609088)
(9.10910910910911,0.964952509276203)
(9.11911911911912,0.965152456362466)
(9.12912912912913,0.965351428566405)
(9.13913913913914,0.965549429578953)
(9.14914914914915,0.965746463083424)
(9.15915915915916,0.965942532755488)
(9.16916916916917,0.966137642263153)
(9.17917917917918,0.966331795266742)
(9.18918918918919,0.966524995418869)
(9.1991991991992,0.966717246364427)
(9.20920920920921,0.966908551740556)
(9.21921921921922,0.967098915176631)
(9.22922922922923,0.967288340294239)
(9.23923923923924,0.96747683070716)
(9.24924924924925,0.967664390021344)
(9.25925925925926,0.967851021834898)
(9.26926926926927,0.968036729738061)
(9.27927927927928,0.968221517313191)
(9.28928928928929,0.968405388134741)
(9.2992992992993,0.968588345769244)
(9.30930930930931,0.968770393775294)
(9.31931931931932,0.968951535703529)
(9.32932932932933,0.969131775096613)
(9.33933933933934,0.96931111548922)
(9.34934934934935,0.969489560408014)
(9.35935935935936,0.969667113371638)
(9.36936936936937,0.969843777890689)
(9.37937937937938,0.970019557467712)
(9.38938938938939,0.970194455597175)
(9.3993993993994,0.970368475765461)
(9.40940940940941,0.970541621450845)
(9.41941941941942,0.970713896123487)
(9.42942942942943,0.970885303245409)
(9.43943943943944,0.971055846270488)
(9.44944944944945,0.971225528644435)
(9.45945945945946,0.971394353804785)
(9.46946946946947,0.971562325180882)
(9.47947947947948,0.971729446193865)
(9.48948948948949,0.971895720256654)
(9.4994994994995,0.97206115077394)
(9.50950950950951,0.972225741142167)
(9.51951951951952,0.972389494749523)
(9.52952952952953,0.972552414975927)
(9.53953953953954,0.972714505193017)
(9.54954954954955,0.972875768764135)
(9.55955955955956,0.973036209044321)
(9.56956956956957,0.973195829380295)
(9.57957957957958,0.973354633110451)
(9.58958958958959,0.973512623564844)
(9.5995995995996,0.973669804065178)
(9.60960960960961,0.973826177924798)
(9.61961961961962,0.973981748448677)
(9.62962962962963,0.974136518933409)
(9.63963963963964,0.974290492667195)
(9.64964964964965,0.974443672929837)
(9.65965965965966,0.974596062992729)
(9.66966966966967,0.974747666118844)
(9.67967967967968,0.974898485562728)
(9.68968968968969,0.975048524570491)
(9.6996996996997,0.975197786379797)
(9.70970970970971,0.975346274219859)
(9.71971971971972,0.975493991311427)
(9.72972972972973,0.975640940866783)
(9.73973973973974,0.975787126089731)
(9.74974974974975,0.975932550175594)
(9.75975975975976,0.9760772163112)
(9.76976976976977,0.976221127674883)
(9.77977977977978,0.97636428743647)
(9.78978978978979,0.976506698757276)
(9.7997997997998,0.976648364790102)
(9.80980980980981,0.976789288679222)
(9.81981981981982,0.976929473560381)
(9.82982982982983,0.977068922560792)
(9.83983983983984,0.977207638799125)
(9.84984984984985,0.977345625385504)
(9.85985985985986,0.977482885421503)
(9.86986986986987,0.977619422000142)
(9.87987987987988,0.97775523820588)
(9.88988988988989,0.977890337114611)
(9.8998998998999,0.97802472179366)
(9.90990990990991,0.978158395301782)
(9.91991991991992,0.978291360689154)
(9.92992992992993,0.978423620997372)
(9.93993993993994,0.978555179259451)
(9.94994994994995,0.97868603849982)
(9.95995995995996,0.978816201734316)
(9.96996996996997,0.978945671970186)
(9.97997997997998,0.979074452206082)
(9.98998998998999,0.97920254543206)
(10,0.979329954629575)

};
\addplot [red, opacity=0.25, mark=square*, mark size=3, mark options={draw=black}, only marks]
coordinates {
(0,-7.91e-101)
(0.1,0.0606461)
(0.2,0.120911)
(0.3,0.180421)
(0.4,0.23882)
(0.5,0.295773)
(0.6,0.350975)
(0.7,0.404157)
(0.8,0.455087)
(0.9,0.503573)
(1,0.549466)
(1.1,0.592661)
(1.2,0.63309)
(1.3,0.670727)
(1.4,0.705579)
(1.5,0.737686)
(1.6,0.767117)
(1.7,0.793963)
(1.8,0.818334)
(1.9,0.840358)
(2,0.86017)
(2.1,0.877916)
(2.2,0.893744)
(2.3,0.907804)
(2.4,0.920243)
(2.5,0.931206)
(2.6,0.940831)
(2.7,0.949252)
(2.8,0.956594)
(2.9,0.962972)
(3,0.968495)
(3.1,0.973262)
(3.2,0.977364)
(3.3,0.980883)
(3.4,0.983892)
(3.5,0.98646)
(3.6,0.988643)
(3.7,0.990495)
(3.8,0.992062)
(3.9,0.993385)
(4,0.994498)
(4.1,0.995433)
(4.2,0.996217)
(4.3,0.996872)
(4.4,0.997418)
(4.5,0.997873)
(4.6,0.99825)
(4.7,0.998563)
(4.8,0.998823)
(4.9,0.999036)
(5,0.999213)
(5.1,0.999358)
(5.2,0.999477)
(5.3,0.999575)
(5.4,0.999655)
(5.5,0.99972)
(5.6,0.999773)
(5.7,0.999817)
(5.8,0.999852)
(5.9,0.99988)
(6,0.999904)
(6.1,0.999923)
(6.2,0.999938)
(6.3,0.99995)
(6.4,0.99996)
(6.5,0.999968)
(6.6,0.999974)
(6.7,0.99998)
(6.8,0.999984)
(6.9,0.999987)
(7,0.99999)
(7.1,0.999992)
(7.2,0.999993)
(7.3,0.999995)
(7.4,0.999996)
(7.5,0.999997)
(7.6,0.999997)
(7.7,0.999998)
(7.8,0.999998)
(7.9,0.999999)
(8,0.999999)
(8.1,0.999999)
(8.2,0.999999)
(8.3,1)
(8.4,1)
(8.5,1)
(8.6,1)
(8.7,1)
(8.8,1)
(8.9,1)
(9,1)
(9.1,1)
(9.2,1)
(9.3,1)
(9.4,1)
(9.5,1)
(9.6,1)
(9.7,1)
(9.8,1)
(9.9,1)
(10,1)

};
\addplot [green!50.0!black, opacity=0.25, mark=square*, mark size=3, mark options={draw=black}, only marks]
coordinates {
(0,0)
(0.1,0.0185323)
(0.2,0.0370544)
(0.3,0.0555565)
(0.4,0.0740285)
(0.5,0.0924604)
(0.6,0.110842)
(0.7,0.129164)
(0.8,0.147417)
(0.9,0.165591)
(1,0.183676)
(1.1,0.201663)
(1.2,0.219543)
(1.3,0.237307)
(1.4,0.254946)
(1.5,0.272452)
(1.6,0.289815)
(1.7,0.307029)
(1.8,0.324085)
(1.9,0.340975)
(2,0.357693)
(2.1,0.37423)
(2.2,0.39058)
(2.3,0.406736)
(2.4,0.422692)
(2.5,0.438443)
(2.6,0.453981)
(2.7,0.469303)
(2.8,0.484403)
(2.9,0.499276)
(3,0.513918)
(3.1,0.528324)
(3.2,0.542491)
(3.3,0.556416)
(3.4,0.570095)
(3.5,0.583526)
(3.6,0.596706)
(3.7,0.609634)
(3.8,0.622306)
(3.9,0.634723)
(4,0.646882)
(4.1,0.658782)
(4.2,0.670425)
(4.3,0.681808)
(4.4,0.692933)
(4.5,0.703799)
(4.6,0.714407)
(4.7,0.724759)
(4.8,0.734855)
(4.9,0.744697)
(5,0.754286)
(5.1,0.763626)
(5.2,0.772717)
(5.3,0.781562)
(5.4,0.790165)
(5.5,0.798527)
(5.6,0.806652)
(5.7,0.814544)
(5.8,0.822206)
(5.9,0.829641)
(6,0.836854)
(6.1,0.843848)
(6.2,0.850627)
(6.3,0.857197)
(6.4,0.863561)
(6.5,0.869724)
(6.6,0.875691)
(6.7,0.881466)
(6.8,0.887054)
(6.9,0.892461)
(7,0.897692)
(7.1,0.902751)
(7.2,0.907644)
(7.3,0.912377)
(7.4,0.916954)
(7.5,0.921382)
(7.6,0.925665)
(7.7,0.929809)
(7.8,0.933821)
(7.9,0.937704)
(8,0.941466)
(8.1,0.945112)
(8.2,0.948647)
(8.3,0.952078)
(8.4,0.95541)
(8.5,0.958649)
(8.6,0.9618)
(8.7,0.964869)
(8.8,0.967863)
(8.9,0.970787)
(9,0.973647)
(9.1,0.976449)
(9.2,0.979198)
(9.3,0.9819)
(9.4,0.984561)
(9.5,0.987188)
(9.6,0.989785)
(9.7,0.992359)
(9.8,0.994916)
(9.9,0.997461)
(10,1)

};
\path [draw=black, fill opacity=0] (axis cs:13,1)--(axis cs:13,1);

\path [draw=black, fill opacity=0] (axis cs:10,13)--(axis cs:10,13);

\path [draw=black, fill opacity=0] (axis cs:13,0)--(axis cs:13,0);

\path [draw=black, fill opacity=0] (axis cs:0,13)--(axis cs:0,13);

\end{axis}

\end{tikzpicture}

	\end{center}
	%\caption{}
	\label{fig:heatDiff}
\end{figure}