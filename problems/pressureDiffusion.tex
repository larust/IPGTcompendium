\section{Pressure Diffusion Benchmark}
\subsection{Problem description}
This example Tests the ability of a code to represent pressure dissipation with a slightly compressible fluid. The problem is defined assuming Darcian flow in a porous medium with Dirichlet conditions on the left and right sides of the \SI[scientific-notation=false, round-precision=2]{100.0}{\metre} long by \SI[scientific-notation=false, round-precision=2]{10.0}{\metre} thick domain (Figure~\ref{fig:presDiffFig}). The initial initial condition is uniform pressure (\SI[scientific-notation=false, round-precision=2]{0.1}{\mega\pascal}). The temperature in the simulation is fixed at \SI[scientific-notation=false, round-precision=2]{50}{\degreeCelsius}. Output data for x-axis pressures are calculated for times of \SIlist[scientific-notation=false, round-precision=2]{2.0; 20.0; 200.0}{\second}. The rock matrix is assumed to be incompressible.

\begin{figure}[h]
	\centering
	\import{./figures/}{theisFig.pdf_tex}
	\caption{Model domain}
	\label{fig:presDiffFig}
\end{figure}

The analytical solution applied is eq. 4.2.2 of Crank (1975) for that of Section 4.3.3 Uniform initial distribution. Surface concentrations different:

\begin{align}
	f(x,t)=&P_1+ (P_2-P_1)\frac{x}{X}+\frac{2}{\pi}\sum_{n=1}^{\infty}\left[ \frac{1}{n}(P_2\cos(n\pi)-P_1)\sin \left( \frac{n\pi x}{X}\right)\exp \left( -\frac{D_{\text{eff}}n^2\pi^2 t}{X^2} \right) \right] \nonumber\\
	+& \frac{4 P_0}{\pi}\sum_{m=0}^{\infty}\left[ \frac{1}{2m+1}\sin\left( \frac{(2m+1)\pi x}{X} \right)\exp \left( -\frac{D_{\text{eff}}(2m+1)^2\pi^2 t}{X^2} \right) \right]
	\label{eq:anaPresDiff}
\end{align}

The solution was computed using Mathcad, Version 15, using parameter values given below.

\begin{table}[h]
	\caption{Dimensions and material properties}
	\begin{center}
	\begin{tabular}{lrcSs}
		 System length & $X$ & $\coloneqq$ &  100.0 & \si{\metre} \\
		 Thickness & $b$ & $\coloneqq$ & 10.0 & \si{\metre} \\
		 Permeability & $k$ & $\coloneqq$ & 1.0e-12 & \si{\metre\squared} \\
		 Porosity of matrix& $\theta_{\text{unit}}$ & $\coloneqq$ & 0.50 &  \\
		 Dynamic viscosity, water & $\mu$ &$\coloneqq$& 0.0005465159967814679 & \si{\pascal\second} \\
		 Fluid compressibility & $\beta$&$\coloneqq$& 4.4173069067307806e-10 & \si{\per\pascal} \\
	\end{tabular}
	\end{center}
	\label{tab:pressDiffDim}
\end{table}
\begin{table}[h]
	\caption{Boundary and initial conditions}
	\begin{center}
	\begin{tabular}{lrcSs}
		Initial pressure & $P_0$ & $\coloneqq$ & 1.0e5 & \si{\pascal} \\
		Pressure, left side & $P_1$ & $\coloneqq$ & 1.0e5 & \si{\pascal} \\
		Pressure, right side & $P_2$ & $\coloneqq$ & 1.1e5 & \si{\pascal} \\
	\end{tabular}
	\end{center}
	\label{tab:pressDiff}
\end{table}
\begin{table}[h]
	\caption{Derived parameters}
	\begin{center}
		\begin{tabular}{lrclcSs}
			Hydraulic diffusivity & $D_{\text{eff}}$ & $\coloneqq$ & $\dfrac{k}{\theta_{\text{unit}}\beta\mu} $ & $=$ & 8.284561816837185 & \si{\metre\squared\per\second} \\
	\end{tabular}
	\end{center}
	\label{tab:pressDiffDerivPar}

\end{table}
\subsection{Files}
\begin{itemize}
	\item Analytical results are stored in \verb|FluidDiffusionBM-analytical.xlsx|
	\item Numerical results are stored in \verb|FluidDiffusionBM-FALCON.xlsx|
	\item The numerical solution input file is \verb|H1_pressure_diffusion.i|
	\item The numerical solution exodus output file is \verb|H1_pressure_diffusion_outt.e|
\end{itemize}

\subsection{Results}
\begin{figure}[h]
	\begin{center}
		\setlength\figureheight{8cm} 
		\setlength\figurewidth{0.80\textwidth} 
		% This file was created by matplotlib v0.1.0.
% Copyright (c) 2010--2014, Nico Schlömer <nico.schloemer@gmail.com>
% All rights reserved.
% 

\begin{tikzpicture}

\begin{axis}[
xlabel={Distance [m]},
ylabel={Pressure [Pa]},
xmin=0, xmax=100,
ymin=98000, ymax=110000,
axis on top,
width=\figurewidth,
height=\figureheight,
legend entries={{Analytic $t=2.00$ s},{Analytic $t=20.0$ s},{Analytic $t=200$ s},{FALCON $t=2.00$ s},{FALCON $t=20.0$ s},{FALCON $t=200$ s}}
]
\addplot [blue]
coordinates {
(0,110000)
(0.1001001001001,109861.264398862)
(0.2002002002002,109722.570739885)
(0.3003003003003,109583.960927196)
(0.4004004004004,109445.476788907)
(0.5005005005005,109307.160039254)
(0.6006006006006,109169.052240901)
(0.700700700700701,109031.194767485)
(0.800800800800801,108893.628766431)
(0.900900900900901,108756.39512212)
(1.001001001001,108619.534419444)
(1.1011011011011,108483.086907817)
(1.2012012012012,108347.092465683)
(1.3013013013013,108211.590565588)
(1.4014014014014,108076.620239844)
(1.5015015015015,107942.220046866)
(1.6016016016016,107808.428038196)
(1.7017017017017,107675.28172629)
(1.8018018018018,107542.818053096)
(1.9019019019019,107411.073359472)
(2.002002002002,107280.083355488)
(2.1021021021021,107149.883091645)
(2.2022022022022,107020.506931063)
(2.3023023023023,106891.988522652)
(2.4024024024024,106764.360775328)
(2.5025025025025,106637.655833283)
(2.6026026026026,106511.905052348)
(2.7027027027027,106387.138977481)
(2.8028028028028,106263.387321397)
(2.9029029029029,106140.678944373)
(3.003003003003,106019.041835236)
(3.1031031031031,105898.503093568)
(3.2032032032032,105779.088913132)
(3.3033033033033,105660.824566542)
(3.4034034034034,105543.73439118)
(3.5035035035035,105427.841776378)
(3.6036036036036,105313.169151866)
(3.7037037037037,105199.737977493)
(3.8038038038038,105087.568734221)
(3.9039039039039,104976.680916401)
(4.004004004004,104867.093025306)
(4.1041041041041,104758.82256395)
(4.2042042042042,104651.886033157)
(4.3043043043043,104546.298928881)
(4.4044044044044,104442.075740772)
(4.5045045045045,104339.229951966)
(4.6046046046046,104237.774040091)
(4.7047047047047,104137.719479456)
(4.8048048048048,104039.076744431)
(4.9049049049049,103941.855313973)
(5.005005005005,103846.063677279)
(5.1051051051051,103751.709340554)
(5.2052052052052,103658.798834857)
(5.3053053053053,103567.337724993)
(5.40540540540541,103477.330619441)
(5.50550550550551,103388.781181262)
(5.60560560560561,103301.692139976)
(5.70570570570571,103216.065304359)
(5.80580580580581,103131.901576142)
(5.90590590590591,103049.200964561)
(6.00600600600601,102967.962601729)
(6.10610610610611,102888.184758796)
(6.20620620620621,102809.86486285)
(6.30630630630631,102732.99951453)
(6.40640640640641,102657.584506304)
(6.50650650650651,102583.614841378)
(6.60660660660661,102511.084753192)
(6.70670670670671,102439.987725466)
(6.80680680680681,102370.316512751)
(6.90690690690691,102302.063161452)
(7.00700700700701,102235.219031268)
(7.10710710710711,102169.774817027)
(7.20720720720721,102105.720570861)
(7.30730730730731,102043.045724689)
(7.40740740740741,101981.739112957)
(7.50750750750751,101921.788995612)
(7.60760760760761,101863.183081258)
(7.70770770770771,101805.908550455)
(7.80780780780781,101749.952079135)
(7.90790790790791,101695.299862082)
(8.00800800800801,101641.937636449)
(8.10810810810811,101589.850705279)
(8.20820820820821,101539.023960979)
(8.30830830830831,101489.441908736)
(8.40840840840841,101441.088689817)
(8.50850850850851,101393.948104741)
(8.60860860860861,101348.003636285)
(8.70870870870871,101303.238472283)
(8.80880880880881,101259.635528208)
(8.90890890890891,101217.177469497)
(9.00900900900901,101175.846733591)
(9.10910910910911,101135.625551675)
(9.20920920920921,101096.495970079)
(9.30930930930931,101058.439871334)
(9.40940940940941,101021.438994843)
(9.50950950950951,100985.474957164)
(9.60960960960961,100950.529271871)
(9.70970970970971,100916.583368981)
(9.80980980980981,100883.618613938)
(9.90990990990991,100851.616326121)
(10.01001001001,100820.557796882)
(10.1101101101101,100790.42430708)
(10.2102102102102,100761.197144119)
(10.3103103103103,100732.857618469)
(10.4104104104104,100705.387079658)
(10.5105105105105,100678.766931739)
(10.6106106106106,100652.978648213)
(10.7107107107107,100628.003786407)
(10.8108108108108,100603.824001308)
(10.9109109109109,100580.421058836)
(11.011011011011,100557.776848576)
(11.1111111111111,100535.873395935)
(11.2112112112112,100514.692873762)
(11.3113113113113,100494.217613402)
(11.4114114114114,100474.430115198)
(11.5115115115115,100455.313058443)
(11.6116116116116,100436.84931079)
(11.7117117117117,100419.021937108)
(11.8118118118118,100401.814207815)
(11.9119119119119,100385.209606667)
(12.012012012012,100369.191838031)
(12.1121121121121,100353.744833638)
(12.2122122122122,100338.852758829)
(12.3123123123123,100324.500018296)
(12.4124124124124,100310.671261341)
(12.5125125125125,100297.351386643)
(12.6126126126126,100284.525546572)
(12.7127127127127,100272.17915102)
(12.8128128128128,100260.297870808)
(12.9129129129129,100248.867640639)
(13.013013013013,100237.874661633)
(13.1131131131131,100227.305403453)
(13.2132132132132,100217.146606024)
(13.3133133133133,100207.385280872)
(13.4134134134134,100198.008712085)
(13.5135135135135,100189.004456913)
(13.6136136136136,100180.360346027)
(13.7137137137137,100172.06448343)
(13.8138138138138,100164.105246067)
(13.9139139139139,100156.471283109)
(14.014014014014,100149.151514953)
(14.1141141141141,100142.135131944)
(14.2142142142142,100135.411592821)
(14.3143143143143,100128.970622917)
(14.4144144144144,100122.802212117)
(14.5145145145145,100116.896612587)
(14.6146146146146,100111.244336288)
(14.7147147147147,100105.836152294)
(14.8148148148148,100100.663083911)
(14.9149149149149,100095.716405634)
(15.015015015015,100090.987639923)
(15.1151151151151,100086.468553837)
(15.2152152152152,100082.15115553)
(15.3153153153153,100078.027690602)
(15.4154154154154,100074.090638352)
(15.5155155155155,100070.332707909)
(15.6156156156156,100066.746834268)
(15.7157157157157,100063.326174249)
(15.8158158158158,100060.064102367)
(15.9159159159159,100056.95420664)
(16.016016016016,100053.990284345)
(16.1161161161161,100051.166337711)
(16.2162162162162,100048.476569588)
(16.3163163163163,100045.915379072)
(16.4164164164164,100043.477357114)
(16.5165165165165,100041.157282104)
(16.6166166166166,100038.950115456)
(16.7167167167167,100036.850997182)
(16.8168168168168,100034.855241479)
(16.9169169169169,100032.958332316)
(17.017017017017,100031.155919052)
(17.1171171171171,100029.443812059)
(17.2172172172172,100027.817978392)
(17.3173173173173,100026.274537475)
(17.4174174174174,100024.809756833)
(17.5175175175175,100023.420047866)
(17.6176176176176,100022.101961666)
(17.7177177177177,100020.852184885)
(17.8178178178178,100019.667535657)
(17.9179179179179,100018.544959572)
(18.018018018018,100017.481525721)
(18.1181181181181,100016.474422786)
(18.2182182182182,100015.520955214)
(18.3183183183183,100014.618539438)
(18.4184184184184,100013.764700187)
(18.5185185185185,100012.957066848)
(18.6186186186186,100012.193369916)
(18.7187187187187,100011.471437507)
(18.8188188188188,100010.789191947)
(18.9189189189189,100010.144646445)
(19.019019019019,100009.535901832)
(19.1191191191191,100008.96114338)
(19.2192192192192,100008.418637703)
(19.3193193193193,100007.906729732)
(19.4194194194194,100007.423839762)
(19.5195195195195,100006.968460593)
(19.6196196196196,100006.539154726)
(19.7197197197197,100006.134551659)
(19.8198198198198,100005.75334524)
(19.9199199199199,100005.394291113)
(20.02002002002,100005.056204226)
(20.1201201201201,100004.737956422)
(20.2202202202202,100004.4384741)
(20.3203203203203,100004.156735955)
(20.4204204204204,100003.89177078)
(20.5205205205205,100003.64265535)
(20.6206206206206,100003.408512368)
(20.7207207207207,100003.188508484)
(20.8208208208208,100002.981852381)
(20.9209209209209,100002.787792926)
(21.021021021021,100002.605617385)
(21.1211211211211,100002.434649708)
(21.2212212212212,100002.274248867)
(21.3213213213213,100002.123807263)
(21.4214214214214,100001.982749188)
(21.5215215215215,100001.850529343)
(21.6216216216216,100001.726631424)
(21.7217217217217,100001.610566747)
(21.8218218218218,100001.501872937)
(21.9219219219219,100001.40011267)
(22.022022022022,100001.304872458)
(22.1221221221221,100001.215761493)
(22.2222222222222,100001.132410528)
(22.3223223223223,100001.054470814)
(22.4224224224224,100000.981613081)
(22.5225225225225,100000.913526551)
(22.6226226226226,100000.849918012)
(22.7227227227227,100000.790510914)
(22.8228228228228,100000.73504452)
(22.9229229229229,100000.683273084)
(23.023023023023,100000.63496507)
(23.1231231231231,100000.589902409)
(23.2232232232232,100000.547879785)
(23.3233233233233,100000.508703956)
(23.4234234234234,100000.472193109)
(23.5235235235235,100000.438176243)
(23.6236236236236,100000.40649258)
(23.7237237237237,100000.376991009)
(23.8238238238238,100000.349529551)
(23.9239239239239,100000.323974854)
(24.024024024024,100000.300201714)
(24.1241241241241,100000.278092614)
(24.2242242242242,100000.257537293)
(24.3243243243243,100000.238432333)
(24.4244244244244,100000.220680766)
(24.5245245245245,100000.204191704)
(24.6246246246246,100000.188879988)
(24.7247247247247,100000.17466585)
(24.8248248248248,100000.161474605)
(24.9249249249249,100000.149236341)
(25.025025025025,100000.137885646)
(25.1251251251251,100000.127361331)
(25.2252252252252,100000.117606183)
(25.3253253253253,100000.10856672)
(25.4254254254254,100000.100192969)
(25.5255255255255,100000.092438249)
(25.6256256256256,100000.085258968)
(25.7257257257257,100000.078614436)
(25.8258258258258,100000.072466681)
(25.9259259259259,100000.066780281)
(26.026026026026,100000.061522204)
(26.1261261261261,100000.056661657)
(26.2262262262262,100000.052169943)
(26.3263263263263,100000.04802033)
(26.4264264264264,100000.04418792)
(26.5265265265265,100000.040649537)
(26.6266266266266,100000.03738361)
(26.7267267267267,100000.03437007)
(26.8268268268268,100000.031590256)
(26.9269269269269,100000.029026813)
(27.027027027027,100000.026663616)
(27.1271271271271,100000.024485681)
(27.2272272272272,100000.02247909)
(27.3273273273273,100000.020630923)
(27.4274274274274,100000.018929187)
(27.5275275275275,100000.017362752)
(27.6276276276276,100000.015921299)
(27.7277277277277,100000.014595255)
(27.8278278278278,100000.01337575)
(27.9279279279279,100000.012254562)
(28.028028028028,100000.011224078)
(28.1281281281281,100000.010277245)
(28.2282282282282,100000.009407535)
(28.3283283283283,100000.00860891)
(28.4284284284284,100000.007875779)
(28.5285285285285,100000.007202976)
(28.6286286286286,100000.006585723)
(28.7287287287287,100000.006019604)
(28.8288288288288,100000.005500541)
(28.9289289289289,100000.005024766)
(29.029029029029,100000.004588801)
(29.1291291291291,100000.004189436)
(29.2292292292292,100000.003823708)
(29.3293293293293,100000.003488886)
(29.4294294294294,100000.003182451)
(29.5295295295295,100000.002902081)
(29.6296296296296,100000.002645635)
(29.7297297297297,100000.002411145)
(29.8298298298298,100000.002196794)
(29.9299299299299,100000.002000913)
(30.03003003003,100000.001821963)
(30.1301301301301,100000.001658532)
(30.2302302302302,100000.001509317)
(30.3303303303303,100000.001373125)
(30.4304304304304,100000.001248855)
(30.5305305305305,100000.001135499)
(30.6306306306306,100000.001032129)
(30.7307307307307,100000.000937894)
(30.8308308308308,100000.000852013)
(30.9309309309309,100000.000773768)
(31.031031031031,100000.000702503)
(31.1311311311311,100000.000637614)
(31.2312312312312,100000.000578549)
(31.3313313313313,100000.000524802)
(31.4314314314314,100000.000475907)
(31.5315315315315,100000.000431441)
(31.6316316316316,100000.000391015)
(31.7317317317317,100000.000354273)
(31.8318318318318,100000.000320888)
(31.9319319319319,100000.000290565)
(32.032032032032,100000.000263029)
(32.1321321321321,100000.000238033)
(32.2322322322322,100000.000215349)
(32.3323323323323,100000.000194769)
(32.4324324324324,100000.000176105)
(32.5325325325325,100000.000159182)
(32.6326326326326,100000.000143843)
(32.7327327327327,100000.000129943)
(32.8328328328328,100000.000117353)
(32.9329329329329,100000.000105951)
(33.033033033033,100000.000095628)
(33.1331331331331,100000.000086286)
(33.2332332332332,100000.000077834)
(33.3333333333333,100000.000070189)
(33.4334334334334,100000.000063276)
(33.5335335335335,100000.000057027)
(33.6336336336336,100000.00005138)
(33.7337337337337,100000.000046279)
(33.8338338338338,100000.000041672)
(33.9339339339339,100000.000037512)
(34.034034034034,100000.000033758)
(34.1341341341341,100000.000030371)
(34.2342342342342,100000.000027315)
(34.3343343343343,100000.000024559)
(34.4344344344344,100000.000022075)
(34.5345345345345,100000.000019837)
(34.6346346346346,100000.00001782)
(34.7347347347347,100000.000016003)
(34.8348348348348,100000.000014368)
(34.9349349349349,100000.000012896)
(35.035035035035,100000.000011571)
(35.1351351351351,100000.000010379)
(35.2352352352352,100000.000009307)
(35.3353353353353,100000.000008344)
(35.4354354354354,100000.000007478)
(35.5355355355355,100000.0000067)
(35.6356356356356,100000.000006001)
(35.7357357357357,100000.000005373)
(35.8358358358358,100000.00000481)
(35.9359359359359,100000.000004304)
(36.036036036036,100000.000003851)
(36.1361361361361,100000.000003444)
(36.2362362362362,100000.000003079)
(36.3363363363363,100000.000002752)
(36.4364364364364,100000.000002459)
(36.5365365365365,100000.000002197)
(36.6366366366366,100000.000001962)
(36.7367367367367,100000.000001751)
(36.8368368368368,100000.000001563)
(36.9369369369369,100000.000001395)
(37.037037037037,100000.000001244)
(37.1371371371371,100000.000001109)
(37.2372372372372,100000.000000989)
(37.3373373373373,100000.000000881)
(37.4374374374374,100000.000000785)
(37.5375375375375,100000.000000699)
(37.6376376376376,100000.000000623)
(37.7377377377377,100000.000000554)
(37.8378378378378,100000.000000493)
(37.9379379379379,100000.000000439)
(38.038038038038,100000.00000039)
(38.1381381381381,100000.000000347)
(38.2382382382382,100000.000000308)
(38.3383383383383,100000.000000274)
(38.4384384384384,100000.000000243)
(38.5385385385385,100000.000000216)
(38.6386386386386,100000.000000192)
(38.7387387387387,100000.00000017)
(38.8388388388388,100000.000000151)
(38.9389389389389,100000.000000134)
(39.039039039039,100000.000000119)
(39.1391391391391,100000.000000105)
(39.2392392392392,100000.000000093)
(39.3393393393393,100000.000000083)
(39.4394394394394,100000.000000073)
(39.5395395395395,100000.000000065)
(39.6396396396396,100000.000000057)
(39.7397397397397,100000.000000051)
(39.8398398398398,100000.000000045)
(39.9399399399399,100000.00000004)
(40.04004004004,100000.000000035)
(40.1401401401401,100000.000000031)
(40.2402402402402,100000.000000027)
(40.3403403403403,100000.000000024)
(40.4404404404404,100000.000000021)
(40.5405405405405,100000.000000019)
(40.6406406406406,100000.000000017)
(40.7407407407407,100000.000000015)
(40.8408408408408,100000.000000013)
(40.9409409409409,100000.000000012)
(41.041041041041,100000.00000001)
(41.1411411411411,100000.000000009)
(41.2412412412412,100000.000000008)
(41.3413413413413,100000.000000007)
(41.4414414414414,100000.000000006)
(41.5415415415415,100000.000000005)
(41.6416416416416,100000.000000005)
(41.7417417417417,100000.000000004)
(41.8418418418418,100000.000000004)
(41.9419419419419,100000.000000003)
(42.042042042042,100000.000000003)
(42.1421421421421,100000.000000003)
(42.2422422422422,100000.000000002)
(42.3423423423423,100000.000000002)
(42.4424424424424,100000.000000002)
(42.5425425425425,100000.000000001)
(42.6426426426426,100000.000000001)
(42.7427427427427,100000.000000001)
(42.8428428428428,100000.000000001)
(42.9429429429429,100000.000000001)
(43.043043043043,100000.000000001)
(43.1431431431431,100000.000000001)
(43.2432432432432,100000.000000001)
(43.3433433433433,100000.000000001)
(43.4434434434434,100000.000000001)
(43.5435435435435,100000)
(43.6436436436436,100000)
(43.7437437437437,100000)
(43.8438438438438,100000)
(43.9439439439439,100000)
(44.044044044044,100000)
(44.1441441441441,100000)
(44.2442442442442,100000)
(44.3443443443443,100000)
(44.4444444444444,100000)
(44.5445445445445,100000)
(44.6446446446446,100000)
(44.7447447447447,100000)
(44.8448448448448,100000)
(44.9449449449449,100000)
(45.045045045045,100000)
(45.1451451451451,100000)
(45.2452452452452,100000)
(45.3453453453453,100000)
(45.4454454454454,100000)
(45.5455455455455,100000)
(45.6456456456456,100000)
(45.7457457457457,100000)
(45.8458458458458,100000)
(45.9459459459459,99999.9999999999)
(46.046046046046,100000)
(46.1461461461461,100000)
(46.2462462462462,100000)
(46.3463463463463,100000)
(46.4464464464464,100000)
(46.5465465465465,100000)
(46.6466466466466,100000)
(46.7467467467467,100000)
(46.8468468468468,100000)
(46.9469469469469,100000)
(47.047047047047,100000)
(47.1471471471471,99999.9999999999)
(47.2472472472472,99999.9999999999)
(47.3473473473473,100000)
(47.4474474474474,100000)
(47.5475475475475,100000)
(47.6476476476476,100000)
(47.7477477477477,100000)
(47.8478478478478,100000)
(47.9479479479479,100000)
(48.048048048048,100000)
(48.1481481481481,100000)
(48.2482482482482,100000)
(48.3483483483483,100000)
(48.4484484484484,100000)
(48.5485485485485,100000)
(48.6486486486486,100000)
(48.7487487487487,100000)
(48.8488488488488,99999.9999999999)
(48.9489489489489,100000)
(49.049049049049,100000)
(49.1491491491491,100000)
(49.2492492492492,100000)
(49.3493493493493,100000)
(49.4494494494494,100000)
(49.5495495495495,99999.9999999999)
(49.6496496496496,100000)
(49.7497497497497,100000)
(49.8498498498498,100000)
(49.9499499499499,100000)
(50.05005005005,100000)
(50.1501501501501,100000)
(50.2502502502502,100000)
(50.3503503503503,100000)
(50.4504504504504,100000)
(50.5505505505505,100000)
(50.6506506506506,100000)
(50.7507507507507,100000)
(50.8508508508508,100000)
(50.9509509509509,99999.9999999999)
(51.051051051051,100000)
(51.1511511511511,100000)
(51.2512512512512,100000)
(51.3513513513513,100000)
(51.4514514514514,100000)
(51.5515515515515,100000)
(51.6516516516516,99999.9999999999)
(51.7517517517517,100000)
(51.8518518518518,100000)
(51.9519519519519,100000)
(52.052052052052,100000)
(52.1521521521521,100000)
(52.2522522522522,100000)
(52.3523523523523,100000)
(52.4524524524524,100000)
(52.5525525525525,100000)
(52.6526526526526,100000)
(52.7527527527527,100000)
(52.8528528528528,100000)
(52.9529529529529,100000)
(53.053053053053,100000)
(53.1531531531531,100000)
(53.2532532532532,100000)
(53.3533533533533,100000)
(53.4534534534534,100000)
(53.5535535535535,100000)
(53.6536536536536,100000)
(53.7537537537537,100000)
(53.8538538538538,100000)
(53.9539539539539,100000)
(54.054054054054,100000)
(54.1541541541541,100000)
(54.2542542542542,100000)
(54.3543543543543,100000)
(54.4544544544544,100000)
(54.5545545545545,100000)
(54.6546546546546,100000)
(54.7547547547547,100000)
(54.8548548548548,100000)
(54.9549549549549,100000)
(55.055055055055,100000)
(55.1551551551551,100000)
(55.2552552552552,100000)
(55.3553553553553,100000)
(55.4554554554554,100000)
(55.5555555555556,100000)
(55.6556556556557,100000)
(55.7557557557558,100000)
(55.8558558558559,100000)
(55.955955955956,100000)
(56.0560560560561,100000)
(56.1561561561562,100000)
(56.2562562562563,100000)
(56.3563563563564,100000)
(56.4564564564565,100000)
(56.5565565565566,100000)
(56.6566566566567,100000)
(56.7567567567568,100000)
(56.8568568568569,100000)
(56.956956956957,100000)
(57.0570570570571,100000)
(57.1571571571572,100000)
(57.2572572572573,100000)
(57.3573573573574,100000)
(57.4574574574575,100000)
(57.5575575575576,100000)
(57.6576576576577,100000)
(57.7577577577578,100000)
(57.8578578578579,100000)
(57.957957957958,100000)
(58.0580580580581,100000)
(58.1581581581582,100000)
(58.2582582582583,100000)
(58.3583583583584,100000)
(58.4584584584585,100000)
(58.5585585585586,100000)
(58.6586586586587,99999.9999999999)
(58.7587587587588,100000)
(58.8588588588589,100000)
(58.958958958959,100000)
(59.0590590590591,100000)
(59.1591591591592,100000)
(59.2592592592593,99999.9999999999)
(59.3593593593594,100000)
(59.4594594594595,100000)
(59.5595595595596,100000)
(59.6596596596597,100000)
(59.7597597597598,100000)
(59.8598598598599,100000)
(59.95995995996,100000)
(60.0600600600601,100000)
(60.1601601601602,100000)
(60.2602602602603,100000)
(60.3603603603604,100000)
(60.4604604604605,100000)
(60.5605605605606,100000)
(60.6606606606607,100000)
(60.7607607607608,100000)
(60.8608608608609,100000)
(60.960960960961,100000)
(61.0610610610611,100000)
(61.1611611611612,100000)
(61.2612612612613,100000)
(61.3613613613614,100000)
(61.4614614614615,100000)
(61.5615615615616,100000)
(61.6616616616617,100000)
(61.7617617617618,100000)
(61.8618618618619,100000)
(61.961961961962,100000)
(62.0620620620621,100000)
(62.1621621621622,100000)
(62.2622622622623,100000)
(62.3623623623624,100000)
(62.4624624624625,100000)
(62.5625625625626,100000)
(62.6626626626627,100000)
(62.7627627627628,100000)
(62.8628628628629,100000)
(62.962962962963,100000)
(63.0630630630631,100000)
(63.1631631631632,100000)
(63.2632632632633,100000)
(63.3633633633634,100000)
(63.4634634634635,100000)
(63.5635635635636,99999.9999999999)
(63.6636636636637,100000)
(63.7637637637638,100000)
(63.8638638638639,100000)
(63.963963963964,100000)
(64.0640640640641,100000)
(64.1641641641641,100000)
(64.2642642642643,100000)
(64.3643643643643,100000)
(64.4644644644645,100000)
(64.5645645645645,100000)
(64.6646646646647,100000)
(64.7647647647647,100000)
(64.8648648648649,100000)
(64.9649649649649,100000)
(65.0650650650651,100000)
(65.1651651651651,100000)
(65.2652652652653,100000)
(65.3653653653653,99999.9999999999)
(65.4654654654655,100000)
(65.5655655655655,100000)
(65.6656656656657,100000)
(65.7657657657657,100000)
(65.8658658658659,99999.9999999999)
(65.9659659659659,100000)
(66.0660660660661,99999.9999999999)
(66.1661661661661,100000)
(66.2662662662663,100000)
(66.3663663663663,99999.9999999999)
(66.4664664664665,100000)
(66.5665665665666,99999.9999999999)
(66.6666666666667,100000)
(66.7667667667668,100000)
(66.8668668668669,100000)
(66.966966966967,100000)
(67.0670670670671,100000)
(67.1671671671672,100000)
(67.2672672672673,100000)
(67.3673673673674,100000)
(67.4674674674675,100000)
(67.5675675675676,99999.9999999999)
(67.6676676676677,100000)
(67.7677677677678,100000)
(67.8678678678679,100000)
(67.967967967968,100000)
(68.0680680680681,100000)
(68.1681681681682,100000)
(68.2682682682683,100000)
(68.3683683683684,100000)
(68.4684684684685,100000)
(68.5685685685686,100000)
(68.6686686686687,100000)
(68.7687687687688,100000)
(68.8688688688689,99999.9999999999)
(68.968968968969,100000)
(69.0690690690691,100000)
(69.1691691691692,100000)
(69.2692692692693,100000)
(69.3693693693694,100000)
(69.4694694694695,100000)
(69.5695695695696,100000)
(69.6696696696697,100000)
(69.7697697697698,100000)
(69.8698698698699,100000)
(69.96996996997,100000)
(70.0700700700701,100000)
(70.1701701701702,99999.9999999999)
(70.2702702702703,99999.9999999999)
(70.3703703703704,99999.9999999999)
(70.4704704704705,100000)
(70.5705705705706,100000)
(70.6706706706707,100000)
(70.7707707707708,100000)
(70.8708708708709,100000)
(70.970970970971,99999.9999999999)
(71.0710710710711,100000)
(71.1711711711712,100000)
(71.2712712712713,100000)
(71.3713713713714,100000)
(71.4714714714715,100000)
(71.5715715715716,100000)
(71.6716716716717,100000)
(71.7717717717718,100000)
(71.8718718718719,100000)
(71.971971971972,99999.9999999999)
(72.0720720720721,100000)
(72.1721721721722,100000)
(72.2722722722723,100000)
(72.3723723723724,100000)
(72.4724724724725,100000)
(72.5725725725726,100000)
(72.6726726726727,100000)
(72.7727727727728,100000)
(72.8728728728729,100000)
(72.972972972973,100000)
(73.0730730730731,100000)
(73.1731731731732,100000)
(73.2732732732733,100000)
(73.3733733733734,100000)
(73.4734734734735,100000)
(73.5735735735736,100000)
(73.6736736736737,100000)
(73.7737737737738,100000)
(73.8738738738739,100000)
(73.973973973974,100000)
(74.0740740740741,100000)
(74.1741741741742,100000)
(74.2742742742743,100000)
(74.3743743743744,99999.9999999999)
(74.4744744744745,100000)
(74.5745745745746,100000)
(74.6746746746747,100000)
(74.7747747747748,100000)
(74.8748748748749,100000)
(74.974974974975,100000)
(75.0750750750751,100000)
(75.1751751751752,99999.9999999999)
(75.2752752752753,99999.9999999999)
(75.3753753753754,99999.9999999999)
(75.4754754754755,100000)
(75.5755755755756,100000)
(75.6756756756757,100000)
(75.7757757757758,100000)
(75.8758758758759,100000)
(75.975975975976,99999.9999999999)
(76.0760760760761,100000)
(76.1761761761762,100000)
(76.2762762762763,99999.9999999999)
(76.3763763763764,100000)
(76.4764764764765,100000)
(76.5765765765766,100000)
(76.6766766766767,100000)
(76.7767767767768,100000)
(76.8768768768769,99999.9999999999)
(76.976976976977,100000)
(77.0770770770771,100000)
(77.1771771771772,100000)
(77.2772772772773,99999.9999999999)
(77.3773773773774,99999.9999999999)
(77.4774774774775,100000)
(77.5775775775776,100000)
(77.6776776776777,99999.9999999999)
(77.7777777777778,100000)
(77.8778778778779,100000)
(77.977977977978,100000)
(78.0780780780781,100000)
(78.1781781781782,100000)
(78.2782782782783,100000)
(78.3783783783784,100000)
(78.4784784784785,100000)
(78.5785785785786,100000)
(78.6786786786787,100000)
(78.7787787787788,100000)
(78.8788788788789,100000)
(78.978978978979,100000)
(79.0790790790791,100000)
(79.1791791791792,100000)
(79.2792792792793,100000)
(79.3793793793794,100000)
(79.4794794794795,99999.9999999999)
(79.5795795795796,100000)
(79.6796796796797,100000)
(79.7797797797798,100000)
(79.8798798798799,100000)
(79.97997997998,99999.9999999999)
(80.0800800800801,100000)
(80.1801801801802,100000)
(80.2802802802803,100000)
(80.3803803803804,100000)
(80.4804804804805,100000)
(80.5805805805806,100000)
(80.6806806806807,100000)
(80.7807807807808,100000)
(80.8808808808809,100000)
(80.980980980981,100000)
(81.0810810810811,100000)
(81.1811811811812,99999.9999999999)
(81.2812812812813,100000)
(81.3813813813814,100000)
(81.4814814814815,99999.9999999999)
(81.5815815815816,100000)
(81.6816816816817,100000)
(81.7817817817818,100000)
(81.8818818818819,100000)
(81.981981981982,100000)
(82.0820820820821,100000)
(82.1821821821822,100000)
(82.2822822822823,100000)
(82.3823823823824,100000)
(82.4824824824825,99999.9999999999)
(82.5825825825826,100000)
(82.6826826826827,100000)
(82.7827827827828,100000)
(82.8828828828829,100000)
(82.982982982983,99999.9999999999)
(83.0830830830831,100000)
(83.1831831831832,100000)
(83.2832832832833,99999.9999999999)
(83.3833833833834,100000)
(83.4834834834835,100000)
(83.5835835835836,100000)
(83.6836836836837,99999.9999999999)
(83.7837837837838,100000)
(83.8838838838839,100000)
(83.983983983984,100000)
(84.0840840840841,99999.9999999999)
(84.1841841841842,100000)
(84.2842842842843,100000)
(84.3843843843844,99999.9999999999)
(84.4844844844845,100000)
(84.5845845845846,100000)
(84.6846846846847,99999.9999999999)
(84.7847847847848,100000)
(84.8848848848849,100000)
(84.984984984985,99999.9999999999)
(85.0850850850851,100000)
(85.1851851851852,100000)
(85.2852852852853,100000)
(85.3853853853854,100000)
(85.4854854854855,100000)
(85.5855855855856,100000)
(85.6856856856857,100000)
(85.7857857857858,100000)
(85.8858858858859,99999.9999999999)
(85.985985985986,100000)
(86.0860860860861,100000)
(86.1861861861862,100000)
(86.2862862862863,99999.9999999999)
(86.3863863863864,100000)
(86.4864864864865,100000)
(86.5865865865866,100000)
(86.6866866866867,100000)
(86.7867867867868,100000)
(86.8868868868869,100000)
(86.986986986987,100000)
(87.0870870870871,100000)
(87.1871871871872,100000)
(87.2872872872873,100000)
(87.3873873873874,100000)
(87.4874874874875,100000)
(87.5875875875876,100000)
(87.6876876876877,100000)
(87.7877877877878,100000)
(87.8878878878879,100000)
(87.987987987988,100000)
(88.0880880880881,100000)
(88.1881881881882,99999.9999999999)
(88.2882882882883,100000)
(88.3883883883884,100000)
(88.4884884884885,100000)
(88.5885885885886,99999.9999999999)
(88.6886886886887,100000)
(88.7887887887888,100000)
(88.8888888888889,100000)
(88.988988988989,100000)
(89.0890890890891,100000)
(89.1891891891892,100000)
(89.2892892892893,100000)
(89.3893893893894,100000)
(89.4894894894895,100000)
(89.5895895895896,100000)
(89.6896896896897,99999.9999999999)
(89.7897897897898,100000)
(89.8898898898899,100000)
(89.98998998999,100000)
(90.0900900900901,100000)
(90.1901901901902,100000)
(90.2902902902903,100000)
(90.3903903903904,100000)
(90.4904904904905,100000)
(90.5905905905906,100000)
(90.6906906906907,100000)
(90.7907907907908,100000)
(90.8908908908909,100000)
(90.990990990991,100000)
(91.0910910910911,100000)
(91.1911911911912,100000)
(91.2912912912913,100000)
(91.3913913913914,99999.9999999999)
(91.4914914914915,100000)
(91.5915915915916,100000)
(91.6916916916917,100000)
(91.7917917917918,100000)
(91.8918918918919,100000)
(91.991991991992,100000)
(92.0920920920921,100000)
(92.1921921921922,100000)
(92.2922922922923,100000)
(92.3923923923924,100000)
(92.4924924924925,100000)
(92.5925925925926,100000)
(92.6926926926927,100000)
(92.7927927927928,100000)
(92.8928928928929,100000)
(92.992992992993,100000)
(93.0930930930931,100000)
(93.1931931931932,100000)
(93.2932932932933,100000)
(93.3933933933934,100000)
(93.4934934934935,100000)
(93.5935935935936,100000)
(93.6936936936937,99999.9999999999)
(93.7937937937938,100000)
(93.8938938938939,100000)
(93.993993993994,100000)
(94.0940940940941,100000)
(94.1941941941942,100000)
(94.2942942942943,100000)
(94.3943943943944,100000)
(94.4944944944945,100000)
(94.5945945945946,100000)
(94.6946946946947,99999.9999999999)
(94.7947947947948,100000)
(94.8948948948949,100000)
(94.994994994995,100000)
(95.0950950950951,100000)
(95.1951951951952,100000)
(95.2952952952953,100000)
(95.3953953953954,100000)
(95.4954954954955,100000)
(95.5955955955956,100000)
(95.6956956956957,100000)
(95.7957957957958,99999.9999999999)
(95.8958958958959,100000)
(95.995995995996,100000)
(96.0960960960961,100000)
(96.1961961961962,100000)
(96.2962962962963,100000)
(96.3963963963964,100000)
(96.4964964964965,100000)
(96.5965965965966,100000)
(96.6966966966967,100000)
(96.7967967967968,100000)
(96.8968968968969,100000)
(96.996996996997,100000)
(97.0970970970971,100000)
(97.1971971971972,100000)
(97.2972972972973,100000)
(97.3973973973974,100000)
(97.4974974974975,100000)
(97.5975975975976,100000)
(97.6976976976977,100000)
(97.7977977977978,100000)
(97.8978978978979,100000)
(97.997997997998,100000)
(98.0980980980981,100000)
(98.1981981981982,100000)
(98.2982982982983,100000)
(98.3983983983984,100000)
(98.4984984984985,100000)
(98.5985985985986,100000)
(98.6986986986987,100000)
(98.7987987987988,100000)
(98.8988988988989,100000)
(98.998998998999,100000)
(99.0990990990991,100000)
(99.1991991991992,100000)
(99.2992992992993,100000)
(99.3993993993994,100000)
(99.4994994994995,100000)
(99.5995995995996,100000)
(99.6996996996997,100000)
(99.7997997997998,100000)
(99.8998998998999,100000)
(100,100000)

};
\addplot [green!50.0!black]
coordinates {
(0,110000)
(0.1001001001001,109956.125960942)
(0.2002002002002,109912.253248482)
(0.3003003003003,109868.383189098)
(0.4004004004004,109824.517109028)
(0.5005005005005,109780.656334148)
(0.6006006006006,109736.802189853)
(0.700700700700701,109692.956000938)
(0.800800800800801,109649.119091475)
(0.900900900900901,109605.292784695)
(1.001001001001,109561.478402868)
(1.1011011011011,109517.677267184)
(1.2012012012012,109473.890697631)
(1.3013013013013,109430.120012879)
(1.4014014014014,109386.366530159)
(1.5015015015015,109342.631565143)
(1.6016016016016,109298.916431827)
(1.7017017017017,109255.222442413)
(1.8018018018018,109211.550907189)
(1.9019019019019,109167.903134413)
(2.002002002002,109124.280430195)
(2.1021021021021,109080.684098377)
(2.2022022022022,109037.11544042)
(2.3023023023023,108993.575755287)
(2.4024024024024,108950.066339324)
(2.5025025025025,108906.588486145)
(2.6026026026026,108863.143486521)
(2.7027027027027,108819.732628259)
(2.8028028028028,108776.35719609)
(2.9029029029029,108733.018471556)
(3.003003003003,108689.717732897)
(3.1031031031031,108646.456254933)
(3.2032032032032,108603.235308958)
(3.3033033033033,108560.056162624)
(3.4034034034034,108516.920079831)
(3.5035035035035,108473.828320613)
(3.6036036036036,108430.782141035)
(3.7037037037037,108387.782793074)
(3.8038038038038,108344.831524517)
(3.9039039039039,108301.929578849)
(4.004004004004,108259.078195146)
(4.1041041041041,108216.278607969)
(4.2042042042042,108173.532047255)
(4.3043043043043,108130.839738212)
(4.4044044044044,108088.202901217)
(4.5045045045045,108045.622751706)
(4.6046046046046,108003.100500075)
(4.7047047047047,107960.637351574)
(4.8048048048048,107918.234506207)
(4.9049049049049,107875.893158631)
(5.005005005005,107833.614498051)
(5.1051051051051,107791.399708126)
(5.2052052052052,107749.249966865)
(5.3053053053053,107707.166446534)
(5.40540540540541,107665.150313553)
(5.50550550550551,107623.202728404)
(5.60560560560561,107581.324845534)
(5.70570570570571,107539.51781326)
(5.80580580580581,107497.782773673)
(5.90590590590591,107456.12086255)
(6.00600600600601,107414.53320926)
(6.10610610610611,107373.020936669)
(6.20620620620621,107331.585161058)
(6.30630630630631,107290.226992025)
(6.40640640640641,107248.947532403)
(6.50650650650651,107207.747878171)
(6.60660660660661,107166.629118367)
(6.70670670670671,107125.592335005)
(6.80680680680681,107084.638602988)
(6.90690690690691,107043.768990026)
(7.00700700700701,107002.984556556)
(7.10710710710711,106962.286355658)
(7.20720720720721,106921.675432977)
(7.30730730730731,106881.15282664)
(7.40740740740741,106840.719567187)
(7.50750750750751,106800.376677484)
(7.60760760760761,106760.125172655)
(7.70770770770771,106719.966060001)
(7.80780780780781,106679.900338935)
(7.90790790790791,106639.9290009)
(8.00800800800801,106600.053029305)
(8.10810810810811,106560.273399453)
(8.20820820820821,106520.591078469)
(8.30830830830831,106481.007025238)
(8.40840840840841,106441.522190333)
(8.50850850850851,106402.137515952)
(8.60860860860861,106362.853935855)
(8.70870870870871,106323.672375299)
(8.80880880880881,106284.593750976)
(8.90890890890891,106245.618970953)
(9.00900900900901,106206.748934614)
(9.10910910910911,106167.984532599)
(9.20920920920921,106129.326646747)
(9.30930930930931,106090.776150045)
(9.40940940940941,106052.333906566)
(9.50950950950951,106014.000771423)
(9.60960960960961,105975.777590709)
(9.70970970970971,105937.665201454)
(9.80980980980981,105899.66443157)
(9.90990990990991,105861.776099804)
(10.01001001001,105824.001015692)
(10.1101101101101,105786.339979511)
(10.2102102102102,105748.793782237)
(10.3103103103103,105711.363205499)
(10.4104104104104,105674.049021537)
(10.5105105105105,105636.851993164)
(10.6106106106106,105599.772873724)
(10.7107107107107,105562.812407051)
(10.8108108108108,105525.971327439)
(10.9109109109109,105489.250359599)
(11.011011011011,105452.650218631)
(11.1111111111111,105416.171609984)
(11.2112112112112,105379.815229428)
(11.3113113113113,105343.581763023)
(11.4114114114114,105307.471887091)
(11.5115115115115,105271.486268182)
(11.6116116116116,105235.625563055)
(11.7117117117117,105199.890418645)
(11.8118118118118,105164.281472045)
(11.9119119119119,105128.799350479)
(12.012012012012,105093.44467128)
(12.1121121121121,105058.218041873)
(12.2122122122122,105023.120059751)
(12.3123123123123,104988.151312461)
(12.4124124124124,104953.312377586)
(12.5125125125125,104918.603822728)
(12.6126126126126,104884.026205497)
(12.7127127127127,104849.580073494)
(12.8128128128128,104815.265964303)
(12.9129129129129,104781.084405478)
(13.013013013013,104747.035914536)
(13.1131131131131,104713.120998948)
(13.2132132132132,104679.34015613)
(13.3133133133133,104645.693873443)
(13.4134134134134,104612.182628183)
(13.5135135135135,104578.806887583)
(13.6136136136136,104545.567108807)
(13.7137137137137,104512.463738953)
(13.8138138138138,104479.49721505)
(13.9139139139139,104446.667964064)
(14.014014014014,104413.976402898)
(14.1141141141141,104381.422938398)
(14.2142142142142,104349.007967357)
(14.3143143143143,104316.731876523)
(14.4144144144144,104284.595042608)
(14.5145145145145,104252.597832293)
(14.6146146146146,104220.740602242)
(14.7147147147147,104189.023699112)
(14.8148148148148,104157.447459567)
(14.9149149149149,104126.012210287)
(15.015015015015,104094.718267991)
(15.1151151151151,104063.565939445)
(15.2152152152152,104032.555521482)
(15.3153153153153,104001.687301023)
(15.4154154154154,103970.961555092)
(15.5155155155155,103940.378550839)
(15.6156156156156,103909.938545562)
(15.7157157157157,103879.641786726)
(15.8158158158158,103849.488511991)
(15.9159159159159,103819.478949236)
(16.016016016016,103789.613316582)
(16.1161161161161,103759.891822422)
(16.2162162162162,103730.314665448)
(16.3163163163163,103700.882034678)
(16.4164164164164,103671.59410949)
(16.5165165165165,103642.45105965)
(16.6166166166166,103613.453045344)
(16.7167167167167,103584.600217212)
(16.8168168168168,103555.892716384)
(16.9169169169169,103527.330674511)
(17.017017017017,103498.914213803)
(17.1171171171171,103470.643447066)
(17.2172172172172,103442.51847774)
(17.3173173173173,103414.539399936)
(17.4174174174174,103386.70629848)
(17.5175175175175,103359.019248949)
(17.6176176176176,103331.478317714)
(17.7177177177177,103304.083561984)
(17.8178178178178,103276.835029847)
(17.9179179179179,103249.732760316)
(18.018018018018,103222.776783374)
(18.1181181181181,103195.967120017)
(18.2182182182182,103169.303782307)
(18.3183183183183,103142.786773411)
(18.4184184184184,103116.416087656)
(18.5185185185185,103090.191710578)
(18.6186186186186,103064.113618968)
(18.7187187187187,103038.181780923)
(18.8188188188188,103012.396155904)
(18.9189189189189,102986.756694779)
(19.019019019019,102961.263339883)
(19.1191191191191,102935.91602507)
(19.2192192192192,102910.714675765)
(19.3193193193193,102885.65920902)
(19.4194194194194,102860.749533574)
(19.5195195195195,102835.985549904)
(19.6196196196196,102811.367150284)
(19.7197197197197,102786.894218845)
(19.8198198198198,102762.56663163)
(19.9199199199199,102738.384256657)
(20.02002002002,102714.346953974)
(20.1201201201201,102690.454575725)
(20.2202202202202,102666.706966204)
(20.3203203203203,102643.103961923)
(20.4204204204204,102619.645391672)
(20.5205205205205,102596.331076579)
(20.6206206206206,102573.160830176)
(20.7207207207207,102550.134458463)
(20.8208208208208,102527.251759969)
(20.9209209209209,102504.512525821)
(21.021021021021,102481.916539806)
(21.1211211211211,102459.463578437)
(21.2212212212212,102437.153411021)
(21.3213213213213,102414.985799724)
(21.4214214214214,102392.960499638)
(21.5215215215215,102371.077258849)
(21.6216216216216,102349.335818506)
(21.7217217217217,102327.735912887)
(21.8218218218218,102306.277269467)
(21.9219219219219,102284.959608993)
(22.022022022022,102263.782645544)
(22.1221221221221,102242.746086609)
(22.2222222222222,102221.849633153)
(22.3223223223223,102201.092979689)
(22.4224224224224,102180.475814346)
(22.5225225225225,102159.997818943)
(22.6226226226226,102139.65866906)
(22.7227227227227,102119.458034107)
(22.8228228228228,102099.395577399)
(22.9229229229229,102079.470956225)
(23.023023023023,102059.683821923)
(23.1231231231231,102040.033819952)
(23.2232232232232,102020.520589962)
(23.3233233233233,102001.143765871)
(23.4234234234234,101981.902975934)
(23.5235235235235,101962.79784282)
(23.6236236236236,101943.827983683)
(23.7237237237237,101924.993010236)
(23.8238238238238,101906.292528826)
(23.9239239239239,101887.726140505)
(24.024024024024,101869.293441108)
(24.1241241241241,101850.994021323)
(24.2242242242242,101832.827466767)
(24.3243243243243,101814.793358061)
(24.4244244244244,101796.891270904)
(24.5245245245245,101779.120776144)
(24.6246246246246,101761.481439857)
(24.7247247247247,101743.972823419)
(24.8248248248248,101726.594483581)
(24.9249249249249,101709.345972543)
(25.025025025025,101692.226838027)
(25.1251251251251,101675.236623357)
(25.2252252252252,101658.374867524)
(25.3253253253253,101641.64110527)
(25.4254254254254,101625.034867156)
(25.5255255255255,101608.555679638)
(25.6256256256256,101592.203065142)
(25.7257257257257,101575.976542136)
(25.8258258258258,101559.875625208)
(25.9259259259259,101543.899825133)
(26.026026026026,101528.048648954)
(26.1261261261261,101512.321600051)
(26.2262262262262,101496.718178218)
(26.3263263263263,101481.237879731)
(26.4264264264264,101465.880197427)
(26.5265265265265,101450.644620774)
(26.6266266266266,101435.530635945)
(26.7267267267267,101420.537725887)
(26.8268268268268,101405.665370401)
(26.9269269269269,101390.913046206)
(27.027027027027,101376.280227018)
(27.1271271271271,101361.766383617)
(27.2272272272272,101347.37098392)
(27.3273273273273,101333.093493053)
(27.4274274274274,101318.933373423)
(27.5275275275275,101304.890084788)
(27.6276276276276,101290.963084325)
(27.7277277277277,101277.151826706)
(27.8278278278278,101263.455764161)
(27.9279279279279,101249.874346555)
(28.028028028028,101236.407021452)
(28.1281281281281,101223.053234188)
(28.2282282282282,101209.812427937)
(28.3283283283283,101196.68404378)
(28.4284284284284,101183.667520776)
(28.5285285285285,101170.762296026)
(28.6286286286286,101157.967804742)
(28.7287287287287,101145.283480316)
(28.8288288288288,101132.708754385)
(28.9289289289289,101120.243056897)
(29.029029029029,101107.885816179)
(29.1291291291291,101095.636459)
(29.2292292292292,101083.49441064)
(29.3293293293293,101071.459094951)
(29.4294294294294,101059.529934425)
(29.5295295295295,101047.706350256)
(29.6296296296296,101035.987762403)
(29.7297297297297,101024.373589657)
(29.8298298298298,101012.8632497)
(29.9299299299299,101001.45615917)
(30.03003003003,100990.15173372)
(30.1301301301301,100978.949388085)
(30.2302302302302,100967.848536136)
(30.3303303303303,100956.848590947)
(30.4304304304304,100945.948964853)
(30.5305305305305,100935.149069508)
(30.6306306306306,100924.448315947)
(30.7307307307307,100913.846114642)
(30.8308308308308,100903.341875565)
(30.9309309309309,100892.935008242)
(31.031031031031,100882.624921809)
(31.1311311311311,100872.411025074)
(31.2312312312312,100862.292726572)
(31.3313313313313,100852.269434616)
(31.4314314314314,100842.34055736)
(31.5315315315315,100832.505502847)
(31.6316316316316,100822.763679071)
(31.7317317317317,100813.114494022)
(31.8318318318318,100803.557355747)
(31.9319319319319,100794.091672399)
(32.032032032032,100784.716852291)
(32.1321321321321,100775.432303948)
(32.2322322322322,100766.237436155)
(32.3323323323323,100757.131658014)
(32.4324324324324,100748.114378988)
(32.5325325325325,100739.185008956)
(32.6326326326326,100730.342958257)
(32.7327327327327,100721.587637745)
(32.8328328328328,100712.91845883)
(32.9329329329329,100704.334833533)
(33.033033033033,100695.836174528)
(33.1331331331331,100687.421895188)
(33.2332332332332,100679.091409635)
(33.3333333333333,100670.844132785)
(33.4334334334334,100662.679480388)
(33.5335335335335,100654.596869078)
(33.6336336336336,100646.595716413)
(33.7337337337337,100638.675440922)
(33.8338338338338,100630.835462142)
(33.9339339339339,100623.075200667)
(34.034034034034,100615.394078185)
(34.1341341341341,100607.791517519)
(34.2342342342342,100600.266942671)
(34.3343343343343,100592.819778857)
(34.4344344344344,100585.449452552)
(34.5345345345345,100578.155391523)
(34.6346346346346,100570.937024872)
(34.7347347347347,100563.793783072)
(34.8348348348348,100556.725098002)
(34.9349349349349,100549.730402987)
(35.035035035035,100542.809132832)
(35.1351351351351,100535.960723859)
(35.2352352352352,100529.18461394)
(35.3353353353353,100522.480242532)
(35.4354354354354,100515.847050713)
(35.5355355355355,100509.284481212)
(35.6356356356356,100502.791978443)
(35.7357357357357,100496.368988538)
(35.8358358358358,100490.014959377)
(35.9359359359359,100483.729340621)
(36.036036036036,100477.511583742)
(36.1361361361361,100471.36114205)
(36.2362362362362,100465.277470727)
(36.3363363363363,100459.260026854)
(36.4364364364364,100453.308269437)
(36.5365365365365,100447.421659439)
(36.6366366366366,100441.599659805)
(36.7367367367367,100435.841735486)
(36.8368368368368,100430.147353473)
(36.9369369369369,100424.515982813)
(37.037037037037,100418.947094642)
(37.1371371371371,100413.440162205)
(37.2372372372372,100407.994660882)
(37.3373373373373,100402.610068211)
(37.4374374374374,100397.28586391)
(37.5375375375375,100392.021529903)
(37.6376376376376,100386.816550337)
(37.7377377377377,100381.670411607)
(37.8378378378378,100376.582602377)
(37.9379379379379,100371.552613599)
(38.038038038038,100366.579938531)
(38.1381381381381,100361.664072762)
(38.2382382382382,100356.804514227)
(38.3383383383383,100352.000763225)
(38.4384384384384,100347.252322439)
(38.5385385385385,100342.558696951)
(38.6386386386386,100337.919394264)
(38.7387387387387,100333.333924312)
(38.8388388388388,100328.80179948)
(38.9389389389389,100324.322534618)
(39.039039039039,100319.895647057)
(39.1391391391391,100315.520656624)
(39.2392392392392,100311.197085653)
(39.3393393393393,100306.924459002)
(39.4394394394394,100302.702304066)
(39.5395395395395,100298.530150787)
(39.6396396396396,100294.407531668)
(39.7397397397397,100290.333981785)
(39.8398398398398,100286.3090388)
(39.9399399399399,100282.332242966)
(40.04004004004,100278.403137144)
(40.1401401401401,100274.521266811)
(40.2402402402402,100270.686180067)
(40.3403403403403,100266.897427647)
(40.4404404404404,100263.154562928)
(40.5405405405405,100259.457141941)
(40.6406406406406,100255.804723371)
(40.7407407407407,100252.196868575)
(40.8408408408408,100248.633141579)
(40.9409409409409,100245.113109092)
(41.041041041041,100241.636340509)
(41.1411411411411,100238.202407915)
(41.2412412412412,100234.810886096)
(41.3413413413413,100231.461352538)
(41.4414414414414,100228.153387436)
(41.5415415415415,100224.886573697)
(41.6416416416416,100221.660496942)
(41.7417417417417,100218.474745512)
(41.8418418418418,100215.328910471)
(41.9419419419419,100212.222585608)
(42.042042042042,100209.155367437)
(42.1421421421421,100206.126855205)
(42.2422422422422,100203.136650889)
(42.3423423423423,100200.184359197)
(42.4424424424424,100197.269587574)
(42.5425425425425,100194.391946197)
(42.6426426426426,100191.551047979)
(42.7427427427427,100188.746508566)
(42.8428428428428,100185.97794634)
(42.9429429429429,100183.244982417)
(43.043043043043,100180.547240644)
(43.1431431431431,100177.884347602)
(43.2432432432432,100175.255932599)
(43.3433433433433,100172.661627674)
(43.4434434434434,100170.101067591)
(43.5435435435435,100167.573889835)
(43.6436436436436,100165.079734615)
(43.7437437437437,100162.618244856)
(43.8438438438438,100160.189066196)
(43.9439439439439,100157.791846985)
(44.044044044044,100155.426238279)
(44.1441441441441,100153.091893836)
(44.2442442442442,100150.788470113)
(44.3443443443443,100148.515626258)
(44.4444444444444,100146.273024109)
(44.5445445445445,100144.060328186)
(44.6446446446446,100141.877205687)
(44.7447447447447,100139.72332648)
(44.8448448448448,100137.5983631)
(44.9449449449449,100135.501990742)
(45.045045045045,100133.433887251)
(45.1451451451451,100131.393733121)
(45.2452452452452,100129.381211485)
(45.3453453453453,100127.396008106)
(45.4454454454454,100125.437811373)
(45.5455455455455,100123.506312292)
(45.6456456456456,100121.601204479)
(45.7457457457457,100119.72218415)
(45.8458458458458,100117.868950115)
(45.9459459459459,100116.041203767)
(46.046046046046,100114.238649078)
(46.1461461461461,100112.460992584)
(46.2462462462462,100110.707943381)
(46.3463463463463,100108.979213115)
(46.4464464464464,100107.274515969)
(46.5465465465465,100105.593568659)
(46.6466466466466,100103.936090419)
(46.7467467467467,100102.301802996)
(46.8468468468468,100100.690430636)
(46.9469469469469,100099.101700076)
(47.047047047047,100097.535340534)
(47.1471471471471,100095.991083695)
(47.2472472472472,100094.468663705)
(47.3473473473473,100092.967817159)
(47.4474474474474,100091.488283086)
(47.5475475475475,100090.029802945)
(47.6476476476476,100088.592120607)
(47.7477477477477,100087.174982348)
(47.8478478478478,100085.778136836)
(47.9479479479479,100084.40133512)
(48.048048048048,100083.044330616)
(48.1481481481481,100081.706879099)
(48.2482482482482,100080.388738688)
(48.3483483483483,100079.089669837)
(48.4484484484484,100077.809435318)
(48.5485485485485,100076.547800215)
(48.6486486486486,100075.304531904)
(48.7487487487487,100074.079400049)
(48.8488488488488,100072.872176584)
(48.9489489489489,100071.682635701)
(49.049049049049,100070.510553838)
(49.1491491491491,100069.355709666)
(49.2492492492492,100068.217884079)
(49.3493493493493,100067.096860174)
(49.4494494494494,100065.992423247)
(49.5495495495495,100064.90436077)
(49.6496496496496,100063.832462388)
(49.7497497497497,100062.776519897)
(49.8498498498498,100061.736327238)
(49.9499499499499,100060.711680476)
(50.05005005005,100059.702377795)
(50.1501501501501,100058.708219477)
(50.2502502502502,100057.729007894)
(50.3503503503503,100056.764547491)
(50.4504504504504,100055.814644773)
(50.5505505505505,100054.879108293)
(50.6506506506506,100053.957748637)
(50.7507507507507,100053.050378411)
(50.8508508508508,100052.156812225)
(50.9509509509509,100051.276866684)
(51.051051051051,100050.410360367)
(51.1511511511511,100049.557113821)
(51.2512512512512,100048.716949542)
(51.3513513513513,100047.889691961)
(51.4514514514514,100047.075167434)
(51.5515515515515,100046.273204225)
(51.6516516516516,100045.483632491)
(51.7517517517517,100044.706284273)
(51.8518518518518,100043.940993474)
(51.9519519519519,100043.187595855)
(52.052052052052,100042.445929013)
(52.1521521521521,100041.715832369)
(52.2522522522522,100040.997147157)
(52.3523523523523,100040.289716407)
(52.4524524524524,100039.593384931)
(52.5525525525525,100038.907999311)
(52.6526526526526,100038.233407884)
(52.7527527527527,100037.569460727)
(52.8528528528528,100036.916009644)
(52.9529529529529,100036.272908153)
(53.053053053053,100035.640011471)
(53.1531531531531,100035.0171765)
(53.2532532532532,100034.404261812)
(53.3533533533533,100033.801127638)
(53.4534534534534,100033.207635853)
(53.5535535535535,100032.62364996)
(53.6536536536536,100032.049035081)
(53.7537537537537,100031.483657936)
(53.8538538538538,100030.927386836)
(53.9539539539539,100030.380091668)
(54.054054054054,100029.841643878)
(54.1541541541541,100029.31191646)
(54.2542542542542,100028.790783941)
(54.3543543543543,100028.278122371)
(54.4544544544544,100027.773809303)
(54.5545545545545,100027.277723785)
(54.6546546546546,100026.789746347)
(54.7547547547547,100026.309758982)
(54.8548548548548,100025.837645136)
(54.9549549549549,100025.373289698)
(55.055055055055,100024.91657898)
(55.1551551551551,100024.467400709)
(55.2552552552552,100024.02564401)
(55.3553553553553,100023.591199398)
(55.4554554554554,100023.163958758)
(55.5555555555556,100022.743815339)
(55.6556556556557,100022.330663736)
(55.7557557557558,100021.924399878)
(55.8558558558559,100021.524921019)
(55.955955955956,100021.132125718)
(56.0560560560561,100020.745913834)
(56.1561561561562,100020.366186508)
(56.2562562562563,100019.992846151)
(56.3563563563564,100019.625796435)
(56.4564564564565,100019.264942275)
(56.5565565565566,100018.910189822)
(56.6566566566567,100018.561446447)
(56.7567567567568,100018.218620729)
(56.8568568568569,100017.881622444)
(56.956956956957,100017.550362553)
(57.0570570570571,100017.224753188)
(57.1571571571572,100016.904707641)
(57.2572572572573,100016.590140354)
(57.3573573573574,100016.280966903)
(57.4574574574575,100015.977103987)
(57.5575575575576,100015.678469422)
(57.6576576576577,100015.38498212)
(57.7577577577578,100015.096562085)
(57.8578578578579,100014.813130396)
(57.957957957958,100014.5346092)
(58.0580580580581,100014.260921698)
(58.1581581581582,100013.991992133)
(58.2582582582583,100013.727745781)
(58.3583583583584,100013.468108937)
(58.4584584584585,100013.213008908)
(58.5585585585586,100012.962373996)
(58.6586586586587,100012.716133493)
(58.7587587587588,100012.474217667)
(58.8588588588589,100012.236557748)
(58.958958958959,100012.003085927)
(59.0590590590591,100011.773735332)
(59.1591591591592,100011.548440031)
(59.2592592592593,100011.327135009)
(59.3593593593594,100011.109756166)
(59.4594594594595,100010.896240305)
(59.5595595595596,100010.686525119)
(59.6596596596597,100010.480549182)
(59.7597597597598,100010.278251939)
(59.8598598598599,100010.079573698)
(59.95995995996,100009.884455616)
(60.0600600600601,100009.692839692)
(60.1601601601602,100009.504668756)
(60.2602602602603,100009.319886458)
(60.3603603603604,100009.138437263)
(60.4604604604605,100008.960266434)
(60.5605605605606,100008.785320029)
(60.6606606606607,100008.61354489)
(60.7607607607608,100008.444888628)
(60.8608608608609,100008.279299624)
(60.960960960961,100008.11672701)
(61.0610610610611,100007.957120666)
(61.1611611611612,100007.800431207)
(61.2612612612613,100007.646609977)
(61.3613613613614,100007.495609038)
(61.4614614614615,100007.347381163)
(61.5615615615616,100007.201879825)
(61.6616616616617,100007.05905919)
(61.7617617617618,100006.918874109)
(61.8618618618619,100006.781280105)
(61.961961961962,100006.646233372)
(62.0620620620621,100006.513690759)
(62.1621621621622,100006.383609766)
(62.2622622622623,100006.255948537)
(62.3623623623624,100006.130665846)
(62.4624624624625,100006.007721095)
(62.5625625625626,100005.887074304)
(62.6626626626627,100005.7686861)
(62.7627627627628,100005.652517715)
(62.8628628628629,100005.538530972)
(62.962962962963,100005.426688282)
(63.0630630630631,100005.316952634)
(63.1631631631632,100005.209287589)
(63.2632632632633,100005.10365727)
(63.3633633633634,100005.000026358)
(63.4634634634635,100004.89836008)
(63.5635635635636,100004.798624208)
(63.6636636636637,100004.700785045)
(63.7637637637638,100004.604809423)
(63.8638638638639,100004.510664695)
(63.963963963964,100004.418318725)
(64.0640640640641,100004.327739884)
(64.1641641641641,100004.238897043)
(64.2642642642643,100004.151759566)
(64.3643643643643,100004.066297302)
(64.4644644644645,100003.982480578)
(64.5645645645645,100003.900280197)
(64.6646646646647,100003.819667426)
(64.7647647647647,100003.740613992)
(64.8648648648649,100003.663092077)
(64.9649649649649,100003.587074309)
(65.0650650650651,100003.512533756)
(65.1651651651651,100003.439443923)
(65.2652652652653,100003.367778742)
(65.3653653653653,100003.29751257)
(65.4654654654655,100003.228620178)
(65.5655655655655,100003.161076751)
(65.6656656656657,100003.094857876)
(65.7657657657657,100003.029939542)
(65.8658658658659,100002.96629813)
(65.9659659659659,100002.90391041)
(66.0660660660661,100002.842753535)
(66.1661661661661,100002.782805034)
(66.2662662662663,100002.724042808)
(66.3663663663663,100002.666445124)
(66.4664664664665,100002.609990611)
(66.5665665665666,100002.554658252)
(66.6666666666667,100002.500427382)
(66.7667667667668,100002.447277682)
(66.8668668668669,100002.395189171)
(66.966966966967,100002.344142205)
(67.0670670670671,100002.294117471)
(67.1671671671672,100002.245095981)
(67.2672672672673,100002.197059066)
(67.3673673673674,100002.149988375)
(67.4674674674675,100002.103865869)
(67.5675675675676,100002.058673813)
(67.6676676676677,100002.014394776)
(67.7677677677678,100001.971011625)
(67.8678678678679,100001.928507517)
(67.967967967968,100001.886865901)
(68.0680680680681,100001.846070508)
(68.1681681681682,100001.806105352)
(68.2682682682683,100001.766954718)
(68.3683683683684,100001.728603166)
(68.4684684684685,100001.691035522)
(68.5685685685686,100001.654236876)
(68.6686686686687,100001.618192577)
(68.7687687687688,100001.582888228)
(68.8688688688689,100001.548309685)
(68.968968968969,100001.51444305)
(69.0690690690691,100001.48127467)
(69.1691691691692,100001.448791129)
(69.2692692692693,100001.416979251)
(69.3693693693694,100001.385826088)
(69.4694694694695,100001.355318924)
(69.5695695695696,100001.325445266)
(69.6696696696697,100001.296192844)
(69.7697697697698,100001.267549604)
(69.8698698698699,100001.239503709)
(69.96996996997,100001.212043532)
(70.0700700700701,100001.185157653)
(70.1701701701702,100001.158834857)
(70.2702702702703,100001.13306413)
(70.3703703703704,100001.107834658)
(70.4704704704705,100001.083135818)
(70.5705705705706,100001.058957183)
(70.6706706706707,100001.03528851)
(70.7707707707708,100001.012119744)
(70.8708708708709,100000.989441014)
(70.970970970971,100000.967242624)
(71.0710710710711,100000.94551506)
(71.1711711711712,100000.924248977)
(71.2712712712713,100000.903435203)
(71.3713713713714,100000.883064734)
(71.4714714714715,100000.863128732)
(71.5715715715716,100000.843618519)
(71.6716716716717,100000.824525578)
(71.7717717717718,100000.805841552)
(71.8718718718719,100000.787558234)
(71.971971971972,100000.769667572)
(72.0720720720721,100000.752161662)
(72.1721721721722,100000.735032749)
(72.2722722722723,100000.718273219)
(72.3723723723724,100000.701875603)
(72.4724724724725,100000.685832571)
(72.5725725725726,100000.67013693)
(72.6726726726727,100000.654781621)
(72.7727727727728,100000.639759721)
(72.8728728728729,100000.625064432)
(72.972972972973,100000.61068909)
(73.0730730730731,100000.596627153)
(73.1731731731732,100000.582872204)
(73.2732732732733,100000.569417949)
(73.3733733733734,100000.556258212)
(73.4734734734735,100000.543386934)
(73.5735735735736,100000.530798175)
(73.6736736736737,100000.518486104)
(73.7737737737738,100000.506445004)
(73.8738738738739,100000.494669269)
(73.973973973974,100000.483153397)
(74.0740740740741,100000.471891995)
(74.1741741741742,100000.460879773)
(74.2742742742743,100000.450111543)
(74.3743743743744,100000.439582217)
(74.4744744744745,100000.429286806)
(74.5745745745746,100000.419220417)
(74.6746746746747,100000.409378255)
(74.7747747747748,100000.399755615)
(74.8748748748749,100000.390347885)
(74.974974974975,100000.381150544)
(75.0750750750751,100000.372159159)
(75.1751751751752,100000.363369383)
(75.2752752752753,100000.354776956)
(75.3753753753754,100000.3463777)
(75.4754754754755,100000.338167522)
(75.5755755755756,100000.330142407)
(75.6756756756757,100000.322298421)
(75.7757757757758,100000.314631708)
(75.8758758758759,100000.307138488)
(75.975975975976,100000.299815057)
(76.0760760760761,100000.292657783)
(76.1761761761762,100000.285663109)
(76.2762762762763,100000.278827546)
(76.3763763763764,100000.272147678)
(76.4764764764765,100000.265620155)
(76.5765765765766,100000.259241697)
(76.6766766766767,100000.253009087)
(76.7767767767768,100000.246919174)
(76.8768768768769,100000.240968872)
(76.976976976977,100000.235155155)
(77.0770770770771,100000.229475061)
(77.1771771771772,100000.223925685)
(77.2772772772773,100000.218504183)
(77.3773773773774,100000.213207769)
(77.4774774774775,100000.208033714)
(77.5775775775776,100000.202979342)
(77.6776776776777,100000.198042036)
(77.7777777777778,100000.193219229)
(77.8778778778779,100000.188508409)
(77.977977977978,100000.183907114)
(78.0780780780781,100000.179412933)
(78.1781781781782,100000.175023507)
(78.2782782782783,100000.170736523)
(78.3783783783784,100000.166549716)
(78.4784784784785,100000.162460869)
(78.5785785785786,100000.158467811)
(78.6786786786787,100000.154568416)
(78.7787787787788,100000.150760602)
(78.8788788788789,100000.14704233)
(78.978978978979,100000.143411605)
(79.0790790790791,100000.139866472)
(79.1791791791792,100000.136405018)
(79.2792792792793,100000.133025371)
(79.3793793793794,100000.129725696)
(79.4794794794795,100000.126504199)
(79.5795795795796,100000.123359122)
(79.6796796796797,100000.120288746)
(79.7797797797798,100000.117291386)
(79.8798798798799,100000.114365395)
(79.97997997998,100000.111509159)
(80.0800800800801,100000.108721101)
(80.1801801801802,100000.105999674)
(80.2802802802803,100000.103343367)
(80.3803803803804,100000.1007507)
(80.4804804804805,100000.098220224)
(80.5805805805806,100000.095750523)
(80.6806806806807,100000.093340209)
(80.7807807807808,100000.090987926)
(80.8808808808809,100000.088692345)
(80.980980980981,100000.086452168)
(81.0810810810811,100000.084266123)
(81.1811811811812,100000.082132968)
(81.2812812812813,100000.080051484)
(81.3813813813814,100000.078020482)
(81.4814814814815,100000.076038797)
(81.5815815815816,100000.07410529)
(81.6816816816817,100000.072218846)
(81.7817817817818,100000.070378377)
(81.8818818818819,100000.068582814)
(81.981981981982,100000.066831117)
(82.0820820820821,100000.065122264)
(82.1821821821822,100000.063455257)
(82.2822822822823,100000.061829122)
(82.3823823823824,100000.060242904)
(82.4824824824825,100000.05869567)
(82.5825825825826,100000.057186506)
(82.6826826826827,100000.055714522)
(82.7827827827828,100000.054278843)
(82.8828828828829,100000.052878618)
(82.982982982983,100000.051513011)
(83.0830830830831,100000.050181206)
(83.1831831831832,100000.048882407)
(83.2832832832833,100000.047615833)
(83.3833833833834,100000.046380721)
(83.4834834834835,100000.045176327)
(83.5835835835836,100000.044001923)
(83.6836836836837,100000.042856794)
(83.7837837837838,100000.041740246)
(83.8838838838839,100000.040651598)
(83.983983983984,100000.039590184)
(84.0840840840841,100000.038555355)
(84.1841841841842,100000.037546474)
(84.2842842842843,100000.03656292)
(84.3843843843844,100000.035604087)
(84.4844844844845,100000.034669381)
(84.5845845845846,100000.033758221)
(84.6846846846847,100000.032870043)
(84.7847847847848,100000.032004291)
(84.8848848848849,100000.031160426)
(84.984984984985,100000.030337917)
(85.0850850850851,100000.029536248)
(85.1851851851852,100000.028754916)
(85.2852852852853,100000.027993425)
(85.3853853853854,100000.027251296)
(85.4854854854855,100000.026528056)
(85.5855855855856,100000.025823247)
(85.6856856856857,100000.025136418)
(85.7857857857858,100000.024467131)
(85.8858858858859,100000.023814957)
(85.985985985986,100000.023179478)
(86.0860860860861,100000.022560283)
(86.1861861861862,100000.021956975)
(86.2862862862863,100000.021369161)
(86.3863863863864,100000.020796462)
(86.4864864864865,100000.020238505)
(86.5865865865866,100000.019694926)
(86.6866866866867,100000.01916537)
(86.7867867867868,100000.018649491)
(86.8868868868869,100000.01814695)
(86.986986986987,100000.017657415)
(87.0870870870871,100000.017180565)
(87.1871871871872,100000.016716084)
(87.2872872872873,100000.016263664)
(87.3873873873874,100000.015823005)
(87.4874874874875,100000.015393813)
(87.5875875875876,100000.014975801)
(87.6876876876877,100000.01456869)
(87.7877877877878,100000.014172206)
(87.8878878878879,100000.013786084)
(87.987987987988,100000.013410062)
(88.0880880880881,100000.013043887)
(88.1881881881882,100000.012687311)
(88.2882882882883,100000.012340091)
(88.3883883883884,100000.012001992)
(88.4884884884885,100000.011672782)
(88.5885885885886,100000.011352236)
(88.6886886886887,100000.011040135)
(88.7887887887888,100000.010736265)
(88.8888888888889,100000.010440415)
(88.988988988989,100000.010152382)
(89.0890890890891,100000.009871967)
(89.1891891891892,100000.009598974)
(89.2892892892893,100000.009333215)
(89.3893893893894,100000.009074503)
(89.4894894894895,100000.008822658)
(89.5895895895896,100000.008577504)
(89.6896896896897,100000.008338869)
(89.7897897897898,100000.008106584)
(89.8898898898899,100000.007880485)
(89.98998998999,100000.007660414)
(90.0900900900901,100000.007446214)
(90.1901901901902,100000.007237732)
(90.2902902902903,100000.007034821)
(90.3903903903904,100000.006837335)
(90.4904904904905,100000.006645133)
(90.5905905905906,100000.006458078)
(90.6906906906907,100000.006276034)
(90.7907907907908,100000.006098871)
(90.8908908908909,100000.005926461)
(90.990990990991,100000.005758678)
(91.0910910910911,100000.005595402)
(91.1911911911912,100000.005436512)
(91.2912912912913,100000.005281895)
(91.3913913913914,100000.005131435)
(91.4914914914915,100000.004985024)
(91.5915915915916,100000.004842553)
(91.6916916916917,100000.004703917)
(91.7917917917918,100000.004569015)
(91.8918918918919,100000.004437747)
(91.991991991992,100000.004310015)
(92.0920920920921,100000.004185724)
(92.1921921921922,100000.004064781)
(92.2922922922923,100000.003947097)
(92.3923923923924,100000.003832583)
(92.4924924924925,100000.003721154)
(92.5925925925926,100000.003612724)
(92.6926926926927,100000.003507213)
(92.7927927927928,100000.003404541)
(92.8928928928929,100000.00330463)
(92.992992992993,100000.003207404)
(93.0930930930931,100000.003112789)
(93.1931931931932,100000.003020713)
(93.2932932932933,100000.002931106)
(93.3933933933934,100000.002843897)
(93.4934934934935,100000.002759022)
(93.5935935935936,100000.002676413)
(93.6936936936937,100000.002596008)
(93.7937937937938,100000.002517743)
(93.8938938938939,100000.002441559)
(93.993993993994,100000.002367396)
(94.0940940940941,100000.002295195)
(94.1941941941942,100000.002224901)
(94.2942942942943,100000.002156458)
(94.3943943943944,100000.002089812)
(94.4944944944945,100000.002024911)
(94.5945945945946,100000.001961704)
(94.6946946946947,100000.001900141)
(94.7947947947948,100000.001840172)
(94.8948948948949,100000.001781749)
(94.994994994995,100000.001724827)
(95.0950950950951,100000.001669359)
(95.1951951951952,100000.001615302)
(95.2952952952953,100000.00156261)
(95.3953953953954,100000.001511243)
(95.4954954954955,100000.001461159)
(95.5955955955956,100000.001412316)
(95.6956956956957,100000.001364675)
(95.7957957957958,100000.001318197)
(95.8958958958959,100000.001272845)
(95.995995995996,100000.001228581)
(96.0960960960961,100000.001185368)
(96.1961961961962,100000.001143172)
(96.2962962962963,100000.001101957)
(96.3963963963964,100000.00106169)
(96.4964964964965,100000.001022336)
(96.5965965965966,100000.000983863)
(96.6966966966967,100000.000946239)
(96.7967967967968,100000.000909432)
(96.8968968968969,100000.000873412)
(96.996996996997,100000.000838148)
(97.0970970970971,100000.000803611)
(97.1971971971972,100000.000769772)
(97.2972972972973,100000.000736601)
(97.3973973973974,100000.00070407)
(97.4974974974975,100000.000672153)
(97.5975975975976,100000.000640821)
(97.6976976976977,100000.000610049)
(97.7977977977978,100000.000579809)
(97.8978978978979,100000.000550075)
(97.997997997998,100000.000520823)
(98.0980980980981,100000.000492026)
(98.1981981981982,100000.000463661)
(98.2982982982983,100000.000435702)
(98.3983983983984,100000.000408125)
(98.4984984984985,100000.000380906)
(98.5985985985986,100000.000354022)
(98.6986986986987,100000.000327448)
(98.7987987987988,100000.000301163)
(98.8988988988989,100000.000275143)
(98.998998998999,100000.000249365)
(99.0990990990991,100000.000223806)
(99.1991991991992,100000.000198445)
(99.2992992992993,100000.000173258)
(99.3993993993994,100000.000148224)
(99.4994994994995,100000.000123321)
(99.5995995995996,100000.000098527)
(99.6996996996997,100000.000073819)
(99.7997997997998,100000.000049176)
(99.8998998998999,100000.000024577)
(100,100000)

};
\addplot [red]
coordinates {
(0,110000)
(0.1001001001001,109986.059352758)
(0.2002002002002,109972.118745245)
(0.3003003003003,109958.17821719)
(0.4004004004004,109944.23780832)
(0.5005005005005,109930.297558363)
(0.6006006006006,109916.357507041)
(0.700700700700701,109902.41769408)
(0.800800800800801,109888.478159197)
(0.900900900900901,109874.538942112)
(1.001001001001,109860.600082536)
(1.1011011011011,109846.661620181)
(1.2012012012012,109832.723594751)
(1.3013013013013,109818.786045948)
(1.4014014014014,109804.849013467)
(1.5015015015015,109790.912536999)
(1.6016016016016,109776.976656227)
(1.7017017017017,109763.04141083)
(1.8018018018018,109749.106840478)
(1.9019019019019,109735.172984836)
(2.002002002002,109721.23988356)
(2.1021021021021,109707.307576297)
(2.2022022022022,109693.376102686)
(2.3023023023023,109679.445502359)
(2.4024024024024,109665.515814937)
(2.5025025025025,109651.587080031)
(2.6026026026026,109637.659337241)
(2.7027027027027,109623.732626159)
(2.8028028028028,109609.806986363)
(2.9029029029029,109595.882457422)
(3.003003003003,109581.959078891)
(3.1031031031031,109568.036890314)
(3.2032032032032,109554.115931222)
(3.3033033033033,109540.196241132)
(3.4034034034034,109526.277859549)
(3.5035035035035,109512.360825961)
(3.6036036036036,109498.445179845)
(3.7037037037037,109484.530960661)
(3.8038038038038,109470.618207854)
(3.9039039039039,109456.706960854)
(4.004004004004,109442.797259073)
(4.1041041041041,109428.88914191)
(4.2042042042042,109414.982648742)
(4.3043043043043,109401.077818933)
(4.4044044044044,109387.174691826)
(4.5045045045045,109373.273306748)
(4.6046046046046,109359.373703007)
(4.7047047047047,109345.47591989)
(4.8048048048048,109331.579996666)
(4.9049049049049,109317.685972584)
(5.005005005005,109303.793886873)
(5.1051051051051,109289.903778739)
(5.2052052052052,109276.01568737)
(5.3053053053053,109262.129651929)
(5.40540540540541,109248.24571156)
(5.50550550550551,109234.363905382)
(5.60560560560561,109220.484272493)
(5.70570570570571,109206.606851967)
(5.80580580580581,109192.731682854)
(5.90590590590591,109178.858804179)
(6.00600600600601,109164.988254945)
(6.10610610610611,109151.120074127)
(6.20620620620621,109137.254300678)
(6.30630630630631,109123.390973521)
(6.40640640640641,109109.530131555)
(6.50650650650651,109095.671813654)
(6.60660660660661,109081.816058663)
(6.70670670670671,109067.962905398)
(6.80680680680681,109054.11239265)
(6.90690690690691,109040.264559181)
(7.00700700700701,109026.419443723)
(7.10710710710711,109012.577084979)
(7.20720720720721,108998.737521624)
(7.30730730730731,108984.900792302)
(7.40740740740741,108971.066935626)
(7.50750750750751,108957.23599018)
(7.60760760760761,108943.407994514)
(7.70770770770771,108929.58298715)
(7.80780780780781,108915.761006575)
(7.90790790790791,108901.942091244)
(8.00800800800801,108888.126279581)
(8.10810810810811,108874.313609974)
(8.20820820820821,108860.50412078)
(8.30830830830831,108846.69785032)
(8.40840840840841,108832.894836881)
(8.50850850850851,108819.095118716)
(8.60860860860861,108805.298734042)
(8.70870870870871,108791.505721039)
(8.80880880880881,108777.716117853)
(8.90890890890891,108763.929962593)
(9.00900900900901,108750.14729333)
(9.10910910910911,108736.368148098)
(9.20920920920921,108722.592564895)
(9.30930930930931,108708.820581678)
(9.40940940940941,108695.052236368)
(9.50950950950951,108681.287566845)
(9.60960960960961,108667.526610951)
(9.70970970970971,108653.769406487)
(9.80980980980981,108640.015991216)
(9.90990990990991,108626.266402858)
(10.01001001001,108612.520679094)
(10.1101101101101,108598.778857562)
(10.2102102102102,108585.040975859)
(10.3103103103103,108571.30707154)
(10.4104104104104,108557.577182117)
(10.5105105105105,108543.85134506)
(10.6106106106106,108530.129597794)
(10.7107107107107,108516.411977702)
(10.8108108108108,108502.698522121)
(10.9109109109109,108488.989268345)
(11.011011011011,108475.284253623)
(11.1111111111111,108461.583515158)
(11.2112112112112,108447.887090106)
(11.3113113113113,108434.19501558)
(11.4114114114114,108420.507328645)
(11.5115115115115,108406.824066318)
(11.6116116116116,108393.145265569)
(11.7117117117117,108379.470963322)
(11.8118118118118,108365.801196451)
(11.9119119119119,108352.136001783)
(12.012012012012,108338.475416096)
(12.1121121121121,108324.819476116)
(12.2122122122122,108311.168218524)
(12.3123123123123,108297.521679947)
(12.4124124124124,108283.879896964)
(12.5125125125125,108270.242906102)
(12.6126126126126,108256.610743837)
(12.7127127127127,108242.983446594)
(12.8128128128128,108229.361050745)
(12.9129129129129,108215.74359261)
(13.013013013013,108202.131108458)
(13.1131131131131,108188.523634502)
(13.2132132132132,108174.921206904)
(13.3133133133133,108161.32386177)
(13.4134134134134,108147.731635154)
(13.5135135135135,108134.144563053)
(13.6136136136136,108120.562681411)
(13.7137137137137,108106.986026116)
(13.8138138138138,108093.414633)
(13.9139139139139,108079.848537839)
(14.014014014014,108066.287776352)
(14.1141141141141,108052.732384201)
(14.2142142142142,108039.182396993)
(14.3143143143143,108025.637850274)
(14.4144144144144,108012.098779535)
(14.5145145145145,107998.565220206)
(14.6146146146146,107985.03720766)
(14.7147147147147,107971.514777211)
(14.8148148148148,107957.997964112)
(14.9149149149149,107944.486803557)
(15.015015015015,107930.981330681)
(15.1151151151151,107917.481580556)
(15.2152152152152,107903.987588195)
(15.3153153153153,107890.499388549)
(15.4154154154154,107877.017016506)
(15.5155155155155,107863.540506894)
(15.6156156156156,107850.069894478)
(15.7157157157157,107836.60521396)
(15.8158158158158,107823.146499978)
(15.9159159159159,107809.693787108)
(16.016016016016,107796.247109861)
(16.1161161161161,107782.806502685)
(16.2162162162162,107769.371999961)
(16.3163163163163,107755.943636009)
(16.4164164164164,107742.521445079)
(16.5165165165165,107729.105461359)
(16.6166166166166,107715.69571897)
(16.7167167167167,107702.292251965)
(16.8168168168168,107688.895094333)
(16.9169169169169,107675.504279994)
(17.017017017017,107662.1198428)
(17.1171171171171,107648.741816538)
(17.2172172172172,107635.370234924)
(17.3173173173173,107622.005131607)
(17.4174174174174,107608.646540167)
(17.5175175175175,107595.294494115)
(17.6176176176176,107581.949026892)
(17.7177177177177,107568.610171868)
(17.8178178178178,107555.277962346)
(17.9179179179179,107541.952431555)
(18.018018018018,107528.633612656)
(18.1181181181181,107515.321538736)
(18.2182182182182,107502.016242812)
(18.3183183183183,107488.71775783)
(18.4184184184184,107475.426116661)
(18.5185185185185,107462.141352106)
(18.6186186186186,107448.863496892)
(18.7187187187187,107435.592583672)
(18.8188188188188,107422.328645027)
(18.9189189189189,107409.071713462)
(19.019019019019,107395.82182141)
(19.1191191191191,107382.579001228)
(19.2192192192192,107369.343285198)
(19.3193193193193,107356.114705527)
(19.4194194194194,107342.893294346)
(19.5195195195195,107329.679083711)
(19.6196196196196,107316.472105602)
(19.7197197197197,107303.272391919)
(19.8198198198198,107290.07997449)
(19.9199199199199,107276.894885063)
(20.02002002002,107263.717155308)
(20.1201201201201,107250.546816818)
(20.2202202202202,107237.383901108)
(20.3203203203203,107224.228439614)
(20.4204204204204,107211.080463693)
(20.5205205205205,107197.940004624)
(20.6206206206206,107184.807093604)
(20.7207207207207,107171.681761753)
(20.8208208208208,107158.564040108)
(20.9209209209209,107145.453959629)
(21.021021021021,107132.351551192)
(21.1211211211211,107119.256845594)
(21.2212212212212,107106.169873549)
(21.3213213213213,107093.090665691)
(21.4214214214214,107080.01925257)
(21.5215215215215,107066.955664656)
(21.6216216216216,107053.899932334)
(21.7217217217217,107040.852085907)
(21.8218218218218,107027.812155597)
(21.9219219219219,107014.780171537)
(22.022022022022,107001.756163782)
(22.1221221221221,106988.7401623)
(22.2222222222222,106975.732196974)
(22.3223223223223,106962.732297603)
(22.4224224224224,106949.740493901)
(22.5225225225225,106936.756815497)
(22.6226226226226,106923.781291934)
(22.7227227227227,106910.813952668)
(22.8228228228228,106897.854827071)
(22.9229229229229,106884.903944425)
(23.023023023023,106871.961333929)
(23.1231231231231,106859.027024693)
(23.2232232232232,106846.101045738)
(23.3233233233233,106833.183426)
(23.4234234234234,106820.274194326)
(23.5235235235235,106807.373379474)
(23.6236236236236,106794.481010115)
(23.7237237237237,106781.597114828)
(23.8238238238238,106768.721722107)
(23.9239239239239,106755.854860352)
(24.024024024024,106742.996557878)
(24.1241241241241,106730.146842906)
(24.2242242242242,106717.30574357)
(24.3243243243243,106704.473287909)
(24.4244244244244,106691.649503876)
(24.5245245245245,106678.83441933)
(24.6246246246246,106666.02806204)
(24.7247247247247,106653.230459681)
(24.8248248248248,106640.441639839)
(24.9249249249249,106627.661630006)
(25.025025025025,106614.890457582)
(25.1251251251251,106602.128149873)
(25.2252252252252,106589.374734094)
(25.3253253253253,106576.630237365)
(25.4254254254254,106563.894686713)
(25.5255255255255,106551.168109072)
(25.6256256256256,106538.450531281)
(25.7257257257257,106525.741980084)
(25.8258258258258,106513.042482132)
(25.9259259259259,106500.352063979)
(26.026026026026,106487.670752085)
(26.1261261261261,106474.998572815)
(26.2262262262262,106462.335552437)
(26.3263263263263,106449.681717125)
(26.4264264264264,106437.037092955)
(26.5265265265265,106424.401705906)
(26.6266266266266,106411.775581864)
(26.7267267267267,106399.158746613)
(26.8268268268268,106386.551225844)
(26.9269269269269,106373.953045148)
(27.027027027027,106361.364230018)
(27.1271271271271,106348.784805852)
(27.2272272272272,106336.214797946)
(27.3273273273273,106323.6542315)
(27.4274274274274,106311.103131616)
(27.5275275275275,106298.561523294)
(27.6276276276276,106286.029431437)
(27.7277277277277,106273.506880848)
(27.8278278278278,106260.99389623)
(27.9279279279279,106248.490502188)
(28.028028028028,106235.996723224)
(28.1281281281281,106223.512583741)
(28.2282282282282,106211.038108041)
(28.3283283283283,106198.573320327)
(28.4284284284284,106186.118244698)
(28.5285285285285,106173.672905153)
(28.6286286286286,106161.23732559)
(28.7287287287287,106148.811529805)
(28.8288288288288,106136.395541491)
(28.9289289289289,106123.98938424)
(29.029029029029,106111.593081541)
(29.1291291291291,106099.206656779)
(29.2292292292292,106086.830133239)
(29.3293293293293,106074.463534101)
(29.4294294294294,106062.106882442)
(29.5295295295295,106049.760201235)
(29.6296296296296,106037.423513349)
(29.7297297297297,106025.096841551)
(29.8298298298298,106012.780208501)
(29.9299299299299,106000.473636757)
(30.03003003003,105988.177148771)
(30.1301301301301,105975.89076689)
(30.2302302302302,105963.614513357)
(30.3303303303303,105951.348410308)
(30.4304304304304,105939.092479776)
(30.5305305305305,105926.846743686)
(30.6306306306306,105914.611223858)
(30.7307307307307,105902.385942007)
(30.8308308308308,105890.170919739)
(30.9309309309309,105877.966178556)
(31.031031031031,105865.771739852)
(31.1311311311311,105853.587624916)
(31.2312312312312,105841.413854926)
(31.3313313313313,105829.250450957)
(31.4314314314314,105817.097433973)
(31.5315315315315,105804.954824834)
(31.6316316316316,105792.822644288)
(31.7317317317317,105780.700912977)
(31.8318318318318,105768.589651436)
(31.9319319319319,105756.48888009)
(32.032032032032,105744.398619255)
(32.1321321321321,105732.318889139)
(32.2322322322322,105720.24970984)
(32.3323323323323,105708.191101349)
(32.4324324324324,105696.143083545)
(32.5325325325325,105684.105676199)
(32.6326326326326,105672.078898971)
(32.7327327327327,105660.062771414)
(32.8328328328328,105648.057312967)
(32.9329329329329,105636.06254296)
(33.033033033033,105624.078480616)
(33.1331331331331,105612.105145041)
(33.2332332332332,105600.142555236)
(33.3333333333333,105588.190730089)
(33.4334334334334,105576.249688374)
(33.5335335335335,105564.319448759)
(33.6336336336336,105552.400029796)
(33.7337337337337,105540.491449927)
(33.8338338338338,105528.593727484)
(33.9339339339339,105516.706880684)
(34.034034034034,105504.830927634)
(34.1341341341341,105492.965886327)
(34.2342342342342,105481.111774646)
(34.3343343343343,105469.268610358)
(34.4344344344344,105457.43641112)
(34.5345345345345,105445.615194475)
(34.6346346346346,105433.804977854)
(34.7347347347347,105422.005778572)
(34.8348348348348,105410.217613834)
(34.9349349349349,105398.44050073)
(35.035035035035,105386.674456235)
(35.1351351351351,105374.919497212)
(35.2352352352352,105363.17564041)
(35.3353353353353,105351.442902463)
(35.4354354354354,105339.721299891)
(35.5355355355355,105328.010849099)
(35.6356356356356,105316.311566378)
(35.7357357357357,105304.623467906)
(35.8358358358358,105292.946569743)
(35.9359359359359,105281.280887836)
(36.036036036036,105269.626438016)
(36.1361361361361,105257.983236001)
(36.2362362362362,105246.351297389)
(36.3363363363363,105234.730637668)
(36.4364364364364,105223.121272206)
(36.5365365365365,105211.523216258)
(36.6366366366366,105199.936484961)
(36.7367367367367,105188.361093337)
(36.8368368368368,105176.797056293)
(36.9369369369369,105165.244388618)
(37.037037037037,105153.703104984)
(37.1371371371371,105142.17321995)
(37.2372372372372,105130.654747954)
(37.3373373373373,105119.147703319)
(37.4374374374374,105107.652100254)
(37.5375375375375,105096.167952845)
(37.6376376376376,105084.695275067)
(37.7377377377377,105073.234080773)
(37.8378378378378,105061.784383702)
(37.9379379379379,105050.346197474)
(38.038038038038,105038.919535591)
(38.1381381381381,105027.504411438)
(38.2382382382382,105016.100838283)
(38.3383383383383,105004.708829275)
(38.4384384384384,104993.328397446)
(38.5385385385385,104981.959555708)
(38.6386386386386,104970.602316856)
(38.7387387387387,104959.256693568)
(38.8388388388388,104947.922698401)
(38.9389389389389,104936.600343796)
(39.039039039039,104925.289642074)
(39.1391391391391,104913.990605437)
(39.2392392392392,104902.70324597)
(39.3393393393393,104891.427575636)
(39.4394394394394,104880.163606283)
(39.5395395395395,104868.911349637)
(39.6396396396396,104857.670817305)
(39.7397397397397,104846.442020777)
(39.8398398398398,104835.224971421)
(39.9399399399399,104824.019680487)
(40.04004004004,104812.826159106)
(40.1401401401401,104801.644418288)
(40.2402402402402,104790.474468925)
(40.3403403403403,104779.316321787)
(40.4404404404404,104768.169987526)
(40.5405405405405,104757.035476674)
(40.6406406406406,104745.912799643)
(40.7407407407407,104734.801966724)
(40.8408408408408,104723.702988088)
(40.9409409409409,104712.615873787)
(41.041041041041,104701.540633752)
(41.1411411411411,104690.477277793)
(41.2412412412412,104679.425815602)
(41.3413413413413,104668.386256747)
(41.4414414414414,104657.358610679)
(41.5415415415415,104646.342886725)
(41.6416416416416,104635.339094094)
(41.7417417417417,104624.347241874)
(41.8418418418418,104613.36733903)
(41.9419419419419,104602.399394409)
(42.042042042042,104591.443416736)
(42.1421421421421,104580.499414613)
(42.2422422422422,104569.567396525)
(42.3423423423423,104558.647370833)
(42.4424424424424,104547.739345777)
(42.5425425425425,104536.843329478)
(42.6426426426426,104525.959329933)
(42.7427427427427,104515.087355019)
(42.8428428428428,104504.227412493)
(42.9429429429429,104493.379509988)
(43.043043043043,104482.543655017)
(43.1431431431431,104471.719854973)
(43.2432432432432,104460.908117125)
(43.3433433433433,104450.108448621)
(43.4434434434434,104439.320856488)
(43.5435435435435,104428.545347632)
(43.6436436436436,104417.781928836)
(43.7437437437437,104407.030606763)
(43.8438438438438,104396.291387952)
(43.9439439439439,104385.564278822)
(44.044044044044,104374.84928567)
(44.1441441441441,104364.14641467)
(44.2442442442442,104353.455671876)
(44.3443443443443,104342.777063218)
(44.4444444444444,104332.110594507)
(44.5445445445445,104321.456271428)
(44.6446446446446,104310.814099549)
(44.7447447447447,104300.184084312)
(44.8448448448448,104289.566231038)
(44.9449449449449,104278.960544927)
(45.045045045045,104268.367031056)
(45.1451451451451,104257.785694381)
(45.2452452452452,104247.216539735)
(45.3453453453453,104236.659571828)
(45.4454454454454,104226.114795251)
(45.5455455455455,104215.58221447)
(45.6456456456456,104205.061833829)
(45.7457457457457,104194.553657552)
(45.8458458458458,104184.057689739)
(45.9459459459459,104173.573934369)
(46.046046046046,104163.102395297)
(46.1461461461461,104152.643076258)
(46.2462462462462,104142.195980863)
(46.3463463463463,104131.761112603)
(46.4464464464464,104121.338474844)
(46.5465465465465,104110.928070833)
(46.6466466466466,104100.529903692)
(46.7467467467467,104090.143976422)
(46.8468468468468,104079.770291903)
(46.9469469469469,104069.408852891)
(47.047047047047,104059.059662021)
(47.1471471471471,104048.722721804)
(47.2472472472472,104038.398034632)
(47.3473473473473,104028.085602773)
(47.4474474474474,104017.785428371)
(47.5475475475475,104007.497513452)
(47.6476476476476,103997.221859918)
(47.7477477477477,103986.958469547)
(47.8478478478478,103976.707343998)
(47.9479479479479,103966.468484806)
(48.048048048048,103956.241893385)
(48.1481481481481,103946.027571026)
(48.2482482482482,103935.825518899)
(48.3483483483483,103925.635738052)
(48.4484484484484,103915.45822941)
(48.5485485485485,103905.292993778)
(48.6486486486486,103895.140031836)
(48.7487487487487,103884.999344146)
(48.8488488488488,103874.870931145)
(48.9489489489489,103864.75479315)
(49.049049049049,103854.650930355)
(49.1491491491491,103844.559342833)
(49.2492492492492,103834.480030536)
(49.3493493493493,103824.412993292)
(49.4494494494494,103814.35823081)
(49.5495495495495,103804.315742676)
(49.6496496496496,103794.285528354)
(49.7497497497497,103784.267587189)
(49.8498498498498,103774.2619184)
(49.9499499499499,103764.26852109)
(50.05005005005,103754.287394236)
(50.1501501501501,103744.318536695)
(50.2502502502502,103734.361947205)
(50.3503503503503,103724.41762438)
(50.4504504504504,103714.485566713)
(50.5505505505505,103704.565772577)
(50.6506506506506,103694.658240224)
(50.7507507507507,103684.762967783)
(50.8508508508508,103674.879953265)
(50.9509509509509,103665.009194557)
(51.051051051051,103655.150689426)
(51.1511511511511,103645.304435521)
(51.2512512512512,103635.470430365)
(51.3513513513513,103625.648671364)
(51.4514514514514,103615.839155803)
(51.5515515515515,103606.041880845)
(51.6516516516516,103596.256843533)
(51.7517517517517,103586.48404079)
(51.8518518518518,103576.723469419)
(51.9519519519519,103566.975126101)
(52.052052052052,103557.239007398)
(52.1521521521521,103547.515109751)
(52.2522522522522,103537.803429482)
(52.3523523523523,103528.103962791)
(52.4524524524524,103518.41670576)
(52.5525525525525,103508.74165435)
(52.6526526526526,103499.078804401)
(52.7527527527527,103489.428151637)
(52.8528528528528,103479.789691658)
(52.9529529529529,103470.163419946)
(53.053053053053,103460.549331864)
(53.1531531531531,103450.947422656)
(53.2532532532532,103441.357687444)
(53.3533533533533,103431.780121234)
(53.4534534534534,103422.214718909)
(53.5535535535535,103412.661475237)
(53.6536536536536,103403.120384864)
(53.7537537537537,103393.591442317)
(53.8538538538538,103384.074642007)
(53.9539539539539,103374.569978222)
(54.054054054054,103365.077445135)
(54.1541541541541,103355.597036797)
(54.2542542542542,103346.128747145)
(54.3543543543543,103336.672569992)
(54.4544544544544,103327.228499037)
(54.5545545545545,103317.796527859)
(54.6546546546546,103308.376649919)
(54.7547547547547,103298.968858559)
(54.8548548548548,103289.573147006)
(54.9549549549549,103280.189508365)
(55.055055055055,103270.817935626)
(55.1551551551551,103261.458421662)
(55.2552552552552,103252.110959225)
(55.3553553553553,103242.775540953)
(55.4554554554554,103233.452159365)
(55.5555555555556,103224.140806862)
(55.6556556556557,103214.841475731)
(55.7557557557558,103205.554158138)
(55.8558558558559,103196.278846135)
(55.955955955956,103187.015531656)
(56.0560560560561,103177.764206518)
(56.1561561561562,103168.524862422)
(56.2562562562563,103159.297490952)
(56.3563563563564,103150.082083576)
(56.4564564564565,103140.878631647)
(56.5565565565566,103131.687126399)
(56.6566566566567,103122.507558952)
(56.7567567567568,103113.339920311)
(56.8568568568569,103104.184201362)
(56.956956956957,103095.040392878)
(57.0570570570571,103085.908485517)
(57.1571571571572,103076.788469818)
(57.2572572572573,103067.680336208)
(57.3573573573574,103058.584074999)
(57.4574574574575,103049.499676386)
(57.5575575575576,103040.427130449)
(57.6576576576577,103031.366427156)
(57.7577577577578,103022.317556357)
(57.8578578578579,103013.28050779)
(57.957957957958,103004.255271077)
(58.0580580580581,102995.241835726)
(58.1581581581582,102986.240191133)
(58.2582582582583,102977.250326576)
(58.3583583583584,102968.272231223)
(58.4584584584585,102959.305894126)
(58.5585585585586,102950.351304224)
(58.6586586586587,102941.408450344)
(58.7587587587588,102932.477321196)
(58.8588588588589,102923.557905382)
(58.958958958959,102914.650191386)
(59.0590590590591,102905.754167582)
(59.1591591591592,102896.869822232)
(59.2592592592593,102887.997143482)
(59.3593593593594,102879.136119368)
(59.4594594594595,102870.286737814)
(59.5595595595596,102861.448986631)
(59.6596596596597,102852.622853518)
(59.7597597597598,102843.808326063)
(59.8598598598599,102835.00539174)
(59.95995995996,102826.214037915)
(60.0600600600601,102817.434251839)
(60.1601601601602,102808.666020654)
(60.2602602602603,102799.909331392)
(60.3603603603604,102791.16417097)
(60.4604604604605,102782.430526197)
(60.5605605605606,102773.708383773)
(60.6606606606607,102764.997730284)
(60.7607607607608,102756.298552207)
(60.8608608608609,102747.610835911)
(60.960960960961,102738.934567651)
(61.0610610610611,102730.269733576)
(61.1611611611612,102721.616319723)
(61.2612612612613,102712.974312021)
(61.3613613613614,102704.343696288)
(61.4614614614615,102695.724458233)
(61.5615615615616,102687.116583459)
(61.6616616616617,102678.520057457)
(61.7617617617618,102669.93486561)
(61.8618618618619,102661.360993193)
(61.961961961962,102652.798425373)
(62.0620620620621,102644.247147208)
(62.1621621621622,102635.707143648)
(62.2622622622623,102627.178399536)
(62.3623623623624,102618.660899608)
(62.4624624624625,102610.15462849)
(62.5625625625626,102601.659570703)
(62.6626626626627,102593.17571066)
(62.7627627627628,102584.703032668)
(62.8628628628629,102576.241520927)
(62.962962962963,102567.791159528)
(63.0630630630631,102559.351932459)
(63.1631631631632,102550.923823601)
(63.2632632632633,102542.506816727)
(63.3633633633634,102534.100895507)
(63.4634634634635,102525.706043504)
(63.5635635635636,102517.322244174)
(63.6636636636637,102508.949480871)
(63.7637637637638,102500.587736841)
(63.8638638638639,102492.236995227)
(63.963963963964,102483.897239067)
(64.0640640640641,102475.568451293)
(64.1641641641641,102467.250614735)
(64.2642642642643,102458.943712116)
(64.3643643643643,102450.647726058)
(64.4644644644645,102442.362639078)
(64.5645645645645,102434.088433589)
(64.6646646646647,102425.8250919)
(64.7647647647647,102417.57259622)
(64.8648648648649,102409.330928652)
(64.9649649649649,102401.100071196)
(65.0650650650651,102392.880005752)
(65.1651651651651,102384.670714116)
(65.2652652652653,102376.472177981)
(65.3653653653653,102368.284378941)
(65.4654654654655,102360.107298484)
(65.5655655655655,102351.940918001)
(65.6656656656657,102343.785218777)
(65.7657657657657,102335.640182)
(65.8658658658659,102327.505788754)
(65.9659659659659,102319.382020025)
(66.0660660660661,102311.268856695)
(66.1661661661661,102303.166279549)
(66.2662662662663,102295.074269271)
(66.3663663663663,102286.992806443)
(66.4664664664665,102278.921871549)
(66.5665665665666,102270.861444975)
(66.6666666666667,102262.811507004)
(66.7667667667668,102254.772037823)
(66.8668668668669,102246.743017519)
(66.966966966967,102238.724426081)
(67.0670670670671,102230.716243399)
(67.1671671671672,102222.718449264)
(67.2672672672673,102214.73102337)
(67.3673673673674,102206.753945313)
(67.4674674674675,102198.787194592)
(67.5675675675676,102190.830750608)
(67.6676676676677,102182.884592665)
(67.7677677677678,102174.94869997)
(67.8678678678679,102167.023051633)
(67.967967967968,102159.107626668)
(68.0680680680681,102151.202403994)
(68.1681681681682,102143.30736243)
(68.2682682682683,102135.422480705)
(68.3683683683684,102127.547737447)
(68.4684684684685,102119.683111191)
(68.5685685685686,102111.828580378)
(68.6686686686687,102103.984123353)
(68.7687687687688,102096.149718365)
(68.8688688688689,102088.32534357)
(68.968968968969,102080.510977031)
(69.0690690690691,102072.706596714)
(69.1691691691692,102064.912180493)
(69.2692692692693,102057.127706149)
(69.3693693693694,102049.353151369)
(69.4694694694695,102041.588493747)
(69.5695695695696,102033.833710784)
(69.6696696696697,102026.088779888)
(69.7697697697698,102018.353678376)
(69.8698698698699,102010.628383472)
(69.96996996997,102002.912872308)
(70.0700700700701,101995.207121925)
(70.1701701701702,101987.511109272)
(70.2702702702703,101979.824811206)
(70.3703703703704,101972.148204496)
(70.4704704704705,101964.481265816)
(70.5705705705706,101956.823971754)
(70.6706706706707,101949.176298805)
(70.7707707707708,101941.538223375)
(70.8708708708709,101933.90972178)
(70.970970970971,101926.290770245)
(71.0710710710711,101918.68134491)
(71.1711711711712,101911.081421821)
(71.2712712712713,101903.490976939)
(71.3713713713714,101895.909986135)
(71.4714714714715,101888.338425191)
(71.5715715715716,101880.776269802)
(71.6716716716717,101873.223495576)
(71.7717717717718,101865.680078031)
(71.8718718718719,101858.145992601)
(71.971971971972,101850.62121463)
(72.0720720720721,101843.105719377)
(72.1721721721722,101835.599482013)
(72.2722722722723,101828.102477626)
(72.3723723723724,101820.614681213)
(72.4724724724725,101813.13606769)
(72.5725725725726,101805.666611885)
(72.6726726726727,101798.206288541)
(72.7727727727728,101790.755072316)
(72.8728728728729,101783.312937784)
(72.972972972973,101775.879859433)
(73.0730730730731,101768.455811669)
(73.1731731731732,101761.040768813)
(73.2732732732733,101753.634705101)
(73.3733733733734,101746.237594687)
(73.4734734734735,101738.849411642)
(73.5735735735736,101731.470129953)
(73.6736736736737,101724.099723526)
(73.7737737737738,101716.738166182)
(73.8738738738739,101709.385431662)
(73.973973973974,101702.041493625)
(74.0740740740741,101694.706325647)
(74.1741741741742,101687.379901224)
(74.2742742742743,101680.06219377)
(74.3743743743744,101672.753176618)
(74.4744744744745,101665.452823023)
(74.5745745745746,101658.161106156)
(74.6746746746747,101650.877999111)
(74.7747747747748,101643.603474899)
(74.8748748748749,101636.337506455)
(74.974974974975,101629.080066633)
(75.0750750750751,101621.831128208)
(75.1751751751752,101614.590663876)
(75.2752752752753,101607.358646256)
(75.3753753753754,101600.135047888)
(75.4754754754755,101592.919841234)
(75.5755755755756,101585.712998677)
(75.6756756756757,101578.514492527)
(75.7757757757758,101571.324295012)
(75.8758758758759,101564.142378285)
(75.975975975976,101556.968714425)
(76.0760760760761,101549.803275431)
(76.1761761761762,101542.646033228)
(76.2762762762763,101535.496959664)
(76.3763763763764,101528.356026514)
(76.4764764764765,101521.223205476)
(76.5765765765766,101514.098468173)
(76.6766766766767,101506.981786154)
(76.7767767767768,101499.873130894)
(76.8768768768769,101492.772473793)
(76.976976976977,101485.679786179)
(77.0770770770771,101478.595039304)
(77.1771771771772,101471.518204349)
(77.2772772772773,101464.449252421)
(77.3773773773774,101457.388154555)
(77.4774774774775,101450.334881714)
(77.5775775775776,101443.289404786)
(77.6776776776777,101436.251694592)
(77.7777777777778,101429.221721878)
(77.8778778778779,101422.199457319)
(77.977977977978,101415.184871521)
(78.0780780780781,101408.177935018)
(78.1781781781782,101401.178618273)
(78.2782782782783,101394.186891681)
(78.3783783783784,101387.202725566)
(78.4784784784785,101380.226090182)
(78.5785785785786,101373.256955713)
(78.6786786786787,101366.295292278)
(78.7787787787788,101359.341069923)
(78.8788788788789,101352.394258628)
(78.978978978979,101345.454828304)
(79.0790790790791,101338.522748795)
(79.1791791791792,101331.597989877)
(79.2792792792793,101324.680521258)
(79.3793793793794,101317.770312582)
(79.4794794794795,101310.867333422)
(79.5795795795796,101303.971553289)
(79.6796796796797,101297.082941625)
(79.7797797797798,101290.201467807)
(79.8798798798799,101283.327101148)
(79.97997997998,101276.459810895)
(80.0800800800801,101269.599566229)
(80.1801801801802,101262.746336267)
(80.2802802802803,101255.900090063)
(80.3803803803804,101249.060796605)
(80.4804804804805,101242.22842482)
(80.5805805805806,101235.40294357)
(80.6806806806807,101228.584321652)
(80.7807807807808,101221.772527805)
(80.8808808808809,101214.9675307)
(80.980980980981,101208.169298951)
(81.0810810810811,101201.377801106)
(81.1811811811812,101194.593005653)
(81.2812812812813,101187.814881021)
(81.3813813813814,101181.043395573)
(81.4814814814815,101174.278517616)
(81.5815815815816,101167.520215394)
(81.6816816816817,101160.768457092)
(81.7817817817818,101154.023210835)
(81.8818818818819,101147.284444688)
(81.981981981982,101140.552126657)
(82.0820820820821,101133.82622469)
(82.1821821821822,101127.106706675)
(82.2822822822823,101120.393540444)
(82.3823823823824,101113.686693767)
(82.4824824824825,101106.986134361)
(82.5825825825826,101100.291829882)
(82.6826826826827,101093.603747932)
(82.7827827827828,101086.921856053)
(82.8828828828829,101080.246121733)
(82.982982982983,101073.576512403)
(83.0830830830831,101066.912995439)
(83.1831831831832,101060.255538159)
(83.2832832832833,101053.604107829)
(83.3833833833834,101046.958671658)
(83.4834834834835,101040.3191968)
(83.5835835835836,101033.685650358)
(83.6836836836837,101027.057999376)
(83.7837837837838,101020.436210849)
(83.8838838838839,101013.820251716)
(83.983983983984,101007.210088863)
(84.0840840840841,101000.605689124)
(84.1841841841842,100994.007019281)
(84.2842842842843,100987.414046063)
(84.3843843843844,100980.826736146)
(84.4844844844845,100974.245056158)
(84.5845845845846,100967.668972671)
(84.6846846846847,100961.098452212)
(84.7847847847848,100954.533461252)
(84.8848848848849,100947.973966214)
(84.984984984985,100941.419933472)
(85.0850850850851,100934.871329348)
(85.1851851851852,100928.328120117)
(85.2852852852853,100921.790272004)
(85.3853853853854,100915.257751185)
(85.4854854854855,100908.730523787)
(85.5855855855856,100902.208555892)
(85.6856856856857,100895.69181353)
(85.7857857857858,100889.180262686)
(85.8858858858859,100882.673869298)
(85.985985985986,100876.172599257)
(86.0860860860861,100869.676418406)
(86.1861861861862,100863.185292544)
(86.2862862862863,100856.699187423)
(86.3863863863864,100850.21806875)
(86.4864864864865,100843.741902185)
(86.5865865865866,100837.270653345)
(86.6866866866867,100830.804287803)
(86.7867867867868,100824.342771085)
(86.8868868868869,100817.886068676)
(86.986986986987,100811.434146015)
(87.0870870870871,100804.986968499)
(87.1871871871872,100798.544501482)
(87.2872872872873,100792.106710274)
(87.3873873873874,100785.673560145)
(87.4874874874875,100779.24501632)
(87.5875875875876,100772.821043986)
(87.6876876876877,100766.401608284)
(87.7877877877878,100759.986674318)
(87.8878878878879,100753.57620715)
(87.987987987988,100747.1701718)
(88.0880880880881,100740.768533249)
(88.1881881881882,100734.371256439)
(88.2882882882883,100727.978306272)
(88.3883883883884,100721.58964761)
(88.4884884884885,100715.205245277)
(88.5885885885886,100708.825064059)
(88.6886886886887,100702.449068702)
(88.7887887887888,100696.077223916)
(88.8888888888889,100689.709494372)
(88.988988988989,100683.345844706)
(89.0890890890891,100676.986239514)
(89.1891891891892,100670.630643358)
(89.2892892892893,100664.279020763)
(89.3893893893894,100657.931336218)
(89.4894894894895,100651.587554175)
(89.5895895895896,100645.247639053)
(89.6896896896897,100638.911555236)
(89.7897897897898,100632.579267071)
(89.8898898898899,100626.250738873)
(89.98998998999,100619.925934922)
(90.0900900900901,100613.604819464)
(90.1901901901902,100607.287356714)
(90.2902902902903,100600.973510851)
(90.3903903903904,100594.663246024)
(90.4904904904905,100588.356526346)
(90.5905905905906,100582.053315903)
(90.6906906906907,100575.753578744)
(90.7907907907908,100569.457278892)
(90.8908908908909,100563.164380334)
(90.990990990991,100556.87484703)
(91.0910910910911,100550.588642907)
(91.1911911911912,100544.305731865)
(91.2912912912913,100538.026077771)
(91.3913913913914,100531.749644465)
(91.4914914914915,100525.476395756)
(91.5915915915916,100519.206295426)
(91.6916916916917,100512.939307228)
(91.7917917917918,100506.675394887)
(91.8918918918919,100500.414522099)
(91.991991991992,100494.156652535)
(92.0920920920921,100487.901749837)
(92.1921921921922,100481.649777622)
(92.2922922922923,100475.400699478)
(92.3923923923924,100469.154478969)
(92.4924924924925,100462.911079632)
(92.5925925925926,100456.670464981)
(92.6926926926927,100450.432598501)
(92.7927927927928,100444.197443656)
(92.8928928928929,100437.964963883)
(92.992992992993,100431.735122597)
(93.0930930930931,100425.507883186)
(93.1931931931932,100419.283209018)
(93.2932932932933,100413.061063437)
(93.3933933933934,100406.841409763)
(93.4934934934935,100400.624211294)
(93.5935935935936,100394.409431306)
(93.6936936936937,100388.197033055)
(93.7937937937938,100381.986979773)
(93.8938938938939,100375.779234671)
(93.993993993994,100369.573760941)
(94.0940940940941,100363.370521754)
(94.1941941941942,100357.16948026)
(94.2942942942943,100350.97059959)
(94.3943943943944,100344.773842854)
(94.4944944944945,100338.579173146)
(94.5945945945946,100332.386553539)
(94.6946946946947,100326.195947087)
(94.7947947947948,100320.007316827)
(94.8948948948949,100313.820625778)
(94.994994994995,100307.635836943)
(95.0950950950951,100301.452913305)
(95.1951951951952,100295.271817832)
(95.2952952952953,100289.092513475)
(95.3953953953954,100282.91496317)
(95.4954954954955,100276.739129835)
(95.5955955955956,100270.564976375)
(95.6956956956957,100264.392465679)
(95.7957957957958,100258.22156062)
(95.8958958958959,100252.052224058)
(95.995995995996,100245.884418839)
(96.0960960960961,100239.718107794)
(96.1961961961962,100233.553253741)
(96.2962962962963,100227.389819486)
(96.3963963963964,100221.227767821)
(96.4964964964965,100215.067061526)
(96.5965965965966,100208.907663368)
(96.6966966966967,100202.749536105)
(96.7967967967968,100196.59264248)
(96.8968968968969,100190.436945228)
(96.996996996997,100184.28240707)
(97.0970970970971,100178.12899072)
(97.1971971971972,100171.976658879)
(97.2972972972973,100165.825374241)
(97.3973973973974,100159.675099488)
(97.4974974974975,100153.525797294)
(97.5975975975976,100147.377430326)
(97.6976976976977,100141.229961239)
(97.7977977977978,100135.083352683)
(97.8978978978979,100128.937567299)
(97.997997997998,100122.79256772)
(98.0980980980981,100116.648316573)
(98.1981981981982,100110.504776478)
(98.2982982982983,100104.361910049)
(98.3983983983984,100098.219679893)
(98.4984984984985,100092.078048612)
(98.5985985985986,100085.936978802)
(98.6986986986987,100079.796433055)
(98.7987987987988,100073.656373957)
(98.8988988988989,100067.516764091)
(98.998998998999,100061.377566034)
(99.0990990990991,100055.238742361)
(99.1991991991992,100049.100255645)
(99.2992992992993,100042.962068451)
(99.3993993993994,100036.824143347)
(99.4994994994995,100030.686442894)
(99.5995995995996,100024.548929655)
(99.6996996996997,100018.411566189)
(99.7997997997998,100012.274315053)
(99.8998998998999,100006.137138804)
(100,100000)

};
\addplot [blue, opacity=0.25, mark=square*, mark size=3, mark options={draw=black}, only marks]
coordinates {
(0,110000)
(1,108291)
(2,106728)
(3,105371)
(4,104233)
(5,103302)
(6,102554)
(7,101961)
(8,101497)
(9,101136)
(10,100859)
(11,100647)
(12,100485)
(13,100363)
(14,100270)
(15,100201)
(16,100149)
(17,100111)
(18,100082)
(19,100060)
(20,100044)
(21,100033)
(22,100024)
(23,100018)
(24,100013)
(25,100009)
(26,100007)
(27,100005)
(28,100004)
(29,100003)
(30,100002)
(31,100001)
(32,100001)
(33,100001)
(34,100001)
(35,100000)
(36,100000)
(37,100000)
(38,100000)
(39,100000)
(40,100000)
(41,100000)
(42,100000)
(43,100000)
(44,100000)
(45,100000)
(46,100000)
(47,100000)
(48,100000)
(49,100000)
(50,100000)
(51,100000)
(52,100000)
(53,100000)
(54,100000)
(55,100000)
(56,100000)
(57,100000)
(58,100000)
(59,100000)
(60,100000)
(61,100000)
(62,100000)
(63,100000)
(64,100000)
(65,100000)
(66,100000)
(67,100000)
(68,100000)
(69,100000)
(70,100000)
(71,100000)
(72,100000)
(73,100000)
(74,100000)
(75,100000)
(76,100000)
(77,100000)
(78,100000)
(79,100000)
(80,100000)
(81,100000)
(82,100000)
(83,100000)
(84,100000)
(85,100000)
(86,100000)
(87,100000)
(88,100000)
(89,100000)
(90,100000)
(91,100000)
(92,100000)
(93,100000)
(94,100000)
(95,100000)
(96,100000)
(97,100000)
(98,100000)
(99,100000)
(100,100000)

};
\addplot [green!50.0!black, opacity=0.25, mark=square*, mark size=3, mark options={draw=black}, only marks]
coordinates {
(0,110000)
(1,109554)
(2,109110)
(3,108668)
(4,108231)
(5,107800)
(6,107375)
(7,106959)
(8,106552)
(9,106155)
(10,105770)
(11,105397)
(12,105037)
(13,104690)
(14,104358)
(15,104040)
(16,103737)
(17,103449)
(18,103176)
(19,102918)
(20,102675)
(21,102447)
(22,102234)
(23,102034)
(24,101849)
(25,101676)
(26,101517)
(27,101370)
(28,101234)
(29,101109)
(30,100995)
(31,100891)
(32,100796)
(33,100710)
(34,100631)
(35,100561)
(36,100497)
(37,100439)
(38,100388)
(39,100342)
(40,100300)
(41,100264)
(42,100231)
(43,100202)
(44,100176)
(45,100154)
(46,100133)
(47,100116)
(48,100100)
(49,100087)
(50,100075)
(51,100065)
(52,100056)
(53,100048)
(54,100041)
(55,100035)
(56,100030)
(57,100026)
(58,100022)
(59,100019)
(60,100016)
(61,100013)
(62,100011)
(63,100010)
(64,100008)
(65,100007)
(66,100006)
(67,100005)
(68,100004)
(69,100003)
(70,100003)
(71,100002)
(72,100002)
(73,100002)
(74,100001)
(75,100001)
(76,100001)
(77,100001)
(78,100001)
(79,100001)
(80,100000)
(81,100000)
(82,100000)
(83,100000)
(84,100000)
(85,100000)
(86,100000)
(87,100000)
(88,100000)
(89,100000)
(90,100000)
(91,100000)
(92,100000)
(93,100000)
(94,100000)
(95,100000)
(96,100000)
(97,100000)
(98,100000)
(99,100000)
(100,100000)

};
\addplot [red, opacity=0.25, mark=square*, mark size=3, mark options={draw=black}, only marks]
coordinates {
(0,110000)
(1,109861)
(2,109721)
(3,109582)
(4,109443)
(5,109304)
(6,109165)
(7,109027)
(8,108889)
(9,108751)
(10,108613)
(11,108476)
(12,108339)
(13,108203)
(14,108067)
(15,107932)
(16,107797)
(17,107663)
(18,107530)
(19,107397)
(20,107265)
(21,107134)
(22,107003)
(23,106874)
(24,106745)
(25,106617)
(26,106490)
(27,106363)
(28,106238)
(29,106114)
(30,105990)
(31,105868)
(32,105747)
(33,105627)
(34,105507)
(35,105389)
(36,105272)
(37,105156)
(38,105042)
(39,104928)
(40,104816)
(41,104705)
(42,104595)
(43,104486)
(44,104378)
(45,104272)
(46,104167)
(47,104063)
(48,103960)
(49,103858)
(50,103758)
(51,103659)
(52,103561)
(53,103464)
(54,103369)
(55,103275)
(56,103182)
(57,103090)
(58,102999)
(59,102910)
(60,102822)
(61,102735)
(62,102649)
(63,102564)
(64,102480)
(65,102397)
(66,102316)
(67,102235)
(68,102156)
(69,102077)
(70,102000)
(71,101923)
(72,101848)
(73,101773)
(74,101700)
(75,101627)
(76,101555)
(77,101484)
(78,101413)
(79,101344)
(80,101275)
(81,101207)
(82,101139)
(83,101072)
(84,101006)
(85,100940)
(86,100875)
(87,100810)
(88,100746)
(89,100682)
(90,100619)
(91,100556)
(92,100494)
(93,100431)
(94,100369)
(95,100307)
(96,100246)
(97,100184)
(98,100123)
(99,100061)
(100,100000)

};
\path [draw=black, fill opacity=0] (axis cs:13,110000)--(axis cs:13,110000);

\path [draw=black, fill opacity=0] (axis cs:100,13)--(axis cs:100,13);

\path [draw=black, fill opacity=0] (axis cs:13,98000)--(axis cs:13,98000);

\path [draw=black, fill opacity=0] (axis cs:0,13)--(axis cs:0,13);

\end{axis}

\end{tikzpicture}
	\end{center}
	%\caption{}
	\label{fig:pressDiff}
\end{figure}