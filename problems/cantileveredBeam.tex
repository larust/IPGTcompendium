\section{Cantilevered Steel Beam Benchmark}
\subsection{Problem description}
This example tests the ability of a code to describe a common mechanical behaviour, the bending of a cantilevered beam with a uniform load applied on its upper face. The problem is defined for a steel \SI[scientific-notation=false, round-precision=2]{1.0}{\metre} long, \SI[scientific-notation=false, round-precision=2]{0.1}{\metre} thick, steel beam. The initial condition is zero displacement. Output data for y-axis displacements are shown for steady state.


\begin{table}
	\caption{Problem dimensions}
	\begin{center}
	\begin{tabular}{lrcSs}
		Length of beam & $L_x$ & $\coloneqq$ & 1.0 & \si{\metre} \\
		Thickness of beam& $h$ & $\coloneqq$ & 0.010 & \si{\metre} \\
		Loading& $q$ & $\coloneqq$ & 1.0e3 & \si{\pascal} \\
	\end{tabular}
	\end{center}
	\label{tab:beamDim}
\end{table}
\begin{table}
	\caption{Material properties}
	\begin{center}
	\begin{tabular}{lrcSs}
		Density of steel& $\rho_r$ & $\coloneqq$ & 7850.0 & \si{\kilogram\per\metre\cubed} \\
		Youngs modulus of steel& $E_r$ & $\coloneqq$ & 2.0e11 & \si{\pascal} \\
		Shear modulus of steel& $G_r$ & $\coloneqq$ & 0.30e11 & \si{\pascal} \\
		Poisson's ratio & $\nu_r$ & $\coloneqq$ & 0.225 &  \\
	\end{tabular}
	\end{center}
	\label{tab:beamMatProps}
\end{table}

%\renewcommand{\arraystretch}{1.5}
\begin{table}[h]
	\caption{Derived parameters}
	\begin{center}
		\begin{tabular}{lrclcSs}
		Total force applied& $F$ & $\coloneqq$ & $qL_x$ & $=$ &1000.0 & \si{\newton\per\metre} \\
		Flexural rigidity of plate& $D$ & $\coloneqq$ & $\dfrac{E_r h^3}{12(1-\nu_2^2)}$ & $=$ &  17555.40926047838& \si{\joule} \\

		Moment of inertia & $I$ & $\coloneqq$ & $\dfrac{D}{E_r}$ & $=$ &  8.7777046302391941e-08 & \si{\metre\tothe{4}} 
		%& $$ & $\coloneqq$ & $ $ & $=$ & & \si{} \\
		%& $$ & $\coloneqq$ & $ $ & $=$ & & \si{} \\
		%& $$ & $\coloneqq$ & $ $ & $=$ & & \si{} \\
	\end{tabular}
	\end{center}
	\label{tab:beamDerivPar}
\end{table}
A reference for the analytical solution is Turcotte and Schubert, page 117. The solution was computed using Mathcad, Version 15, using parameter values and equations shown below. The numerical solution was computed with the FALCON simulator from INL. Specified dimensions and properties for the material (steel) are highlighted above.

\subsection{Files}
\begin{itemize}
	\item Analytical results are stored in \verb|CantileveredSteelBeamBM-analytical.xlsx|
	\item Numerical results are stored in \verb|CantileveredSteelBeamBM-FALCON.xlsx|
	\item The numerical solution input file is \verb|CantileveredSteelBeamBM.i|
	\item The numerical solution exodus output file is \verb|CantileveredSteelBeamBM_out.e|
\end{itemize}

\subsection{Results}
\begin{table}[h]
	\caption{Analytical solution of the beam problem}
	\begin{center}
		\begin{tabular}{lrcl}
			Bending moment& $M(x)$ & $\coloneqq$ & $\dfrac{1}{2}q(L_x-x)^2$ \\
			Displacement solution& $w(x)$ & $\coloneqq$ & $\dfrac{qx^2}{D}\left( \dfrac{x^2}{24}-\dfrac{L_x x}{6}+\dfrac{L_x^2}{4} \right)$  \\
			Maximum bending stress& $\sigma_{xx,\text{max}}$ & $\coloneqq$ & $-\dfrac{6M(x)}{h^2}$ \\
	\end{tabular}
	\end{center}
	\label{tab:beamSolPar}
\end{table}

\begin{figure}[h]
	\begin{center}
		\setlength\figureheight{8cm} 
		\setlength\figurewidth{0.8\textwidth} 
		% This file was created by matplotlib v0.1.0.
% Copyright (c) 2010--2014, Nico Schlömer <nico.schloemer@gmail.com>
% All rights reserved.
% 

\begin{tikzpicture}

\begin{axis}[
xlabel={Fractional position along plate of length L},
ylabel={Displacement [mm]},
xmin=0, xmax=1,
ymin=-8, ymax=0,
axis on top,
width=\figurewidth,
height=\figureheight
]
\addplot [red]
coordinates {
(0,0)
(0.001001001001001,-1.42596291235562e-05)
(0.002002002002002,-5.7000455936455e-05)
(0.003003003003003,-0.000128165432495232)
(0.004004004004004,-0.000227697568047344)
(0.005005005005005,-0.000355539929031167)
(0.00600600600600601,-0.000511635639076)
(0.00700700700700701,-0.000695927879002062)
(0.00800800800800801,-0.000908359886820492)
(0.00900900900900901,-0.00114887495773335)
(0.01001001001001,-0.00141741644413362)
(0.011011011011011,-0.0017139277556052)
(0.012012012012012,-0.00203835235892291)
(0.013013013013013,-0.0023906337780525)
(0.014014014014014,-0.00277071559415063)
(0.015015015015015,-0.00317854144556489)
(0.016016016016016,-0.00361405502783377)
(0.017017017017017,-0.00407720009368672)
(0.018018018018018,-0.00456792045304407)
(0.019019019019019,-0.0050861599730171)
(0.02002002002002,-0.00563186257790798)
(0.021021021021021,-0.00620497224920984)
(0.022022022022022,-0.0068054330256067)
(0.023023023023023,-0.00743318900297351)
(0.024024024024024,-0.00808818433437615)
(0.025025025025025,-0.0087703632300714)
(0.026026026026026,-0.00947966995750698)
(0.027027027027027,-0.0102160488413215)
(0.028028028028028,-0.0109794442633446)
(0.029029029029029,-0.0117698006625967)
(0.03003003003003,-0.0125870625352891)
(0.031031031031031,-0.0134311744348243)
(0.032032032032032,-0.0143020809717954)
(0.033033033033033,-0.0151997268139866)
(0.034034034034034,-0.016124056686373)
(0.035035035035035,-0.0170750153711206)
(0.036036036036036,-0.0180525477075862)
(0.037037037037037,-0.0190565985923179)
(0.038038038038038,-0.0200871129790542)
(0.039039039039039,-0.0211440358787249)
(0.04004004004004,-0.0222273123594507)
(0.041041041041041,-0.0233368875465429)
(0.042042042042042,-0.0244727066225041)
(0.043043043043043,-0.0256347148270277)
(0.044044044044044,-0.0268228574569979)
(0.045045045045045,-0.0280370798664898)
(0.046046046046046,-0.0292773274667698)
(0.047047047047047,-0.0305435457262948)
(0.048048048048048,-0.0318356801707127)
(0.049049049049049,-0.0331536763828625)
(0.05005005005005,-0.034497480002774)
(0.0510510510510511,-0.0358670367276679)
(0.0520520520520521,-0.0372622923119558)
(0.0530530530530531,-0.0386831925672404)
(0.0540540540540541,-0.0401296833623152)
(0.0550550550550551,-0.0416017106231645)
(0.0560560560560561,-0.0430992203329636)
(0.0570570570570571,-0.0446221585320789)
(0.0580580580580581,-0.0461704713180676)
(0.0590590590590591,-0.0477441048456776)
(0.0600600600600601,-0.0493430053268481)
(0.0610610610610611,-0.050967119030709)
(0.0620620620620621,-0.0526163922835811)
(0.0630630630630631,-0.0542907714689762)
(0.0640640640640641,-0.055990203027597)
(0.0650650650650651,-0.0577146334573372)
(0.0660660660660661,-0.0594640093132813)
(0.0670670670670671,-0.0612382772077048)
(0.0680680680680681,-0.0630373838100739)
(0.0690690690690691,-0.0648612758470461)
(0.0700700700700701,-0.0667099001024696)
(0.0710710710710711,-0.0685832034173836)
(0.0720720720720721,-0.070481132690018)
(0.0730730730730731,-0.0724036348757939)
(0.0740740740740741,-0.0743506569873231)
(0.0750750750750751,-0.0763221460944086)
(0.0760760760760761,-0.0783180493240441)
(0.0770770770770771,-0.0803383138604142)
(0.0780780780780781,-0.0823828869448946)
(0.0790790790790791,-0.0844517158760517)
(0.0800800800800801,-0.0865447480096429)
(0.0810810810810811,-0.0886619307586168)
(0.0820820820820821,-0.0908032115931124)
(0.0830830830830831,-0.0929685380404601)
(0.0840840840840841,-0.0951578576851809)
(0.0850850850850851,-0.0973711181689868)
(0.0860860860860861,-0.0996082671907809)
(0.0870870870870871,-0.101869252506657)
(0.0880880880880881,-0.1041540219299)
(0.0890890890890891,-0.106462523330985)
(0.0900900900900901,-0.10879470463758)
(0.0910910910910911,-0.111150513834541)
(0.0920920920920921,-0.113529898963918)
(0.0930930930930931,-0.115932808124949)
(0.0940940940940941,-0.118359189474065)
(0.0950950950950951,-0.120808991224888)
(0.0960960960960961,-0.123282161648228)
(0.0970970970970971,-0.125778649072091)
(0.0980980980980981,-0.128298401881668)
(0.0990990990990991,-0.130841368519346)
(0.1001001001001,-0.1334074974847)
(0.101101101101101,-0.135996737334498)
(0.102102102102102,-0.138609036682696)
(0.103103103103103,-0.141244344200443)
(0.104104104104104,-0.143902608616079)
(0.105105105105105,-0.146583778715135)
(0.106106106106106,-0.149287803340331)
(0.107107107107107,-0.15201463139158)
(0.108108108108108,-0.154764211825985)
(0.109109109109109,-0.157536493657841)
(0.11011011011011,-0.160331425958632)
(0.111111111111111,-0.163148957857034)
(0.112112112112112,-0.165989038538915)
(0.113113113113113,-0.168851617247331)
(0.114114114114114,-0.171736643282533)
(0.115115115115115,-0.174644066001959)
(0.116116116116116,-0.17757383482024)
(0.117117117117117,-0.180525899209197)
(0.118118118118118,-0.183500208697844)
(0.119119119119119,-0.186496712872383)
(0.12012012012012,-0.189515361376209)
(0.121121121121121,-0.192556103909906)
(0.122122122122122,-0.195618890231251)
(0.123123123123123,-0.198703670155211)
(0.124124124124124,-0.201810393553944)
(0.125125125125125,-0.204939010356799)
(0.126126126126126,-0.208089470550315)
(0.127127127127127,-0.211261724178224)
(0.128128128128128,-0.214455721341446)
(0.129129129129129,-0.217671412198095)
(0.13013013013013,-0.220908746963474)
(0.131131131131131,-0.224167675910077)
(0.132132132132132,-0.227448149367589)
(0.133133133133133,-0.230750117722888)
(0.134134134134134,-0.234073531420039)
(0.135135135135135,-0.237418340960302)
(0.136136136136136,-0.240784496902124)
(0.137137137137137,-0.244171949861147)
(0.138138138138138,-0.2475806505102)
(0.139139139139139,-0.251010549579306)
(0.14014014014014,-0.254461597855678)
(0.141141141141141,-0.257933746183718)
(0.142142142142142,-0.261426945465021)
(0.143143143143143,-0.264941146658373)
(0.144144144144144,-0.268476300779751)
(0.145145145145145,-0.272032358902321)
(0.146146146146146,-0.275609272156442)
(0.147147147147147,-0.279206991729662)
(0.148148148148148,-0.282825468866723)
(0.149149149149149,-0.286464654869554)
(0.15015015015015,-0.290124501097278)
(0.151151151151151,-0.293804958966207)
(0.152152152152152,-0.297505979949846)
(0.153153153153153,-0.301227515578889)
(0.154154154154154,-0.304969517441221)
(0.155155155155155,-0.308731937181919)
(0.156156156156156,-0.31251472650325)
(0.157157157157157,-0.316317837164673)
(0.158158158158158,-0.320141220982838)
(0.159159159159159,-0.323984829831583)
(0.16016016016016,-0.32784861564194)
(0.161161161161161,-0.331732530402132)
(0.162162162162162,-0.335636526157571)
(0.163163163163163,-0.339560555010861)
(0.164164164164164,-0.343504569121797)
(0.165165165165165,-0.347468520707365)
(0.166166166166166,-0.35145236204174)
(0.167167167167167,-0.355456045456292)
(0.168168168168168,-0.359479523339577)
(0.169169169169169,-0.363522748137346)
(0.17017017017017,-0.367585672352539)
(0.171171171171171,-0.371668248545286)
(0.172172172172172,-0.375770429332912)
(0.173173173173173,-0.379892167389927)
(0.174174174174174,-0.384033415448036)
(0.175175175175175,-0.388194126296135)
(0.176176176176176,-0.392374252780309)
(0.177177177177177,-0.396573747803835)
(0.178178178178178,-0.40079256432718)
(0.179179179179179,-0.405030655368003)
(0.18018018018018,-0.409287974001153)
(0.181181181181181,-0.413564473358672)
(0.182182182182182,-0.417860106629789)
(0.183183183183183,-0.422174827060929)
(0.184184184184184,-0.426508587955703)
(0.185185185185185,-0.430861342674916)
(0.186186186186186,-0.435233044636563)
(0.187187187187187,-0.43962364731583)
(0.188188188188188,-0.444033104245093)
(0.189189189189189,-0.448461369013921)
(0.19019019019019,-0.452908395269073)
(0.191191191191191,-0.457374136714497)
(0.192192192192192,-0.461858547111335)
(0.193193193193193,-0.466361580277918)
(0.194194194194194,-0.470883190089768)
(0.195195195195195,-0.475423330479599)
(0.196196196196196,-0.479981955437314)
(0.197197197197197,-0.48455901901001)
(0.198198198198198,-0.489154475301972)
(0.199199199199199,-0.493768278474678)
(0.2002002002002,-0.498400382746794)
(0.201201201201201,-0.50305074239418)
(0.202202202202202,-0.507719311749886)
(0.203203203203203,-0.512406045204152)
(0.204204204204204,-0.517110897204411)
(0.205205205205205,-0.521833822255284)
(0.206206206206206,-0.526574774918585)
(0.207207207207207,-0.531333709813318)
(0.208208208208208,-0.536110581615679)
(0.209209209209209,-0.540905345059054)
(0.21021021021021,-0.54571795493402)
(0.211211211211211,-0.550548366088345)
(0.212212212212212,-0.555396533426988)
(0.213213213213213,-0.560262411912099)
(0.214214214214214,-0.565145956563019)
(0.215215215215215,-0.570047122456279)
(0.216216216216216,-0.574965864725603)
(0.217217217217217,-0.579902138561903)
(0.218218218218218,-0.584855899213284)
(0.219219219219219,-0.589827101985042)
(0.22022022022022,-0.594815702239663)
(0.221221221221221,-0.599821655396825)
(0.222222222222222,-0.604844916933394)
(0.223223223223223,-0.609885442383432)
(0.224224224224224,-0.614943187338186)
(0.225225225225225,-0.6200181074461)
(0.226226226226226,-0.625110158412804)
(0.227227227227227,-0.630219296001121)
(0.228228228228228,-0.635345476031064)
(0.229229229229229,-0.64048865437984)
(0.23023023023023,-0.645648786981842)
(0.231231231231231,-0.650825829828658)
(0.232232232232232,-0.656019738969065)
(0.233233233233233,-0.661230470509031)
(0.234234234234234,-0.666457980611716)
(0.235235235235235,-0.671702225497469)
(0.236236236236236,-0.676963161443832)
(0.237237237237237,-0.682240744785537)
(0.238238238238238,-0.687534931914506)
(0.239239239239239,-0.692845679279854)
(0.24024024024024,-0.698172943387885)
(0.241241241241241,-0.703516680802094)
(0.242242242242242,-0.70887684814317)
(0.243243243243243,-0.714253402088988)
(0.244244244244244,-0.719646299374618)
(0.245245245245245,-0.725055496792318)
(0.246246246246246,-0.73048095119154)
(0.247247247247247,-0.735922619478924)
(0.248248248248248,-0.741380458618302)
(0.249249249249249,-0.746854425630697)
(0.25025025025025,-0.752344477594325)
(0.251251251251251,-0.757850571644588)
(0.252252252252252,-0.763372664974084)
(0.253253253253253,-0.768910714832598)
(0.254254254254254,-0.774464678527109)
(0.255255255255255,-0.780034513421785)
(0.256256256256256,-0.785620176937985)
(0.257257257257257,-0.791221626554261)
(0.258258258258258,-0.796838819806353)
(0.259259259259259,-0.802471714287193)
(0.26026026026026,-0.808120267646905)
(0.261261261261261,-0.813784437592802)
(0.262262262262262,-0.819464181889391)
(0.263263263263263,-0.825159458358366)
(0.264264264264264,-0.830870224878615)
(0.265265265265265,-0.836596439386216)
(0.266266266266266,-0.842338059874436)
(0.267267267267267,-0.848095044393736)
(0.268268268268268,-0.853867351051766)
(0.269269269269269,-0.859654938013368)
(0.27027027027027,-0.865457763500574)
(0.271271271271271,-0.871275785792608)
(0.272272272272272,-0.877108963225883)
(0.273273273273273,-0.882957254194005)
(0.274274274274274,-0.888820617147769)
(0.275275275275275,-0.894699010595164)
(0.276276276276276,-0.900592393101366)
(0.277277277277277,-0.906500723288745)
(0.278278278278278,-0.912423959836859)
(0.279279279279279,-0.918362061482461)
(0.28028028028028,-0.924314987019491)
(0.281281281281281,-0.930282695299083)
(0.282282282282282,-0.936265145229559)
(0.283283283283283,-0.942262295776433)
(0.284284284284284,-0.948274105962412)
(0.285285285285285,-0.954300534867391)
(0.286286286286286,-0.960341541628457)
(0.287287287287287,-0.966397085439889)
(0.288288288288288,-0.972467125553155)
(0.289289289289289,-0.978551621276915)
(0.29029029029029,-0.984650531977021)
(0.291291291291291,-0.990763817076513)
(0.292292292292292,-0.996891436055625)
(0.293293293293293,-1.00303334845178)
(0.294294294294294,-1.00918951385959)
(0.295295295295295,-1.01535989193087)
(0.296296296296296,-1.0215444423746)
(0.297297297297297,-1.02774312495698)
(0.298298298298298,-1.03395589950139)
(0.299299299299299,-1.04018272588838)
(0.3003003003003,-1.04642356405574)
(0.301301301301301,-1.05267837399839)
(0.302302302302302,-1.0589471157685)
(0.303303303303303,-1.06522974947538)
(0.304304304304304,-1.07152623528556)
(0.305305305305305,-1.07783653342277)
(0.306306306306306,-1.08416060416789)
(0.307307307307307,-1.09049840785904)
(0.308308308308308,-1.09684990489149)
(0.309309309309309,-1.10321505571773)
(0.31031031031031,-1.10959382084742)
(0.311311311311311,-1.11598616084742)
(0.312312312312312,-1.12239203634179)
(0.313313313313313,-1.12881140801177)
(0.314314314314314,-1.13524423659579)
(0.315315315315315,-1.14169048288948)
(0.316316316316316,-1.14815010774564)
(0.317317317317317,-1.15462307207429)
(0.318318318318318,-1.16110933684261)
(0.319319319319319,-1.16760886307501)
(0.32032032032032,-1.17412161185305)
(0.321321321321321,-1.18064754431551)
(0.322322322322322,-1.18718662165835)
(0.323323323323323,-1.19373880513471)
(0.324324324324324,-1.20030405605495)
(0.325325325325325,-1.20688233578659)
(0.326326326326326,-1.21347360575435)
(0.327327327327327,-1.22007782744016)
(0.328328328328328,-1.22669496238311)
(0.329329329329329,-1.23332497217951)
(0.33033033033033,-1.23996781848284)
(0.331331331331331,-1.24662346300378)
(0.332332332332332,-1.2532918675102)
(0.333333333333333,-1.25997299382716)
(0.334334334334334,-1.26666680383691)
(0.335335335335335,-1.2733732594789)
(0.336336336336336,-1.28009232274975)
(0.337337337337337,-1.2868239557033)
(0.338338338338338,-1.29356812045055)
(0.339339339339339,-1.30032477915971)
(0.34034034034034,-1.30709389405618)
(0.341341341341341,-1.31387542742255)
(0.342342342342342,-1.32066934159859)
(0.343343343343343,-1.32747559898128)
(0.344344344344344,-1.33429416202477)
(0.345345345345345,-1.34112499324042)
(0.346346346346346,-1.34796805519677)
(0.347347347347347,-1.35482331051955)
(0.348348348348348,-1.36169072189169)
(0.349349349349349,-1.3685702520533)
(0.35035035035035,-1.37546186380168)
(0.351351351351351,-1.38236551999135)
(0.352352352352352,-1.38928118353397)
(0.353353353353353,-1.39620881739844)
(0.354354354354354,-1.40314838461081)
(0.355355355355355,-1.41009984825436)
(0.356356356356356,-1.41706317146953)
(0.357357357357357,-1.42403831745397)
(0.358358358358358,-1.43102524946251)
(0.359359359359359,-1.43802393080717)
(0.36036036036036,-1.44503432485717)
(0.361361361361361,-1.45205639503892)
(0.362362362362362,-1.459090104836)
(0.363363363363363,-1.46613541778922)
(0.364364364364364,-1.47319229749655)
(0.365365365365365,-1.48026070761315)
(0.366366366366366,-1.4873406118514)
(0.367367367367367,-1.49443197398083)
(0.368368368368368,-1.5015347578282)
(0.369369369369369,-1.50864892727744)
(0.37037037037037,-1.51577444626967)
(0.371371371371371,-1.52291127880321)
(0.372372372372372,-1.53005938893356)
(0.373373373373373,-1.53721874077342)
(0.374374374374374,-1.54438929849268)
(0.375375375375375,-1.55157102631842)
(0.376376376376376,-1.55876388853491)
(0.377377377377377,-1.5659678494836)
(0.378378378378378,-1.57318287356316)
(0.379379379379379,-1.58040892522942)
(0.38038038038038,-1.58764596899541)
(0.381381381381381,-1.59489396943137)
(0.382382382382382,-1.60215289116471)
(0.383383383383383,-1.60942269888003)
(0.384384384384384,-1.61670335731913)
(0.385385385385385,-1.623994831281)
(0.386386386386386,-1.63129708562182)
(0.387387387387387,-1.63861008525495)
(0.388388388388388,-1.64593379515097)
(0.389389389389389,-1.65326818033762)
(0.39039039039039,-1.66061320589984)
(0.391391391391391,-1.66796883697977)
(0.392392392392392,-1.67533503877674)
(0.393393393393393,-1.68271177654725)
(0.394394394394394,-1.69009901560502)
(0.395395395395395,-1.69749672132094)
(0.396396396396396,-1.7049048591231)
(0.397397397397397,-1.71232339449678)
(0.398398398398398,-1.71975229298446)
(0.399399399399399,-1.72719152018578)
(0.4004004004004,-1.73464104175761)
(0.401401401401401,-1.74210082341398)
(0.402402402402402,-1.74957083092614)
(0.403403403403403,-1.7570510301225)
(0.404404404404404,-1.76454138688868)
(0.405405405405405,-1.77204186716749)
(0.406406406406406,-1.77955243695893)
(0.407407407407407,-1.78707306232018)
(0.408408408408408,-1.79460370936563)
(0.409409409409409,-1.80214434426683)
(0.41041041041041,-1.80969493325257)
(0.411411411411411,-1.81725544260878)
(0.412412412412412,-1.82482583867861)
(0.413413413413413,-1.8324060878624)
(0.414414414414414,-1.83999615661766)
(0.415415415415415,-1.84759601145912)
(0.416416416416416,-1.85520561895869)
(0.417417417417417,-1.86282494574545)
(0.418418418418418,-1.8704539585057)
(0.419419419419419,-1.87809262398293)
(0.42042042042042,-1.88574090897779)
(0.421421421421421,-1.89339878034814)
(0.422422422422422,-1.90106620500906)
(0.423423423423423,-1.90874314993277)
(0.424424424424424,-1.9164295821487)
(0.425425425425425,-1.9241254687435)
(0.426426426426426,-1.93183077686096)
(0.427427427427427,-1.9395454737021)
(0.428428428428428,-1.94726952652512)
(0.429429429429429,-1.9550029026454)
(0.43043043043043,-1.96274556943553)
(0.431431431431431,-1.97049749432527)
(0.432432432432432,-1.97825864480159)
(0.433433433433433,-1.98602898840864)
(0.434434434434434,-1.99380849274776)
(0.435435435435435,-2.00159712547749)
(0.436436436436436,-2.00939485431355)
(0.437437437437437,-2.01720164702886)
(0.438438438438438,-2.02501747145352)
(0.439439439439439,-2.03284229547483)
(0.44044044044044,-2.04067608703729)
(0.441441441441441,-2.04851881414257)
(0.442442442442442,-2.05637044484954)
(0.443443443443443,-2.06423094727427)
(0.444444444444444,-2.07210028959)
(0.445445445445445,-2.07997844002719)
(0.446446446446446,-2.08786536687345)
(0.447447447447447,-2.09576103847363)
(0.448448448448448,-2.10366542322974)
(0.449449449449449,-2.11157848960098)
(0.45045045045045,-2.11950020610375)
(0.451451451451451,-2.12743054131164)
(0.452452452452452,-2.13536946385543)
(0.453453453453453,-2.14331694242309)
(0.454454454454454,-2.15127294575979)
(0.455455455455455,-2.15923744266788)
(0.456456456456456,-2.16721040200689)
(0.457457457457457,-2.17519179269357)
(0.458458458458458,-2.18318158370184)
(0.459459459459459,-2.19117974406281)
(0.46046046046046,-2.1991862428648)
(0.461461461461461,-2.2072010492533)
(0.462462462462462,-2.215224132431)
(0.463463463463463,-2.22325546165778)
(0.464464464464464,-2.23129500625071)
(0.465465465465465,-2.23934273558406)
(0.466466466466466,-2.24739861908928)
(0.467467467467467,-2.255462626255)
(0.468468468468468,-2.26353472662707)
(0.469469469469469,-2.27161488980852)
(0.47047047047047,-2.27970308545955)
(0.471471471471471,-2.28779928329757)
(0.472472472472472,-2.29590345309719)
(0.473473473473473,-2.30401556469019)
(0.474474474474474,-2.31213558796555)
(0.475475475475475,-2.32026349286945)
(0.476476476476476,-2.32839924940524)
(0.477477477477477,-2.33654282763349)
(0.478478478478478,-2.34469419767192)
(0.479479479479479,-2.35285332969548)
(0.48048048048048,-2.3610201939363)
(0.481481481481481,-2.36919476068369)
(0.482482482482482,-2.37737700028415)
(0.483483483483483,-2.38556688314139)
(0.484484484484485,-2.3937643797163)
(0.485485485485485,-2.40196946052695)
(0.486486486486486,-2.41018209614862)
(0.487487487487487,-2.41840225721377)
(0.488488488488488,-2.42662991441205)
(0.48948948948949,-2.43486503849031)
(0.49049049049049,-2.44310760025258)
(0.491491491491491,-2.45135757056009)
(0.492492492492492,-2.45961492033125)
(0.493493493493493,-2.46787962054168)
(0.494494494494495,-2.47615164222416)
(0.495495495495495,-2.4844309564687)
(0.496496496496497,-2.49271753442246)
(0.497497497497497,-2.50101134728983)
(0.498498498498498,-2.50931236633236)
(0.4994994994995,-2.51762056286881)
(0.500500500500501,-2.52593590827512)
(0.501501501501502,-2.53425837398443)
(0.502502502502503,-2.54258793148706)
(0.503503503503503,-2.55092455233053)
(0.504504504504504,-2.55926820811954)
(0.505505505505506,-2.56761887051601)
(0.506506506506507,-2.57597651123901)
(0.507507507507508,-2.58434110206482)
(0.508508508508508,-2.59271261482693)
(0.509509509509509,-2.60109102141598)
(0.510510510510511,-2.60947629377984)
(0.511511511511512,-2.61786840392355)
(0.512512512512513,-2.62626732390934)
(0.513513513513513,-2.63467302585664)
(0.514514514514514,-2.64308548194206)
(0.515515515515516,-2.65150466439942)
(0.516516516516517,-2.65993054551972)
(0.517517517517518,-2.66836309765114)
(0.518518518518518,-2.67680229319906)
(0.519519519519519,-2.68524810462605)
(0.520520520520521,-2.69370050445189)
(0.521521521521522,-2.70215946525352)
(0.522522522522523,-2.71062495966508)
(0.523523523523523,-2.71909696037791)
(0.524524524524524,-2.72757544014054)
(0.525525525525526,-2.73606037175869)
(0.526526526526527,-2.74455172809526)
(0.527527527527528,-2.75304948207035)
(0.528528528528528,-2.76155360666126)
(0.529529529529529,-2.77006407490245)
(0.530530530530531,-2.77858085988562)
(0.531531531531532,-2.7871039347596)
(0.532532532532533,-2.79563327273047)
(0.533533533533533,-2.80416884706147)
(0.534534534534534,-2.81271063107302)
(0.535535535535536,-2.82125859814277)
(0.536536536536537,-2.82981272170551)
(0.537537537537538,-2.83837297525327)
(0.538538538538539,-2.84693933233524)
(0.539539539539539,-2.85551176655782)
(0.540540540540541,-2.86409025158458)
(0.541541541541542,-2.87267476113629)
(0.542542542542543,-2.88126526899092)
(0.543543543543544,-2.88986174898362)
(0.544544544544544,-2.89846417500673)
(0.545545545545546,-2.9070725210098)
(0.546546546546547,-2.91568676099955)
(0.547547547547548,-2.92430686903989)
(0.548548548548549,-2.93293281925194)
(0.54954954954955,-2.94156458581399)
(0.550550550550551,-2.95020214296153)
(0.551551551551552,-2.95884546498725)
(0.552552552552553,-2.96749452624102)
(0.553553553553554,-2.97614930112989)
(0.554554554554555,-2.98480976411813)
(0.555555555555556,-2.99347588972718)
(0.556556556556557,-3.00214765253567)
(0.557557557557558,-3.01082502717943)
(0.558558558558559,-3.01950798835147)
(0.55955955955956,-3.02819651080202)
(0.560560560560561,-3.03689056933846)
(0.561561561561562,-3.04559013882539)
(0.562562562562563,-3.05429519418459)
(0.563563563563564,-3.06300571039502)
(0.564564564564565,-3.07172166249286)
(0.565565565565566,-3.08044302557146)
(0.566566566566567,-3.08916977478135)
(0.567567567567568,-3.09790188533029)
(0.568568568568569,-3.10663933248319)
(0.56956956956957,-3.11538209156217)
(0.570570570570571,-3.12413013794654)
(0.571571571571572,-3.1328834470728)
(0.572572572572573,-3.14164199443465)
(0.573573573573574,-3.15040575558295)
(0.574574574574575,-3.15917470612579)
(0.575575575575576,-3.16794882172843)
(0.576576576576577,-3.17672807811332)
(0.577577577577578,-3.18551245106011)
(0.578578578578579,-3.19430191640562)
(0.57957957957958,-3.2030964500439)
(0.580580580580581,-3.21189602792616)
(0.581581581581582,-3.2207006260608)
(0.582582582582583,-3.22951022051343)
(0.583583583583584,-3.23832478740683)
(0.584584584584585,-3.247144302921)
(0.585585585585586,-3.25596874329309)
(0.586586586586587,-3.26479808481748)
(0.587587587587588,-3.27363230384571)
(0.588588588588589,-3.28247137678654)
(0.58958958958959,-3.29131528010589)
(0.590590590590591,-3.30016399032691)
(0.591591591591592,-3.3090174840299)
(0.592592592592593,-3.31787573785237)
(0.593593593593594,-3.32673872848902)
(0.594594594594595,-3.33560643269175)
(0.595595595595596,-3.34447882726963)
(0.596596596596597,-3.35335588908894)
(0.597597597597598,-3.36223759507314)
(0.598598598598599,-3.37112392220289)
(0.5995995995996,-3.38001484751603)
(0.600600600600601,-3.38891034810759)
(0.601601601601602,-3.39781040112981)
(0.602602602602603,-3.40671498379211)
(0.603603603603604,-3.41562407336108)
(0.604604604604605,-3.42453764716054)
(0.605605605605606,-3.43345568257146)
(0.606606606606607,-3.44237815703204)
(0.607607607607608,-3.45130504803765)
(0.608608608608609,-3.46023633314085)
(0.60960960960961,-3.46917198995139)
(0.610610610610611,-3.47811199613623)
(0.611611611611612,-3.48705632941948)
(0.612612612612613,-3.4960049675825)
(0.613613613613614,-3.50495788846378)
(0.614614614614615,-3.51391506995905)
(0.615615615615616,-3.5228764900212)
(0.616616616616617,-3.53184212666032)
(0.617617617617618,-3.54081195794369)
(0.618618618618619,-3.54978596199579)
(0.61961961961962,-3.55876411699828)
(0.620620620620621,-3.56774640119002)
(0.621621621621622,-3.57673279286705)
(0.622622622622623,-3.5857232703826)
(0.623623623623624,-3.59471781214711)
(0.624624624624625,-3.60371639662819)
(0.625625625625626,-3.61271900235066)
(0.626626626626627,-3.62172560789651)
(0.627627627627628,-3.63073619190493)
(0.628628628628629,-3.6397507330723)
(0.62962962962963,-3.64876921015221)
(0.630630630630631,-3.65779160195541)
(0.631631631631632,-3.66681788734985)
(0.632632632632633,-3.67584804526069)
(0.633633633633634,-3.68488205467025)
(0.634634634634635,-3.69391989461808)
(0.635635635635636,-3.70296154420088)
(0.636636636636637,-3.71200698257256)
(0.637637637637638,-3.72105618894423)
(0.638638638638639,-3.73010914258418)
(0.63963963963964,-3.73916582281788)
(0.640640640640641,-3.74822620902801)
(0.641641641641642,-3.75729028065445)
(0.642642642642643,-3.76635801719423)
(0.643643643643644,-3.7754293982016)
(0.644644644644645,-3.78450440328801)
(0.645645645645646,-3.79358301212209)
(0.646646646646647,-3.80266520442964)
(0.647647647647648,-3.81175095999368)
(0.648648648648649,-3.82084025865441)
(0.64964964964965,-3.82993308030921)
(0.650650650650651,-3.83902940491269)
(0.651651651651652,-3.8481292124766)
(0.652652652652653,-3.8572324830699)
(0.653653653653654,-3.86633919681877)
(0.654654654654655,-3.87544933390653)
(0.655655655655656,-3.88456287457373)
(0.656656656656657,-3.8936797991181)
(0.657657657657658,-3.90280008789456)
(0.658658658658659,-3.9119237213152)
(0.65965965965966,-3.92105067984934)
(0.660660660660661,-3.93018094402347)
(0.661661661661662,-3.93931449442127)
(0.662662662662663,-3.94845131168361)
(0.663663663663664,-3.95759137650855)
(0.664664664664665,-3.96673466965136)
(0.665665665665666,-3.97588117192447)
(0.666666666666667,-3.98503086419753)
(0.667667667667668,-3.99418372739736)
(0.668668668668669,-4.00333974250798)
(0.66966966966967,-4.0124988905706)
(0.670670670670671,-4.02166115268363)
(0.671671671671672,-4.03082651000265)
(0.672672672672673,-4.03999494374044)
(0.673673673673674,-4.04916643516698)
(0.674674674674675,-4.05834096560944)
(0.675675675675676,-4.06751851645216)
(0.676676676676677,-4.0766990691367)
(0.677677677677678,-4.08588260516179)
(0.678678678678679,-4.09506910608337)
(0.67967967967968,-4.10425855351454)
(0.680680680680681,-4.11345092912562)
(0.681681681681682,-4.12264621464411)
(0.682682682682683,-4.1318443918547)
(0.683683683683684,-4.14104544259928)
(0.684684684684685,-4.15024934877691)
(0.685685685685686,-4.15945609234387)
(0.686686686686687,-4.1686656553136)
(0.687687687687688,-4.17787801975676)
(0.688688688688689,-4.18709316780118)
(0.68968968968969,-4.19631108163189)
(0.690690690690691,-4.20553174349111)
(0.691691691691692,-4.21475513567825)
(0.692692692692693,-4.22398124054991)
(0.693693693693694,-4.23321004051988)
(0.694694694694695,-4.24244151805914)
(0.695695695695696,-4.25167565569587)
(0.696696696696697,-4.26091243601543)
(0.697697697697698,-4.27015184166039)
(0.698698698698699,-4.27939385533047)
(0.6996996996997,-4.28863845978263)
(0.700700700700701,-4.29788563783099)
(0.701701701701702,-4.30713537234687)
(0.702702702702703,-4.31638764625878)
(0.703703703703704,-4.32564244255242)
(0.704704704704705,-4.33489974427068)
(0.705705705705706,-4.34415953451365)
(0.706706706706707,-4.3534217964386)
(0.707707707707708,-4.36268651325999)
(0.708708708708709,-4.37195366824948)
(0.70970970970971,-4.38122324473592)
(0.710710710710711,-4.39049522610534)
(0.711711711711712,-4.39976959580098)
(0.712712712712713,-4.40904633732324)
(0.713713713713714,-4.41832543422975)
(0.714714714714715,-4.42760687013531)
(0.715715715715716,-4.4368906287119)
(0.716716716716717,-4.4461766936887)
(0.717717717717718,-4.4554650488521)
(0.718718718718719,-4.46475567804566)
(0.71971971971972,-4.47404856517013)
(0.720720720720721,-4.48334369418346)
(0.721721721721722,-4.49264104910079)
(0.722722722722723,-4.50194061399444)
(0.723723723723724,-4.51124237299394)
(0.724724724724725,-4.520546310286)
(0.725725725725726,-4.52985241011451)
(0.726726726726727,-4.53916065678057)
(0.727727727727728,-4.54847103464247)
(0.728728728728729,-4.55778352811567)
(0.72972972972973,-4.56709812167284)
(0.730730730730731,-4.57641479984385)
(0.731731731731732,-4.58573354721572)
(0.732732732732733,-4.59505434843271)
(0.733733733733734,-4.60437718819624)
(0.734734734734735,-4.61370205126493)
(0.735735735735736,-4.6230289224546)
(0.736736736736737,-4.63235778663823)
(0.737737737737738,-4.64168862874603)
(0.738738738738739,-4.65102143376538)
(0.73973973973974,-4.66035618674085)
(0.740740740740741,-4.66969287277421)
(0.741741741741742,-4.67903147702442)
(0.742742742742743,-4.68837198470761)
(0.743743743743744,-4.69771438109713)
(0.744744744744745,-4.70705865152351)
(0.745745745745746,-4.71640478137447)
(0.746746746746747,-4.72575275609491)
(0.747747747747748,-4.73510256118695)
(0.748748748748749,-4.74445418220987)
(0.74974974974975,-4.75380760478015)
(0.750750750750751,-4.76316281457147)
(0.751751751751752,-4.7725197973147)
(0.752752752752753,-4.78187853879789)
(0.753753753753754,-4.79123902486629)
(0.754754754754755,-4.80060124142233)
(0.755755755755756,-4.80996517442565)
(0.756756756756757,-4.81933080989307)
(0.757757757757758,-4.82869813389859)
(0.758758758758759,-4.83806713257342)
(0.75975975975976,-4.84743779210594)
(0.760760760760761,-4.85681009874176)
(0.761761761761762,-4.86618403878363)
(0.762762762762763,-4.87555959859152)
(0.763763763763764,-4.8849367645826)
(0.764764764764765,-4.8943155232312)
(0.765765765765766,-4.90369586106887)
(0.766766766766767,-4.91307776468433)
(0.767767767767768,-4.92246122072351)
(0.768768768768769,-4.93184621588951)
(0.76976976976977,-4.94123273694265)
(0.770770770770771,-4.9506207707004)
(0.771771771771772,-4.96001030403745)
(0.772772772772773,-4.96940132388568)
(0.773773773773774,-4.97879381723416)
(0.774774774774775,-4.98818777112914)
(0.775775775775776,-4.99758317267406)
(0.776776776776777,-5.00698000902958)
(0.777777777777778,-5.0163782674135)
(0.778778778778779,-5.02577793510087)
(0.77977977977978,-5.03517899942388)
(0.780780780780781,-5.04458144777193)
(0.781781781781782,-5.05398526759163)
(0.782782782782783,-5.06339044638676)
(0.783783783783784,-5.07279697171828)
(0.784784784784785,-5.08220483120437)
(0.785785785785786,-5.09161401252038)
(0.786786786786787,-5.10102450339886)
(0.787787787787788,-5.11043629162955)
(0.788788788788789,-5.11984936505937)
(0.78978978978979,-5.12926371159246)
(0.790790790790791,-5.13867931919011)
(0.791791791791792,-5.14809617587084)
(0.792792792792793,-5.15751426971032)
(0.793793793793794,-5.16693358884146)
(0.794794794794795,-5.17635412145432)
(0.795795795795796,-5.18577585579616)
(0.796796796796797,-5.19519878017145)
(0.797797797797798,-5.20462288294184)
(0.798798798798799,-5.21404815252616)
(0.7997997997998,-5.22347457740043)
(0.800800800800801,-5.23290214609789)
(0.801801801801802,-5.24233084720894)
(0.802802802802803,-5.25176066938119)
(0.803803803803804,-5.26119160131942)
(0.804804804804805,-5.27062363178562)
(0.805805805805806,-5.28005674959898)
(0.806806806806807,-5.28949094363584)
(0.807807807807808,-5.29892620282978)
(0.808808808808809,-5.30836251617153)
(0.80980980980981,-5.31779987270904)
(0.810810810810811,-5.32723826154743)
(0.811811811811812,-5.33667767184903)
(0.812812812812813,-5.34611809283335)
(0.813813813813814,-5.35555951377709)
(0.814814814814815,-5.36500192401414)
(0.815815815815816,-5.37444531293559)
(0.816816816816817,-5.38388966998971)
(0.817817817817818,-5.39333498468197)
(0.818818818818819,-5.40278124657503)
(0.81981981981982,-5.41222844528873)
(0.820820820820821,-5.42167657050011)
(0.821821821821822,-5.43112561194341)
(0.822822822822823,-5.44057555941003)
(0.823823823823824,-5.4500264027486)
(0.824824824824825,-5.45947813186492)
(0.825825825825826,-5.46893073672198)
(0.826826826826827,-5.47838420733996)
(0.827827827827828,-5.48783853379624)
(0.828828828828829,-5.49729370622539)
(0.82982982982983,-5.50674971481916)
(0.830830830830831,-5.5162065498265)
(0.831831831831832,-5.52566420155356)
(0.832832832832833,-5.53512266036365)
(0.833833833833834,-5.5445819166773)
(0.834834834834835,-5.55404196097222)
(0.835835835835836,-5.56350278378332)
(0.836836836836837,-5.57296437570269)
(0.837837837837838,-5.58242672737961)
(0.838838838838839,-5.59188982952056)
(0.83983983983984,-5.6013536728892)
(0.840840840840841,-5.6108182483064)
(0.841841841841842,-5.62028354665019)
(0.842842842842843,-5.62974955885583)
(0.843843843843844,-5.63921627591573)
(0.844844844844845,-5.64868368887952)
(0.845845845845846,-5.65815178885401)
(0.846846846846847,-5.66762056700321)
(0.847847847847848,-5.67709001454831)
(0.848848848848849,-5.68656012276769)
(0.84984984984985,-5.69603088299692)
(0.850850850850851,-5.70550228662879)
(0.851851851851852,-5.71497432511323)
(0.852852852852853,-5.7244469899574)
(0.853853853853854,-5.73392027272565)
(0.854854854854855,-5.74339416503949)
(0.855855855855856,-5.75286865857765)
(0.856856856856857,-5.76234374507605)
(0.857857857857858,-5.77181941632778)
(0.858858858858859,-5.78129566418314)
(0.85985985985986,-5.79077248054961)
(0.860860860860861,-5.80024985739187)
(0.861861861861862,-5.80972778673179)
(0.862862862862863,-5.81920626064842)
(0.863863863863864,-5.82868527127802)
(0.864864864864865,-5.83816481081401)
(0.865865865865866,-5.84764487150704)
(0.866866866866867,-5.85712544566492)
(0.867867867867868,-5.86660652565266)
(0.868868868868869,-5.87608810389247)
(0.86986986986987,-5.88557017286375)
(0.870870870870871,-5.89505272510306)
(0.871871871871872,-5.90453575320421)
(0.872872872872873,-5.91401924981814)
(0.873873873873874,-5.92350320765301)
(0.874874874874875,-5.93298761947418)
(0.875875875875876,-5.94247247810419)
(0.876876876876877,-5.95195777642276)
(0.877877877877878,-5.96144350736682)
(0.878878878878879,-5.97092966393047)
(0.87987987987988,-5.98041623916503)
(0.880880880880881,-5.98990322617898)
(0.881881881881882,-5.99939061813801)
(0.882882882882883,-6.008878408265)
(0.883883883883884,-6.01836658984001)
(0.884884884884885,-6.02785515620031)
(0.885885885885886,-6.03734410074033)
(0.886886886886887,-6.04683341691172)
(0.887887887887888,-6.05632309822331)
(0.888888888888889,-6.06581313824112)
(0.88988988988989,-6.07530353058837)
(0.890890890890891,-6.08479426894545)
(0.891891891891892,-6.09428534704996)
(0.892892892892893,-6.10377675869668)
(0.893893893893894,-6.11326849773761)
(0.894894894894895,-6.12276055808189)
(0.895895895895896,-6.13225293369588)
(0.896896896896897,-6.14174561860314)
(0.897897897897898,-6.15123860688441)
(0.898898898898899,-6.16073189267761)
(0.8998998998999,-6.17022547017787)
(0.900900900900901,-6.1797193336375)
(0.901901901901902,-6.189213477366)
(0.902902902902903,-6.19870789573006)
(0.903903903903904,-6.20820258315359)
(0.904904904904905,-6.21769753411764)
(0.905905905905906,-6.22719274316048)
(0.906906906906907,-6.23668820487758)
(0.907907907907908,-6.24618391392159)
(0.908908908908909,-6.25567986500233)
(0.90990990990991,-6.26517605288685)
(0.910910910910911,-6.27467247239937)
(0.911911911911912,-6.28416911842129)
(0.912912912912913,-6.29366598589122)
(0.913913913913914,-6.30316306980496)
(0.914914914914915,-6.31266036521549)
(0.915915915915916,-6.32215786723298)
(0.916916916916917,-6.33165557102481)
(0.917917917917918,-6.34115347181553)
(0.918918918918919,-6.3506515648869)
(0.91991991991992,-6.36014984557785)
(0.920920920920921,-6.3696483092845)
(0.921921921921922,-6.3791469514602)
(0.922922922922923,-6.38864576761544)
(0.923923923923924,-6.39814475331793)
(0.924924924924925,-6.40764390419256)
(0.925925925925926,-6.41714321592143)
(0.926926926926927,-6.4266426842438)
(0.927927927927928,-6.43614230495615)
(0.928928928928929,-6.44564207391213)
(0.92992992992993,-6.45514198702259)
(0.930930930930931,-6.46464204025557)
(0.931931931931932,-6.47414222963631)
(0.932932932932933,-6.48364255124722)
(0.933933933933934,-6.49314300122792)
(0.934934934934935,-6.5026435757752)
(0.935935935935936,-6.51214427114308)
(0.936936936936937,-6.52164508364273)
(0.937937937937938,-6.53114600964252)
(0.938938938938939,-6.54064704556803)
(0.93993993993994,-6.55014818790202)
(0.940940940940941,-6.55964943318443)
(0.941941941941942,-6.5691507780124)
(0.942942942942943,-6.57865221904026)
(0.943943943943944,-6.58815375297955)
(0.944944944944945,-6.59765537659896)
(0.945945945945946,-6.60715708672441)
(0.946946946946947,-6.61665888023898)
(0.947947947947948,-6.62616075408297)
(0.948948948948949,-6.63566270525385)
(0.94994994994995,-6.64516473080629)
(0.950950950950951,-6.65466682785214)
(0.951951951951952,-6.66416899356046)
(0.952952952952953,-6.67367122515749)
(0.953953953953954,-6.68317351992666)
(0.954954954954955,-6.69267587520858)
(0.955955955955956,-6.70217828840108)
(0.956956956956957,-6.71168075695916)
(0.957957957957958,-6.721183278395)
(0.958958958958959,-6.73068585027801)
(0.95995995995996,-6.74018847023475)
(0.960960960960961,-6.74969113594899)
(0.961961961961962,-6.7591938451617)
(0.962962962962963,-6.76869659567101)
(0.963963963963964,-6.77819938533228)
(0.964964964964965,-6.78770221205803)
(0.965965965965966,-6.79720507381798)
(0.966966966966967,-6.80670796863905)
(0.967967967967968,-6.81621089460535)
(0.968968968968969,-6.82571384985816)
(0.96996996996997,-6.83521683259598)
(0.970970970970971,-6.84471984107447)
(0.971971971971972,-6.85422287360652)
(0.972972972972973,-6.86372592856217)
(0.973973973973974,-6.87322900436867)
(0.974974974974975,-6.88273209951048)
(0.975975975975976,-6.89223521252921)
(0.976976976976977,-6.90173834202369)
(0.977977977977978,-6.91124148664994)
(0.978978978978979,-6.92074464512116)
(0.97997997997998,-6.93024781620774)
(0.980980980980981,-6.93975099873727)
(0.981981981981982,-6.94925419159453)
(0.982982982982983,-6.95875739372148)
(0.983983983983984,-6.96826060411729)
(0.984984984984985,-6.9777638218383)
(0.985985985985986,-6.98726704599806)
(0.986986986986987,-6.99677027576729)
(0.987987987987988,-7.00627351037392)
(0.988988988988989,-7.01577674910306)
(0.98998998998999,-7.02527999129702)
(0.990990990990991,-7.03478323635529)
(0.991991991991992,-7.04428648373456)
(0.992992992992993,-7.0537897329487)
(0.993993993993994,-7.06329298356879)
(0.994994994994995,-7.07279623522308)
(0.995995995995996,-7.08229948759702)
(0.996996996996997,-7.09180274043326)
(0.997997997997998,-7.10130599353162)
(0.998998998998999,-7.11080924674913)
(1,-7.1203125)

};
\path [draw=black, fill opacity=0] (axis cs:13,0)--(axis cs:13,0);

\path [draw=black, fill opacity=0] (axis cs:1,13)--(axis cs:1,13);

\path [draw=black, fill opacity=0] (axis cs:13,-8)--(axis cs:13,-8);

\path [draw=black, fill opacity=0] (axis cs:0,13)--(axis cs:0,13);

\end{axis}

\end{tikzpicture}
	\end{center}
	%\caption{Fractional position along a plate of length L}
	\label{fig:BeamDisp}
\end{figure}
\begin{figure}[h]
	\begin{center}
		\setlength\figureheight{8cm} 
		\setlength\figurewidth{0.8\textwidth} 
		% This file was created by matplotlib v0.1.0.
% Copyright (c) 2010--2014, Nico Schlömer <nico.schloemer@gmail.com>
% All rights reserved.
% 

\begin{tikzpicture}

\begin{axis}[
xlabel={Fractional position along plate of length L},
ylabel={Fiber stress, relative to loading [$\cdot10^3$]},
xmin=0, xmax=1,
ymin=-30, ymax=0,
axis on top,
width=\figurewidth,
height=\figureheight
]
\addplot [blue]
coordinates {
(0,-30)
(0.001001001001001,-29.9399700000301)
(0.002002002002002,-29.8800001202404)
(0.003003003003003,-29.8200903606309)
(0.004004004004004,-29.7602407212017)
(0.005005005005005,-29.7004512019527)
(0.00600600600600601,-29.640721802884)
(0.00700700700700701,-29.5810525239955)
(0.00800800800800801,-29.5214433652872)
(0.00900900900900901,-29.4618943267592)
(0.01001001001001,-29.4024054084114)
(0.011011011011011,-29.3429766102439)
(0.012012012012012,-29.2836079322566)
(0.013013013013013,-29.2242993744495)
(0.014014014014014,-29.1650509368227)
(0.015015015015015,-29.1058626193761)
(0.016016016016016,-29.0467344221098)
(0.017017017017017,-28.9876663450237)
(0.018018018018018,-28.9286583881178)
(0.019019019019019,-28.8697105513922)
(0.02002002002002,-28.8108228348469)
(0.021021021021021,-28.7519952384817)
(0.022022022022022,-28.6932277622968)
(0.023023023023023,-28.6345204062922)
(0.024024024024024,-28.5758731704678)
(0.025025025025025,-28.5172860548236)
(0.026026026026026,-28.4587590593597)
(0.027027027027027,-28.400292184076)
(0.028028028028028,-28.3418854289725)
(0.029029029029029,-28.2835387940493)
(0.03003003003003,-28.2252522793063)
(0.031031031031031,-28.1670258847436)
(0.032032032032032,-28.1088596103611)
(0.033033033033033,-28.0507534561589)
(0.034034034034034,-27.9927074221369)
(0.035035035035035,-27.9347215082951)
(0.036036036036036,-27.8767957146336)
(0.037037037037037,-27.8189300411523)
(0.038038038038038,-27.7611244878512)
(0.039039039039039,-27.7033790547304)
(0.04004004004004,-27.6456937417898)
(0.041041041041041,-27.5880685490295)
(0.042042042042042,-27.5305034764494)
(0.043043043043043,-27.4729985240496)
(0.044044044044044,-27.41555369183)
(0.045045045045045,-27.3581689797906)
(0.046046046046046,-27.3008443879315)
(0.047047047047047,-27.2435799162526)
(0.048048048048048,-27.1863755647539)
(0.049049049049049,-27.1292313334355)
(0.05005005005005,-27.0721472222974)
(0.0510510510510511,-27.0151232313394)
(0.0520520520520521,-26.9581593605618)
(0.0530530530530531,-26.9012556099643)
(0.0540540540540541,-26.8444119795471)
(0.0550550550550551,-26.7876284693101)
(0.0560560560560561,-26.7309050792534)
(0.0570570570570571,-26.6742418093769)
(0.0580580580580581,-26.6176386596807)
(0.0590590590590591,-26.5610956301647)
(0.0600600600600601,-26.5046127208289)
(0.0610610610610611,-26.4481899316734)
(0.0620620620620621,-26.3918272626981)
(0.0630630630630631,-26.3355247139031)
(0.0640640640640641,-26.2792822852883)
(0.0650650650650651,-26.2230999768537)
(0.0660660660660661,-26.1669777885994)
(0.0670670670670671,-26.1109157205253)
(0.0680680680680681,-26.0549137726315)
(0.0690690690690691,-25.9989719449179)
(0.0700700700700701,-25.9430902373845)
(0.0710710710710711,-25.8872686500314)
(0.0720720720720721,-25.8315071828585)
(0.0730730730730731,-25.7758058358659)
(0.0740740740740741,-25.7201646090535)
(0.0750750750750751,-25.6645835024213)
(0.0760760760760761,-25.6090625159694)
(0.0770770770770771,-25.5536016496977)
(0.0780780780780781,-25.4982009036063)
(0.0790790790790791,-25.4428602776951)
(0.0800800800800801,-25.3875797719642)
(0.0810810810810811,-25.3323593864134)
(0.0820820820820821,-25.277199121043)
(0.0830830830830831,-25.2220989758527)
(0.0840840840840841,-25.1670589508427)
(0.0850850850850851,-25.112079046013)
(0.0860860860860861,-25.0571592613635)
(0.0870870870870871,-25.0022995968942)
(0.0880880880880881,-24.9475000526052)
(0.0890890890890891,-24.8927606284964)
(0.0900900900900901,-24.8380813245678)
(0.0910910910910911,-24.7834621408195)
(0.0920920920920921,-24.7289030772514)
(0.0930930930930931,-24.6744041338636)
(0.0940940940940941,-24.619965310656)
(0.0950950950950951,-24.5655866076287)
(0.0960960960960961,-24.5112680247815)
(0.0970970970970971,-24.4570095621147)
(0.0980980980980981,-24.402811219628)
(0.0990990990990991,-24.3486729973216)
(0.1001001001001,-24.2945948951955)
(0.101101101101101,-24.2405769132496)
(0.102102102102102,-24.1866190514839)
(0.103103103103103,-24.1327213098985)
(0.104104104104104,-24.0788836884933)
(0.105105105105105,-24.0251061872683)
(0.106106106106106,-23.9713888062236)
(0.107107107107107,-23.9177315453592)
(0.108108108108108,-23.8641344046749)
(0.109109109109109,-23.810597384171)
(0.11011011011011,-23.7571204838472)
(0.111111111111111,-23.7037037037037)
(0.112112112112112,-23.6503470437404)
(0.113113113113113,-23.5970505039574)
(0.114114114114114,-23.5438140843546)
(0.115115115115115,-23.4906377849321)
(0.116116116116116,-23.4375216056898)
(0.117117117117117,-23.3844655466277)
(0.118118118118118,-23.3314696077459)
(0.119119119119119,-23.2785337890443)
(0.12012012012012,-23.225658090523)
(0.121121121121121,-23.1728425121818)
(0.122122122122122,-23.120087054021)
(0.123123123123123,-23.0673917160404)
(0.124124124124124,-23.01475649824)
(0.125125125125125,-22.9621814006198)
(0.126126126126126,-22.9096664231799)
(0.127127127127127,-22.8572115659203)
(0.128128128128128,-22.8048168288409)
(0.129129129129129,-22.7524822119417)
(0.13013013013013,-22.7002077152227)
(0.131131131131131,-22.647993338684)
(0.132132132132132,-22.5958390823256)
(0.133133133133133,-22.5437449461473)
(0.134134134134134,-22.4917109301494)
(0.135135135135135,-22.4397370343316)
(0.136136136136136,-22.3878232586941)
(0.137137137137137,-22.3359696032369)
(0.138138138138138,-22.2841760679599)
(0.139139139139139,-22.2324426528631)
(0.14014014014014,-22.1807693579465)
(0.141141141141141,-22.1291561832102)
(0.142142142142142,-22.0776031286542)
(0.143143143143143,-22.0261101942784)
(0.144144144144144,-21.9746773800828)
(0.145145145145145,-21.9233046860675)
(0.146146146146146,-21.8719921122324)
(0.147147147147147,-21.8207396585775)
(0.148148148148148,-21.7695473251029)
(0.149149149149149,-21.7184151118085)
(0.15015015015015,-21.6673430186944)
(0.151151151151151,-21.6163310457605)
(0.152152152152152,-21.5653791930068)
(0.153153153153153,-21.5144874604334)
(0.154154154154154,-21.4636558480402)
(0.155155155155155,-21.4128843558273)
(0.156156156156156,-21.3621729837946)
(0.157157157157157,-21.3115217319422)
(0.158158158158158,-21.2609306002699)
(0.159159159159159,-21.210399588778)
(0.16016016016016,-21.1599286974662)
(0.161161161161161,-21.1095179263347)
(0.162162162162162,-21.0591672753835)
(0.163163163163163,-21.0088767446125)
(0.164164164164164,-20.9586463340217)
(0.165165165165165,-20.9084760436112)
(0.166166166166166,-20.8583658733809)
(0.167167167167167,-20.8083158233308)
(0.168168168168168,-20.758325893461)
(0.169169169169169,-20.7083960837715)
(0.17017017017017,-20.6585263942621)
(0.171171171171171,-20.608716824933)
(0.172172172172172,-20.5589673757842)
(0.173173173173173,-20.5092780468156)
(0.174174174174174,-20.4596488380272)
(0.175175175175175,-20.4100797494191)
(0.176176176176176,-20.3605707809912)
(0.177177177177177,-20.3111219327436)
(0.178178178178178,-20.2617332046761)
(0.179179179179179,-20.212404596789)
(0.18018018018018,-20.1631361090821)
(0.181181181181181,-20.1139277415554)
(0.182182182182182,-20.0647794942089)
(0.183183183183183,-20.0156913670427)
(0.184184184184184,-19.9666633600567)
(0.185185185185185,-19.917695473251)
(0.186186186186186,-19.8687877066255)
(0.187187187187187,-19.8199400601803)
(0.188188188188188,-19.7711525339153)
(0.189189189189189,-19.7224251278305)
(0.19019019019019,-19.673757841926)
(0.191191191191191,-19.6251506762017)
(0.192192192192192,-19.5766036306577)
(0.193193193193193,-19.5281167052939)
(0.194194194194194,-19.4796899001103)
(0.195195195195195,-19.431323215107)
(0.196196196196196,-19.3830166502839)
(0.197197197197197,-19.3347702056411)
(0.198198198198198,-19.2865838811785)
(0.199199199199199,-19.2384576768961)
(0.2002002002002,-19.190391592794)
(0.201201201201201,-19.1423856288721)
(0.202202202202202,-19.0944397851305)
(0.203203203203203,-19.0465540615691)
(0.204204204204204,-18.9987284581879)
(0.205205205205205,-18.950962974987)
(0.206206206206206,-18.9032576119663)
(0.207207207207207,-18.8556123691259)
(0.208208208208208,-18.8080272464657)
(0.209209209209209,-18.7605022439857)
(0.21021021021021,-18.713037361686)
(0.211211211211211,-18.6656325995665)
(0.212212212212212,-18.6182879576273)
(0.213213213213213,-18.5710034358683)
(0.214214214214214,-18.5237790342895)
(0.215215215215215,-18.476614752891)
(0.216216216216216,-18.4295105916728)
(0.217217217217217,-18.3824665506347)
(0.218218218218218,-18.3354826297769)
(0.219219219219219,-18.2885588290994)
(0.22022022022022,-18.2416951486021)
(0.221221221221221,-18.194891588285)
(0.222222222222222,-18.1481481481481)
(0.223223223223223,-18.1014648281916)
(0.224224224224224,-18.0548416284152)
(0.225225225225225,-18.0082785488191)
(0.226226226226226,-17.9617755894032)
(0.227227227227227,-17.9153327501676)
(0.228228228228228,-17.8689500311122)
(0.229229229229229,-17.822627432237)
(0.23023023023023,-17.7763649535421)
(0.231231231231231,-17.7301625950275)
(0.232232232232232,-17.684020356693)
(0.233233233233233,-17.6379382385388)
(0.234234234234234,-17.5919162405649)
(0.235235235235235,-17.5459543627712)
(0.236236236236236,-17.5000526051577)
(0.237237237237237,-17.4542109677245)
(0.238238238238238,-17.4084294504715)
(0.239239239239239,-17.3627080533987)
(0.24024024024024,-17.3170467765062)
(0.241241241241241,-17.271445619794)
(0.242242242242242,-17.2259045832619)
(0.243243243243243,-17.1804236669102)
(0.244244244244244,-17.1350028707386)
(0.245245245245245,-17.0896421947473)
(0.246246246246246,-17.0443416389362)
(0.247247247247247,-16.9991012033054)
(0.248248248248248,-16.9539208878548)
(0.249249249249249,-16.9088006925845)
(0.25025025025025,-16.8637406174944)
(0.251251251251251,-16.8187406625845)
(0.252252252252252,-16.7738008278549)
(0.253253253253253,-16.7289211133055)
(0.254254254254254,-16.6841015189363)
(0.255255255255255,-16.6393420447474)
(0.256256256256256,-16.5946426907388)
(0.257257257257257,-16.5500034569104)
(0.258258258258258,-16.5054243432622)
(0.259259259259259,-16.4609053497942)
(0.26026026026026,-16.4164464765065)
(0.261261261261261,-16.3720477233991)
(0.262262262262262,-16.3277090904719)
(0.263263263263263,-16.2834305777249)
(0.264264264264264,-16.2392121851581)
(0.265265265265265,-16.1950539127716)
(0.266266266266266,-16.1509557605654)
(0.267267267267267,-16.1069177285393)
(0.268268268268268,-16.0629398166936)
(0.269269269269269,-16.019022025028)
(0.27027027027027,-15.9751643535427)
(0.271271271271271,-15.9313668022377)
(0.272272272272272,-15.8876293711129)
(0.273273273273273,-15.8439520601683)
(0.274274274274274,-15.8003348694039)
(0.275275275275275,-15.7567777988198)
(0.276276276276276,-15.713280848416)
(0.277277277277277,-15.6698440181924)
(0.278278278278278,-15.626467308149)
(0.279279279279279,-15.5831507182859)
(0.28028028028028,-15.539894248603)
(0.281281281281281,-15.4966978991003)
(0.282282282282282,-15.4535616697779)
(0.283283283283283,-15.4104855606357)
(0.284284284284284,-15.3674695716738)
(0.285285285285285,-15.3245137028921)
(0.286286286286286,-15.2816179542906)
(0.287287287287287,-15.2387823258694)
(0.288288288288288,-15.1960068176284)
(0.289289289289289,-15.1532914295677)
(0.29029029029029,-15.1106361616872)
(0.291291291291291,-15.068041013987)
(0.292292292292292,-15.0255059864669)
(0.293293293293293,-14.9830310791272)
(0.294294294294294,-14.9406162919676)
(0.295295295295295,-14.8982616249884)
(0.296296296296296,-14.8559670781893)
(0.297297297297297,-14.8137326515705)
(0.298298298298298,-14.7715583451319)
(0.299299299299299,-14.7294441588736)
(0.3003003003003,-14.6873900927955)
(0.301301301301301,-14.6453961468977)
(0.302302302302302,-14.60346232118)
(0.303303303303303,-14.5615886156427)
(0.304304304304304,-14.5197750302855)
(0.305305305305305,-14.4780215651087)
(0.306306306306306,-14.436328220112)
(0.307307307307307,-14.3946949952956)
(0.308308308308308,-14.3531218906594)
(0.309309309309309,-14.3116089062035)
(0.31031031031031,-14.2701560419278)
(0.311311311311311,-14.2287632978324)
(0.312312312312312,-14.1874306739172)
(0.313313313313313,-14.1461581701822)
(0.314314314314314,-14.1049457866275)
(0.315315315315315,-14.063793523253)
(0.316316316316316,-14.0227013800587)
(0.317317317317317,-13.9816693570447)
(0.318318318318318,-13.940697454211)
(0.319319319319319,-13.8997856715574)
(0.32032032032032,-13.8589340090842)
(0.321321321321321,-13.8181424667911)
(0.322322322322322,-13.7774110446783)
(0.323323323323323,-13.7367397427458)
(0.324324324324324,-13.6961285609934)
(0.325325325325325,-13.6555774994213)
(0.326326326326326,-13.6150865580295)
(0.327327327327327,-13.5746557368179)
(0.328328328328328,-13.5342850357865)
(0.329329329329329,-13.4939744549354)
(0.33033033033033,-13.4537239942645)
(0.331331331331331,-13.4135336537739)
(0.332332332332332,-13.3734034334635)
(0.333333333333333,-13.3333333333333)
(0.334334334334334,-13.2933233533834)
(0.335335335335335,-13.2533734936137)
(0.336336336336336,-13.2134837540243)
(0.337337337337337,-13.1736541346151)
(0.338338338338338,-13.1338846353861)
(0.339339339339339,-13.0941752563374)
(0.34034034034034,-13.0545259974689)
(0.341341341341341,-13.0149368587807)
(0.342342342342342,-12.9754078402727)
(0.343343343343343,-12.9359389419449)
(0.344344344344344,-12.8965301637974)
(0.345345345345345,-12.8571815058302)
(0.346346346346346,-12.8178929680431)
(0.347347347347347,-12.7786645504363)
(0.348348348348348,-12.7394962530098)
(0.349349349349349,-12.7003880757635)
(0.35035035035035,-12.6613400186974)
(0.351351351351351,-12.6223520818115)
(0.352352352352352,-12.5834242651059)
(0.353353353353353,-12.5445565685806)
(0.354354354354354,-12.5057489922355)
(0.355355355355355,-12.4670015360706)
(0.356356356356356,-12.428314200086)
(0.357357357357357,-12.3896869842816)
(0.358358358358358,-12.3511198886574)
(0.359359359359359,-12.3126129132135)
(0.36036036036036,-12.2741660579498)
(0.361361361361361,-12.2357793228664)
(0.362362362362362,-12.1974527079632)
(0.363363363363363,-12.1591862132403)
(0.364364364364364,-12.1209798386976)
(0.365365365365365,-12.0828335843351)
(0.366366366366366,-12.0447474501529)
(0.367367367367367,-12.0067214361509)
(0.368368368368368,-11.9687555423291)
(0.369369369369369,-11.9308497686876)
(0.37037037037037,-11.8930041152263)
(0.371371371371371,-11.8552185819453)
(0.372372372372372,-11.8174931688445)
(0.373373373373373,-11.779827875924)
(0.374374374374374,-11.7422227031837)
(0.375375375375375,-11.7046776506236)
(0.376376376376376,-11.6671927182438)
(0.377377377377377,-11.6297679060442)
(0.378378378378378,-11.5924032140248)
(0.379379379379379,-11.5550986421857)
(0.38038038038038,-11.5178541905269)
(0.381381381381381,-11.4806698590482)
(0.382382382382382,-11.4435456477498)
(0.383383383383383,-11.4064815566317)
(0.384384384384384,-11.3694775856938)
(0.385385385385385,-11.3325337349361)
(0.386386386386386,-11.2956500043587)
(0.387387387387387,-11.2588263939615)
(0.388388388388388,-11.2220629037446)
(0.389389389389389,-11.1853595337079)
(0.39039039039039,-11.1487162838514)
(0.391391391391391,-11.1121331541752)
(0.392392392392392,-11.0756101446792)
(0.393393393393393,-11.0391472553635)
(0.394394394394394,-11.002744486228)
(0.395395395395395,-10.9664018372727)
(0.396396396396396,-10.9301193084977)
(0.397397397397397,-10.8938968999029)
(0.398398398398398,-10.8577346114884)
(0.399399399399399,-10.8216324432541)
(0.4004004004004,-10.7855903952)
(0.401401401401401,-10.7496084673262)
(0.402402402402402,-10.7136866596326)
(0.403403403403403,-10.6778249721193)
(0.404404404404404,-10.6420234047862)
(0.405405405405405,-10.6062819576333)
(0.406406406406406,-10.5706006306607)
(0.407407407407407,-10.5349794238683)
(0.408408408408408,-10.4994183372562)
(0.409409409409409,-10.4639173708243)
(0.41041041041041,-10.4284765245726)
(0.411411411411411,-10.3930957985012)
(0.412412412412412,-10.35777519261)
(0.413413413413413,-10.3225147068991)
(0.414414414414414,-10.2873143413684)
(0.415415415415415,-10.2521740960179)
(0.416416416416416,-10.2170939708477)
(0.417417417417417,-10.1820739658577)
(0.418418418418418,-10.147114081048)
(0.419419419419419,-10.1122143164185)
(0.42042042042042,-10.0773746719693)
(0.421421421421421,-10.0425951477003)
(0.422422422422422,-10.0078757436115)
(0.423423423423423,-9.97321645970295)
(0.424424424424424,-9.93861729597465)
(0.425425425425425,-9.9040782524266)
(0.426426426426426,-9.86959932905879)
(0.427427427427427,-9.83518052587122)
(0.428428428428428,-9.80082184286388)
(0.429429429429429,-9.76652328003679)
(0.43043043043043,-9.73228483738994)
(0.431431431431431,-9.69810651492334)
(0.432432432432432,-9.66398831263696)
(0.433433433433433,-9.62993023053083)
(0.434434434434434,-9.59593226860494)
(0.435435435435435,-9.56199442685929)
(0.436436436436436,-9.52811670529389)
(0.437437437437437,-9.49429910390871)
(0.438438438438438,-9.46054162270378)
(0.439439439439439,-9.4268442616791)
(0.44044044044044,-9.39320702083464)
(0.441441441441441,-9.35962990017044)
(0.442442442442442,-9.32611289968647)
(0.443443443443443,-9.29265601938275)
(0.444444444444444,-9.25925925925926)
(0.445445445445445,-9.22592261931601)
(0.446446446446446,-9.19264609955301)
(0.447447447447447,-9.15942969997024)
(0.448448448448448,-9.12627342056771)
(0.449449449449449,-9.09317726134543)
(0.45045045045045,-9.06014122230338)
(0.451451451451451,-9.02716530344158)
(0.452452452452452,-8.99424950476001)
(0.453453453453453,-8.96139382625869)
(0.454454454454454,-8.92859826793761)
(0.455455455455455,-8.89586282979676)
(0.456456456456456,-8.86318751183616)
(0.457457457457457,-8.83057231405579)
(0.458458458458458,-8.79801723645567)
(0.459459459459459,-8.76552227903579)
(0.46046046046046,-8.73308744179615)
(0.461461461461461,-8.70071272473675)
(0.462462462462462,-8.66839812785759)
(0.463463463463463,-8.63614365115867)
(0.464464464464464,-8.60394929463999)
(0.465465465465465,-8.57181505830154)
(0.466466466466466,-8.53974094214335)
(0.467467467467467,-8.50772694616538)
(0.468468468468468,-8.47577307036767)
(0.469469469469469,-8.44387931475019)
(0.47047047047047,-8.41204567931295)
(0.471471471471471,-8.38027216405595)
(0.472472472472472,-8.34855876897919)
(0.473473473473473,-8.31690549408267)
(0.474474474474474,-8.28531233936639)
(0.475475475475475,-8.25377930483035)
(0.476476476476476,-8.22230639047456)
(0.477477477477477,-8.190893596299)
(0.478478478478478,-8.15954092230369)
(0.479479479479479,-8.12824836848861)
(0.48048048048048,-8.09701593485377)
(0.481481481481481,-8.06584362139918)
(0.482482482482482,-8.03473142812482)
(0.483483483483483,-8.00367935503071)
(0.484484484484485,-7.97268740211683)
(0.485485485485485,-7.9417555693832)
(0.486486486486486,-7.9108838568298)
(0.487487487487487,-7.88007226445665)
(0.488488488488488,-7.84932079226374)
(0.48948948948949,-7.81862944025106)
(0.49049049049049,-7.78799820841863)
(0.491491491491491,-7.75742709676644)
(0.492492492492492,-7.72691610529448)
(0.493493493493493,-7.69646523400277)
(0.494494494494495,-7.6660744828913)
(0.495495495495495,-7.63574385196007)
(0.496496496496497,-7.60547334120908)
(0.497497497497497,-7.57526295063833)
(0.498498498498498,-7.54511268024782)
(0.4994994994995,-7.51502253003755)
(0.500500500500501,-7.48499250000751)
(0.501501501501502,-7.45502259015772)
(0.502502502502503,-7.42511280048818)
(0.503503503503503,-7.39526313099887)
(0.504504504504504,-7.3654735816898)
(0.505505505505506,-7.33574415256097)
(0.506506506506507,-7.30607484361238)
(0.507507507507508,-7.27646565484403)
(0.508508508508508,-7.24691658625593)
(0.509509509509509,-7.21742763784806)
(0.510510510510511,-7.18799880962043)
(0.511511511511512,-7.15863010157304)
(0.512512512512513,-7.1293215137059)
(0.513513513513513,-7.10007304601899)
(0.514514514514514,-7.07088469851233)
(0.515515515515516,-7.0417564711859)
(0.516516516516517,-7.01268836403971)
(0.517517517517518,-6.98368037707377)
(0.518518518518518,-6.95473251028807)
(0.519519519519519,-6.9258447636826)
(0.520520520520521,-6.89701713725738)
(0.521521521521522,-6.86824963101239)
(0.522522522522523,-6.83954224494765)
(0.523523523523523,-6.81089497906315)
(0.524524524524524,-6.78230783335889)
(0.525525525525526,-6.75378080783486)
(0.526526526526527,-6.72531390249108)
(0.527527527527528,-6.69690711732754)
(0.528528528528528,-6.66856045234424)
(0.529529529529529,-6.64027390754118)
(0.530530530530531,-6.61204748291835)
(0.531531531531532,-6.58388117847577)
(0.532532532532533,-6.55577499421343)
(0.533533533533533,-6.52772893013133)
(0.534534534534534,-6.49974298622947)
(0.535535535535536,-6.47181716250785)
(0.536536536536537,-6.44395145896647)
(0.537537537537538,-6.41614587560534)
(0.538538538538539,-6.38840041242444)
(0.539539539539539,-6.36071506942378)
(0.540540540540541,-6.33308984660336)
(0.541541541541542,-6.30552474396318)
(0.542542542542543,-6.27801976150324)
(0.543543543543544,-6.25057489922355)
(0.544544544544544,-6.22319015712409)
(0.545545545545546,-6.19586553520487)
(0.546546546546547,-6.1686010334659)
(0.547547547547548,-6.14139665190716)
(0.548548548548549,-6.11425239052867)
(0.54954954954955,-6.08716824933041)
(0.550550550550551,-6.06014422831239)
(0.551551551551552,-6.03318032747462)
(0.552552552552553,-6.00627654681709)
(0.553553553553554,-5.9794328863398)
(0.554554554554555,-5.95264934604274)
(0.555555555555556,-5.92592592592593)
(0.556556556556557,-5.89926262598935)
(0.557557557557558,-5.87265944623302)
(0.558558558558559,-5.84611638665693)
(0.55955955955956,-5.81963344726107)
(0.560560560560561,-5.79321062804546)
(0.561561561561562,-5.76684792901009)
(0.562562562562563,-5.74054535015496)
(0.563563563563564,-5.71430289148007)
(0.564564564564565,-5.68812055298542)
(0.565565565565566,-5.66199833467101)
(0.566566566566567,-5.63593623653684)
(0.567567567567568,-5.60993425858291)
(0.568568568568569,-5.58399240080922)
(0.56956956956957,-5.55811066321577)
(0.570570570570571,-5.53228904580256)
(0.571571571571572,-5.50652754856959)
(0.572572572572573,-5.48082617151686)
(0.573573573573574,-5.45518491464437)
(0.574574574574575,-5.42960377795213)
(0.575575575575576,-5.40408276144012)
(0.576576576576577,-5.37862186510835)
(0.577577577577578,-5.35322108895682)
(0.578578578578579,-5.32788043298554)
(0.57957957957958,-5.30259989719449)
(0.580580580580581,-5.27737948158369)
(0.581581581581582,-5.25221918615312)
(0.582582582582583,-5.2271190109028)
(0.583583583583584,-5.20207895583271)
(0.584584584584585,-5.17709902094286)
(0.585585585585586,-5.15217920623326)
(0.586586586586587,-5.1273195117039)
(0.587587587587588,-5.10251993735477)
(0.588588588588589,-5.07778048318589)
(0.58958958958959,-5.05310114919724)
(0.590590590590591,-5.02848193538884)
(0.591591591591592,-5.00392284176068)
(0.592592592592593,-4.97942386831276)
(0.593593593593594,-4.95498501504508)
(0.594594594594595,-4.93060628195763)
(0.595595595595596,-4.90628766905043)
(0.596596596596597,-4.88202917632347)
(0.597597597597598,-4.85783080377675)
(0.598598598598599,-4.83369255141027)
(0.5995995995996,-4.80961441922403)
(0.600600600600601,-4.78559640721803)
(0.601601601601602,-4.76163851539227)
(0.602602602602603,-4.73774074374675)
(0.603603603603604,-4.71390309228147)
(0.604604604604605,-4.69012556099643)
(0.605605605605606,-4.66640814989163)
(0.606606606606607,-4.64275085896707)
(0.607607607607608,-4.61915368822276)
(0.608608608608609,-4.59561663765868)
(0.60960960960961,-4.57213970727484)
(0.610610610610611,-4.54872289707124)
(0.611611611611612,-4.52536620704789)
(0.612612612612613,-4.50206963720477)
(0.613613613613614,-4.4788331875419)
(0.614614614614615,-4.45565685805926)
(0.615615615615616,-4.43254064875687)
(0.616616616616617,-4.40948455963471)
(0.617617617617618,-4.3864885906928)
(0.618618618618619,-4.36355274193112)
(0.61961961961962,-4.34067701334968)
(0.620620620620621,-4.31786140494849)
(0.621621621621622,-4.29510591672754)
(0.622622622622623,-4.27241054868683)
(0.623623623623624,-4.24977530082635)
(0.624624624624625,-4.22720017314612)
(0.625625625625626,-4.20468516564613)
(0.626626626626627,-4.18223027832637)
(0.627627627627628,-4.15983551118686)
(0.628628628628629,-4.13750086422759)
(0.62962962962963,-4.11522633744856)
(0.630630630630631,-4.09301193084977)
(0.631631631631632,-4.07085764443122)
(0.632632632632633,-4.04876347819291)
(0.633633633633634,-4.02672943213484)
(0.634634634634635,-4.00475550625701)
(0.635635635635636,-3.98284170055942)
(0.636636636636637,-3.96098801504207)
(0.637637637637638,-3.93919444970496)
(0.638638638638639,-3.91746100454809)
(0.63963963963964,-3.89578767957146)
(0.640640640640641,-3.87417447477508)
(0.641641641641642,-3.85262139015893)
(0.642642642642643,-3.83112842572302)
(0.643643643643644,-3.80969558146735)
(0.644644644644645,-3.78832285739193)
(0.645645645645646,-3.76701025349674)
(0.646646646646647,-3.74575776978179)
(0.647647647647648,-3.72456540624709)
(0.648648648648649,-3.70343316289262)
(0.64964964964965,-3.6823610397184)
(0.650650650650651,-3.66134903672441)
(0.651651651651652,-3.64039715391067)
(0.652652652652653,-3.61950539127716)
(0.653653653653654,-3.5986737488239)
(0.654654654654655,-3.57790222655088)
(0.655655655655656,-3.55719082445809)
(0.656656656656657,-3.53653954254555)
(0.657657657657658,-3.51594838081325)
(0.658658658658659,-3.49541733926118)
(0.65965965965966,-3.47494641788936)
(0.660660660660661,-3.45453561669778)
(0.661661661661662,-3.43418493568644)
(0.662662662662663,-3.41389437485534)
(0.663663663663664,-3.39366393420447)
(0.664664664664665,-3.37349361373385)
(0.665665665665666,-3.35338341344347)
(0.666666666666667,-3.33333333333333)
(0.667667667667668,-3.31334337340343)
(0.668668668668669,-3.29341353365377)
(0.66966966966967,-3.27354381408435)
(0.670670670670671,-3.25373421469518)
(0.671671671671672,-3.23398473548624)
(0.672672672672673,-3.21429537645754)
(0.673673673673674,-3.19466613760908)
(0.674674674674675,-3.17509701894086)
(0.675675675675676,-3.15558802045289)
(0.676676676676677,-3.13613914214515)
(0.677677677677678,-3.11675038401765)
(0.678678678678679,-3.09742174607039)
(0.67967967967968,-3.07815322830338)
(0.680680680680681,-3.0589448307166)
(0.681681681681682,-3.03979655331007)
(0.682682682682683,-3.02070839608377)
(0.683683683683684,-3.00168035903772)
(0.684684684684685,-2.9827124421719)
(0.685685685685686,-2.96380464548633)
(0.686686686686687,-2.94495696898099)
(0.687687687687688,-2.9261694126559)
(0.688688688688689,-2.90744197651104)
(0.68968968968969,-2.88877466054643)
(0.690690690690691,-2.87016746476206)
(0.691691691691692,-2.85162038915793)
(0.692692692692693,-2.83313343373403)
(0.693693693693694,-2.81470659849038)
(0.694694694694695,-2.79633988342697)
(0.695695695695696,-2.7780332885438)
(0.696696696696697,-2.75978681384087)
(0.697697697697698,-2.74160045931818)
(0.698698698698699,-2.72347422497573)
(0.6996996996997,-2.70540811081352)
(0.700700700700701,-2.68740211683155)
(0.701701701701702,-2.66945624302982)
(0.702702702702703,-2.65157048940833)
(0.703703703703704,-2.63374485596708)
(0.704704704704705,-2.61597934270607)
(0.705705705705706,-2.5982739496253)
(0.706706706706707,-2.58062867672477)
(0.707707707707708,-2.56304352400448)
(0.708708708708709,-2.54551849146444)
(0.70970970970971,-2.52805357910463)
(0.710710710710711,-2.51064878692506)
(0.711711711711712,-2.49330411492574)
(0.712712712712713,-2.47601956310665)
(0.713713713713714,-2.4587951314678)
(0.714714714714715,-2.4416308200092)
(0.715715715715716,-2.42452662873083)
(0.716716716716717,-2.40748255763271)
(0.717717717717718,-2.39049860671482)
(0.718718718718719,-2.37357477597718)
(0.71971971971972,-2.35671106541977)
(0.720720720720721,-2.33990747504261)
(0.721721721721722,-2.32316400484569)
(0.722722722722723,-2.306480654829)
(0.723723723723724,-2.28985742499256)
(0.724724724724725,-2.27329431533636)
(0.725725725725726,-2.2567913258604)
(0.726726726726727,-2.24034845656467)
(0.727727727727728,-2.22396570744919)
(0.728728728728729,-2.20764307851395)
(0.72972972972973,-2.19138056975895)
(0.730730730730731,-2.17517818118419)
(0.731731731731732,-2.15903591278967)
(0.732732732732733,-2.14295376457539)
(0.733733733733734,-2.12693173654135)
(0.734734734734735,-2.11096982868755)
(0.735735735735736,-2.09506804101399)
(0.736736736736737,-2.07922637352067)
(0.737737737737738,-2.06344482620759)
(0.738738738738739,-2.04772339907475)
(0.73973973973974,-2.03206209212215)
(0.740740740740741,-2.01646090534979)
(0.741741741741742,-2.00091983875768)
(0.742742742742743,-1.9854388923458)
(0.743743743743744,-1.97001806611416)
(0.744744744744745,-1.95465736006277)
(0.745745745745746,-1.93935677419161)
(0.746746746746747,-1.92411630850069)
(0.747747747747748,-1.90893596299002)
(0.748748748748749,-1.89381573765958)
(0.74974974974975,-1.87875563250939)
(0.750750750750751,-1.86375564753943)
(0.751751751751752,-1.84881578274972)
(0.752752752752753,-1.83393603814024)
(0.753753753753754,-1.81911641371101)
(0.754754754754755,-1.80435690946201)
(0.755755755755756,-1.78965752539326)
(0.756756756756757,-1.77501826150475)
(0.757757757757758,-1.76043911779647)
(0.758758758758759,-1.74592009426844)
(0.75975975975976,-1.73146119092065)
(0.760760760760761,-1.7170624077531)
(0.761761761761762,-1.70272374476579)
(0.762762762762763,-1.68844520195872)
(0.763763763763764,-1.67422677933188)
(0.764764764764765,-1.66006847688529)
(0.765765765765766,-1.64597029461894)
(0.766766766766767,-1.63193223253283)
(0.767767767767768,-1.61795429062696)
(0.768768768768769,-1.60403646890133)
(0.76976976976977,-1.59017876735594)
(0.770770770770771,-1.5763811859908)
(0.771771771771772,-1.56264372480589)
(0.772772772772773,-1.54896638380122)
(0.773773773773774,-1.53534916297679)
(0.774774774774775,-1.5217920623326)
(0.775775775775776,-1.50829508186866)
(0.776776776776777,-1.49485822158495)
(0.777777777777778,-1.48148148148148)
(0.778778778778779,-1.46816486155826)
(0.77977977977978,-1.45490836181527)
(0.780780780780781,-1.44171198225252)
(0.781781781781782,-1.42857572287002)
(0.782782782782783,-1.41549958366775)
(0.783783783783784,-1.40248356464573)
(0.784784784784785,-1.38952766580394)
(0.785785785785786,-1.3766318871424)
(0.786786786786787,-1.36379622866109)
(0.787787787787788,-1.35102069036003)
(0.788788788788789,-1.33830527223921)
(0.78978978978979,-1.32564997429862)
(0.790790790790791,-1.31305479653828)
(0.791791791791792,-1.30051973895818)
(0.792792792792793,-1.28804480155832)
(0.793793793793794,-1.27562998433869)
(0.794794794794795,-1.26327528729931)
(0.795795795795796,-1.25098071044017)
(0.796796796796797,-1.23874625376127)
(0.797797797797798,-1.22657191726261)
(0.798798798798799,-1.21445770094419)
(0.7997997997998,-1.20240360480601)
(0.800800800800801,-1.19040962884807)
(0.801801801801802,-1.17847577307037)
(0.802802802802803,-1.16660203747291)
(0.803803803803804,-1.15478842205569)
(0.804804804804805,-1.14303492681871)
(0.805805805805806,-1.13134155176197)
(0.806806806806807,-1.11970829688547)
(0.807807807807808,-1.10813516218922)
(0.808808808808809,-1.0966221476732)
(0.80980980980981,-1.08516925333742)
(0.810810810810811,-1.07377647918188)
(0.811811811811812,-1.06244382520659)
(0.812812812812813,-1.05117129141153)
(0.813813813813814,-1.03995887779672)
(0.814814814814815,-1.02880658436214)
(0.815815815815816,-1.0177144111078)
(0.816816816816817,-1.00668235803371)
(0.817817817817818,-0.995710425139855)
(0.818818818818819,-0.98479861242624)
(0.81981981981982,-0.973946919892866)
(0.820820820820821,-0.963155347539731)
(0.821821821821822,-0.952423895366838)
(0.822822822822823,-0.941752563374185)
(0.823823823823824,-0.931141351561772)
(0.824824824824825,-0.9205902599296)
(0.825825825825826,-0.910099288477666)
(0.826826826826827,-0.899668437205975)
(0.827827827827828,-0.889297706114523)
(0.828828828828829,-0.878987095203312)
(0.82982982982983,-0.868736604472341)
(0.830830830830831,-0.858546233921609)
(0.831831831831832,-0.848415983551119)
(0.832832832832833,-0.838345853360869)
(0.833833833833834,-0.828335843350859)
(0.834834834834835,-0.818385953521089)
(0.835835835835836,-0.808496183871559)
(0.836836836836837,-0.79866653440227)
(0.837837837837838,-0.788897005113221)
(0.838838838838839,-0.779187596004413)
(0.83983983983984,-0.769538307075845)
(0.840840840840841,-0.759949138327516)
(0.841841841841842,-0.750420089759429)
(0.842842842842843,-0.740951161371582)
(0.843843843843844,-0.731542353163975)
(0.844844844844845,-0.722193665136609)
(0.845845845845846,-0.712905097289481)
(0.846846846846847,-0.703676649622595)
(0.847847847847848,-0.69450832213595)
(0.848848848848849,-0.685400114829545)
(0.84984984984985,-0.676352027703379)
(0.850850850850851,-0.667364060757454)
(0.851851851851852,-0.658436213991769)
(0.852852852852853,-0.649568487406325)
(0.853853853853854,-0.640760881001122)
(0.854854854854855,-0.632013394776158)
(0.855855855855856,-0.623326028731434)
(0.856856856856857,-0.614698782866951)
(0.857857857857858,-0.606131657182708)
(0.858858858858859,-0.597624651678706)
(0.85985985985986,-0.589177766354944)
(0.860860860860861,-0.580791001211421)
(0.861861861861862,-0.57246435624814)
(0.862862862862863,-0.564197831465099)
(0.863863863863864,-0.555991426862298)
(0.864864864864865,-0.547845142439737)
(0.865865865865866,-0.539758978197416)
(0.866866866866867,-0.531732934135337)
(0.867867867867868,-0.523767010253497)
(0.868868868868869,-0.515861206551898)
(0.86986986986987,-0.508015523030538)
(0.870870870870871,-0.500229959689419)
(0.871871871871872,-0.492504516528541)
(0.872872872872873,-0.484839193547903)
(0.873873873873874,-0.477233990747504)
(0.874874874874875,-0.469688908127346)
(0.875875875875876,-0.462203945687429)
(0.876876876876877,-0.454779103427752)
(0.877877877877878,-0.447414381348315)
(0.878878878878879,-0.440109779449119)
(0.87987987987988,-0.432865297730162)
(0.880880880880881,-0.425680936191447)
(0.881881881881882,-0.418556694832971)
(0.882882882882883,-0.411492573654736)
(0.883883883883884,-0.404488572656741)
(0.884884884884885,-0.397544691838986)
(0.885885885885886,-0.390660931201472)
(0.886886886886887,-0.383837290744198)
(0.887887887887888,-0.377073770467164)
(0.888888888888889,-0.370370370370371)
(0.88988988988989,-0.363727090453817)
(0.890890890890891,-0.357143930717504)
(0.891891891891892,-0.350620891161432)
(0.892892892892893,-0.3441579717856)
(0.893893893893894,-0.337755172590008)
(0.894894894894895,-0.331412493574655)
(0.895895895895896,-0.325129934739544)
(0.896896896896897,-0.318907496084673)
(0.897897897897898,-0.312745177610043)
(0.898898898898899,-0.306642979315652)
(0.8998998998999,-0.300600901201502)
(0.900900900900901,-0.294618943267592)
(0.901901901901902,-0.288697105513922)
(0.902902902902903,-0.282835387940493)
(0.903903903903904,-0.277033790547304)
(0.904904904904905,-0.271292313334355)
(0.905905905905906,-0.265610956301647)
(0.906906906906907,-0.259989719449179)
(0.907907907907908,-0.254428602776951)
(0.908908908908909,-0.248927606284964)
(0.90990990990991,-0.243486729973216)
(0.910910910910911,-0.23810597384171)
(0.911911911911912,-0.232785337890443)
(0.912912912912913,-0.227524822119417)
(0.913913913913914,-0.222324426528631)
(0.914914914914915,-0.217184151118085)
(0.915915915915916,-0.21210399588778)
(0.916916916916917,-0.207083960837715)
(0.917917917917918,-0.20212404596789)
(0.918918918918919,-0.197224251278306)
(0.91991991991992,-0.192384576768961)
(0.920920920920921,-0.187605022439857)
(0.921921921921922,-0.182885588290994)
(0.922922922922923,-0.178226274322371)
(0.923923923923924,-0.173627080533988)
(0.924924924924925,-0.169088006925845)
(0.925925925925926,-0.164609053497942)
(0.926926926926927,-0.16019022025028)
(0.927927927927928,-0.155831507182859)
(0.928928928928929,-0.151532914295677)
(0.92992992992993,-0.147294441588736)
(0.930930930930931,-0.143116089062035)
(0.931931931931932,-0.138997856715574)
(0.932932932932933,-0.134939744549354)
(0.933933933933934,-0.130941752563374)
(0.934934934934935,-0.127003880757634)
(0.935935935935936,-0.123126129132135)
(0.936936936936937,-0.119308497686876)
(0.937937937937938,-0.115550986421857)
(0.938938938938939,-0.111853595337079)
(0.93993993993994,-0.108216324432541)
(0.940940940940941,-0.104639173708243)
(0.941941941941942,-0.101122143164185)
(0.942942942942943,-0.0976652328003681)
(0.943943943943944,-0.0942684426167908)
(0.944944944944945,-0.0909317726134542)
(0.945945945945946,-0.0876552227903579)
(0.946946946946947,-0.0844387931475019)
(0.947947947947948,-0.0812824836848862)
(0.948948948948949,-0.0781862944025105)
(0.94994994994995,-0.0751502253003754)
(0.950950950950951,-0.0721742763784806)
(0.951951951951952,-0.0692584476368261)
(0.952952952952953,-0.0664027390754119)
(0.953953953953954,-0.0636071506942377)
(0.954954954954955,-0.0608716824933041)
(0.955955955955956,-0.0581963344726108)
(0.956956956956957,-0.0555811066321578)
(0.957957957957958,-0.053025998971945)
(0.958958958958959,-0.0505310114919724)
(0.95995995995996,-0.0480961441922403)
(0.960960960960961,-0.0457213970727484)
(0.961961961961962,-0.0434067701334969)
(0.962962962962963,-0.0411522633744857)
(0.963963963963964,-0.0389578767957146)
(0.964964964964965,-0.0368236103971839)
(0.965965965965966,-0.0347494641788936)
(0.966966966966967,-0.0327354381408436)
(0.967967967967968,-0.0307815322830339)
(0.968968968968969,-0.0288877466054643)
(0.96996996996997,-0.0270540811081351)
(0.970970970970971,-0.0252805357910463)
(0.971971971971972,-0.0235671106541978)
(0.972972972972973,-0.0219138056975896)
(0.973973973973974,-0.0203206209212215)
(0.974974974974975,-0.0187875563250938)
(0.975975975975976,-0.0173146119092065)
(0.976976976976977,-0.0159017876735595)
(0.977977977977978,-0.0145490836181528)
(0.978978978978979,-0.0132564997429862)
(0.97997997997998,-0.0120240360480601)
(0.980980980980981,-0.0108516925333742)
(0.981981981981982,-0.0097394691989287)
(0.982982982982983,-0.00868736604472346)
(0.983983983983984,-0.00769538307075842)
(0.984984984984985,-0.00676352027703379)
(0.985985985985986,-0.00589177766354945)
(0.986986986986987,-0.00508015523030541)
(0.987987987987988,-0.00432865297730167)
(0.988988988988989,-0.00363727090453815)
(0.98998998998999,-0.00300600901201502)
(0.990990990990991,-0.00243486729973217)
(0.991991991991992,-0.00192384576768963)
(0.992992992992993,-0.00147294441588734)
(0.993993993993994,-0.0010821632443254)
(0.994994994994995,-0.000751502253003754)
(0.995995995995996,-0.000480961441922408)
(0.996996996996997,-0.000270540811081359)
(0.997997997997998,-0.000120240360480595)
(0.998998998998999,-3.00600901201488e-05)
(1,0)

};
\path [draw=black, fill opacity=0] (axis cs:13,0)--(axis cs:13,0);

\path [draw=black, fill opacity=0] (axis cs:1,13)--(axis cs:1,13);

\path [draw=black, fill opacity=0] (axis cs:13,-30)--(axis cs:13,-30);

\path [draw=black, fill opacity=0] (axis cs:0,13)--(axis cs:0,13);

\end{axis}

\end{tikzpicture}
	\end{center}
	%\caption{Fractional position along a plate of length L}
	\label{fig:BeamStress}
\end{figure}
