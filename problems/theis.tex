\section{Theis Problem Benchmark}
\subsection{Problem description}

This benchmark problem provides the solution describing the pressure changes that occur when fluid is produced from a symmetric, infinite, homogeneous, isotropic aquifer of uniform thickness.  The problem tests the ability of a code to conserve mass and model a compressible fluid and tests mass conservation equation and source boundary conditions. The governing equation considered is

\begin{equation}
	\frac{\partial p}{\partial t} = \frac{T}{S}\left(\frac{\partial^2 p}{\partial r^2}+\frac{1}{r}\frac{\partial p}{\partial r} \right)
	\label{eq:Theis}
\end{equation}

The problem is defined for a disk-shaped domain with Neumann condition representing pumping on the inside of the domain at well radius, $r_{\text{well}}$ with outer radial dimension sufficiently distant to effect a semi-infinite domain (Figure~\ref{fig:theisFig}). The Fluid properties are prescribed for \SI[scientific-notation=false,round-precision=2]{50}{\degreeCelsius}. Output data for x-axis pressures are calculated for times of \SIlist[scientific-notation=false, round-precision=2]{2.0; 20.0; 200.0}{\second}. %$10^6$ and $10^7$ seconds.

\begin{figure}[h]
	\centering
	%\input{theisFig.pdf_tex}
	\import{./figures/}{theisFig.pdf_tex}
	\caption{Model domain}
	\label{fig:theisFig}
\end{figure}

The analytical solution of this problem introduces the similarity variable
\begin{equation}
	u=\frac{r^2 S}{4Tt}
	\label{eq:simVar}
\end{equation}
which is then used to determine the drawdown
\begin{equation}
	s=\frac{Q_v}{4\pi T}f(u)
	\label{eq:drawdown}
\end{equation}
where $f(u)$ is the exponential integral, defined as
\begin{equation}
	f(u)=\int_u^\infty \frac{e^{-t}}{t}dt.
	\label{eq:expInt}
\end{equation}
The analytical solution is shown on Figure~\ref{fig:Theis} using parameters from Tables~\ref{tab:TheisMatprop} and~\ref{tab:TheisFluprop}.

The numerical solution was computed with the FALCON simulator from INL. Specified properties for the homogeneous solid and fluid are given in Tables~\ref{tab:TheisMatprop} and~\ref{tab:TheisFluprop}. FALCON computes fluid properties (viscosity, density) from lookup tables based on the IAWPS-97 equation of state. The FALCON-calculated values match those used for calculation of the analytical solution.


\begin{table}[h]
	\caption{Reservoir properties}
	\begin{center}
	\begin{tabular}{lrcSs}
		Radius of injection well & $r_{\text{well}}$ & $\coloneqq$ & 1.0 & \si{\metre} \\
		Radius of outer limit & $r_{\text{outer}}$ &$\coloneqq$& 100.0 & \si{\metre} \\
		Thickness & $b$&$\coloneqq$& 1.0 & \si{\metre} \\
		Permeability & $k$&$\coloneqq$& 9.869233e-13 & \si{\metre\squared} \\
		Porosity of matrix & $\theta_{\text{unit}}$&$\coloneqq$& 0.50 & - \\
	\end{tabular}
	\end{center}
	\label{tab:TheisMatprop}
\end{table}
\begin{table}[h]
	\caption{Fluid properties}
	\begin{center}
	\begin{tabular}{lrcSs}
		Ambient fluid pressure & $p_{\text{amb}}$ & $\coloneqq$ & 1.0e5 & \si{\pascal} \\
		Ambient fluid temperature & $T_{\text{amb}}$ & $\coloneqq$ & 323.15 & \si{\kelvin} \\
		Volumetric flow rate & $Q_v$ & $\coloneqq$ & 1.6666666666666667e-05 & \si{\metre\cubed\per\second} \\
		Dynamic viscosity, water & $\mu$ &$\coloneqq$& 0.0005465159967814679 & \si{\pascal\second} \\
		Fluid compressibility & $\beta$&$\coloneqq$& 4.4173069067307806e-10 & \si{\per\pascal} \\
	\end{tabular}
	\end{center}
	\label{tab:TheisFluprop}
\end{table}

\setlength{\tabcolsep}{3pt}
\renewcommand{\arraystretch}{1.5}
\begin{table}[h]
	\caption{Derived parameters}
	\begin{center}
		\begin{tabular}{lrclcSs}
		Hydraulic diffusivity & $D_{\text{eff}}$ & $\coloneqq$ & $ \dfrac{k}{\theta_{\text{unit}}\beta\mu}$ & $=$ &8.17622708732695&\si{\metre\squared\per\second} \\
		Transmissivity of unit & $T_{\text{unit}}$ & $\coloneqq$ & $k b $& $=$ &9.869233e-13& \si{\metre\cubed} \\
		Storativity of unit & $S_{\text{unit}}$ & $\coloneqq$ & $b\theta_{\text{unit}}\beta$& $=$ &2.2086534533653903e-10&\si{\metre\per\pascal} \\
		Implied hydraulic $K$ & $K_{g}$ & $\coloneqq$ & $\dfrac{k\rho g}{\mu}$& $=$ &1.749739091752754e-05&\si{\metre\per\day} \\
		Injection velocity & $v_{\text{inj}}$ & $\coloneqq$ & $\dfrac{Q_v}{2\pi r_{\text{well}}b}$& $=$ &2.6525823848649226e-06&\si{\metre\per\second} \\
%		Total volume of fluid injected & $V_{\text{inj}}$ & $\coloneqq$ & $Q_v t_3 $& $=$ &0.0033333333333333335&\si{\metre\cubed} \\
		Reynolds number at well & $\text{Re}$ & $\coloneqq$ & $\dfrac{2 r_{\text{well}}\rho v_{\text{inj}}}{\mu}$& $=$ &9.591092816142062& \\

		Transmissivity & $T$ & $\coloneqq$ & $\dfrac{1}{\mu}T_{\text{unit}}\rho g $& $=$ & 1.7735287141919426e-05& \si{\metre\squared\per\second} \\

		Storativity & $S$ & $\coloneqq$ & $S_{\text{unit}}\rho g$& $=$ &  2.140763450623906e-06&  -
	\end{tabular}
	\end{center}
	\label{tab:TheisDerivPar}
\end{table}

\subsection{Files}
\begin{itemize}
	\item Analytical results are stored in \verb|TheisProblemBM-analytical.xlsx|
	\item Numerical results are stored in \verb|TheisProblemBM-FALCON.xlsx|
	\item The numerical solution input file is \verb|TheisProblem.i|
	\item The numerical solution exodus output file is \verb|TheisProblem_out.e|
\end{itemize}
%col1=['Hydraulic diffusivity','Transmissivity of unit','Storativity of unit','Implied hydraulic $K$','Injection velocity','Total volume of fluid injected','Reynolds number at well']
\newpage
\subsection{Results}

\begin{figure}[h!]
	\begin{center}
		\setlength\figureheight{8cm} 
		\setlength\figurewidth{0.8\textwidth} 
		% This file was created by matplotlib v0.1.0.
% Copyright (c) 2010--2014, Nico Schlömer <nico.schloemer@gmail.com>
% All rights reserved.
% 

\begin{tikzpicture}

\begin{axis}[
xlabel={Radius [m]},
ylabel={Pressure [Pa]},
xmin=0, xmax=100,
ymin=92000, ymax=100000,
axis on top,
width=\figurewidth,
height=\figureheight,
legend entries={{Analytic $t=2.00$ s},{Analytic $t=20.0$ s},{Analytic $t=200$ s},{FALCON $t=2.00$ s},{FALCON $t=20.0$ s},{FALCON $t=200$ s}},
legend style={at={(0.97,0.03)}, anchor=south east}
]
\addplot [blue]
coordinates {
(1,97367.6362798124)
(1.0990990990991,97502.3614683289)
(1.1981981981982,97625.0442903485)
(1.2972972972973,97737.5687403099)
(1.3963963963964,97841.402423637)
(1.4954954954955,97937.7109927388)
(1.59459459459459,98027.4357914484)
(1.69369369369369,98111.3480484977)
(1.79279279279279,98190.0876373111)
(1.89189189189189,98264.1913933262)
(1.99099099099099,98334.1141928049)
(2.09009009009009,98400.2449057917)
(2.18918918918919,98462.9186497429)
(2.28828828828829,98522.4263276375)
(2.38738738738739,98579.0221420455)
(2.48648648648649,98632.9295795382)
(2.58558558558558,98684.3462244358)
(2.68468468468468,98733.4476662769)
(2.78378378378378,98780.390698256)
(2.88288288288288,98825.3159555319)
(2.98198198198198,98868.3501070625)
(3.08108108108108,98909.6076885876)
(3.18018018018018,98949.1926449573)
(3.27927927927928,98987.199635342)
(3.37837837837838,99023.7151436985)
(3.47747747747748,99058.8184282891)
(3.57657657657658,99092.5823373949)
(3.67567567567568,99125.0740131767)
(3.77477477477477,99156.3555015405)
(3.87387387387387,99186.4842826307)
(3.97297297297297,99215.5137339834)
(4.07207207207207,99243.4935362991)
(4.17117117117117,99270.4700301166)
(4.27027027027027,99296.486530311)
(4.36936936936937,99321.5836042224)
(4.46846846846847,99345.7993183163)
(4.56756756756757,99369.1694575196)
(4.66666666666667,99391.7277207564)
(4.76576576576576,99413.5058956884)
(4.86486486486486,99434.5340152331)
(4.96396396396396,99454.840498069)
(5.06306306306306,99474.4522750323)
(5.16216216216216,99493.3949030508)
(5.26126126126126,99511.6926680431)
(5.36036036036036,99529.3686780226)
(5.45945945945946,99546.4449474908)
(5.55855855855856,99562.9424740628)
(5.65765765765766,99578.8813081577)
(5.75675675675676,99594.2806164792)
(5.85585585585585,99609.1587399311)
(5.95495495495495,99623.5332465318)
(6.05405405405405,99637.4209798304)
(6.15315315315315,99650.8381032684)
(6.25225225225225,99663.8001408814)
(6.35135135135135,99676.3220146927)
(6.45045045045045,99688.4180791119)
(6.54954954954955,99700.1021526181)
(6.64864864864865,99711.3875469792)
(6.74774774774775,99722.2870942314)
(6.84684684684685,99732.8131716208)
(6.94594594594594,99742.9777246881)
(7.04504504504504,99752.792288661)
(7.14414414414414,99762.2680083008)
(7.24324324324324,99771.4156563367)
(7.34234234234234,99780.2456506085)
(7.44144144144144,99788.7680700273)
(7.54054054054054,99796.9926694516)
(7.63963963963964,99804.9288935708)
(7.73873873873874,99812.5858898762)
(7.83783783783784,99819.9725207951)
(7.93693693693694,99827.0973750544)
(8.03603603603603,99833.9687783369)
(8.13513513513513,99840.5948032855)
(8.23423423423423,99846.9832789074)
(8.33333333333333,99853.1417994254)
(8.43243243243243,99859.0777326189)
(8.53153153153153,99864.7982276948)
(8.63063063063063,99870.3102227233)
(8.72972972972973,99875.6204516732)
(8.82882882882883,99880.7354510755)
(8.92792792792793,99885.6615663445)
(9.02702702702702,99890.4049577812)
(9.12612612612612,99894.9716062828)
(9.22522522522522,99899.3673187796)
(9.32432432432432,99903.5977334202)
(9.42342342342342,99907.6683245215)
(9.52252252252252,99911.584407302)
(9.62162162162162,99915.351142413)
(9.72072072072072,99918.9735402817)
(9.81981981981982,99922.45646528)
(9.91891891891892,99925.8046397307)
(10.018018018018,99929.0226477619)
(10.1171171171171,99932.11493902)
(10.2162162162162,99935.0858322511)
(10.3153153153153,99937.9395187586)
(10.4144144144144,99940.6800657455)
(10.5135135135135,99943.3114195488)
(10.6126126126126,99945.8374087718)
(10.7117117117117,99948.2617473224)
(10.8108108108108,99950.5880373598)
(10.9099099099099,99952.8197721592)
(11.009009009009,99954.9603388946)
(11.1081081081081,99957.013021348)
(11.2072072072072,99958.9810025468)
(11.3063063063063,99960.8673673332)
(11.4054054054054,99962.6751048702)
(11.5045045045045,99964.4071110864)
(11.6036036036036,99966.0661910622)
(11.7027027027027,99967.6550613607)
(11.8018018018018,99969.1763523054)
(11.9009009009009,99970.6326102068)
(12,99972.0262995399)
(12.0990990990991,99973.3598050743)
(12.1981981981982,99974.6354339591)
(12.2972972972973,99975.8554177635)
(12.3963963963964,99977.0219144745)
(12.4954954954955,99978.1370104541)
(12.5945945945946,99979.2027223547)
(12.6936936936937,99980.2209989972)
(12.7927927927928,99981.1937232091)
(12.8918918918919,99982.1227136267)
(12.990990990991,99983.0097264598)
(13.0900900900901,99983.8564572214)
(13.1891891891892,99984.6645424217)
(13.2882882882883,99985.4355612281)
(13.3873873873874,99986.1710370911)
(13.4864864864865,99986.8724393373)
(13.5855855855856,99987.541184729)
(13.6846846846847,99988.1786389919)
(13.7837837837838,99988.7861183112)
(13.8828828828829,99989.3648907955)
(13.981981981982,99989.9161779104)
(14.0810810810811,99990.4411558807)
(14.1801801801802,99990.9409570634)
(14.2792792792793,99991.4166712895)
(14.3783783783784,99991.8693471769)
(14.4774774774775,99992.2999934141)
(14.5765765765766,99992.7095800147)
(14.6756756756757,99993.0990395431)
(14.7747747747748,99993.4692683127)
(14.8738738738739,99993.8211275547)
(14.972972972973,99994.1554445609)
(15.0720720720721,99994.4730137978)
(15.1711711711712,99994.7745979944)
(15.2702702702703,99995.060929203)
(15.3693693693694,99995.3327098336)
(15.4684684684685,99995.5906136621)
(15.5675675675676,99995.8352868135)
(15.6666666666667,99996.0673487183)
(15.7657657657658,99996.2873930451)
(15.8648648648649,99996.495988608)
(15.963963963964,99996.6936802494)
(16.0630630630631,99996.880989699)
(16.1621621621622,99997.0584164088)
(16.2612612612613,99997.2264383657)
(16.3603603603604,99997.3855128795)
(16.4594594594595,99997.5360773502)
(16.5585585585586,99997.6785500112)
(16.6576576576577,99997.8133306524)
(16.7567567567568,99997.9408013206)
(16.8558558558559,99998.0613269992)
(16.954954954955,99998.1752562672)
(17.054054054054,99998.2829219383)
(17.1531531531531,99998.3846416793)
(17.2522522522522,99998.4807186095)
(17.3513513513513,99998.5714418809)
(17.4504504504504,99998.6570872395)
(17.5495495495495,99998.7379175687)
(17.6486486486486,99998.814183414)
(17.7477477477477,99998.886123491)
(17.8468468468468,99998.9539651756)
(17.9459459459459,99999.017924978)
(18.045045045045,99999.0782089999)
(18.1441441441441,99999.1350133759)
(18.2432432432432,99999.1885246995)
(18.3423423423423,99999.2389204337)
(18.4414414414414,99999.2863693071)
(18.5405405405405,99999.3310316953)
(18.6396396396396,99999.3730599882)
(18.7387387387387,99999.4125989443)
(18.8378378378378,99999.4497860311)
(18.9369369369369,99999.4847517529)
(19.036036036036,99999.5176199659)
(19.1351351351351,99999.5485081813)
(19.2342342342342,99999.5775278566)
(19.3333333333333,99999.6047846756)
(19.4324324324324,99999.6303788165)
(19.5315315315315,99999.6544052105)
(19.6306306306306,99999.6769537892)
(19.7297297297297,99999.6981097214)
(19.8288288288288,99999.7179536418)
(19.9279279279279,99999.7365618684)
(20.027027027027,99999.7540066118)
(20.1261261261261,99999.7703561757)
(20.2252252252252,99999.7856751483)
(20.3243243243243,99999.8000245861)
(20.4234234234234,99999.8134621889)
(20.5225225225225,99999.8260424685)
(20.6216216216216,99999.8378169085)
(20.7207207207207,99999.8488341178)
(20.8198198198198,99999.8591399775)
(20.9189189189189,99999.8687777802)
(21.018018018018,99999.877788364)
(21.1171171171171,99999.8862102398)
(21.2162162162162,99999.8940797126)
(21.3153153153153,99999.9014309976)
(21.4144144144144,99999.9082963308)
(21.5135135135135,99999.9147060738)
(21.6126126126126,99999.9206888143)
(21.7117117117117,99999.9262714617)
(21.8108108108108,99999.9314793373)
(21.9099099099099,99999.9363362616)
(22.009009009009,99999.9408646354)
(22.1081081081081,99999.9450855188)
(22.2072072072072,99999.9490187049)
(22.3063063063063,99999.9526827908)
(22.4054054054054,99999.9560952438)
(22.5045045045045,99999.9592724658)
(22.6036036036036,99999.9622298531)
(22.7027027027027,99999.9649818536)
(22.8018018018018,99999.9675420214)
(22.9009009009009,99999.9699230681)
(23,99999.9721369112)
(23.0990990990991,99999.9741947207)
(23.1981981981982,99999.9761069625)
(23.2972972972973,99999.9778834401)
(23.3963963963964,99999.9795333333)
(23.4954954954955,99999.9810652355)
(23.5945945945946,99999.9824871889)
(23.6936936936937,99999.9838067169)
(23.7927927927928,99999.9850308561)
(23.8918918918919,99999.9861661855)
(23.990990990991,99999.9872188543)
(24.0900900900901,99999.9881946083)
(24.1891891891892,99999.9890988149)
(24.2882882882883,99999.989936486)
(24.3873873873874,99999.9907123007)
(24.4864864864865,99999.9914306257)
(24.5855855855856,99999.992095535)
(24.6846846846847,99999.9927108282)
(24.7837837837838,99999.9932800484)
(24.8828828828829,99999.9938064978)
(24.981981981982,99999.9942932538)
(25.0810810810811,99999.9947431828)
(25.1801801801802,99999.9951589544)
(25.2792792792793,99999.995543054)
(25.3783783783784,99999.9958977944)
(25.4774774774775,99999.9962253278)
(25.5765765765766,99999.9965276557)
(25.6756756756757,99999.9968066395)
(25.7747747747748,99999.9970640091)
(25.8738738738739,99999.9973013724)
(25.972972972973,99999.9975202227)
(26.0720720720721,99999.9977219473)
(26.1711711711712,99999.9979078337)
(26.2702702702703,99999.9980790774)
(26.3693693693694,99999.9982367871)
(26.4684684684685,99999.9983819917)
(26.5675675675676,99999.9985156448)
(26.6666666666667,99999.9986386307)
(26.7657657657658,99999.9987517689)
(26.8648648648649,99999.9988558184)
(26.963963963964,99999.9989514824)
(27.0630630630631,99999.9990394117)
(27.1621621621622,99999.9991202088)
(27.2612612612613,99999.9991944314)
(27.3603603603604,99999.9992625949)
(27.4594594594595,99999.9993251764)
(27.5585585585586,99999.9993826167)
(27.6576576576577,99999.9994353233)
(27.7567567567568,99999.9994836724)
(27.8558558558559,99999.9995280118)
(27.954954954955,99999.9995686626)
(28.054054054054,99999.9996059209)
(28.1531531531531,99999.9996400602)
(28.2522522522522,99999.9996713328)
(28.3513513513513,99999.9996999711)
(28.4504504504504,99999.9997261898)
(28.5495495495495,99999.9997501863)
(28.6486486486486,99999.9997721427)
(28.7477477477477,99999.9997922269)
(28.8468468468468,99999.9998105933)
(28.9459459459459,99999.9998273839)
(29.045045045045,99999.9998427296)
(29.1441441441441,99999.9998567507)
(29.2432432432432,99999.999869558)
(29.3423423423423,99999.9998812532)
(29.4414414414414,99999.9998919298)
(29.5405405405405,99999.9999016737)
(29.6396396396396,99999.999910564)
(29.7387387387387,99999.999918673)
(29.8378378378378,99999.9999260673)
(29.9369369369369,99999.999932808)
(30.036036036036,99999.9999389511)
(30.1351351351351,99999.999944548)
(30.2342342342342,99999.9999496458)
(30.3333333333333,99999.9999542876)
(30.4324324324324,99999.9999585131)
(30.5315315315315,99999.9999623586)
(30.6306306306306,99999.999965857)
(30.7297297297297,99999.999969039)
(30.8288288288288,99999.9999719323)
(30.9279279279279,99999.9999745622)
(31.027027027027,99999.9999769522)
(31.1261261261261,99999.9999791235)
(31.2252252252252,99999.9999810954)
(31.3243243243243,99999.9999828859)
(31.4234234234234,99999.9999845112)
(31.5225225225225,99999.999985986)
(31.6216216216216,99999.9999873239)
(31.7207207207207,99999.9999885374)
(31.8198198198198,99999.9999896375)
(31.9189189189189,99999.9999906347)
(32.018018018018,99999.9999915383)
(32.1171171171171,99999.9999923568)
(32.2162162162162,99999.9999930981)
(32.3153153153153,99999.9999937693)
(32.4144144144144,99999.9999943767)
(32.5135135135135,99999.9999949264)
(32.6126126126126,99999.9999954236)
(32.7117117117117,99999.9999958733)
(32.8108108108108,99999.9999962798)
(32.9099099099099,99999.9999966472)
(33.009009009009,99999.9999969792)
(33.1081081081081,99999.999997279)
(33.2072072072072,99999.9999975498)
(33.3063063063063,99999.9999977943)
(33.4054054054054,99999.9999980149)
(33.5045045045045,99999.999998214)
(33.6036036036036,99999.9999983935)
(33.7027027027027,99999.9999985554)
(33.8018018018018,99999.9999987014)
(33.9009009009009,99999.9999988329)
(34,99999.9999989514)
(34.0990990990991,99999.9999990582)
(34.1981981981982,99999.9999991543)
(34.2972972972973,99999.9999992408)
(34.3963963963964,99999.9999993187)
(34.4954954954955,99999.9999993887)
(34.5945945945946,99999.9999994517)
(34.6936936936937,99999.9999995084)
(34.7927927927928,99999.9999995593)
(34.8918918918919,99999.999999605)
(34.990990990991,99999.9999996461)
(35.0900900900901,99999.9999996831)
(35.1891891891892,99999.9999997162)
(35.2882882882883,99999.999999746)
(35.3873873873874,99999.9999997727)
(35.4864864864865,99999.9999997966)
(35.5855855855856,99999.9999998181)
(35.6846846846847,99999.9999998373)
(35.7837837837838,99999.9999998546)
(35.8828828828829,99999.9999998701)
(35.981981981982,99999.9999998839)
(36.0810810810811,99999.9999998963)
(36.1801801801802,99999.9999999074)
(36.2792792792793,99999.9999999174)
(36.3783783783784,99999.9999999263)
(36.4774774774775,99999.9999999342)
(36.5765765765766,99999.9999999413)
(36.6756756756757,99999.9999999477)
(36.7747747747748,99999.9999999534)
(36.8738738738739,99999.9999999584)
(36.972972972973,99999.999999963)
(37.0720720720721,99999.999999967)
(37.1711711711712,99999.9999999706)
(37.2702702702703,99999.9999999739)
(37.3693693693694,99999.9999999767)
(37.4684684684685,99999.9999999793)
(37.5675675675676,99999.9999999816)
(37.6666666666667,99999.9999999836)
(37.7657657657658,99999.9999999855)
(37.8648648648649,99999.9999999871)
(37.963963963964,99999.9999999885)
(38.0630630630631,99999.9999999898)
(38.1621621621622,99999.9999999909)
(38.2612612612613,99999.999999992)
(38.3603603603604,99999.9999999929)
(38.4594594594595,99999.9999999937)
(38.5585585585586,99999.9999999944)
(38.6576576576577,99999.999999995)
(38.7567567567568,99999.9999999956)
(38.8558558558559,99999.9999999961)
(38.9549549549549,99999.9999999965)
(39.054054054054,99999.9999999969)
(39.1531531531531,99999.9999999973)
(39.2522522522522,99999.9999999976)
(39.3513513513513,99999.9999999979)
(39.4504504504504,99999.9999999981)
(39.5495495495495,99999.9999999983)
(39.6486486486486,99999.9999999985)
(39.7477477477477,99999.9999999987)
(39.8468468468468,99999.9999999989)
(39.9459459459459,99999.999999999)
(40.045045045045,99999.9999999991)
(40.1441441441441,99999.9999999992)
(40.2432432432432,99999.9999999993)
(40.3423423423423,99999.9999999994)
(40.4414414414414,99999.9999999995)
(40.5405405405405,99999.9999999995)
(40.6396396396396,99999.9999999996)
(40.7387387387387,99999.9999999996)
(40.8378378378378,99999.9999999997)
(40.9369369369369,99999.9999999997)
(41.036036036036,99999.9999999998)
(41.1351351351351,99999.9999999998)
(41.2342342342342,99999.9999999998)
(41.3333333333333,99999.9999999998)
(41.4324324324324,99999.9999999999)
(41.5315315315315,99999.9999999999)
(41.6306306306306,99999.9999999999)
(41.7297297297297,99999.9999999999)
(41.8288288288288,99999.9999999999)
(41.9279279279279,99999.9999999999)
(42.027027027027,99999.9999999999)
(42.1261261261261,99999.9999999999)
(42.2252252252252,99999.9999999999)
(42.3243243243243,100000)
(42.4234234234234,100000)
(42.5225225225225,100000)
(42.6216216216216,100000)
(42.7207207207207,100000)
(42.8198198198198,100000)
(42.9189189189189,100000)
(43.018018018018,100000)
(43.1171171171171,100000)
(43.2162162162162,100000)
(43.3153153153153,100000)
(43.4144144144144,100000)
(43.5135135135135,100000)
(43.6126126126126,100000)
(43.7117117117117,100000)
(43.8108108108108,100000)
(43.9099099099099,100000)
(44.009009009009,100000)
(44.1081081081081,100000)
(44.2072072072072,100000)
(44.3063063063063,100000)
(44.4054054054054,100000)
(44.5045045045045,100000)
(44.6036036036036,100000)
(44.7027027027027,100000)
(44.8018018018018,100000)
(44.9009009009009,100000)
(45,100000)
(45.0990990990991,100000)
(45.1981981981982,100000)
(45.2972972972973,100000)
(45.3963963963964,100000)
(45.4954954954955,100000)
(45.5945945945946,100000)
(45.6936936936937,100000)
(45.7927927927928,100000)
(45.8918918918919,100000)
(45.990990990991,100000)
(46.0900900900901,100000)
(46.1891891891892,100000)
(46.2882882882883,100000)
(46.3873873873874,100000)
(46.4864864864865,100000)
(46.5855855855856,100000)
(46.6846846846847,100000)
(46.7837837837838,100000)
(46.8828828828829,100000)
(46.981981981982,100000)
(47.0810810810811,100000)
(47.1801801801802,100000)
(47.2792792792793,100000)
(47.3783783783784,100000)
(47.4774774774775,100000)
(47.5765765765766,100000)
(47.6756756756757,100000)
(47.7747747747748,100000)
(47.8738738738739,100000)
(47.972972972973,100000)
(48.0720720720721,100000)
(48.1711711711712,100000)
(48.2702702702703,100000)
(48.3693693693694,100000)
(48.4684684684685,100000)
(48.5675675675676,100000)
(48.6666666666667,100000)
(48.7657657657658,100000)
(48.8648648648649,100000)
(48.963963963964,100000)
(49.0630630630631,100000)
(49.1621621621622,100000)
(49.2612612612613,100000)
(49.3603603603604,100000)
(49.4594594594595,100000)
(49.5585585585586,100000)
(49.6576576576577,100000)
(49.7567567567568,100000)
(49.8558558558559,100000)
(49.9549549549549,100000)
(50.054054054054,100000)
(50.1531531531531,100000)
(50.2522522522522,100000)
(50.3513513513513,100000)
(50.4504504504504,100000)
(50.5495495495495,100000)
(50.6486486486486,100000)
(50.7477477477477,100000)
(50.8468468468468,100000)
(50.9459459459459,100000)
(51.045045045045,100000)
(51.1441441441441,100000)
(51.2432432432432,100000)
(51.3423423423423,100000)
(51.4414414414414,100000)
(51.5405405405405,100000)
(51.6396396396396,100000)
(51.7387387387387,100000)
(51.8378378378378,100000)
(51.9369369369369,100000)
(52.036036036036,100000)
(52.1351351351351,100000)
(52.2342342342342,100000)
(52.3333333333333,100000)
(52.4324324324324,100000)
(52.5315315315315,100000)
(52.6306306306306,100000)
(52.7297297297297,100000)
(52.8288288288288,100000)
(52.9279279279279,100000)
(53.027027027027,100000)
(53.1261261261261,100000)
(53.2252252252252,100000)
(53.3243243243243,100000)
(53.4234234234234,100000)
(53.5225225225225,100000)
(53.6216216216216,100000)
(53.7207207207207,100000)
(53.8198198198198,100000)
(53.9189189189189,100000)
(54.018018018018,100000)
(54.1171171171171,100000)
(54.2162162162162,100000)
(54.3153153153153,100000)
(54.4144144144144,100000)
(54.5135135135135,100000)
(54.6126126126126,100000)
(54.7117117117117,100000)
(54.8108108108108,100000)
(54.9099099099099,100000)
(55.009009009009,100000)
(55.1081081081081,100000)
(55.2072072072072,100000)
(55.3063063063063,100000)
(55.4054054054054,100000)
(55.5045045045045,100000)
(55.6036036036036,100000)
(55.7027027027027,100000)
(55.8018018018018,100000)
(55.9009009009009,100000)
(56,100000)
(56.0990990990991,100000)
(56.1981981981982,100000)
(56.2972972972973,100000)
(56.3963963963964,100000)
(56.4954954954955,100000)
(56.5945945945946,100000)
(56.6936936936937,100000)
(56.7927927927928,100000)
(56.8918918918919,100000)
(56.990990990991,100000)
(57.0900900900901,100000)
(57.1891891891892,100000)
(57.2882882882883,100000)
(57.3873873873874,100000)
(57.4864864864865,100000)
(57.5855855855856,100000)
(57.6846846846847,100000)
(57.7837837837838,100000)
(57.8828828828829,100000)
(57.981981981982,100000)
(58.0810810810811,100000)
(58.1801801801802,100000)
(58.2792792792793,100000)
(58.3783783783784,100000)
(58.4774774774775,100000)
(58.5765765765766,100000)
(58.6756756756757,100000)
(58.7747747747748,100000)
(58.8738738738739,100000)
(58.972972972973,100000)
(59.0720720720721,100000)
(59.1711711711712,100000)
(59.2702702702703,100000)
(59.3693693693694,100000)
(59.4684684684685,100000)
(59.5675675675676,100000)
(59.6666666666667,100000)
(59.7657657657658,100000)
(59.8648648648649,100000)
(59.963963963964,100000)
(60.0630630630631,100000)
(60.1621621621622,100000)
(60.2612612612613,100000)
(60.3603603603604,100000)
(60.4594594594595,100000)
(60.5585585585586,100000)
(60.6576576576577,100000)
(60.7567567567568,100000)
(60.8558558558559,100000)
(60.9549549549549,100000)
(61.054054054054,100000)
(61.1531531531531,100000)
(61.2522522522522,100000)
(61.3513513513513,100000)
(61.4504504504504,100000)
(61.5495495495495,100000)
(61.6486486486486,100000)
(61.7477477477477,100000)
(61.8468468468468,100000)
(61.9459459459459,100000)
(62.045045045045,100000)
(62.1441441441441,100000)
(62.2432432432432,100000)
(62.3423423423423,100000)
(62.4414414414414,100000)
(62.5405405405405,100000)
(62.6396396396396,100000)
(62.7387387387387,100000)
(62.8378378378378,100000)
(62.9369369369369,100000)
(63.036036036036,100000)
(63.1351351351351,100000)
(63.2342342342342,100000)
(63.3333333333333,100000)
(63.4324324324324,100000)
(63.5315315315315,100000)
(63.6306306306306,100000)
(63.7297297297297,100000)
(63.8288288288288,100000)
(63.9279279279279,100000)
(64.027027027027,100000)
(64.1261261261261,100000)
(64.2252252252252,100000)
(64.3243243243243,100000)
(64.4234234234234,100000)
(64.5225225225225,100000)
(64.6216216216216,100000)
(64.7207207207207,100000)
(64.8198198198198,100000)
(64.9189189189189,100000)
(65.018018018018,100000)
(65.1171171171171,100000)
(65.2162162162162,100000)
(65.3153153153153,100000)
(65.4144144144144,100000)
(65.5135135135135,100000)
(65.6126126126126,100000)
(65.7117117117117,100000)
(65.8108108108108,100000)
(65.9099099099099,100000)
(66.009009009009,100000)
(66.1081081081081,100000)
(66.2072072072072,100000)
(66.3063063063063,100000)
(66.4054054054054,100000)
(66.5045045045045,100000)
(66.6036036036036,100000)
(66.7027027027027,100000)
(66.8018018018018,100000)
(66.9009009009009,100000)
(67,100000)
(67.0990990990991,100000)
(67.1981981981982,100000)
(67.2972972972973,100000)
(67.3963963963964,100000)
(67.4954954954955,100000)
(67.5945945945946,100000)
(67.6936936936937,100000)
(67.7927927927928,100000)
(67.8918918918919,100000)
(67.990990990991,100000)
(68.0900900900901,100000)
(68.1891891891892,100000)
(68.2882882882883,100000)
(68.3873873873874,100000)
(68.4864864864865,100000)
(68.5855855855856,100000)
(68.6846846846847,100000)
(68.7837837837838,100000)
(68.8828828828829,100000)
(68.981981981982,100000)
(69.0810810810811,100000)
(69.1801801801802,100000)
(69.2792792792793,100000)
(69.3783783783784,100000)
(69.4774774774775,100000)
(69.5765765765766,100000)
(69.6756756756757,100000)
(69.7747747747748,100000)
(69.8738738738739,100000)
(69.972972972973,100000)
(70.0720720720721,100000)
(70.1711711711712,100000)
(70.2702702702703,100000)
(70.3693693693694,100000)
(70.4684684684685,100000)
(70.5675675675676,100000)
(70.6666666666667,100000)
(70.7657657657657,100000)
(70.8648648648649,100000)
(70.9639639639639,100000)
(71.0630630630631,100000)
(71.1621621621621,100000)
(71.2612612612613,100000)
(71.3603603603603,100000)
(71.4594594594595,100000)
(71.5585585585585,100000)
(71.6576576576576,100000)
(71.7567567567567,100000)
(71.8558558558558,100000)
(71.9549549549549,100000)
(72.054054054054,100000)
(72.1531531531531,100000)
(72.2522522522522,100000)
(72.3513513513513,100000)
(72.4504504504504,100000)
(72.5495495495495,100000)
(72.6486486486486,100000)
(72.7477477477477,100000)
(72.8468468468468,100000)
(72.9459459459459,100000)
(73.045045045045,100000)
(73.1441441441441,100000)
(73.2432432432432,100000)
(73.3423423423423,100000)
(73.4414414414414,100000)
(73.5405405405405,100000)
(73.6396396396396,100000)
(73.7387387387387,100000)
(73.8378378378378,100000)
(73.9369369369369,100000)
(74.036036036036,100000)
(74.1351351351351,100000)
(74.2342342342342,100000)
(74.3333333333333,100000)
(74.4324324324324,100000)
(74.5315315315315,100000)
(74.6306306306306,100000)
(74.7297297297297,100000)
(74.8288288288288,100000)
(74.9279279279279,100000)
(75.027027027027,100000)
(75.1261261261261,100000)
(75.2252252252252,100000)
(75.3243243243243,100000)
(75.4234234234234,100000)
(75.5225225225225,100000)
(75.6216216216216,100000)
(75.7207207207207,100000)
(75.8198198198198,100000)
(75.9189189189189,100000)
(76.018018018018,100000)
(76.1171171171171,100000)
(76.2162162162162,100000)
(76.3153153153153,100000)
(76.4144144144144,100000)
(76.5135135135135,100000)
(76.6126126126126,100000)
(76.7117117117117,100000)
(76.8108108108108,100000)
(76.9099099099099,100000)
(77.009009009009,100000)
(77.1081081081081,100000)
(77.2072072072072,100000)
(77.3063063063063,100000)
(77.4054054054054,100000)
(77.5045045045045,100000)
(77.6036036036036,100000)
(77.7027027027027,100000)
(77.8018018018018,100000)
(77.9009009009009,100000)
(78,100000)
(78.0990990990991,100000)
(78.1981981981982,100000)
(78.2972972972973,100000)
(78.3963963963964,100000)
(78.4954954954955,100000)
(78.5945945945946,100000)
(78.6936936936937,100000)
(78.7927927927928,100000)
(78.8918918918919,100000)
(78.990990990991,100000)
(79.0900900900901,100000)
(79.1891891891892,100000)
(79.2882882882883,100000)
(79.3873873873874,100000)
(79.4864864864865,100000)
(79.5855855855856,100000)
(79.6846846846847,100000)
(79.7837837837838,100000)
(79.8828828828829,100000)
(79.981981981982,100000)
(80.0810810810811,100000)
(80.1801801801802,100000)
(80.2792792792793,100000)
(80.3783783783784,100000)
(80.4774774774775,100000)
(80.5765765765766,100000)
(80.6756756756757,100000)
(80.7747747747748,100000)
(80.8738738738739,100000)
(80.972972972973,100000)
(81.0720720720721,100000)
(81.1711711711712,100000)
(81.2702702702703,100000)
(81.3693693693694,100000)
(81.4684684684685,100000)
(81.5675675675676,100000)
(81.6666666666667,100000)
(81.7657657657657,100000)
(81.8648648648649,100000)
(81.9639639639639,100000)
(82.0630630630631,100000)
(82.1621621621621,100000)
(82.2612612612613,100000)
(82.3603603603603,100000)
(82.4594594594595,100000)
(82.5585585585585,100000)
(82.6576576576576,100000)
(82.7567567567567,100000)
(82.8558558558558,100000)
(82.9549549549549,100000)
(83.054054054054,100000)
(83.1531531531531,100000)
(83.2522522522522,100000)
(83.3513513513513,100000)
(83.4504504504504,100000)
(83.5495495495495,100000)
(83.6486486486486,100000)
(83.7477477477477,100000)
(83.8468468468468,100000)
(83.9459459459459,100000)
(84.045045045045,100000)
(84.1441441441441,100000)
(84.2432432432432,100000)
(84.3423423423423,100000)
(84.4414414414414,100000)
(84.5405405405405,100000)
(84.6396396396396,100000)
(84.7387387387387,100000)
(84.8378378378378,100000)
(84.9369369369369,100000)
(85.036036036036,100000)
(85.1351351351351,100000)
(85.2342342342342,100000)
(85.3333333333333,100000)
(85.4324324324324,100000)
(85.5315315315315,100000)
(85.6306306306306,100000)
(85.7297297297297,100000)
(85.8288288288288,100000)
(85.9279279279279,100000)
(86.027027027027,100000)
(86.1261261261261,100000)
(86.2252252252252,100000)
(86.3243243243243,100000)
(86.4234234234234,100000)
(86.5225225225225,100000)
(86.6216216216216,100000)
(86.7207207207207,100000)
(86.8198198198198,100000)
(86.9189189189189,100000)
(87.018018018018,100000)
(87.1171171171171,100000)
(87.2162162162162,100000)
(87.3153153153153,100000)
(87.4144144144144,100000)
(87.5135135135135,100000)
(87.6126126126126,100000)
(87.7117117117117,100000)
(87.8108108108108,100000)
(87.9099099099099,100000)
(88.009009009009,100000)
(88.1081081081081,100000)
(88.2072072072072,100000)
(88.3063063063063,100000)
(88.4054054054054,100000)
(88.5045045045045,100000)
(88.6036036036036,100000)
(88.7027027027027,100000)
(88.8018018018018,100000)
(88.9009009009009,100000)
(89,100000)
(89.0990990990991,100000)
(89.1981981981982,100000)
(89.2972972972973,100000)
(89.3963963963964,100000)
(89.4954954954955,100000)
(89.5945945945946,100000)
(89.6936936936937,100000)
(89.7927927927928,100000)
(89.8918918918919,100000)
(89.990990990991,100000)
(90.0900900900901,100000)
(90.1891891891892,100000)
(90.2882882882883,100000)
(90.3873873873874,100000)
(90.4864864864865,100000)
(90.5855855855856,100000)
(90.6846846846847,100000)
(90.7837837837838,100000)
(90.8828828828829,100000)
(90.981981981982,100000)
(91.0810810810811,100000)
(91.1801801801802,100000)
(91.2792792792793,100000)
(91.3783783783784,100000)
(91.4774774774775,100000)
(91.5765765765766,100000)
(91.6756756756757,100000)
(91.7747747747748,100000)
(91.8738738738739,100000)
(91.972972972973,100000)
(92.0720720720721,100000)
(92.1711711711712,100000)
(92.2702702702703,100000)
(92.3693693693694,100000)
(92.4684684684685,100000)
(92.5675675675676,100000)
(92.6666666666667,100000)
(92.7657657657657,100000)
(92.8648648648649,100000)
(92.9639639639639,100000)
(93.0630630630631,100000)
(93.1621621621621,100000)
(93.2612612612613,100000)
(93.3603603603603,100000)
(93.4594594594595,100000)
(93.5585585585585,100000)
(93.6576576576576,100000)
(93.7567567567567,100000)
(93.8558558558558,100000)
(93.9549549549549,100000)
(94.054054054054,100000)
(94.1531531531531,100000)
(94.2522522522522,100000)
(94.3513513513513,100000)
(94.4504504504504,100000)
(94.5495495495495,100000)
(94.6486486486486,100000)
(94.7477477477477,100000)
(94.8468468468468,100000)
(94.9459459459459,100000)
(95.045045045045,100000)
(95.1441441441441,100000)
(95.2432432432432,100000)
(95.3423423423423,100000)
(95.4414414414414,100000)
(95.5405405405405,100000)
(95.6396396396396,100000)
(95.7387387387387,100000)
(95.8378378378378,100000)
(95.9369369369369,100000)
(96.036036036036,100000)
(96.1351351351351,100000)
(96.2342342342342,100000)
(96.3333333333333,100000)
(96.4324324324324,100000)
(96.5315315315315,100000)
(96.6306306306306,100000)
(96.7297297297297,100000)
(96.8288288288288,100000)
(96.9279279279279,100000)
(97.027027027027,100000)
(97.1261261261261,100000)
(97.2252252252252,100000)
(97.3243243243243,100000)
(97.4234234234234,100000)
(97.5225225225225,100000)
(97.6216216216216,100000)
(97.7207207207207,100000)
(97.8198198198198,100000)
(97.9189189189189,100000)
(98.018018018018,100000)
(98.1171171171171,100000)
(98.2162162162162,100000)
(98.3153153153153,100000)
(98.4144144144144,100000)
(98.5135135135135,100000)
(98.6126126126126,100000)
(98.7117117117117,100000)
(98.8108108108108,100000)
(98.9099099099099,100000)
(99.009009009009,100000)
(99.1081081081081,100000)
(99.2072072072072,100000)
(99.3063063063063,100000)
(99.4054054054054,100000)
(99.5045045045045,100000)
(99.6036036036036,100000)
(99.7027027027027,100000)
(99.8018018018018,100000)
(99.9009009009009,100000)
(100,100000)

};
\addplot [green!50.0!black]
coordinates {
(1,95708.4342251685)
(1.0990990990991,95845.1882763508)
(1.1981981981982,95970.0875261663)
(1.2972972972973,96085.014897654)
(1.3963963963964,96191.4368370766)
(1.4954954954955,96290.5177510587)
(1.59459459459459,96383.1976522458)
(1.69369369369369,96470.2463540798)
(1.79279279279279,96552.3022319704)
(1.89189189189189,96629.9005420707)
(1.99099099099099,96703.4945016167)
(2.09009009009009,96773.4712434854)
(2.18918918918919,96840.1640714927)
(2.28828828828829,96903.8620002408)
(2.38738738738739,96964.8172709856)
(2.48648648648649,97023.2513379084)
(2.58558558558558,97079.3596837829)
(2.68468468468468,97133.3157294212)
(2.78378378378378,97185.2740341385)
(2.88288288288288,97235.3729361409)
(2.98198198198198,97283.7367464852)
(3.08108108108108,97330.4775842312)
(3.18018018018018,97375.6969209801)
(3.27927927927928,97419.4868883319)
(3.37837837837838,97461.9313906343)
(3.47747747747748,97503.1070568149)
(3.57657657657658,97543.0840584364)
(3.67567567567568,97581.9268159235)
(3.77477477477477,97619.6946108169)
(3.87387387387387,97656.4421186727)
(3.97297297297297,97692.2198746372)
(4.07207207207207,97727.0746816534)
(4.17117117117117,97761.0499695769)
(4.27027027027027,97794.1861121215)
(4.36936936936937,97826.5207074388)
(4.46846846846847,97858.088827227)
(4.56756756756757,97888.9232385135)
(4.66666666666667,97919.0546016288)
(4.76576576576576,97948.5116473761)
(4.86486486486486,97977.321335966)
(4.96396396396396,98005.5089999216)
(5.06306306306306,98033.0984728578)
(5.16216216216216,98060.1122057759)
(5.26126126126126,98086.5713722995)
(5.36036036036036,98112.4959640898)
(5.45945945945946,98137.9048775194)
(5.55855855855856,98162.8159925487)
(5.65765765765766,98187.2462446319)
(5.75675675675676,98211.2116903783)
(5.85585585585585,98234.7275676095)
(5.95495495495495,98257.8083503756)
(6.05405405405405,98280.4677994305)
(6.15315315315315,98302.7190086088)
(6.25225225225225,98324.5744474969)
(6.35135135135135,98346.0460007475)
(6.45045045045045,98367.1450043509)
(6.54954954954955,98387.8822791394)
(6.64864864864865,98408.2681617763)
(6.74774774774775,98428.3125334509)
(6.84684684684685,98448.0248464817)
(6.94594594594594,98467.4141490072)
(7.04504504504504,98486.4891079274)
(7.14414414414414,98505.2580302418)
(7.24324324324324,98523.728882918)
(7.34234234234234,98541.9093114093)
(7.44144144144144,98559.80665693)
(7.54054054054054,98577.4279725885)
(7.63963963963964,98594.7800384656)
(7.73873873873874,98611.8693757214)
(7.83783783783784,98628.7022598037)
(7.93693693693694,98645.2847328264)
(8.03603603603603,98661.622615179)
(8.13513513513513,98677.7215164252)
(8.23423423423423,98693.5868455397)
(8.33333333333333,98709.2238205334)
(8.43243243243243,98724.6374775087)
(8.53153153153153,98739.8326791855)
(8.63063063063063,98754.8141229343)
(8.72972972972973,98769.5863483506)
(8.82882882882883,98784.1537444006)
(8.92792792792793,98798.5205561687)
(9.02702702702702,98812.6908912313)
(9.12612612612612,98826.6687256827)
(9.22522522522522,98840.4579098347)
(9.32432432432432,98854.062173612)
(9.42342342342342,98867.4851316612)
(9.52252252252252,98880.7302881929)
(9.62162162162162,98893.8010415717)
(9.72072072072072,98906.7006886719)
(9.81981981981982,98919.4324290107)
(9.91891891891892,98931.9993686747)
(10.018018018018,98944.4045240496)
(10.1171171171171,98956.6508253665)
(10.2162162162162,98968.7411200755)
(10.3153153153153,98980.678176055)
(10.4144144144144,98992.4646846682)
(10.5135135135135,99004.1032636736)
(10.6126126126126,99015.5964599994)
(10.7117117117117,99026.9467523878)
(10.8108108108108,99038.1565539182)
(10.9099099099099,99049.2282144136)
(11.009009009009,99060.1640227392)
(11.1081081081081,99070.9662089964)
(11.2072072072072,99081.63694662)
(11.3063063063063,99092.178354382)
(11.4054054054054,99102.5924983084)
(11.5045045045045,99112.8813935118)
(11.6036036036036,99123.0470059456)
(11.7027027027027,99133.0912540835)
(11.8018018018018,99143.0160105273)
(11.9009009009009,99152.8231035475)
(12,99162.5143185595)
(12.0990990990991,99172.0913995388)
(12.1981981981982,99181.5560503783)
(12.2972972972973,99190.9099361904)
(12.3963963963964,99200.1546845561)
(12.4954954954955,99209.2918867245)
(12.5945945945946,99218.3230987643)
(12.6936936936937,99227.2498426705)
(12.7927927927928,99236.0736074263)
(12.8918918918919,99244.7958500259)
(12.990990990991,99253.4179964557)
(13.0900900900901,99261.9414426391)
(13.1891891891892,99270.3675553456)
(13.2882882882883,99278.6976730642)
(13.3873873873874,99286.9331068459)
(13.4864864864865,99295.0751411132)
(13.5855855855856,99303.125034441)
(13.6846846846847,99311.0840203076)
(13.7837837837838,99318.9533078197)
(13.8828828828829,99326.7340824098)
(13.981981981982,99334.4275065094)
(14.0810810810811,99342.034720198)
(14.1801801801802,99349.5568418289)
(14.2792792792793,99356.9949686332)
(14.3783783783784,99364.3501773021)
(14.4774774774775,99371.6235245498)
(14.5765765765766,99378.8160476562)
(14.6756756756757,99385.9287649914)
(14.7747747747748,99392.9626765221)
(14.8738738738739,99399.9187643011)
(14.972972972973,99406.7979929395)
(15.0720720720721,99413.6013100646)
(15.1711711711712,99420.3296467607)
(15.2702702702703,99426.9839179972)
(15.3693693693694,99433.5650230409)
(15.4684684684685,99440.0738458561)
(15.5675675675676,99446.5112554911)
(15.6666666666667,99452.8781064524)
(15.7657657657658,99459.1752390666)
(15.8648648648649,99465.4034798313)
(15.963963963964,99471.5636417545)
(16.0630630630631,99477.6565246831)
(16.1621621621622,99483.6829156216)
(16.2612612612613,99489.6435890406)
(16.3603603603604,99495.5393071753)
(16.4594594594595,99501.370820316)
(16.5585585585586,99507.1388670882)
(16.6576576576577,99512.8441747251)
(16.7567567567568,99518.4874593317)
(16.8558558558559,99524.069426141)
(16.954954954955,99529.5907697617)
(17.054054054054,99535.0521744198)
(17.1531531531531,99540.4543141922)
(17.2522522522522,99545.7978532333)
(17.3513513513513,99551.0834459957)
(17.4504504504504,99556.3117374433)
(17.5495495495495,99561.4833632598)
(17.6486486486486,99566.5989500493)
(17.7477477477477,99571.6591155327)
(17.8468468468468,99576.6644687376)
(17.9459459459459,99581.6156101831)
(18.045045045045,99586.5131320592)
(18.1441441441441,99591.3576184016)
(18.2432432432432,99596.1496452606)
(18.3423423423423,99600.8897808667)
(18.4414414414414,99605.5785857902)
(18.5405405405405,99610.2166130973)
(18.6396396396396,99614.8044085017)
(18.7387387387387,99619.3425105118)
(18.8378378378378,99623.8314505741)
(18.9369369369369,99628.2717532129)
(19.036036036036,99632.6639361656)
(19.1351351351351,99637.0085105153)
(19.2342342342342,99641.3059808188)
(19.3333333333333,99645.5568452322)
(19.4324324324324,99649.7615956326)
(19.5315315315315,99653.9207177365)
(19.6306306306306,99658.0346912155)
(19.7297297297297,99662.1039898088)
(19.8288288288288,99666.1290814329)
(19.9279279279279,99670.110428288)
(20.027027027027,99674.0484869624)
(20.1261261261261,99677.9437085334)
(20.2252252252252,99681.7965386665)
(20.3243243243243,99685.6074177117)
(20.4234234234234,99689.3767807967)
(20.5225225225225,99693.105057919)
(20.6216216216216,99696.7926740351)
(20.7207207207207,99700.4400491471)
(20.8198198198198,99704.0475983876)
(20.9189189189189,99707.6157321027)
(21.018018018018,99711.1448559326)
(21.1171171171171,99714.6353708903)
(21.2162162162162,99718.0876734383)
(21.3153153153153,99721.5021555638)
(21.4144144144144,99724.8792048517)
(21.5135135135135,99728.2192045563)
(21.6126126126126,99731.5225336705)
(21.7117117117117,99734.7895669945)
(21.8108108108108,99738.0206752018)
(21.9099099099099,99741.2162249041)
(22.009009009009,99744.3765787152)
(22.1081081081081,99747.5020953122)
(22.2072072072072,99750.5931294965)
(22.3063063063063,99753.6500322526)
(22.4054054054054,99756.6731508063)
(22.5045045045045,99759.6628286802)
(22.6036036036036,99762.61940575)
(22.7027027027027,99765.5432182974)
(22.8018018018018,99768.4345990634)
(22.9009009009009,99771.2938772997)
(23,99774.1213788187)
(23.0990990990991,99776.9174260435)
(23.1981981981982,99779.682338055)
(23.2972972972973,99782.4164306403)
(23.3963963963964,99785.1200163377)
(23.4954954954955,99787.7934044824)
(23.5945945945946,99790.4369012503)
(23.6936936936937,99793.0508097015)
(23.7927927927928,99795.6354298225)
(23.8918918918919,99798.1910585672)
(23.990990990991,99800.7179898978)
(24.0900900900901,99803.2165148243)
(24.1891891891892,99805.6869214435)
(24.2882882882883,99808.1294949766)
(24.3873873873874,99810.5445178069)
(24.4864864864865,99812.932269516)
(24.5855855855856,99815.2930269195)
(24.6846846846847,99817.6270641024)
(24.7837837837838,99819.9346524528)
(24.8828828828829,99822.2160606961)
(24.981981981982,99824.4715549275)
(25.0810810810811,99826.7013986446)
(25.1801801801802,99828.9058527791)
(25.2792792792793,99831.0851757273)
(25.3783783783784,99833.2396233815)
(25.4774774774775,99835.3694491589)
(25.5765765765766,99837.4749040314)
(25.6756756756757,99839.5562365544)
(25.7747747747748,99841.6136928947)
(25.8738738738739,99843.6475168583)
(25.972972972973,99845.6579499177)
(26.0720720720721,99847.645231238)
(26.1711711711712,99849.6095977039)
(26.2702702702703,99851.5512839445)
(26.3693693693694,99853.4705223592)
(26.4684684684685,99855.3675431421)
(26.5675675675676,99857.2425743062)
(26.6666666666667,99859.0958417076)
(26.7657657657658,99860.9275690689)
(26.8648648648649,99862.737978002)
(26.963963963964,99864.5272880309)
(27.0630630630631,99866.2957166142)
(27.1621621621622,99868.0434791665)
(27.2612612612613,99869.7707890802)
(27.3603603603604,99871.4778577465)
(27.4594594594595,99873.1648945765)
(27.5585585585586,99874.8321070213)
(27.6576576576577,99876.479700592)
(27.7567567567568,99878.1078788801)
(27.8558558558559,99879.7168435761)
(27.954954954955,99881.3067944894)
(28.054054054054,99882.8779295667)
(28.1531531531531,99884.4304449107)
(28.2522522522522,99885.9645347983)
(28.3513513513513,99887.4803916986)
(28.4504504504504,99888.9782062904)
(28.5495495495495,99890.4581674799)
(28.6486486486486,99891.9204624178)
(28.7477477477477,99893.365276516)
(28.8468468468468,99894.7927934645)
(28.9459459459459,99896.2031952473)
(29.045045045045,99897.5966621594)
(29.1441441441441,99898.9733728219)
(29.2432432432432,99900.333504198)
(29.3423423423423,99901.6772316088)
(29.4414414414414,99903.004728748)
(29.5405405405405,99904.3161676971)
(29.6396396396396,99905.6117189405)
(29.7387387387387,99906.8915513799)
(29.8378378378378,99908.1558323486)
(29.9369369369369,99909.404727626)
(30.036036036036,99910.6384014516)
(30.1351351351351,99911.8570165386)
(30.2342342342342,99913.0607340878)
(30.3333333333333,99914.2497138011)
(30.4324324324324,99915.4241138946)
(30.5315315315315,99916.5840911122)
(30.6306306306306,99917.7298007381)
(30.7297297297297,99918.8613966098)
(30.8288288288288,99919.9790311308)
(30.9279279279279,99921.0828552832)
(31.027027027027,99922.1730186399)
(31.1261261261261,99923.2496693768)
(31.2252252252252,99924.312954285)
(31.3243243243243,99925.3630187826)
(31.4234234234234,99926.4000069265)
(31.5225225225225,99927.4240614243)
(31.6216216216216,99928.4353236453)
(31.7207207207207,99929.4339336324)
(31.8198198198198,99930.4200301129)
(31.9189189189189,99931.3937505101)
(32.018018018018,99932.355230954)
(32.1171171171171,99933.3046062923)
(32.2162162162162,99934.242010101)
(32.3153153153153,99935.1675746954)
(32.4144144144144,99936.0814311402)
(32.5135135135135,99936.9837092604)
(32.6126126126126,99937.8745376514)
(32.7117117117117,99938.7540436891)
(32.8108108108108,99939.6223535402)
(32.9099099099099,99940.4795921722)
(33.009009009009,99941.3258833633)
(33.1081081081081,99942.1613497122)
(33.2072072072072,99942.9861126477)
(33.3063063063063,99943.8002924387)
(33.4054054054054,99944.6040082033)
(33.5045045045045,99945.3973779186)
(33.6036036036036,99946.1805184298)
(33.7027027027027,99946.9535454597)
(33.8018018018018,99947.7165736176)
(33.9009009009009,99948.4697164087)
(34,99949.2130862431)
(34.0990990990991,99949.9467944445)
(34.1981981981982,99950.6709512593)
(34.2972972972973,99951.3856658651)
(34.3963963963964,99952.0910463798)
(34.4954954954955,99952.7871998701)
(34.5945945945946,99953.4742323598)
(34.6936936936937,99954.1522488386)
(34.7927927927928,99954.8213532704)
(34.8918918918919,99955.4816486016)
(34.990990990991,99956.1332367694)
(35.0900900900901,99956.7762187102)
(35.1891891891892,99957.4106943675)
(35.2882882882883,99958.0367626999)
(35.3873873873874,99958.6545216895)
(35.4864864864865,99959.2640683495)
(35.5855855855856,99959.8654987324)
(35.6846846846847,99960.4589079373)
(35.7837837837838,99961.0443901183)
(35.8828828828829,99961.6220384918)
(35.981981981982,99962.1919453443)
(36.0810810810811,99962.7542020398)
(36.1801801801802,99963.3088990277)
(36.2792792792793,99963.85612585)
(36.3783783783784,99964.3959711485)
(36.4774774774775,99964.9285226729)
(36.5765765765766,99965.4538672874)
(36.6756756756757,99965.9720909782)
(36.7747747747748,99966.4832788607)
(36.8738738738739,99966.9875151869)
(36.972972972973,99967.4848833519)
(37.0720720720721,99967.9754659018)
(37.1711711711712,99968.4593445399)
(37.2702702702703,99968.936600134)
(37.3693693693694,99969.4073127234)
(37.4684684684685,99969.8715615257)
(37.5675675675676,99970.3294249435)
(37.6666666666667,99970.7809805712)
(37.7657657657658,99971.2263052016)
(37.8648648648649,99971.6654748331)
(37.963963963964,99972.0985646757)
(38.0630630630631,99972.5256491576)
(38.1621621621622,99972.9468019324)
(38.2612612612613,99973.3620958848)
(38.3603603603604,99973.7716031376)
(38.4594594594595,99974.1753950579)
(38.5585585585586,99974.5735422632)
(38.6576576576577,99974.9661146282)
(38.7567567567568,99975.353181291)
(38.8558558558559,99975.7348106589)
(38.9549549549549,99976.111070415)
(39.054054054054,99976.4820275244)
(39.1531531531531,99976.8477482399)
(39.2522522522522,99977.2082981085)
(39.3513513513513,99977.5637419774)
(39.4504504504504,99977.9141439996)
(39.5495495495495,99978.2595676404)
(39.6486486486486,99978.6000756827)
(39.7477477477477,99978.9357302337)
(39.8468468468468,99979.2665927298)
(39.9459459459459,99979.5927239432)
(40.045045045045,99979.9141839872)
(40.1441441441441,99980.2310323222)
(40.2432432432432,99980.543327761)
(40.3423423423423,99980.851128475)
(40.4414414414414,99981.1544919994)
(40.5405405405405,99981.4534752389)
(40.6396396396396,99981.7481344735)
(40.7387387387387,99982.0385253634)
(40.8378378378378,99982.3247029551)
(40.9369369369369,99982.6067216867)
(41.036036036036,99982.8846353929)
(41.1351351351351,99983.1584973109)
(41.2342342342342,99983.4283600854)
(41.3333333333333,99983.6942757741)
(41.4324324324324,99983.9562958528)
(41.5315315315315,99984.2144712207)
(41.6306306306306,99984.4688522058)
(41.7297297297297,99984.7194885695)
(41.8288288288288,99984.9664295125)
(41.9279279279279,99985.2097236792)
(42.027027027027,99985.4494191633)
(42.1261261261261,99985.6855635124)
(42.2252252252252,99985.9182037334)
(42.3243243243243,99986.1473862971)
(42.4234234234234,99986.3731571433)
(42.5225225225225,99986.5955616859)
(42.6216216216216,99986.8146448172)
(42.7207207207207,99987.0304509135)
(42.8198198198198,99987.2430238394)
(42.9189189189189,99987.4524069524)
(43.018018018018,99987.6586431082)
(43.1171171171171,99987.8617746652)
(43.2162162162162,99988.0618434888)
(43.3153153153153,99988.2588909567)
(43.4144144144144,99988.4529579629)
(43.5135135135135,99988.6440849227)
(43.6126126126126,99988.832311777)
(43.7117117117117,99989.017677997)
(43.8108108108108,99989.2002225886)
(43.9099099099099,99989.379984097)
(44.009009009009,99989.5570006108)
(44.1081081081081,99989.7313097668)
(44.2072072072072,99989.902948754)
(44.3063063063063,99990.0719543183)
(44.4054054054054,99990.2383627665)
(44.5045045045045,99990.4022099707)
(44.6036036036036,99990.5635313727)
(44.7027027027027,99990.722361988)
(44.8018018018018,99990.87873641)
(44.9009009009009,99991.0326888143)
(45,99991.1842529627)
(45.0990990990991,99991.3334622076)
(45.1981981981982,99991.4803494954)
(45.2972972972973,99991.6249473714)
(45.3963963963964,99991.7672879831)
(45.4954954954955,99991.9074030847)
(45.5945945945946,99992.0453240406)
(45.6936936936937,99992.1810818298)
(45.7927927927928,99992.3147070494)
(45.8918918918919,99992.4462299187)
(45.990990990991,99992.5756802831)
(46.0900900900901,99992.7030876174)
(46.1891891891892,99992.8284810304)
(46.2882882882883,99992.9518892681)
(46.3873873873874,99993.0733407175)
(46.4864864864865,99993.1928634104)
(46.5855855855856,99993.310485027)
(46.6846846846847,99993.4262328997)
(46.7837837837838,99993.5401340166)
(46.8828828828829,99993.6522150249)
(46.981981981982,99993.7625022349)
(47.0810810810811,99993.8710216233)
(47.1801801801802,99993.9777988365)
(47.2792792792793,99994.0828591944)
(47.3783783783784,99994.1862276939)
(47.4774774774775,99994.287929012)
(47.5765765765766,99994.3879875094)
(47.6756756756757,99994.4864272339)
(47.7747747747748,99994.5832719237)
(47.8738738738739,99994.6785450106)
(47.972972972973,99994.7722696238)
(48.0720720720721,99994.8644685923)
(48.1711711711712,99994.9551644489)
(48.2702702702703,99995.0443794332)
(48.3693693693694,99995.1321354946)
(48.4684684684685,99995.2184542958)
(48.5675675675676,99995.3033572154)
(48.6666666666667,99995.3868653518)
(48.7657657657658,99995.4689995254)
(48.8648648648649,99995.5497802826)
(48.963963963964,99995.629227898)
(49.0630630630631,99995.7073623778)
(49.1621621621622,99995.7842034629)
(49.2612612612613,99995.8597706316)
(49.3603603603604,99995.9340831027)
(49.4594594594595,99996.0071598383)
(49.5585585585586,99996.0790195468)
(49.6576576576577,99996.1496806858)
(49.7567567567568,99996.2191614647)
(49.8558558558559,99996.2874798478)
(49.9549549549549,99996.3546535567)
(50.054054054054,99996.4207000737)
(50.1531531531531,99996.485636644)
(50.2522522522522,99996.5494802784)
(50.3513513513513,99996.6122477564)
(50.4504504504504,99996.6739556286)
(50.5495495495495,99996.7346202194)
(50.6486486486486,99996.7942576295)
(50.7477477477477,99996.8528837388)
(50.8468468468468,99996.9105142088)
(50.9459459459459,99996.9671644848)
(51.045045045045,99997.0228497992)
(51.1441441441441,99997.0775851734)
(51.2432432432432,99997.1313854205)
(51.3423423423423,99997.1842651477)
(51.4414414414414,99997.2362387588)
(51.5405405405405,99997.2873204565)
(51.6396396396396,99997.3375242451)
(51.7387387387387,99997.3868639324)
(51.8378378378378,99997.4353531325)
(51.9369369369369,99997.4830052678)
(52.036036036036,99997.5298335716)
(52.1351351351351,99997.57585109)
(52.2342342342342,99997.6210706847)
(52.3333333333333,99997.6655050346)
(52.4324324324324,99997.7091666385)
(52.5315315315315,99997.7520678173)
(52.6306306306306,99997.7942207158)
(52.7297297297297,99997.8356373052)
(52.8288288288288,99997.876329385)
(52.9279279279279,99997.9163085855)
(53.027027027027,99997.9555863694)
(53.1261261261261,99997.9941740342)
(53.2252252252252,99998.0320827141)
(53.3243243243243,99998.0693233822)
(53.4234234234234,99998.1059068523)
(53.5225225225225,99998.1418437812)
(53.6216216216216,99998.1771446701)
(53.7207207207207,99998.2118198673)
(53.8198198198198,99998.2458795697)
(53.9189189189189,99998.2793338246)
(54.018018018018,99998.3121925319)
(54.1171171171171,99998.3444654458)
(54.2162162162162,99998.3761621768)
(54.3153153153153,99998.4072921935)
(54.4144144144144,99998.4378648241)
(54.5135135135135,99998.4678892586)
(54.6126126126126,99998.4973745506)
(54.7117117117117,99998.5263296188)
(54.8108108108108,99998.5547632487)
(54.9099099099099,99998.5826840947)
(55.009009009009,99998.6101006816)
(55.1081081081081,99998.637021406)
(55.2072072072072,99998.6634545386)
(55.3063063063063,99998.6894082253)
(55.4054054054054,99998.7148904891)
(55.5045045045045,99998.7399092316)
(55.6036036036036,99998.7644722349)
(55.7027027027027,99998.7885871626)
(55.8018018018018,99998.812261562)
(55.9009009009009,99998.8355028653)
(56,99998.8583183911)
(56.0990990990991,99998.8807153462)
(56.1981981981982,99998.9027008267)
(56.2972972972973,99998.9242818199)
(56.3963963963964,99998.9454652055)
(56.4954954954955,99998.9662577571)
(56.5945945945946,99998.9866661437)
(56.6936936936937,99999.006696931)
(56.7927927927928,99999.0263565831)
(56.8918918918919,99999.0456514634)
(56.990990990991,99999.0645878363)
(57.0900900900901,99999.0831718688)
(57.1891891891892,99999.1014096311)
(57.2882882882883,99999.1193070986)
(57.3873873873874,99999.136870153)
(57.4864864864865,99999.1541045835)
(57.5855855855856,99999.1710160882)
(57.6846846846847,99999.1876102752)
(57.7837837837838,99999.2038926642)
(57.8828828828829,99999.2198686872)
(57.981981981982,99999.2355436903)
(58.0810810810811,99999.2509229346)
(58.1801801801802,99999.2660115973)
(58.2792792792793,99999.280814773)
(58.3783783783784,99999.2953374751)
(58.4774774774775,99999.3095846365)
(58.5765765765766,99999.3235611112)
(58.6756756756757,99999.3372716749)
(58.7747747747748,99999.3507210267)
(58.8738738738739,99999.3639137897)
(58.972972972973,99999.3768545123)
(59.0720720720721,99999.3895476695)
(59.1711711711712,99999.4019976633)
(59.2702702702703,99999.4142088247)
(59.3693693693694,99999.4261854136)
(59.4684684684685,99999.437931621)
(59.5675675675676,99999.449451569)
(59.6666666666667,99999.4607493125)
(59.7657657657658,99999.4718288397)
(59.8648648648649,99999.4826940734)
(59.963963963964,99999.4933488719)
(60.0630630630631,99999.5037970297)
(60.1621621621622,99999.5140422788)
(60.2612612612613,99999.5240882893)
(60.3603603603604,99999.5339386706)
(60.4594594594595,99999.5435969721)
(60.5585585585586,99999.5530666841)
(60.6576576576577,99999.5623512388)
(60.7567567567568,99999.5714540111)
(60.8558558558559,99999.5803783194)
(60.9549549549549,99999.5891274265)
(61.054054054054,99999.5977045406)
(61.1531531531531,99999.6061128156)
(61.2522522522522,99999.6143553525)
(61.3513513513513,99999.6224352001)
(61.4504504504504,99999.6303553553)
(61.5495495495495,99999.6381187647)
(61.6486486486486,99999.6457283244)
(61.7477477477477,99999.6531868818)
(61.8468468468468,99999.6604972353)
(61.9459459459459,99999.6676621361)
(62.045045045045,99999.674684288)
(62.1441441441441,99999.6815663486)
(62.2432432432432,99999.6883109302)
(62.3423423423423,99999.6949206)
(62.4414414414414,99999.7013978809)
(62.5405405405405,99999.7077452526)
(62.6396396396396,99999.713965152)
(62.7387387387387,99999.7200599736)
(62.8378378378378,99999.7260320706)
(62.9369369369369,99999.7318837553)
(63.036036036036,99999.7376173)
(63.1351351351351,99999.743234937)
(63.2342342342342,99999.7487388601)
(63.3333333333333,99999.7541312245)
(63.4324324324324,99999.7594141478)
(63.5315315315315,99999.7645897105)
(63.6306306306306,99999.7696599564)
(63.7297297297297,99999.7746268935)
(63.8288288288288,99999.7794924944)
(63.9279279279279,99999.7842586967)
(64.027027027027,99999.788927404)
(64.1261261261261,99999.793500486)
(64.2252252252252,99999.7979797793)
(64.3243243243243,99999.8023670879)
(64.4234234234234,99999.8066641835)
(64.5225225225225,99999.8108728064)
(64.6216216216216,99999.8149946657)
(64.7207207207207,99999.81903144)
(64.8198198198198,99999.8229847778)
(64.9189189189189,99999.8268562979)
(65.018018018018,99999.8306475902)
(65.1171171171171,99999.8343602158)
(65.2162162162162,99999.8379957077)
(65.3153153153153,99999.841555571)
(65.4144144144144,99999.8450412839)
(65.5135135135135,99999.8484542976)
(65.6126126126126,99999.8517960369)
(65.7117117117117,99999.8550679007)
(65.8108108108108,99999.8582712626)
(65.9099099099099,99999.8614074709)
(66.009009009009,99999.8644778493)
(66.1081081081081,99999.8674836974)
(66.2072072072072,99999.8704262907)
(66.3063063063063,99999.8733068815)
(66.4054054054054,99999.876126699)
(66.5045045045045,99999.8788869498)
(66.6036036036036,99999.8815888179)
(66.7027027027027,99999.8842334659)
(66.8018018018018,99999.8868220345)
(66.9009009009009,99999.8893556433)
(67,99999.8918353913)
(67.0990990990991,99999.8942623567)
(67.1981981981982,99999.896637598)
(67.2972972972973,99999.8989621536)
(67.3963963963964,99999.9012370427)
(67.4954954954955,99999.9034632653)
(67.5945945945946,99999.9056418026)
(67.6936936936937,99999.9077736175)
(67.7927927927928,99999.9098596546)
(67.8918918918919,99999.9119008408)
(67.990990990991,99999.9138980856)
(68.0900900900901,99999.9158522811)
(68.1891891891892,99999.9177643027)
(68.2882882882883,99999.9196350089)
(68.3873873873874,99999.9214652422)
(68.4864864864865,99999.9232558289)
(68.5855855855856,99999.9250075796)
(68.6846846846847,99999.9267212894)
(68.7837837837838,99999.9283977381)
(68.8828828828829,99999.9300376909)
(68.981981981982,99999.9316418979)
(69.0810810810811,99999.9332110951)
(69.1801801801802,99999.9347460042)
(69.2792792792793,99999.9362473329)
(69.3783783783784,99999.9377157755)
(69.4774774774775,99999.9391520127)
(69.5765765765766,99999.940556712)
(69.6756756756757,99999.941930528)
(69.7747747747748,99999.9432741026)
(69.8738738738739,99999.9445880653)
(69.972972972973,99999.9458730331)
(70.0720720720721,99999.9471296112)
(70.1711711711712,99999.9483583929)
(70.2702702702703,99999.9495599598)
(70.3693693693694,99999.9507348823)
(70.4684684684685,99999.9518837192)
(70.5675675675676,99999.9530070189)
(70.6666666666667,99999.9541053185)
(70.7657657657657,99999.9551791447)
(70.8648648648649,99999.9562290138)
(70.9639639639639,99999.9572554319)
(71.0630630630631,99999.958258895)
(71.1621621621621,99999.9592398894)
(71.2612612612613,99999.9601988915)
(71.3603603603603,99999.9611363686)
(71.4594594594595,99999.9620527782)
(71.5585585585585,99999.962948569)
(71.6576576576576,99999.9638241807)
(71.7567567567567,99999.9646800441)
(71.8558558558558,99999.9655165813)
(71.9549549549549,99999.9663342063)
(72.054054054054,99999.9671333242)
(72.1531531531531,99999.9679143325)
(72.2522522522522,99999.9686776202)
(72.3513513513513,99999.9694235688)
(72.4504504504504,99999.9701525519)
(72.5495495495495,99999.9708649356)
(72.6486486486486,99999.9715610785)
(72.7477477477477,99999.9722413321)
(72.8468468468468,99999.9729060404)
(72.9459459459459,99999.9735555408)
(73.045045045045,99999.9741901635)
(73.1441441441441,99999.974810232)
(73.2432432432432,99999.9754160634)
(73.3423423423423,99999.976007968)
(73.4414414414414,99999.9765862499)
(73.5405405405405,99999.977151207)
(73.6396396396396,99999.9777031309)
(73.7387387387387,99999.9782423072)
(73.8378378378378,99999.9787690158)
(73.9369369369369,99999.9792835306)
(74.036036036036,99999.97978612)
(74.1351351351351,99999.9802770466)
(74.2342342342342,99999.9807565678)
(74.3333333333333,99999.9812249355)
(74.4324324324324,99999.9816823963)
(74.5315315315315,99999.9821291919)
(74.6306306306306,99999.9825655587)
(74.7297297297297,99999.9829917282)
(74.8288288288288,99999.9834079271)
(74.9279279279279,99999.9838143775)
(75.027027027027,99999.9842112964)
(75.1261261261261,99999.9845988968)
(75.2252252252252,99999.9849773867)
(75.3243243243243,99999.98534697)
(75.4234234234234,99999.9857078462)
(75.5225225225225,99999.9860602107)
(75.6216216216216,99999.9864042545)
(75.7207207207207,99999.9867401649)
(75.8198198198198,99999.9870681248)
(75.9189189189189,99999.9873883136)
(76.018018018018,99999.9877009066)
(76.1171171171171,99999.9880060755)
(76.2162162162162,99999.9883039884)
(76.3153153153153,99999.9885948096)
(76.4144144144144,99999.9888787)
(76.5135135135135,99999.9891558169)
(76.6126126126126,99999.9894263146)
(76.7117117117117,99999.9896903437)
(76.8108108108108,99999.9899480517)
(76.9099099099099,99999.9901995828)
(77.009009009009,99999.9904450784)
(77.1081081081081,99999.9906846764)
(77.2072072072072,99999.990918512)
(77.3063063063063,99999.9911467174)
(77.4054054054054,99999.991369422)
(77.5045045045045,99999.9915867523)
(77.6036036036036,99999.9917988319)
(77.7027027027027,99999.9920057821)
(77.8018018018018,99999.9922077211)
(77.9009009009009,99999.9924047648)
(78,99999.9925970265)
(78.0990990990991,99999.992784617)
(78.1981981981982,99999.9929676447)
(78.2972972972973,99999.9931462155)
(78.3963963963964,99999.993320433)
(78.4954954954955,99999.9934903987)
(78.5945945945946,99999.9936562116)
(78.6936936936937,99999.9938179687)
(78.7927927927928,99999.9939757648)
(78.8918918918919,99999.9941296925)
(78.990990990991,99999.9942798425)
(79.0900900900901,99999.9944263033)
(79.1891891891892,99999.9945691616)
(79.2882882882883,99999.9947085021)
(79.3873873873874,99999.9948444076)
(79.4864864864865,99999.9949769591)
(79.5855855855856,99999.9951062357)
(79.6846846846847,99999.9952323147)
(79.7837837837838,99999.9953552719)
(79.8828828828829,99999.9954751811)
(79.981981981982,99999.9955921146)
(80.0810810810811,99999.9957061431)
(80.1801801801802,99999.9958173356)
(80.2792792792793,99999.9959257596)
(80.3783783783784,99999.996031481)
(80.4774774774775,99999.9961345643)
(80.5765765765766,99999.9962350726)
(80.6756756756757,99999.9963330673)
(80.7747747747748,99999.9964286087)
(80.8738738738739,99999.9965217555)
(80.972972972973,99999.9966125653)
(81.0720720720721,99999.9967010941)
(81.1711711711712,99999.9967873969)
(81.2702702702703,99999.9968715272)
(81.3693693693694,99999.9969535374)
(81.4684684684685,99999.9970334787)
(81.5675675675676,99999.9971114011)
(81.6666666666667,99999.9971873534)
(81.7657657657657,99999.9972613834)
(81.8648648648649,99999.9973335377)
(81.9639639639639,99999.9974038618)
(82.0630630630631,99999.9974724002)
(82.1621621621621,99999.9975391964)
(82.2612612612613,99999.9976042928)
(82.3603603603603,99999.9976677309)
(82.4594594594595,99999.9977295513)
(82.5585585585585,99999.9977897933)
(82.6576576576576,99999.9978484958)
(82.7567567567567,99999.9979056964)
(82.8558558558558,99999.997961432)
(82.9549549549549,99999.9980157386)
(83.054054054054,99999.9980686513)
(83.1531531531531,99999.9981202044)
(83.2522522522522,99999.9981704315)
(83.3513513513513,99999.9982193652)
(83.4504504504504,99999.9982670376)
(83.5495495495495,99999.9983134799)
(83.6486486486486,99999.9983587225)
(83.7477477477477,99999.9984027951)
(83.8468468468468,99999.9984457268)
(83.9459459459459,99999.998487546)
(84.045045045045,99999.9985282803)
(84.1441441441441,99999.9985679568)
(84.2432432432432,99999.9986066018)
(84.3423423423423,99999.998644241)
(84.4414414414414,99999.9986808998)
(84.5405405405405,99999.9987166024)
(84.6396396396396,99999.998751373)
(84.7387387387387,99999.9987852349)
(84.8378378378378,99999.9988182109)
(84.9369369369369,99999.9988503232)
(85.036036036036,99999.9988815937)
(85.1351351351351,99999.9989120435)
(85.2342342342342,99999.9989416933)
(85.3333333333333,99999.9989705633)
(85.4324324324324,99999.9989986733)
(85.5315315315315,99999.9990260424)
(85.6306306306306,99999.9990526895)
(85.7297297297297,99999.9990786328)
(85.8288288288288,99999.9991038902)
(85.9279279279279,99999.9991284792)
(86.027027027027,99999.9991524168)
(86.1261261261261,99999.9991757196)
(86.2252252252252,99999.9991984037)
(86.3243243243243,99999.999220485)
(86.4234234234234,99999.9992419789)
(86.5225225225225,99999.9992629004)
(86.6216216216216,99999.9992832641)
(86.7207207207207,99999.9993030845)
(86.8198198198198,99999.9993223754)
(86.9189189189189,99999.9993411504)
(87.018018018018,99999.9993594228)
(87.1171171171171,99999.9993772057)
(87.2162162162162,99999.9993945115)
(87.3153153153153,99999.9994113527)
(87.4144144144144,99999.9994277412)
(87.5135135135135,99999.9994436887)
(87.6126126126126,99999.9994592068)
(87.7117117117117,99999.9994743064)
(87.8108108108108,99999.9994889985)
(87.9099099099099,99999.9995032937)
(88.009009009009,99999.9995172023)
(88.1081081081081,99999.9995307343)
(88.2072072072072,99999.9995438997)
(88.3063063063063,99999.9995567079)
(88.4054054054054,99999.9995691683)
(88.5045045045045,99999.99958129)
(88.6036036036036,99999.9995930819)
(88.7027027027027,99999.9996045526)
(88.8018018018018,99999.9996157106)
(88.9009009009009,99999.999626564)
(89,99999.9996371209)
(89.0990990990991,99999.9996473892)
(89.1981981981982,99999.9996573763)
(89.2972972972973,99999.9996670897)
(89.3963963963964,99999.9996765367)
(89.4954954954955,99999.9996857243)
(89.5945945945946,99999.9996946594)
(89.6936936936937,99999.9997033486)
(89.7927927927928,99999.9997117985)
(89.8918918918919,99999.9997200155)
(89.990990990991,99999.9997280057)
(90.0900900900901,99999.9997357752)
(90.1891891891892,99999.9997433298)
(90.2882882882883,99999.9997506754)
(90.3873873873874,99999.9997578174)
(90.4864864864865,99999.9997647613)
(90.5855855855856,99999.9997715125)
(90.6846846846847,99999.999778076)
(90.7837837837838,99999.999784457)
(90.8828828828829,99999.9997906603)
(90.981981981982,99999.9997966906)
(91.0810810810811,99999.9998025528)
(91.1801801801802,99999.9998082512)
(91.2792792792793,99999.9998137903)
(91.3783783783784,99999.9998191744)
(91.4774774774775,99999.9998244077)
(91.5765765765766,99999.9998294942)
(91.6756756756757,99999.999834438)
(91.7747747747748,99999.9998392429)
(91.8738738738739,99999.9998439127)
(91.972972972973,99999.999848451)
(92.0720720720721,99999.9998528615)
(92.1711711711712,99999.9998571475)
(92.2702702702703,99999.9998613126)
(92.3693693693694,99999.9998653599)
(92.4684684684685,99999.9998692928)
(92.5675675675676,99999.9998731143)
(92.6666666666667,99999.9998768275)
(92.7657657657657,99999.9998804353)
(92.8648648648649,99999.9998839407)
(92.9639639639639,99999.9998873464)
(93.0630630630631,99999.9998906552)
(93.1621621621621,99999.9998938698)
(93.2612612612613,99999.9998969928)
(93.3603603603603,99999.9999000266)
(93.4594594594595,99999.9999029737)
(93.5585585585585,99999.9999058366)
(93.6576576576576,99999.9999086176)
(93.7567567567567,99999.9999113189)
(93.8558558558558,99999.9999139427)
(93.9549549549549,99999.9999164912)
(94.054054054054,99999.9999189665)
(94.1531531531531,99999.9999213706)
(94.2522522522522,99999.9999237055)
(94.3513513513513,99999.9999259732)
(94.4504504504504,99999.9999281754)
(94.5495495495495,99999.999930314)
(94.6486486486486,99999.9999323909)
(94.7477477477477,99999.9999344077)
(94.8468468468468,99999.999936366)
(94.9459459459459,99999.9999382677)
(95.045045045045,99999.9999401141)
(95.1441441441441,99999.999941907)
(95.2432432432432,99999.9999436477)
(95.3423423423423,99999.9999453378)
(95.4414414414414,99999.9999469787)
(95.5405405405405,99999.9999485718)
(95.6396396396396,99999.9999501184)
(95.7387387387387,99999.9999516198)
(95.8378378378378,99999.9999530774)
(95.9369369369369,99999.9999544923)
(96.036036036036,99999.9999558657)
(96.1351351351351,99999.9999571989)
(96.2342342342342,99999.999958493)
(96.3333333333333,99999.9999597491)
(96.4324324324324,99999.9999609682)
(96.5315315315315,99999.9999621515)
(96.6306306306306,99999.9999632999)
(96.7297297297297,99999.9999644145)
(96.8288288288288,99999.9999654962)
(96.9279279279279,99999.9999665459)
(97.027027027027,99999.9999675647)
(97.1261261261261,99999.9999685532)
(97.2252252252252,99999.9999695125)
(97.3243243243243,99999.9999704434)
(97.4234234234234,99999.9999713466)
(97.5225225225225,99999.999972223)
(97.6216216216216,99999.9999730733)
(97.7207207207207,99999.9999738984)
(97.8198198198198,99999.9999746988)
(97.9189189189189,99999.9999754754)
(98.018018018018,99999.9999762289)
(98.1171171171171,99999.9999769598)
(98.2162162162162,99999.9999776689)
(98.3153153153153,99999.9999783567)
(98.4144144144144,99999.999979024)
(98.5135135135135,99999.9999796712)
(98.6126126126126,99999.999980299)
(98.7117117117117,99999.999980908)
(98.8108108108108,99999.9999814987)
(98.9099099099099,99999.9999820716)
(99.009009009009,99999.9999826272)
(99.1081081081081,99999.9999831661)
(99.2072072072072,99999.9999836887)
(99.3063063063063,99999.9999841956)
(99.4054054054054,99999.9999846871)
(99.5045045045045,99999.9999851637)
(99.6036036036036,99999.999985626)
(99.7027027027027,99999.9999860742)
(99.8018018018018,99999.9999865088)
(99.9009009009009,99999.9999869302)
(100,99999.9999873388)

};
\addplot [red]
coordinates {
(1,94040.4138215227)
(1.0990990990991,94177.3724368646)
(1.1981981981982,94302.4955254516)
(1.2972972972973,94417.6659993897)
(1.3963963963964,94524.3502930705)
(1.4954954954955,94623.7128003101)
(1.59459459459459,94716.6935200095)
(1.69369369369369,94804.0622509315)
(1.79279279279279,94886.4573528734)
(1.89189189189189,94964.4140654446)
(1.99099099099099,95038.3855884071)
(2.09009009009009,95108.7590362357)
(2.18918918918919,95175.8676934181)
(2.28828828828829,95240.0005543033)
(2.38738738738739,95301.4098389717)
(2.48648648648649,95360.3169795088)
(2.58558558558558,95416.9174356743)
(2.68468468468468,95471.3846043504)
(2.78378378378378,95523.8730200087)
(2.88288288288288,95574.520995101)
(2.98198198198198,95623.4528140208)
(3.08108108108108,95670.7805682591)
(3.18018018018018,95716.6057009449)
(3.27927927927928,95761.0203143057)
(3.37837837837838,95804.1082824196)
(3.47747747747748,95845.9462030495)
(3.57657657657658,95886.6042157024)
(3.67567567567568,95926.1467078592)
(3.77477477477477,95964.6329272314)
(3.87387387387387,96002.1175146648)
(3.97297297297297,96038.6509697167)
(4.07207207207207,96074.2800588664)
(4.17117117117117,96109.0481746349)
(4.27027027027027,96142.9956525338)
(4.36936936936937,96176.1600516485)
(4.46846846846847,96208.5764037516)
(4.56756756756757,96240.2774350885)
(4.66666666666667,96271.2937643555)
(4.76576576576576,96301.6540798744)
(4.86486486486486,96331.3852985296)
(4.96396396396396,96360.5127086794)
(5.06306306306306,96389.0600989378)
(5.16216216216216,96417.0498744749)
(5.26126126126126,96444.5031622564)
(5.36036036036036,96471.4399064638)
(5.45945945945946,96497.8789551726)
(5.55855855855856,96523.8381392336)
(5.65765765765766,96549.3343441836)
(5.75675675675676,96574.3835759115)
(5.85585585585585,96599.0010207202)
(5.95495495495495,96623.201100348)
(6.05405405405405,96646.9975224491)
(6.15315315315315,96670.4033269756)
(6.25225225225225,96693.430928853)
(6.35135135135135,96716.0921573013)
(6.45045045045045,96738.3982921103)
(6.54954954954955,96760.3600971503)
(6.64864864864865,96781.9878513662)
(6.74774774774775,96803.2913774781)
(6.84684684684685,96824.2800685901)
(6.94594594594594,96844.9629128864)
(7.04504504504504,96865.3485165791)
(7.14414414414414,96885.4451252517)
(7.24324324324324,96905.2606437334)
(7.34234234234234,96924.8026546226)
(7.44144144144144,96944.0784355687)
(7.54054054054054,96963.0949754104)
(7.63963963963964,96981.8589892607)
(7.73873873873874,97000.3769326194)
(7.83783783783784,97018.6550145883)
(7.93693693693694,97036.6992102553)
(8.03603603603603,97054.5152723107)
(8.13513513513513,97072.1087419514)
(8.23423423423423,97089.4849591249)
(8.33333333333333,97106.6490721604)
(8.43243243243243,97123.6060468311)
(8.53153153153153,97140.3606748863)
(8.63063063063063,97156.9175820912)
(8.72972972972973,97173.2812358078)
(8.82882882882883,97189.4559521481)
(8.92792792792793,97205.445902727)
(9.02702702702702,97221.2551210443)
(9.12612612612612,97236.8875085161)
(9.22522522522522,97252.3468401826)
(9.32432432432432,97267.6367701091)
(9.42342342342342,97282.760836503)
(9.52252252252252,97297.7224665621)
(9.62162162162162,97312.5249810724)
(9.72072072072072,97327.1715987702)
(9.81981981981982,97341.6654404827)
(9.91891891891892,97356.0095330617)
(10.018018018018,97370.2068131201)
(10.1171171171171,97384.2601305863)
(10.2162162162162,97398.1722520838)
(10.3153153153153,97411.9458641487)
(10.4144144144144,97425.5835762932)
(10.5135135135135,97439.0879239233)
(10.6126126126126,97452.461371121)
(10.7117117117117,97465.7063132955)
(10.8108108108108,97478.8250797139)
(10.9099099099099,97491.8199359155)
(11.009009009009,97504.693086017)
(11.1081081081081,97517.4466749151)
(11.2072072072072,97530.0827903902)
(11.3063063063063,97542.6034651184)
(11.4054054054054,97555.0106785954)
(11.5045045045045,97567.3063589771)
(11.6036036036036,97579.492384841)
(11.7027027027027,97591.5705868737)
(11.8018018018018,97603.5427494858)
(11.9009009009009,97615.4106123607)
(12,97627.1758719382)
(12.0990990990991,97638.8401828373)
(12.1981981981982,97650.4051592213)
(12.2972972972973,97661.8723761075)
(12.3963963963964,97673.2433706242)
(12.4954954954955,97684.5196432175)
(12.5945945945946,97695.7026588112)
(12.6936936936937,97706.7938479202)
(12.7927927927928,97717.7946077216)
(12.8918918918919,97728.7063030836)
(12.990990990991,97739.5302675556)
(13.0900900900901,97750.2678043205)
(13.1891891891892,97760.920187111)
(13.2882882882883,97771.4886610917)
(13.3873873873874,97781.9744437084)
(13.4864864864865,97792.378725506)
(13.5855855855856,97802.7026709168)
(13.6846846846847,97812.9474190197)
(13.7837837837838,97823.1140842721)
(13.8828828828829,97833.2037572158)
(13.981981981982,97843.2175051579)
(14.0810810810811,97853.156372827)
(14.1801801801802,97863.0213830071)
(14.2792792792793,97872.8135371493)
(14.3783783783784,97882.5338159616)
(14.4774774774775,97892.1831799798)
(14.5765765765766,97901.7625701176)
(14.6756756756757,97911.2729081989)
(14.7747747747748,97920.7150974717)
(14.8738738738739,97930.0900231054)
(14.972972972973,97939.3985526709)
(15.0720720720721,97948.6415366051)
(15.1711711711712,97957.8198086609)
(15.2702702702703,97966.934186341)
(15.3693693693694,97975.9854713195)
(15.4684684684685,97984.9744498481)
(15.5675675675676,97993.9018931513)
(15.6666666666667,98002.768557807)
(15.7657657657658,98011.5751861171)
(15.8648648648649,98020.3225064647)
(15.963963963964,98029.0112336614)
(16.0630630630631,98037.6420692835)
(16.1621621621622,98046.2157019974)
(16.2612612612613,98054.732807876)
(16.3603603603604,98063.1940507048)
(16.4594594594595,98071.6000822783)
(16.5585585585586,98079.9515426891)
(16.6576576576577,98088.2490606066)
(16.7567567567568,98096.4932535487)
(16.8558558558559,98104.6847281445)
(16.954954954955,98112.8240803904)
(17.054054054054,98120.9118958977)
(17.1531531531531,98128.9487501337)
(17.2522522522522,98136.9352086555)
(17.3513513513513,98144.8718273371)
(17.4504504504504,98152.7591525907)
(17.5495495495495,98160.5977215803)
(17.6486486486486,98168.3880624312)
(17.7477477477477,98176.1306944318)
(17.8468468468468,98183.826128231)
(17.9459459459459,98191.4748660295)
(18.045045045045,98199.0774017663)
(18.1441441441441,98206.6342212997)
(18.2432432432432,98214.1458025836)
(18.3423423423423,98221.6126158388)
(18.4414414414414,98229.03512372)
(18.5405405405405,98236.4137814781)
(18.6396396396396,98243.7490371181)
(18.7387387387387,98251.0413315531)
(18.8378378378378,98258.2910987542)
(18.9369369369369,98265.4987658958)
(19.036036036036,98272.6647534984)
(19.1351351351351,98279.7894755664)
(19.2342342342342,98286.8733397231)
(19.3333333333333,98293.9167473419)
(19.4324324324324,98300.9200936744)
(19.5315315315315,98307.8837679752)
(19.6306306306306,98314.8081536229)
(19.7297297297297,98321.6936282393)
(19.8288288288288,98328.5405638045)
(19.9279279279279,98335.3493267696)
(20.027027027027,98342.1202781668)
(20.1261261261261,98348.8537737159)
(20.2252252252252,98355.5501639296)
(20.3243243243243,98362.2097942149)
(20.4234234234234,98368.8330049727)
(20.5225225225225,98375.4201316949)
(20.6216216216216,98381.9715050593)
(20.7207207207207,98388.4874510217)
(20.8198198198198,98394.9682909063)
(20.9189189189189,98401.4143414937)
(21.018018018018,98407.8259151074)
(21.1171171171171,98414.203319697)
(21.2162162162162,98420.5468589209)
(21.3153153153153,98426.8568322262)
(21.4144144144144,98433.1335349267)
(21.5135135135135,98439.3772582798)
(21.6126126126126,98445.5882895608)
(21.7117117117117,98451.7669121363)
(21.8108108108108,98457.9134055349)
(21.9099099099099,98464.0280455177)
(22.009009009009,98470.1111041457)
(22.1081081081081,98476.1628498469)
(22.2072072072072,98482.1835474812)
(22.3063063063063,98488.1734584039)
(22.4054054054054,98494.1328405285)
(22.5045045045045,98500.0619483868)
(22.6036036036036,98505.9610331892)
(22.7027027027027,98511.8303428825)
(22.8018018018018,98517.6701222067)
(22.9009009009009,98523.4806127514)
(23,98529.2620530096)
(23.0990990990991,98535.0146784314)
(23.1981981981982,98540.7387214763)
(23.2972972972973,98546.4344116642)
(23.3963963963964,98552.1019756253)
(23.4954954954955,98557.741637149)
(23.5945945945946,98563.3536172323)
(23.6936936936937,98568.9381341261)
(23.7927927927928,98574.4954033816)
(23.8918918918919,98580.025637895)
(23.990990990991,98585.5290479519)
(24.0900900900901,98591.0058412699)
(24.1891891891892,98596.4562230414)
(24.2882882882883,98601.8803959747)
(24.3873873873874,98607.2785603347)
(24.4864864864865,98612.6509139824)
(24.5855855855856,98617.997652414)
(24.6846846846847,98623.3189687993)
(24.7837837837838,98628.6150540184)
(24.8828828828829,98633.886096699)
(24.981981981982,98639.132283252)
(25.0810810810811,98644.3537979067)
(25.1801801801802,98649.5508227453)
(25.2792792792793,98654.7235377367)
(25.3783783783784,98659.87212077)
(25.4774774774775,98664.9967476863)
(25.5765765765766,98670.0975923113)
(25.6756756756757,98675.1748264862)
(25.7747747747748,98680.2286200983)
(25.8738738738739,98685.2591411114)
(25.972972972973,98690.266555595)
(26.0720720720721,98695.2510277532)
(26.1711711711712,98700.2127199533)
(26.2702702702703,98705.1517927538)
(26.3693693693694,98710.0684049311)
(26.4684684684685,98714.9627135069)
(26.5675675675676,98719.8348737744)
(26.6666666666667,98724.6850393237)
(26.7657657657658,98729.5133620678)
(26.8648648648649,98734.319992267)
(26.963963963964,98739.1050785534)
(27.0630630630631,98743.8687679551)
(27.1621621621622,98748.6112059192)
(27.2612612612613,98753.3325363356)
(27.3603603603604,98758.0329015591)
(27.4594594594595,98762.7124424319)
(27.5585585585586,98767.3712983057)
(27.6576576576577,98772.0096070625)
(27.7567567567568,98776.6275051364)
(27.8558558558559,98781.225127534)
(27.954954954955,98785.8026078547)
(28.054054054054,98790.3600783107)
(28.1531531531531,98794.8976697469)
(28.2522522522522,98799.4155116598)
(28.3513513513513,98803.9137322165)
(28.4504504504504,98808.3924582738)
(28.5495495495495,98812.8518153958)
(28.6486486486486,98817.2919278723)
(28.7477477477477,98821.7129187365)
(28.8468468468468,98826.1149097819)
(28.9459459459459,98830.4980215801)
(29.045045045045,98834.8623734969)
(29.1441441441441,98839.2080837091)
(29.2432432432432,98843.5352692207)
(29.3423423423423,98847.8440458786)
(29.4414414414414,98852.1345283887)
(29.5405405405405,98856.4068303308)
(29.6396396396396,98860.6610641742)
(29.7387387387387,98864.8973412922)
(29.8378378378378,98869.1157719769)
(29.9369369369369,98873.3164654536)
(30.036036036036,98877.4995298951)
(30.1351351351351,98881.6650724352)
(30.2342342342342,98885.813199183)
(30.3333333333333,98889.9440152356)
(30.4324324324324,98894.0576246923)
(30.5315315315315,98898.1541306666)
(30.6306306306306,98902.2336352999)
(30.7297297297297,98906.2962397737)
(30.8288288288288,98910.3420443219)
(30.9279279279279,98914.3711482434)
(31.027027027027,98918.3836499136)
(31.1261261261261,98922.3796467968)
(31.2252252252252,98926.3592354574)
(31.3243243243243,98930.3225115715)
(31.4234234234234,98934.2695699383)
(31.5225225225225,98938.200504491)
(31.6216216216216,98942.1154083076)
(31.7207207207207,98946.014373622)
(31.8198198198198,98949.8974918344)
(31.9189189189189,98953.7648535218)
(32.018018018018,98957.6165484479)
(32.1171171171171,98961.4526655737)
(32.2162162162162,98965.2732930672)
(32.3153153153153,98969.0785183129)
(32.4144144144144,98972.868427922)
(32.5135135135135,98976.6431077416)
(32.6126126126126,98980.4026428638)
(32.7117117117117,98984.1471176354)
(32.8108108108108,98987.8766156666)
(32.9099099099099,98991.5912198399)
(33.009009009009,98995.2910123192)
(33.1081081081081,98998.9760745582)
(33.2072072072072,99002.646487309)
(33.3063063063063,99006.3023306305)
(33.4054054054054,99009.9436838966)
(33.5045045045045,99013.5706258045)
(33.6036036036036,99017.1832343825)
(33.7027027027027,99020.7815869984)
(33.8018018018018,99024.3657603665)
(33.9009009009009,99027.9358305561)
(34,99031.4918729986)
(34.0990990990991,99035.0339624949)
(34.1981981981982,99038.5621732232)
(34.2972972972973,99042.0765787456)
(34.3963963963964,99045.5772520159)
(34.4954954954955,99049.0642653863)
(34.5945945945946,99052.5376906141)
(34.6936936936937,99055.997598869)
(34.7927927927928,99059.4440607396)
(34.8918918918919,99062.8771462401)
(34.990990990991,99066.2969248167)
(35.0900900900901,99069.7034653542)
(35.1891891891892,99073.0968361825)
(35.2882882882883,99076.4771050825)
(35.3873873873874,99079.8443392925)
(35.4864864864865,99083.1986055142)
(35.5855855855856,99086.5399699191)
(35.6846846846847,99089.8684981539)
(35.7837837837838,99093.1842553464)
(35.8828828828829,99096.4873061117)
(35.981981981982,99099.7777145575)
(36.0810810810811,99103.0555442897)
(36.1801801801802,99106.3208584183)
(36.2792792792793,99109.5737195624)
(36.3783783783784,99112.8141898557)
(36.4774774774775,99116.042330952)
(36.5765765765766,99119.2582040303)
(36.6756756756757,99122.4618697999)
(36.7747747747748,99125.6533885055)
(36.8738738738739,99128.8328199324)
(36.972972972973,99132.0002234114)
(37.0720720720721,99135.1556578233)
(37.1711711711712,99138.2991816042)
(37.2702702702703,99141.4308527502)
(37.3693693693694,99144.5507288219)
(37.4684684684685,99147.6588669491)
(37.5675675675676,99150.7553238352)
(37.6666666666667,99153.8401557623)
(37.7657657657658,99156.913418595)
(37.8648648648649,99159.9751677848)
(37.963963963964,99163.0254583752)
(38.0630630630631,99166.064345005)
(38.1621621621622,99169.0918819131)
(38.2612612612613,99172.1081229427)
(38.3603603603604,99175.1131215449)
(38.4594594594595,99178.1069307834)
(38.5585585585586,99181.0896033379)
(38.6576576576577,99184.0611915087)
(38.7567567567568,99187.02174722)
(38.8558558558559,99189.9713220241)
(38.9549549549549,99192.9099671051)
(39.054054054054,99195.8377332826)
(39.1531531531531,99198.7546710156)
(39.2522522522522,99201.660830406)
(39.3513513513513,99204.5562612022)
(39.4504504504504,99207.4410128026)
(39.5495495495495,99210.3151342594)
(39.6486486486486,99213.1786742819)
(39.7477477477477,99216.0316812399)
(39.8468468468468,99218.8742031672)
(39.9459459459459,99221.7062877646)
(40.045045045045,99224.5279824039)
(40.1441441441441,99227.3393341305)
(40.2432432432432,99230.1403896669)
(40.3423423423423,99232.9311954159)
(40.4414414414414,99235.7117974638)
(40.5405405405405,99238.4822415833)
(40.6396396396396,99241.2425732367)
(40.7387387387387,99243.9928375789)
(40.8378378378378,99246.7330794604)
(40.9369369369369,99249.4633434304)
(41.036036036036,99252.1836737394)
(41.1351351351351,99254.8941143425)
(41.2342342342342,99257.5947089017)
(41.3333333333333,99260.2855007894)
(41.4324324324324,99262.9665330905)
(41.5315315315315,99265.6378486057)
(41.6306306306306,99268.2994898539)
(41.7297297297297,99270.9514990749)
(41.8288288288288,99273.5939182323)
(41.9279279279279,99276.2267890157)
(42.027027027027,99278.8501528437)
(42.1261261261261,99281.4640508662)
(42.2252252252252,99284.0685239672)
(42.3243243243243,99286.6636127667)
(42.4234234234234,99289.249357624)
(42.5225225225225,99291.8257986394)
(42.6216216216216,99294.3929756572)
(42.7207207207207,99296.9509282675)
(42.8198198198198,99299.4996958092)
(42.9189189189189,99302.0393173716)
(43.018018018018,99304.5698317974)
(43.1171171171171,99307.0912776845)
(43.2162162162162,99309.6036933884)
(43.3153153153153,99312.1071170245)
(43.4144144144144,99314.6015864701)
(43.5135135135135,99317.0871393668)
(43.6126126126126,99319.5638131225)
(43.7117117117117,99322.0316449134)
(43.8108108108108,99324.4906716865)
(43.9099099099099,99326.9409301611)
(44.009009009009,99329.3824568315)
(44.1081081081081,99331.8152879684)
(44.2072072072072,99334.2394596214)
(44.3063063063063,99336.6550076206)
(44.4054054054054,99339.0619675788)
(44.5045045045045,99341.4603748933)
(44.6036036036036,99343.8502647479)
(44.7027027027027,99346.2316721149)
(44.8018018018018,99348.6046317565)
(44.9009009009009,99350.9691782273)
(45,99353.3253458756)
(45.0990990990991,99355.6731688455)
(45.1981981981982,99358.0126810786)
(45.2972972972973,99360.3439163158)
(45.3963963963964,99362.6669080988)
(45.4954954954955,99364.9816897724)
(45.5945945945946,99367.2882944856)
(45.6936936936937,99369.5867551935)
(45.7927927927928,99371.8771046593)
(45.8918918918919,99374.1593754554)
(45.990990990991,99376.4335999655)
(46.0900900900901,99378.6998103858)
(46.1891891891892,99380.9580387271)
(46.2882882882883,99383.2083168159)
(46.3873873873874,99385.4506762963)
(46.4864864864865,99387.6851486314)
(46.5855855855856,99389.9117651048)
(46.6846846846847,99392.1305568222)
(46.7837837837838,99394.341554713)
(46.8828828828829,99396.5447895314)
(46.981981981982,99398.7402918583)
(47.0810810810811,99400.9280921026)
(47.1801801801802,99403.1082205026)
(47.2792792792793,99405.2807071273)
(47.3783783783784,99407.4455818781)
(47.4774774774775,99409.6028744899)
(47.5765765765766,99411.7526145328)
(47.6756756756757,99413.894831413)
(47.7747747747748,99416.0295543746)
(47.8738738738739,99418.1568125008)
(47.972972972973,99420.2766347149)
(48.0720720720721,99422.3890497823)
(48.1711711711712,99424.4940863109)
(48.2702702702703,99426.5917727531)
(48.3693693693694,99428.6821374069)
(48.4684684684685,99430.7652084168)
(48.5675675675676,99432.8410137754)
(48.6666666666667,99434.9095813247)
(48.7657657657658,99436.9709387569)
(48.8648648648649,99439.025113616)
(48.963963963964,99441.0721332986)
(49.0630630630631,99443.1120250556)
(49.1621621621622,99445.1448159928)
(49.2612612612613,99447.1705330726)
(49.3603603603604,99449.1892031146)
(49.4594594594595,99451.200852797)
(49.5585585585586,99453.205508658)
(49.6576576576577,99455.2031970963)
(49.7567567567568,99457.1939443725)
(49.8558558558559,99459.1777766105)
(49.9549549549549,99461.1547197978)
(50.054054054054,99463.1247997875)
(50.1531531531531,99465.0880422985)
(50.2522522522522,99467.0444729171)
(50.3513513513513,99468.9941170977)
(50.4504504504504,99470.9370001642)
(50.5495495495495,99472.8731473106)
(50.6486486486486,99474.8025836021)
(50.7477477477477,99476.7253339764)
(50.8468468468468,99478.6414232441)
(50.9459459459459,99480.5508760903)
(51.045045045045,99482.4537170752)
(51.1441441441441,99484.349970635)
(51.2432432432432,99486.239661083)
(51.3423423423423,99488.1228126107)
(51.4414414414414,99489.9994492883)
(51.5405405405405,99491.8695950657)
(51.6396396396396,99493.7332737739)
(51.7387387387387,99495.5905091252)
(51.8378378378378,99497.4413247145)
(51.9369369369369,99499.2857440203)
(52.036036036036,99501.1237904051)
(52.1351351351351,99502.9554871166)
(52.2342342342342,99504.7808572886)
(52.3333333333333,99506.5999239414)
(52.4324324324324,99508.4127099834)
(52.5315315315315,99510.219238211)
(52.6306306306306,99512.0195313104)
(52.7297297297297,99513.8136118576)
(52.8288288288288,99515.6015023195)
(52.9279279279279,99517.3832250549)
(53.027027027027,99519.158802315)
(53.1261261261261,99520.9282562444)
(53.2252252252252,99522.6916088816)
(53.3243243243243,99524.44888216)
(53.4234234234234,99526.2000979086)
(53.5225225225225,99527.9452778528)
(53.6216216216216,99529.684443615)
(53.7207207207207,99531.4176167155)
(53.8198198198198,99533.144818573)
(53.9189189189189,99534.8660705057)
(54.018018018018,99536.5813937317)
(54.1171171171171,99538.2908093699)
(54.2162162162162,99539.9943384403)
(54.3153153153153,99541.6920018654)
(54.4144144144144,99543.3838204703)
(54.5135135135135,99545.0698149837)
(54.6126126126126,99546.7500060383)
(54.7117117117117,99548.4244141717)
(54.8108108108108,99550.0930598272)
(54.9099099099099,99551.7559633539)
(55.009009009009,99553.413145008)
(55.1081081081081,99555.0646249532)
(55.2072072072072,99556.710423261)
(55.3063063063063,99558.3505599119)
(55.4054054054054,99559.9850547957)
(55.5045045045045,99561.6139277121)
(55.6036036036036,99563.2371983716)
(55.7027027027027,99564.8548863958)
(55.8018018018018,99566.4670113181)
(55.9009009009009,99568.0735925844)
(56,99569.6746495536)
(56.0990990990991,99571.2702014982)
(56.1981981981982,99572.860267605)
(56.2972972972973,99574.4448669754)
(56.3963963963964,99576.0240186264)
(56.4954954954955,99577.5977414907)
(56.5945945945946,99579.1660544176)
(56.6936936936937,99580.7289761734)
(56.7927927927928,99582.286525442)
(56.8918918918919,99583.8387208256)
(56.990990990991,99585.3855808449)
(57.0900900900901,99586.9271239399)
(57.1891891891892,99588.4633684704)
(57.2882882882883,99589.9943327165)
(57.3873873873874,99591.520034879)
(57.4864864864865,99593.0404930803)
(57.5855855855856,99594.5557253645)
(57.6846846846847,99596.065749698)
(57.7837837837838,99597.5705839701)
(57.8828828828829,99599.0702459937)
(57.981981981982,99600.5647535053)
(58.0810810810811,99602.0541241659)
(58.1801801801802,99603.5383755613)
(58.2792792792793,99605.0175252028)
(58.3783783783784,99606.4915905274)
(58.4774774774775,99607.9605888983)
(58.5765765765766,99609.4245376056)
(58.6756756756757,99610.8834538667)
(58.7747747747748,99612.3373548266)
(58.8738738738739,99613.7862575585)
(58.972972972973,99615.2301790642)
(59.0720720720721,99616.6691362745)
(59.1711711711712,99618.1031460499)
(59.2702702702703,99619.5322251807)
(59.3693693693694,99620.9563903877)
(59.4684684684685,99622.3756583225)
(59.5675675675676,99623.790045568)
(59.6666666666667,99625.1995686388)
(59.7657657657658,99626.6042439815)
(59.8648648648649,99628.0040879754)
(59.963963963964,99629.3991169328)
(60.0630630630631,99630.7893470992)
(60.1621621621622,99632.174794654)
(60.2612612612613,99633.5554757109)
(60.3603603603604,99634.9314063179)
(60.4594594594595,99636.3026024584)
(60.5585585585586,99637.6690800509)
(60.6576576576577,99639.0308549498)
(60.7567567567568,99640.3879429457)
(60.8558558558559,99641.7403597656)
(60.9549549549549,99643.0881210739)
(61.054054054054,99644.4312424717)
(61.1531531531531,99645.7697394984)
(61.2522522522522,99647.1036276312)
(61.3513513513513,99648.4329222857)
(61.4504504504504,99649.7576388166)
(61.5495495495495,99651.0777925174)
(61.6486486486486,99652.3933986215)
(61.7477477477477,99653.7044723022)
(61.8468468468468,99655.0110286727)
(61.9459459459459,99656.3130827873)
(62.045045045045,99657.6106496409)
(62.1441441441441,99658.9037441699)
(62.2432432432432,99660.1923812525)
(62.3423423423423,99661.4765757085)
(62.4414414414414,99662.7563423004)
(62.5405405405405,99664.0316957333)
(62.6396396396396,99665.3026506552)
(62.7387387387387,99666.5692216576)
(62.8378378378378,99667.8314232755)
(62.9369369369369,99669.089269988)
(63.036036036036,99670.3427762185)
(63.1351351351351,99671.5919563351)
(63.2342342342342,99672.8368246508)
(63.3333333333333,99674.0773954239)
(63.4324324324324,99675.3136828581)
(63.5315315315315,99676.5457011032)
(63.6306306306306,99677.7734642552)
(63.7297297297297,99678.9969863565)
(63.8288288288288,99680.2162813963)
(63.9279279279279,99681.431363311)
(64.027027027027,99682.6422459844)
(64.1261261261261,99683.8489432479)
(64.2252252252252,99685.0514688809)
(64.3243243243243,99686.2498366112)
(64.4234234234234,99687.444060115)
(64.5225225225225,99688.6341530174)
(64.6216216216216,99689.8201288928)
(64.7207207207207,99691.0020012648)
(64.8198198198198,99692.1797836069)
(64.9189189189189,99693.3534893423)
(65.018018018018,99694.5231318447)
(65.1171171171171,99695.6887244381)
(65.2162162162162,99696.8502803976)
(65.3153153153153,99698.007812949)
(65.4144144144144,99699.1613352697)
(65.5135135135135,99700.3108604885)
(65.6126126126126,99701.4564016862)
(65.7117117117117,99702.5979718955)
(65.8108108108108,99703.7355841017)
(65.9099099099099,99704.8692512426)
(66.009009009009,99705.9989862089)
(66.1081081081081,99707.1248018443)
(66.2072072072072,99708.2467109462)
(66.3063063063063,99709.3647262652)
(66.4054054054054,99710.4788605061)
(66.5045045045045,99711.5891263277)
(66.6036036036036,99712.6955363431)
(66.7027027027027,99713.7981031201)
(66.8018018018018,99714.8968391813)
(66.9009009009009,99715.9917570042)
(67,99717.0828690219)
(67.0990990990991,99718.1701876228)
(67.1981981981982,99719.2537251513)
(67.2972972972973,99720.3334939075)
(67.3963963963964,99721.4095061479)
(67.4954954954955,99722.4817740856)
(67.5945945945946,99723.55030989)
(67.6936936936937,99724.6151256877)
(67.7927927927928,99725.6762335623)
(67.8918918918919,99726.7336455548)
(67.990990990991,99727.7873736637)
(68.0900900900901,99728.8374298453)
(68.1891891891892,99729.8838260139)
(68.2882882882883,99730.9265740418)
(68.3873873873874,99731.9656857601)
(68.4864864864865,99733.0011729582)
(68.5855855855856,99734.0330473844)
(68.6846846846847,99735.0613207461)
(68.7837837837838,99736.0860047099)
(68.8828828828829,99737.1071109017)
(68.981981981982,99738.1246509073)
(69.0810810810811,99739.1386362722)
(69.1801801801802,99740.1490785019)
(69.2792792792793,99741.1559890622)
(69.3783783783784,99742.1593793793)
(69.4774774774775,99743.1592608401)
(69.5765765765766,99744.1556447923)
(69.6756756756757,99745.1485425445)
(69.7747747747748,99746.1379653665)
(69.8738738738739,99747.1239244898)
(69.972972972973,99748.106431107)
(70.0720720720721,99749.0854963729)
(70.1711711711712,99750.061131404)
(70.2702702702703,99751.0333472789)
(70.3693693693694,99752.0021550387)
(70.4684684684685,99752.9675656869)
(70.5675675675676,99753.9295901896)
(70.6666666666667,99754.8882394759)
(70.7657657657657,99755.8435244378)
(70.8648648648649,99756.7954559305)
(70.9639639639639,99757.7440447726)
(71.0630630630631,99758.6893017464)
(71.1621621621621,99759.6312375977)
(71.2612612612613,99760.5698630364)
(71.3603603603603,99761.5051887362)
(71.4594594594595,99762.4372253352)
(71.5585585585585,99763.3659834359)
(71.6576576576576,99764.2914736053)
(71.7567567567567,99765.2137063753)
(71.8558558558558,99766.1326922423)
(71.9549549549549,99767.0484416681)
(72.054054054054,99767.9609650797)
(72.1531531531531,99768.8702728692)
(72.2522522522522,99769.7763753945)
(72.3513513513513,99770.6792829792)
(72.4504504504504,99771.5790059124)
(72.5495495495495,99772.4755544495)
(72.6486486486486,99773.3689388121)
(72.7477477477477,99774.2591691879)
(72.8468468468468,99775.1462557312)
(72.9459459459459,99776.0302085627)
(73.045045045045,99776.9110377703)
(73.1441441441441,99777.7887534083)
(73.2432432432432,99778.6633654983)
(73.3423423423423,99779.5348840292)
(73.4414414414414,99780.403318957)
(73.5405405405405,99781.2686802054)
(73.6396396396396,99782.1309776657)
(73.7387387387387,99782.9902211969)
(73.8378378378378,99783.8464206259)
(73.9369369369369,99784.6995857478)
(74.036036036036,99785.5497263258)
(74.1351351351351,99786.3968520915)
(74.2342342342342,99787.240972745)
(74.3333333333333,99788.0820979548)
(74.4324324324324,99788.9202373585)
(74.5315315315315,99789.7554005624)
(74.6306306306306,99790.5875971416)
(74.7297297297297,99791.4168366409)
(74.8288288288288,99792.2431285739)
(74.9279279279279,99793.0664824238)
(75.027027027027,99793.8869076434)
(75.1261261261261,99794.7044136551)
(75.2252252252252,99795.5190098511)
(75.3243243243243,99796.3307055937)
(75.4234234234234,99797.139510215)
(75.5225225225225,99797.9454330175)
(75.6216216216216,99798.7484832739)
(75.7207207207207,99799.5486702275)
(75.8198198198198,99800.3460030919)
(75.9189189189189,99801.1404910516)
(76.018018018018,99801.9321432619)
(76.1171171171171,99802.720968849)
(76.2162162162162,99803.50697691)
(76.3153153153153,99804.2901765135)
(76.4144144144144,99805.0705766991)
(76.5135135135135,99805.8481864778)
(76.6126126126126,99806.6230148325)
(76.7117117117117,99807.3950707173)
(76.8108108108108,99808.1643630582)
(76.9099099099099,99808.9309007533)
(77.009009009009,99809.6946926724)
(77.1081081081081,99810.4557476576)
(77.2072072072072,99811.214074523)
(77.3063063063063,99811.9696820552)
(77.4054054054054,99812.7225790133)
(77.5045045045045,99813.4727741287)
(77.6036036036036,99814.2202761057)
(77.7027027027027,99814.9650936213)
(77.8018018018018,99815.7072353253)
(77.9009009009009,99816.4467098407)
(78,99817.1835257632)
(78.0990990990991,99817.9176916622)
(78.1981981981982,99818.64921608)
(78.2972972972973,99819.3781075326)
(78.3963963963964,99820.1043745092)
(78.4954954954955,99820.828025473)
(78.5945945945946,99821.5490688606)
(78.6936936936937,99822.2675130826)
(78.7927927927928,99822.9833665235)
(78.8918918918919,99823.6966375419)
(78.990990990991,99824.4073344703)
(79.0900900900901,99825.1154656157)
(79.1891891891892,99825.8210392593)
(79.2882882882883,99826.5240636568)
(79.3873873873874,99827.2245470382)
(79.4864864864865,99827.9224976085)
(79.5855855855856,99828.6179235472)
(79.6846846846847,99829.3108330085)
(79.7837837837838,99830.0012341218)
(79.8828828828829,99830.6891349913)
(79.981981981982,99831.3745436965)
(80.0810810810811,99832.0574682919)
(80.1801801801802,99832.7379168073)
(80.2792792792793,99833.4158972481)
(80.3783783783784,99834.091417595)
(80.4774774774775,99834.7644858042)
(80.5765765765766,99835.4351098078)
(80.6756756756757,99836.1032975135)
(80.7747747747748,99836.7690568049)
(80.8738738738739,99837.4323955414)
(80.972972972973,99838.0933215586)
(81.0720720720721,99838.7518426681)
(81.1711711711712,99839.4079666577)
(81.2702702702703,99840.0617012917)
(81.3693693693694,99840.7130543104)
(81.4684684684685,99841.3620334307)
(81.5675675675676,99842.0086463463)
(81.6666666666667,99842.6529007273)
(81.7657657657657,99843.2948042204)
(81.8648648648649,99843.9343644493)
(81.9639639639639,99844.5715890146)
(82.0630630630631,99845.2064854937)
(82.1621621621621,99845.8390614412)
(82.2612612612613,99846.4693243888)
(82.3603603603603,99847.0972818454)
(82.4594594594595,99847.7229412972)
(82.5585585585585,99848.3463102076)
(82.6576576576576,99848.9673960179)
(82.7567567567567,99849.5862061464)
(82.8558558558558,99850.2027479895)
(82.9549549549549,99850.8170289211)
(83.054054054054,99851.4290562927)
(83.1531531531531,99852.0388374338)
(83.2522522522522,99852.6463796521)
(83.3513513513513,99853.2516902328)
(83.4504504504504,99853.8547764396)
(83.5495495495495,99854.4556455141)
(83.6486486486486,99855.0543046763)
(83.7477477477477,99855.6507611246)
(83.8468468468468,99856.2450220354)
(83.9459459459459,99856.837094564)
(84.045045045045,99857.4269858441)
(84.1441441441441,99858.0147029878)
(84.2432432432432,99858.6002530862)
(84.3423423423423,99859.183643209)
(84.4414414414414,99859.7648804048)
(84.5405405405405,99860.3439717009)
(84.6396396396396,99860.920924104)
(84.7387387387387,99861.4957445994)
(84.8378378378378,99862.0684401517)
(84.9369369369369,99862.6390177049)
(85.036036036036,99863.2074841819)
(85.1351351351351,99863.7738464851)
(85.2342342342342,99864.3381114964)
(85.3333333333333,99864.900286077)
(85.4324324324324,99865.4603770677)
(85.5315315315315,99866.018391289)
(85.6306306306306,99866.5743355409)
(85.7297297297297,99867.1282166034)
(85.8288288288288,99867.680041236)
(85.9279279279279,99868.2298161783)
(86.027027027027,99868.7775481497)
(86.1261261261261,99869.3232438498)
(86.2252252252252,99869.8669099581)
(86.3243243243243,99870.4085531343)
(86.4234234234234,99870.9481800184)
(86.5225225225225,99871.4857972306)
(86.6216216216216,99872.0214113713)
(86.7207207207207,99872.5550290215)
(86.8198198198198,99873.0866567427)
(86.9189189189189,99873.6163010766)
(87.018018018018,99874.143968546)
(87.1171171171171,99874.6696656539)
(87.2162162162162,99875.1933988843)
(87.3153153153153,99875.7151747017)
(87.4144144144144,99876.2349995518)
(87.5135135135135,99876.7528798611)
(87.6126126126126,99877.2688220367)
(87.7117117117117,99877.7828324673)
(87.8108108108108,99878.2949175223)
(87.9099099099099,99878.8050835524)
(88.009009009009,99879.3133368893)
(88.1081081081081,99879.8196838464)
(88.2072072072072,99880.3241307179)
(88.3063063063063,99880.8266837799)
(88.4054054054054,99881.3273492895)
(88.5045045045045,99881.8261334856)
(88.6036036036036,99882.3230425885)
(88.7027027027027,99882.8180828001)
(88.8018018018018,99883.3112603043)
(88.9009009009009,99883.8025812663)
(89,99884.2920518334)
(89.0990990990991,99884.7796781345)
(89.1981981981982,99885.2654662807)
(89.2972972972973,99885.7494223648)
(89.3963963963964,99886.2315524617)
(89.4954954954955,99886.7118626285)
(89.5945945945946,99887.1903589043)
(89.6936936936937,99887.6670473103)
(89.7927927927928,99888.1419338502)
(89.8918918918919,99888.6150245098)
(89.990990990991,99889.0863252573)
(90.0900900900901,99889.5558420433)
(90.1891891891892,99890.0235808008)
(90.2882882882883,99890.4895474454)
(90.3873873873874,99890.9537478752)
(90.4864864864865,99891.416187971)
(90.5855855855856,99891.8768735963)
(90.6846846846847,99892.335810597)
(90.7837837837838,99892.7930048022)
(90.8828828828829,99893.2484620237)
(90.981981981982,99893.7021880559)
(91.0810810810811,99894.1541886765)
(91.1801801801802,99894.604469646)
(91.2792792792793,99895.053036708)
(91.3783783783784,99895.4998955891)
(91.4774774774775,99895.9450519992)
(91.5765765765766,99896.3885116311)
(91.6756756756757,99896.8302801611)
(91.7747747747748,99897.2703632487)
(91.8738738738739,99897.7087665367)
(91.972972972973,99898.1454956513)
(92.0720720720721,99898.5805562022)
(92.1711711711712,99899.0139537825)
(92.2702702702703,99899.4456939689)
(92.3693693693694,99899.8757823216)
(92.4684684684685,99900.3042243845)
(92.5675675675676,99900.731025685)
(92.6666666666667,99901.1561917347)
(92.7657657657657,99901.5797280283)
(92.8648648648649,99902.0016400449)
(92.9639639639639,99902.4219332471)
(93.0630630630631,99902.8406130817)
(93.1621621621621,99903.2576849792)
(93.2612612612613,99903.6731543542)
(93.3603603603603,99904.0870266055)
(93.4594594594595,99904.4993071158)
(93.5585585585585,99904.9100012521)
(93.6576576576576,99905.3191143654)
(93.7567567567567,99905.7266517911)
(93.8558558558558,99906.132618849)
(93.9549549549549,99906.5370208429)
(94.054054054054,99906.9398630612)
(94.1531531531531,99907.3411507767)
(94.2522522522522,99907.7408892466)
(94.3513513513513,99908.1390837128)
(94.4504504504504,99908.5357394014)
(94.5495495495495,99908.9308615234)
(94.6486486486486,99909.3244552744)
(94.7477477477477,99909.7165258345)
(94.8468468468468,99910.1070783688)
(94.9459459459459,99910.4961180269)
(95.045045045045,99910.8836499435)
(95.1441441441441,99911.2696792379)
(95.2432432432432,99911.6542110145)
(95.3423423423423,99912.0372503625)
(95.4414414414414,99912.4188023562)
(95.5405405405405,99912.7988720548)
(95.6396396396396,99913.1774645027)
(95.7387387387387,99913.5545847295)
(95.8378378378378,99913.9302377496)
(95.9369369369369,99914.3044285629)
(96.036036036036,99914.6771621546)
(96.1351351351351,99915.0484434948)
(96.2342342342342,99915.4182775394)
(96.3333333333333,99915.7866692293)
(96.4324324324324,99916.153623491)
(96.5315315315315,99916.5191452364)
(96.6306306306306,99916.8832393628)
(96.7297297297297,99917.245910753)
(96.8288288288288,99917.6071642757)
(96.9279279279279,99917.9670047848)
(97.027027027027,99918.3254371201)
(97.1261261261261,99918.6824661069)
(97.2252252252252,99919.0380965564)
(97.3243243243243,99919.3923332655)
(97.4234234234234,99919.7451810169)
(97.5225225225225,99920.096644579)
(97.6216216216216,99920.4467287064)
(97.7207207207207,99920.7954381394)
(97.8198198198198,99921.1427776043)
(97.9189189189189,99921.4887518133)
(98.018018018018,99921.8333654649)
(98.1171171171171,99922.1766232435)
(98.2162162162162,99922.5185298195)
(98.3153153153153,99922.8590898496)
(98.4144144144144,99923.1983079768)
(98.5135135135135,99923.5361888302)
(98.6126126126126,99923.872737025)
(98.7117117117117,99924.207957163)
(98.8108108108108,99924.5418538322)
(98.9099099099099,99924.8744316069)
(99.009009009009,99925.205695048)
(99.1081081081081,99925.5356487027)
(99.2072072072072,99925.8642971047)
(99.3063063063063,99926.1916447743)
(99.4054054054054,99926.5176962184)
(99.5045045045045,99926.8424559303)
(99.6036036036036,99927.1659283901)
(99.7027027027027,99927.4881180645)
(99.8018018018018,99927.809029407)
(99.9009009009009,99928.1286668576)
(100,99928.4470348434)

};
\addplot [blue, opacity=0.25, mark=square*, mark size=3, mark options={draw=black}, only marks]
coordinates {
(0,95643.8)
(1,97771)
(2,98668.5)
(3,99128.7)
(4,99405.7)
(5,99588.3)
(6,99707.3)
(7,99791.4)
(8,99851)
(9,99893.4)
(10,99923.6)
(11,99945.3)
(12,99960.8)
(13,99972.1)
(14,99979.9)
(15,99985.4)
(16,99989.5)
(17,99992.5)
(18,99994.7)
(19,99996.1)
(20,99997.2)
(21,99998)
(22,99998.5)
(23,99999)
(24,99999.3)
(25,99999.5)
(26,99999.6)
(27,99999.7)
(28,99999.8)
(29,99999.9)
(30,99999.9)
(31,99999.9)
(32,99999.9)
(33,100000)
(34,100000)
(35,100000)
(36,100000)
(37,100000)
(38,100000)
(39,100000)
(40,100000)
(41,100000)
(42,100000)
(43,100000)
(44,100000)
(45,100000)
(46,100000)
(47,100000)
(48,100000)
(49,100000)
(50,100000)
(51,100000)
(52,100000)
(53,100000)
(54,100000)
(55,100000)
(56,100000)
(57,100000)
(58,100000)
(59,100000)
(60,100000)
(61,100000)
(62,100000)
(63,100000)
(64,100000)
(65,100000)
(66,100000)
(67,100000)
(68,100000)
(69,100000)
(70,100000)
(71,100000)
(72,100000)
(73,100000)
(74,100000)
(75,100000)
(76,100000)
(77,100000)
(78,100000)
(79,100000)
(80,100000)
(81,100000)
(82,100000)
(83,100000)
(84,100000)
(85,100000)
(86,100000)
(87,100000)
(88,100000)
(89,100000)
(90,100000)
(91,100000)
(92,100000)
(93,100000)
(94,100000)
(95,100000)
(96,100000)
(97,100000)
(98,100000)
(99,100000)
(100,100000)

};
\addplot [green!50.0!black, opacity=0.25, mark=square*, mark size=3, mark options={draw=black}, only marks]
coordinates {
(0,94024.6)
(1,96180.7)
(2,97130.6)
(3,97662.4)
(4,98024.1)
(5,98299.1)
(6,98514.2)
(7,98693.9)
(8,98846.7)
(9,98978.3)
(10,99092.8)
(11,99193.3)
(12,99282.1)
(13,99361.1)
(14,99430.9)
(15,99492.3)
(16,99547.6)
(17,99597.7)
(18,99642.7)
(19,99681.6)
(20,99717.4)
(21,99749.9)
(22,99777.6)
(23,99803.5)
(24,99826.2)
(25,99846.3)
(26,99865)
(27,99880.6)
(28,99895.6)
(29,99907.7)
(30,99919.5)
(31,99929.1)
(32,99938.4)
(33,99945.8)
(34,99953.2)
(35,99958.8)
(36,99964.5)
(37,99968.9)
(38,99973.1)
(39,99976.8)
(40,99979.8)
(41,99982.8)
(42,99985)
(43,99987.1)
(44,99988.9)
(45,99990.4)
(46,99991.9)
(47,99993)
(48,99994)
(49,99995)
(50,99995.6)
(51,99996.3)
(52,99996.9)
(53,99997.3)
(54,99997.7)
(55,99998.1)
(56,99998.4)
(57,99998.6)
(58,99998.9)
(59,99999)
(60,99999.2)
(61,99999.3)
(62,99999.4)
(63,99999.5)
(64,99999.6)
(65,99999.7)
(66,99999.7)
(67,99999.8)
(68,99999.8)
(69,99999.8)
(70,99999.9)
(71,99999.9)
(72,99999.9)
(73,99999.9)
(74,99999.9)
(75,99999.9)
(76,100000)
(77,100000)
(78,100000)
(79,100000)
(80,100000)
(81,100000)
(82,100000)
(83,100000)
(84,100000)
(85,100000)
(86,100000)
(87,100000)
(88,100000)
(89,100000)
(90,100000)
(91,100000)
(92,100000)
(93,100000)
(94,100000)
(95,100000)
(96,100000)
(97,100000)
(98,100000)
(99,100000)
(100,100000)

};
\addplot [red, opacity=0.25, mark=square*, mark size=3, mark options={draw=black}, only marks]
coordinates {
(0,92572.9)
(1,94730.4)
(2,95683.1)
(3,96219.4)
(4,96587.1)
(5,96869.9)
(6,97094.6)
(7,97285.5)
(8,97451)
(9,97596.7)
(10,97726.8)
(11,97844.2)
(12,97951.2)
(13,98049.5)
(14,98139.7)
(15,98222.7)
(16,98300.4)
(17,98373.6)
(18,98442.4)
(19,98506)
(20,98566.7)
(21,98624.7)
(22,98678.4)
(23,98730.3)
(24,98779.3)
(25,98825.9)
(26,98871.1)
(27,98913)
(28,98954.4)
(29,98992.7)
(30,99030.6)
(31,99065.9)
(32,99100.8)
(33,99133.3)
(34,99165.7)
(35,99195.5)
(36,99225.4)
(37,99253.2)
(38,99280.6)
(39,99306.9)
(40,99332)
(41,99356.9)
(42,99379.9)
(43,99402.8)
(44,99424.8)
(45,99445.7)
(46,99466.7)
(47,99486)
(48,99505.1)
(49,99524.1)
(50,99541.5)
(51,99558.8)
(52,99575.9)
(53,99591.6)
(54,99607.3)
(55,99623)
(56,99637.2)
(57,99651.3)
(58,99665.5)
(59,99678.5)
(60,99691.3)
(61,99704.1)
(62,99716.3)
(63,99727.8)
(64,99739.2)
(65,99750.7)
(66,99761.1)
(67,99771.4)
(68,99781.7)
(69,99791.7)
(70,99800.9)
(71,99810.1)
(72,99819.3)
(73,99828.2)
(74,99836.4)
(75,99844.7)
(76,99852.9)
(77,99860.8)
(78,99868.2)
(79,99875.6)
(80,99883)
(81,99890.3)
(82,99896.9)
(83,99903.6)
(84,99910.2)
(85,99916.8)
(86,99923.1)
(87,99929.1)
(88,99935.1)
(89,99941.1)
(90,99947.1)
(91,99952.6)
(92,99958)
(93,99963.5)
(94,99969)
(95,99974.5)
(96,99979.6)
(97,99984.7)
(98,99989.8)
(99,99994.9)
(100,100000)

};
\path [draw=black, fill opacity=0] (axis cs:13,100000)--(axis cs:13,100000);

\path [draw=black, fill opacity=0] (axis cs:100,13)--(axis cs:100,13);

\path [draw=black, fill opacity=0] (axis cs:13,92000)--(axis cs:13,92000);

\path [draw=black, fill opacity=0] (axis cs:0,13)--(axis cs:0,13);

\end{axis}

\end{tikzpicture}

	\end{center}
	\caption{The solution to the Theis problem}
	\label{fig:Theis}
\end{figure}
